\input texinfo  @c -*-texinfo-*-
@setfilename ../info/emacs
@ifinfo
This file documents the GNU Emacs editor.

Copyright (C) 1985 Richard M. Stallman.

Permission is granted to make and distribute verbatim copies of
this manual provided the copyright notice and this permission notice
are preserved on all copies.

@ignore
Permission is granted to process this file through Tex and print the
results, provided the printed document carries copying permission
notice identical to this one except for the removal of this paragraph
(this paragraph not being relevant to the printed manual).

@end ignore
Permission is granted to copy and distribute modified versions of this
manual under the conditions for verbatim copying, provided also that the
sections entitled ``The GNU Manifesto'', ``Distribution'' and ``GNU Emacs
General Public License'' are included exactly as in the original, and
provided that the entire resulting derived work is distributed under the
terms of a permission notice identical to this one.

Permission is granted to copy and distribute translations of this manual
into another language, under the above conditions for modified versions,
except that the sections entitled ``The GNU Manifesto'', ``Distribution''
and ``GNU Emacs General Public License'' may be included in a translation
approved by the author instead of in the original English.
@end ifinfo
@c
@setchapternewpage odd
@settitle GNU Emacs Manual
@c
@titlepage
@sp 6
@center @titlefont{GNU Emacs Manual}
@sp 4
@center Fourth Edition, Emacs Version 17
@sp 1
@center February 1986
@sp 5
@center Richard Stallman
@page
@vskip 0pt plus 1filll
Copyright @copyright{} 1985 Richard M. Stallman.

Permission is granted to make and distribute verbatim copies of
this manual provided the copyright notice and this permission notice
are preserved on all copies.

Permission is granted to copy and distribute modified versions of this
manual under the conditions for verbatim copying, provided also that the
sections entitled ``The GNU Manifesto'', ``Distribution'' and ``GNU Emacs
General Public License'' are included exactly as in the original, and
provided that the entire resulting derived work is distributed under the
terms of a permission notice identical to this one.

Permission is granted to copy and distribute translations of this manual
into another language, under the above conditions for modified versions,
except that the sections entitled ``The GNU Manifesto'', ``Distribution''
and ``GNU Emacs General Public License'' may be included in a translation
approved by the author instead of in the original English.
@end titlepage
@page
@ifinfo
@node Top, Distrib,, (DIR)

The Emacs Editor
****************

Emacs is the extensible, customizable, self-documenting real-time
display editor.  This Info file describes how to edit with Emacs
and some of how to customize it, but not how to extend it.

@end ifinfo
@menu
* Distrib::     How to get the latest Emacs distribution.
* License::     The GNU Emacs General Public License gives you permission
		to redistribute GNU Emacs on certain terms; and also
		explains that there is no warranty.
* Intro::       An introduction to Emacs concepts.
* Glossary::    The glossary.
* Manifesto::   What's GNU?  Gnu's Not Unix!

Indexes, nodes containing large menus
* Key Index::      An item for each standard Emacs key sequence.
* Command Index::  An item for each command name.
* Variable Index:: An item for each documented variable.
* Concept Index::  An item for each concept.

Important General Concepts
* Screen::      How to interpret what you see on the screen.
* Characters::  Emacs's character sets for file contents and for keyboard.
* Keys::        Key sequences: what you type to request one editing action.
* Commands::    Commands: named functions run by key sequences to do editing.
* Entering Emacs::  Starting Emacs from the shell.
* Exiting::     Stopping or killing Emacs.
* Basic::       The most basic editing commands.
* Undo::        Undoing recently made changes in the text.
* Minibuffer::  Entering arguments that are prompted for.
* M-x::         Invoking commands by their names.
* Help::        Commands for asking Emacs about its commands.

Important Text-Changing Commands
* Mark::        The mark: how to delimit a ``region'' of text.
* Killing::     Killing text.
* Yanking::     Recovering killed text.  Moving text.
* Accumulating Text::
                Other ways of copying text.
* Rectangles::  Operating on the text inside a rectangle on the screen.
* Registers::   Saving a text string or a location in the buffer.
* Display::     Controlling what text is displayed.
* Search::      Finding or replacing occurrences of a string.
* Fixit::       Commands especially useful for fixing typos.

Larger Units of Text
* Files::       All about handling files.
* Buffers::     Multiple buffers; editing several files at once.
* Windows::     Viewing two pieces of text at once.

Advanced Features
* Major Modes:: Text mode vs. Lisp mode vs. C mode ...
* Indentation:: Editing the white space at the beginnings of lines.
* Text::        Commands and modes for editing English.
* Programs::    Commands and modes for editing programs.
* Running::     Compiling, running and debugging programs.
* Abbrevs::     How to define text abbreviations to reduce
                 the number of characters you must type.
* Picture::     Editing pictures made up of characters
                 using the quarter-plane screen model.
* Sending Mail::Sending mail in Emacs.
* Rmail::       Reading mail in Emacs.
* Recursive Edit::
                A command can allow you to do editing
                 "within the command".  This is called a
                 `recursive editing level'.
* Narrowing::   Restricting display and editing to a portion
                 of the buffer.
* Shell::       Executing shell commands from Emacs.
* Dissociated Press::  Dissociating text for fun.
* Amusements::         Various games and hacks.
* Customization::      Modifying the behavior of Emacs.

Recovery from Problems.
* Quitting::    Quitting and aborting.
* Lossage::     What to do if Emacs is hung or malfunctioning.
* Bugs::        How and when to report a bug.

Here are some other nodes which are really inferiors of the ones
already listed, mentioned here so you can get to them in one step:

Subnodes of Screen
* Point::	The place in the text where editing commands operate.
* Echo Area::   Short messages appear at the bottom of the screen.
* Mode Line::	Interpreting the mode line.

Subnodes of Basic
* Blank Lines:: Commands to make or delete blank lines.
* Continuation Lines:: Lines too wide for the screen.
* Position Info::      What page, line, row, or column is point on?
* Arguments::   Giving numeric arguments to commands.

Subnodes of Minibuffer
* Minibuffer File::    Entering file names with the minibuffer.
* Minibuffer Edit::    How to edit in the minibuffer.
* Completion::  An abbreviation facility for minibuffer input.
* Repetition::  Re-executing previous commands that used the minibuffer.

Subnodes of Mark
* Mark Ring::   Previous mark positions saved so you can go back there.

Subnodes of Registers
* RegPos::      Saving positions in registers.
* RegText::     Saving text in registers.
* RegRect::     Saving rectangles in registers.

Subnodes of Display
* Selective Display::      Hiding lines with lots of indentation.
* Display Vars::           Information on variables for customizing display.

Subnodes of Search
* Incremental Search::     Search happens as you type the string.
* Nonincremental Search::  Specify entire string and then search.
* Word Search:: 	   Search for sequence of words.
* Regexp Search::	   Search for match for a regexp.
* Regexps::     	   Syntax of regular expressions.
* Search Case::		   To ignore case while searching, or not.
* Replace::     	   Search, and replace some or all matches.
* Unconditional Replace::  Everything about replacement except for querying.
* Query Replace::          How to use querying.
* Other Repeating Search:: Operating on all matches for some regexp.

Subnodes of Fixit
* Kill Errors:: Commands to kill a batch of recently entered text.
* Transpose::   Exchanging two characters, words, lines, lists...
* Fixing Case:: Correcting case of last word entered.
* Spelling::    Apply spelling checker to a word, or a whole file.

Subnodes of Files
* File Names::  How to type and edit file name arguments.
* Visiting::    Visiting a file prepares Emacs to edit the file.
* Saving::      Saving makes your changes permanent.
* Backup::      How Emacs saves the old version of your file.
* Interlocking::How Emacs protects against simultaneous editing
                 of one file by two users.
* Reverting::   Reverting cancels all the changes not saved.
* Auto Save::   Auto Save periodically protects against loss of data.
* ListDir::     Listing the contents of a file directory.
* Dired::       ``Editing'' a directory to delete, rename, etc.
                 the files in it.
* Misc File Ops:: Other things you can do on files.

Subnodes of Buffers
* Select Buffer::   Creating a new buffer or reselecting an old one.
* List Buffers::    Getting a list of buffers that exist.
* Misc Buffer::     Renaming; changing read-only status.
* Kill Buffer::     Killing buffers you no longer need.
* Several Buffers:: How to go through the list of all buffers
                     and operate variously on several of them.

Subnodes of Indentation
* Indentation Commands:: Various commands and techniques for indentation.
* Tab Stops::   You can set arbitrary "tab stops" and then
                 indent to the next tab stop when you want to.
* Just Spaces:: You can request indentation using just spaces.

Subnodes of Text
* Text Mode::   The major mode for editing text files.
* Nroff Mode::  The major mode for editing input to the formatter nroff.
* TeX Mode::    The major mode for editing input to the formatter TeX.
* Outline Mode::The major mode for editing outlines.
* Words::       Moving over and killing words.
* Sentences::   Moving over and killing sentences.
* Paragraphs::	Moving over paragraphs.
* Pages::	Moving over pages.
* Filling::     Filling or justifying text
* Case::        Changing the case of text

Subnodes of Programs
* Program Modes::       Major modes for editing programs.
* Lists::       Expressions with balanced parentheses.
                 There are editing commands to operate on them.
* Defuns::      Each program is made up of separate functions.
                 There are editing commands to operate on them.
* Grinding::    Adjusting indentation to show the nesting.
* Matching::    Insertion of a close-delimiter flashes matching open.
* Comments::    Inserting, illing and aligning comments.
* Balanced Editing::    Inserting two matching parentheses at once, etc.
* Documentation::       Getting documentation of functions you plan to call.
* Change Log::  Maintaining a change history for your program.
* Tags::        Go direct to any function in your program in one
                 command.  Tags remembers which file it is in.

Subnodes of Running
* Compilation::       Compiling programs in languages other than Lisp
                       (C, Pascal, etc.)
* Lisp Modes::        Various modes for editing Lisp programs, with
                       different facilities for running the Lisp programs.
* Lisp Libraries::    Creating Lisp programs to run in Emacs.
* Lisp Interaction::  Executing Lisp in an Emacs buffer.
* Lisp Eval::         Executing a single Lisp expression in Emacs.
* Lisp Debug::        Debugging Lisp programs running in Emacs.
* External Lisp::     Communicating through Emacs with a separate Lisp.

Subnodes of Abbrevs
* Defining Abbrevs::  Defining an abbrev, so it will expand when typed.
* Expanding Abbrevs:: Controlling expansion: prefixes, canceling expansion.
* Editing Abbrevs::   Viewing or editing the entire list of defined abbrevs.
* Saving Abbrevs::    Saving the entire list of abbrevs for another session.

Subnodes of Picture
* Basic Picture::     Basic concepts and simple commands of Picture Mode.
* Insert in Picture:: Controlling direction of cursor motion
                       after "self-inserting" characters.
* Tabs in Picture::   Various features for tab stops and indentation.
* Rectangles in Picture:: Clearing and superimposing rectangles.

Subnodes of Rmail::
* Rmail Scrolling::   Scrolling through a message.
* Rmail Motion::      Moving to another message.
* Rmail Deletion::    Deleting and expunging messages.
* Rmail Inbox::       How mail gets into the Rmail file.
* Rmail Files::       Using multiple Rmail files.
* Rmail Labels::      Classifying messages by labeling them.
* Rmail Summary::     Summaries show brief info on many messages.
* Rmail Reply::       Sending replies to messages you are viewing.
* Rmail Editing::     Editing message text and headers in Rmail.
* Rmail Digest::      Extracting the messages from a digest message.

Subnodes of Customization
* Minor Modes::       Each minor mode is one feature you can turn on
                       independently of any others.
* Variables::         Many Emacs commands examine Emacs variables
                       to decide what to do; by setting variables,
                       you can control their functioning.
* Examining::         Examining or setting one variable's value.
* Edit Options::      Examining or editing list of all variables' values.
* Locals::            Per-buffer values of variables.
* File Variables::    How files can specify variable values.
* Keyboard Macros::   A keyboard macro records a sequence of keystrokes
                       to be replayed with a single command.
* Key Bindings::      The keymaps say what command each key runs.
                       By changing them, you can "redefine keys".
* Keymaps::           Definition of the keymap data structure.
* Rebinding::         How to redefine one key's meaning conveniently.
* Disabling::         Disabling a command means confirmation is required
                       before it can be executed.  This is done to protect
                       beginners from surprises.
* Syntax::            The syntax table controls how words and expressions
                       are parsed.
* Init File::         How to write common customizations in the `.emacs' file.

Subnodes of Lossage (and recovery)
* Stuck Recursive::   `[...]' in mode line around the parentheses.
* Screen Garbled::    Garbage on the screen.
* Text Garbled::      Garbage in the text.
* Unasked-for Search::Spontaneous entry to incremental search.
* Emergency Escape::  Emergency escape---
                       What to do if Emacs stops responding.
* Total Frustration:: When you are at your wits' end.
@end menu

@iftex
@unnumbered Preface

  This manual documents the use and simple customization of the
Emacs editor.  The reader is not expected to be a programmer.  Even simple
customizations do not require programming skill, but the user who is not
interested in customizing can ignore the scattered customization hints.

  This is primarily a reference manual, but can also be used as a
primer.  However, I recommend that the newcomer first use the on-line,
learn-by-doing tutorial, which you get by running Emacs and typing
@kbd{C-h t}.  With it, you learn Emacs by using Emacs on a specially
designed file which describes commands, tells you when to try them,
and then explains the results you see.  This gives a more vivid
introduction than a printed manual.

  On first reading, you need not make any attempt to memorize chapters one
and two, which describe the notational conventions of the manual and the
general appearance of the Emacs display screen.  It is enough to be aware
of what questions are answered in these chapters, so you can refer back
when you later become interested in the answers.  After reading chapter
four you should practice the commands there.  The next few chapters
describe fundamental techniques and concepts that are referred to again and
again.  It is best to understand them thoroughly, experimenting with them
if necessary.

  To find the documentation on a particular command, look in the
index.  Keys (character commands) and command names have separate
indexes just for them.  There is also a glossary, with a cross
reference for each term.

@ignore
  If you know vaguely what the command
does, look in the command summary.  The command summary contains a line or
two about each command, and a cross reference to the section of the
manual that describes the command in more detail; related commands
are grouped together.
@end ignore

  This manual comes in two forms: the published form and the Info form.
The Info form is for on-line perusal with the INFO program; it is
distributed along with GNU Emacs.  Both forms contain substantially the
same text and are generated from a common source file, which is distributed
along with GNU Emacs.

  GNU Emacs is a member of the Emacs editor family.  There are many Emacs
editors, all sharing common principles of organization.  For information on
the underlying philosophy of Emacs and the lessons learned from its
development, write for a copy of AI memo 519a, ``Emacs, the Extensible,
Customizable Self-Documenting Display Editor'', to

@display
Publications Department
Artificial Intelligence Lab
545 Tech Square
Cambridge, MA 02139
@end display

At last report they charge $2.25 per copy.
@end iftex

@node Distrib, License, Top, Top
@unnumbered Distribution

GNU Emacs is @dfn{free}; this means that everyone is free to use it and
free to redistribute it on a free basis.  GNU Emacs is not in the public
domain; it is copyrighted and there are restrictions on its distribution,
but these restrictions are designed to permit everything that a good
cooperating citizen would want to do.  What is not allowed is to try to
prevent others from further sharing any version of GNU Emacs that they
might get from you.  The precise conditions are found in the GNU Emacs
General Public License that comes with Emacs and also appears following
this section.

The easiest way to get a copy of GNU Emacs is from someone else who has it.
You need not ask for permission to do so, or tell any one else; just copy
it.

If you have access to the Internet, you can get the latest distribution
version of GNU Emacs from host @file{prep.ai.mit.edu} using anonymous
login.  See the file @file{/u2/emacs/GETTING.GNU.SOFTWARE} on that host
to find out about your options for copying and which files to use.

You may also eventually receive GNU Emacs when you buy a computer.
Computer manufacturers are free to distribute copies on the same terms that
apply to everyone else.  These terms require them to give you the full
sources, including whatever changes they may have made, and to permit you
to redistribute the GNU Emacs received from them under the usual terms of
the General Public License.  In other words, the program must be free for
you when you get it, not just free for the manufacturer.

If you cannot get a copy in any of those ways, you can order one from the
Free Software Foundation.  Though Emacs itself is free, our distribution
service is not.  An order form is included at the end of the manual, in
manuals printed by the Foundation.  It is also included in the file
@file{etc/DISTRIB} in the Emacs distribution.  For further information,
write to

@display
Free Software Foundation
1000 Mass Ave
Cambridge, MA 02138
@end display

The income from distribution fees goes to support the foundation's
purpose: the development of more free software to distribute just like
GNU Emacs.

If you find GNU Emacs useful, we urge you to @b{send a donation} to the Free
Software Foundation.  This will help support development of the rest of the
GNU system, and other useful software beyond that.  Subject to approval of
our application for a tax exemption, your donation will be tax deductible.

@node License, Intro, Distrib, Top
@unnumbered GNU Emacs General Public License

  The license agreements of most software companies keep you at the
mercy of those companies.  By contrast, our general public license is
intended to give everyone the right to share GNU Emacs.  To make
sure that you get the rights we want you to have, we need to make
restrictions that forbid anyone to deny you these rights or to ask you
to surrender the rights.  Hence this license agreement.

  Specifically, we want to make sure that you have the right to give
away copies of Emacs, that you receive source code or else can get it
if you want it, that you can change Emacs or use pieces of it in new
free programs, and that you know you can do these things.

  To make sure that everyone has such rights, we have to forbid you to
deprive anyone else of these rights.  For example, if you distribute
copies of Emacs, you must give the recipients all the rights that you
have.  You must make sure that they, too, receive or can get the
source code.  And you must tell them their rights.

  Also, for our own protection, we must make certain that everyone
finds out that there is no warranty for GNU Emacs.  If Emacs is
modified by someone else and passed on, we want its recipients to know
that what they have is not what we distributed, so that any problems
introduced by others will not reflect on our reputation.

  Therefore we (Richard Stallman and the Free Software Foundation, Inc.)@:
make the following terms which say what you must do to be allowed to
distribute or change GNU Emacs.

@unnumberedsec Copying Policies

@enumerate
@item
You may copy and distribute verbatim copies of GNU Emacs source
code as you receive it, in any medium, provided that you conspicuously
and appropriately publish on each file a valid copyright notice such
as ``Copyright @copyright{} 1985 Richard M. Stallman'', containing the year of
last change and name of copyright holder for the file in question;
keep intact the notices on all files that refer to this License
Agreement and to the absence of any warranty; and give any other
recipients of the GNU Emacs program a copy of this License Agreement
along with the program.

@item
You may modify your copy or copies of GNU Emacs source code or
any portion of it, and copy and distribute such modifications under
the terms of Paragraph 1 above, provided that you also do the following:

@enumerate
@item
cause the modified files to carry prominent notices stating
who last changed such files and the date of any change; and

@item
cause the whole of any work that you distribute or publish,
that in whole or in part contains or is a derivative of GNU Emacs
or any part thereof, to be freely distributed
and licensed to all third parties on terms identical to those
contained in this License Agreement (except that you may choose
to grant more extensive warranty protection to third parties,
at your option).

@item
if the modified program serves as a text editor, cause it
when started running in the simplest and usual way, to print
an announcement including a valid copyright notice (``Copyright
@copyright{}'', the year of authorship, and all copyright owners' names),
saying that there is no warranty (or else, saying that you provide
a warranty) and that users may redistribute the program under
these conditions, and telling the user how to view a copy of
this License Agreement.
@end enumerate

@item
You may copy and distribute GNU Emacs or any portion of it in
compiled, executable or object code form under the terms of Paragraphs
1 and 2 above provided that you do the following:

@enumerate
@item
cause each such copy of GNU Emacs to be accompanied by the
corresponding machine-readable source code; or

@item
cause each such copy of GNU Emacs to be accompanied by a written
offer, with no time limit, to give any third party free (except
for a nominal shipping charge) machine readable copy of the
corresponding source code; or

@item
in the case of a recipient of GNU Emacs in compiled, executable
or object code form (without the corresponding source code) you
shall cause copies you distribute to be accompanied by a copy of
the written offer of source code which you received along with
the copy of GNU Emacs.
@end enumerate

@item
You may not copy, sublicense, distribute or transfer GNU Emacs except
as expressly provided under this License Agreement.  Any attempt
otherwise to copy, sublicense, distribute or transfer GNU Emacs is
void and your rights to use GNU Emacs under this License agreement
shall be automatically terminated.  However, parties who have received
computer software programs from you with this License Agreement will
not have their licenses terminated so long as such parties remain in
full compliance.
@end enumerate

Your comments and suggestions about our licensing policies and our
software are welcome!  Please contact the Free Software Foundation, Inc.,
1000 Mass Ave, Cambridge, MA 02138, or call (617) 876-3296.

@iftex
@vfil
@eject
@end iftex
@unnumberedsec NO WARRANTY

  BECAUSE GNU EMACS IS LICENSED FREE OF CHARGE, WE PROVIDE ABSOLUTELY
NO WARRANTY, TO THE EXTENT PERMITTED BY APPLICABLE STATE LAW.  EXCEPT
WHEN OTHERWISE STATED IN WRITING, FREE SOFTWARE FOUNDATION, INC,
RICHARD M. STALLMAN AND/OR OTHER PARTIES PROVIDE GNU EMACS ``AS IS''
WITHOUT WARRANTY OF ANY KIND, EITHER EXPRESSED OR IMPLIED, INCLUDING,
BUT NOT LIMITED TO, THE IMPLIED WARRANTIES OF MERCHANTABILITY AND
FITNESS FOR A PARTICULAR PURPOSE.  THE ENTIRE RISK AS TO THE QUALITY
AND PERFORMANCE OF THE PROGRAM IS WITH YOU.  SHOULD THE GNU EMACS
PROGRAM PROVE DEFECTIVE, YOU ASSUME THE COST OF ALL NECESSARY
SERVICING, REPAIR OR CORRECTION.

 IN NO EVENT UNLESS REQUIRED BY APPLICABLE LAW WILL FREE SOFTWARE
FOUNDATION, INC., RICHARD M. STALLMAN, AND/OR ANY OTHER PARTY WHO MAY
MODIFY AND REDISTRIBUTE GNU EMACS AS PERMITTED ABOVE, BE LIABLE TO YOU
FOR DAMAGES, INCLUDING ANY LOST PROFITS, LOST MONIES, OR OTHER
SPECIAL, INCIDENTAL OR CONSEQUENTIAL DAMAGES ARISING OUT OF THE USE OR
INABILITY TO USE (INCLUDING BUT NOT LIMITED TO LOSS OF DATA OR DATA
BEING RENDERED INACCURATE OR LOSSES SUSTAINED BY THIRD PARTIES OR A
FAILURE OF THE PROGRAM TO OPERATE WITH PROGRAMS NOT DISTRIBUTED BY
FREE SOFTWARE FOUNDATION, INC.) THE PROGRAM, EVEN IF YOU HAVE BEEN
ADVISED OF THE POSSIBILITY OF SUCH DAMAGES, OR FOR ANY CLAIM BY ANY
OTHER PARTY.

@node Intro, Glossary, License, Top
@unnumbered Introduction

  You are about to read about GNU Emacs, the Unix/GNU incarnation of the
advanced, self-documenting, customizable, extensible real-time display
editor Emacs.  (The `G' in `GNU' is not silent.)

  We say that Emacs is a @dfn{display} editor because normally the text
being edited is visible on the screen and is updated automatically as you
type your commands.  @xref{Screen,Display}.

  We call it a @dfn{real-time} editor because the display is updated very
frequently, usually after each character or pair of characters you
type.  This minimizes the amount of information you must keep in your
head as you edit.  @xref{Basic,Real-time,Basic Editing}.

  We call Emacs advanced because it provides facilities that go beyond
simple insertion and deletion: filling of text; automatic indentation of
programs; viewing two or more files at once; and dealing in terms of
characters, words, lines, sentences, paragraphs, and pages, as well as
expressions and comments in several different programming languages.  It is
much easier to type one command meaning ``go to the end of the paragraph''
than to find that spot with simple cursor keys.

  @dfn{Self-documenting} means that at any time you can type a special
character, @kbd{Control-h}, to find out what your options are.  You can
also use it to find out what any command does, or to find all the commands
that pertain to a topic.  @xref{Help}.

  @dfn{Customizable} means that you can change the definitions of Emacs
commands in little ways.  For example, if you use a programming language in
which comments start with @samp{<**} and end with @samp{**>}, you can tell
the Emacs comment manipulation commands to use those strings
(@pxref{Comments}).  Another sort of customization is rearrangement of the
command set.  For example, if you prefer the four basic cursor motion
commands (up, down, left and right) on keys in a diamond pattern on the
keyboard, you can have it.  @xref{Customization}.

  @dfn{Extensible} means that you can go beyond simple customization and
write entirely new commands, programs in the Lisp language to be run by
Emacs's own Lisp interpreter.  Emacs is an ``on-line extensible'' system,
which means that it is divided into many functions that call each other,
any of which can be redefined in the middle of an editing session.  Any
part of Emacs can be replaced without making a separate copy of all of
Emacs.  Most of the editing commands of Emacs are written in Lisp already;
the few exceptions could have been written in Lisp but are written in C for
efficiency.  Although only a programmer can write an extension, anybody can
use it afterward.

@node Screen, Characters, Concept Index, Top

@chapter The Organization of the Screen
@cindex screen

  Emacs divides the screen into several areas, each of which contains
its own sorts of information.  The biggest area, of course, is the one
in which you usually see the text you are editing.

  When you are using Emacs, the screen is divided into a number of
@dfn{windows}.  Initially there is one text window occupying all but the
last line, plus the special @dfn{echo area} or @dfn{minibuffer window} in
the last line.  The text window can be subdivided horizontally or
vertically into multiple text windows, each of which can be used for a
different file (@pxref{Windows}).  The window that the cursor is in is the
@dfn{selected window}, in which editing takes place.  The other windows are
just for reference unless you select one of them.

  Each text window's last line is a @dfn{mode line} which describes what is
going on in that window.  It is in inverse video if the terminal supports
that, and contains text that starts like @samp{-----Emacs:@: @var{something}}.  Its
purpose is to indicate what buffer is being displayed in the window above
it; what major and minor modes are in use; and whether the buffer's text
has been changed.

@menu
* Point::	The place in the text where editing commands operate.
* Echo Area::   Short messages appear at the bottom of the screen.
* Mode Line::	Interpreting the mode line.
@end menu

@node Point, Echo Area, Screen, Screen
@section Point
@cindex point
@cindex cursor

  When Emacs is running, the terminal's cursor shows the location at
which editing commands will take effect.  This location is called
@dfn{point}.  Other commands move point through the text, so that you
can edit at different places in it.

  While the cursor appears to point @var{at} a character, point should be
thought of as @var{between} two characters; it points @var{before} the character
that the cursor appears on top of.  Sometimes people speak of ``the
cursor'' when they mean ``point'', or speak of commands that move point as
``cursor motion'' commands.

  Terminals have only one cursor, and when output is in progress it must
appear where the typing is being done.  This does not mean that point is
moving.  It is only that Emacs has no way to show you the location of point
except when the terminal is idle.

  Each Emacs buffer has its own point location.  A buffer that is not being
displayed remembers where point is so that it can be seen when you look at
that buffer again.

  When there are multiple text windows, each window has its own point
location.  The cursor shows the location of point in the selected window.
This also is how you can tell which window is selected.  If the same buffer
appears in more than one window, point can be moved in each window
independently.

  The term `point' comes from the character @samp{.}, which was the
command in TECO (the language in which the original Emacs was written)
for accessing the value now called `point'.

@node Echo Area, Mode Line, Point, Screen
@section The Echo Area
@cindex echo area

  The line at the bottom of the screen (below the mode line) is the
@dfn{echo area}.  It is used to display small amounts of text for several
purposes.

  @dfn{Echoing} means printing out the characters that you type.  Emacs
does not echo single-character keys, and does not echo any keys if you type
the characters with no long pause, but if you pause for more than a second
in the middle of a multi-character key, then all the characters typed so
far are echoed.  This is intended to @dfn{prompt} you for the rest of the
key.  Once the beginning of a key has been echoed, all the rest is echoed
as soon as it is typed; so either the entire key or none of it is echoed.
This behavior is designed to give confident users fast response, while
giving hesitant users maximum feedback.  This behavior is controlled by a
variable you can change (@pxref{Display Vars}).

  If a command cannot be executed, it may print an @dfn{error message} in
the echo area.  Error messages are accompanied by a beep or by flashing the
screen.  Also, any input you have typed ahead is thrown away when an error
happens.

  Some commands print informative messages in the echo area.  These
messages look much like error messages, but they are not announced with a
beep and do not throw away input.  Sometimes the message tells you what the
command has done, when it is not obvious from looking at the text being
edited.  Sometimes the sole purpose of a command is to print a message
giving you specific information.  For example, the command @kbd{C-x =} is
used to print a message describing the character position of point in the
text and its current column in the window.  Commands that take a long time
often display messages ending in @samp{@dots{}} while they are working, and
add @samp{done} at the end when they are finished.

  The echo area is also used to display the @dfn{minibuffer}, a window that
is used for reading arguments to commands, such as the name of a file to be
edited.  When the minibuffer is in use, the echo area begins with a prompt
string that ends with a colon; also, the cursor appears in that line
because it is the selected window.  You can always get out of the
minibuffer by typing @kbd{C-g}.  @xref{Minibuffer}.

@node Mode Line,, Echo Area, Screen
@section The Mode Line
@cindex mode line
@cindex top level

  Each text window's last line is a @dfn{mode line} which describes what is
going on in that window.  When there is only one text window, the mode line
appears right above the echo area.  The mode line is in inverse video if
the terminal supports that, starts and ends with dashes, and contains text
like @samp{Emacs:@: @var{something}}.

  If a mode line has something else in place of @samp{Emacs:@: @var{something}},
then the window above it is in a special subsystem such as Rmail.  The mode
line then indicates the status of the subsystem.

  Normally, the mode line has the following appearance:

@example
--@var{ch}-Emacs: @var{buf}      (@var{major} @var{minor})----@var{pos}------
@end example

@noindent
This serves to indicate various information about the buffer being
displayed in the window: the buffer's name, what major and minor modes are
in use, whether the buffer's text has been changed, and how far down the
buffer you are currently looking.  The top level mode line has this format:

  @var{ch} contains two stars @samp{**} if the text in the buffer has been
edited (the buffer is ``modified''), or @samp{--} if the buffer has not been
edited.  Exception: for a read-only buffer, it is @samp.

  @var{buf} is the name of the window's chosen @dfn{buffer}.  The chosen buffer
in the selected window (the window that the cursor is in) is also Emacs's
selected buffer, the one that editing takes place in.  When we speak of
what some command does to ``the buffer'', we are talking about the
currently selected buffer.  @xref{Buffers}.

  @var{major} is the name of the @dfn{major mode} in effect in the buffer.  At
any time, each buffer is in one and only one of the possible major modes.
The major modes available include Fundamental mode (the least specialized),
Text mode, Lisp mode, C mode, and others.  @xref{Major Modes}, for details
of how the modes differ and how to select one.@refill

  @var{minor} is a list of some of the @dfn{minor modes} that are turned on
at the moment in the window's chosen buffer.  @samp{Fill} means that Auto
Fill mode is on.  @samp{Abbrev} means that Word Abbrev mode is on.
@samp{Overwrite} means that Overwrite mode is on.  @xref{Minor Modes}, for
more information.  @samp{Narrow} means that the buffer being displayed has
editing restricted to only a portion of its text.  This is not really a
minor mode, but is like one.  @xref{Narrowing}.@refill

  @var{pos} tells you whether there is additional text above the top of the
screen, or below the bottom.  If your file is small and it is all on the
screen, @var{pos} is @samp{All}.  Otherwise, it is @samp{Top} if you are
looking at the beginning of the file, @samp{Bot} if you are looking at the
end of the file, or @samp{@var{nn}%}, where @var{nn} is the percentage of
the file above the top of the screen.@refill

  Some other information about the state of Emacs can also be displayed
among the minor modes.  @samp{Def} means that a keyboard macro is being
defined; although this is not exactly a minor mode, it is still useful to
be reminded about.  @xref{Keyboard Macros}.

  In addition, if Emacs is currently inside a recursive editing level,
square brackets (@samp{[@dots{}]}) appear around the parentheses that
surround the modes.  If Emacs is in one recursive editing level within
another, double square brackets appear, and so on.  Since this information
pertains to Emacs in general and not to any one buffer, the square brackets
appear in every mode line on the screen or not in any of them.
@xref{Recursive Edit}.@refill

@findex display-time
  Emacs can optionally display the time and system load in all mode lines.
To enable this feature, type @kbd{M-x display-time}.  The information added
to the mode line usually appears after the file name, before the mode names
and their parentheses.  It looks like this:

@example
@var{hh}:@var{mm}pm @var{l.ll} [@var{d}]
@end example

@noindent
@var{hh} and @var{mm} are the hour and minute, followed always by @samp{am}
or @samp{pm}.  @var{l.ll} is the average number of running processes in the
whole system recently.  @var{d} is an approximate index of the ratio of
disk activity to cpu activity for all users.

The word @samp{Mail} appears after the load level if there is mail for
you that you have not read yet.

@vindex mode-line-inverse-video
  Customization note: the variable @code{mode-line-inverse-video} controls
whether the mode line is displayed in inverse video (assuming the terminal
supports it); @code{nil} means no inverse video.  The default is @code{t}.

@iftex
@chapter Characters, Keys and Commands

  This chapter explains the character set used by Emacs for input commands
and for the contents of files, and also explains the concepts of
@dfn{keys} and @dfn{commands} which are necessary for understanding how
your keyboard input is understood by Emacs.
@end iftex

@node Characters, Keys, Screen, Top
@section The Emacs Character Set
@cindex character set
@cindex ASCII

  GNU Emacs uses the ASCII character set, which defines 128 different
character codes.  Some of these codes are assigned graphic symbols such
as @samp{a} and @samp{=}; the rest are control characters, such as
@kbd{Control-a} (also called @kbd{C-a} for short).  @kbd{C-a} gets its name
from the fact that you type it by holding down the @key{CTRL} key and
then pressing @kbd{a}.  There is no distinction between @kbd{C-a} and
@kbd{C-A}; they are the same character.@refill

  Some control characters have special names, and special keys you can
type them with: @key{RET}, @key{TAB}, @key{LFD}, @key{DEL} and @key{ESC}.
The space character is usually referred to below as @key{SPC}, even though
strictly speaking it is a graphic character whose graphic happens to be
blank.@refill

  Emacs extends the 7-bit ASCII code to an 8-bit code by adding an extra
bit to each character.  This makes 256 possible command characters.  The
additional bit is called Meta.  Any ASCII character can be made Meta;
examples of Meta characters include @kbd{Meta-a} (@kbd{M-a}, for short),
@kbd{M-A} (not the same character as @kbd{M-a}, but those two characters
normally have the same meaning in Emacs), @kbd{M-@key{RET}}, and
@kbd{M-C-a}.  For traditional reasons, @kbd{M-C-a} is usually called
@kbd{C-M-a}; logically speaking, the order in which the modifier keys
@key{CTRL} and @key{META} are mentioned does not matter.@refill

@cindex Control
@cindex Meta
@cindex C-
@cindex M-
  Some terminals have a @key{META} key, and allow you to type Meta
characters by holding this key down.  Thus, @kbd{Meta-a} is typed by
holding down @key{META} and pressing @kbd{a}.  Such a key is not always
labeled @key{META}, however, as it is usually a special option from the
manufacturer.  If there is no @key{META} key, you can still type Meta
characters using two-character sequences starting with @key{ESC}.  Thus, to
enter @kbd{M-a}, you could type @kbd{@key{ESC} a}.  To enter @kbd{C-M-a},
you would type @kbd{@key{ESC} C-a}.  @key{ESC} is allowed on terminals with
Meta keys, too, in case you have formed a habit of doing it.@refill

@vindex meta-flag
  Emacs believes the terminal has a @key{META} key if the variable
@code{meta-flag} is non-@code{nil}.  Normally this is set automatically
according to the termcap entry for your terminal type.  However, sometimes
the termcap entry is wrong, and then it is useful to set this variable
yourself.

  Emacs buffers also use an 8-bit character set, because bytes have 8 bits,
but only the ASCII characters are considered meaningful.  ASCII graphic
characters in Emacs buffers are displayed with their graphics.  @key{LFD}
is the same as a newline character; it is displayed by starting a new line.
@key{TAB} is displayed by moving to the next tab stop column (usually every
8 columns).  Other control characters are displayed as a caret (@samp{^})
followed by the non-control version of the character; thus, @kbd{C-a} is
displayed as @samp{^A}.  Non-ASCII characters 128 and up are displayed with
octal escape sequences; thus, character code 243 (octal), also called
@kbd{M-#} when used as an input character, is displayed as @samp{\243}.

@node Keys, Commands, Characters, Top
@section Keys

@cindex key
  A @dfn{key}---short for @dfn{key sequence}---is a sequence of characters
that is all part of specifying a single Emacs command to be run.  If the
characters are enough to specify a command, they form a @dfn{complete key}.

@kindex C-c
@kindex C-x
@kindex C-h
@kindex ESC
  A single character is always a key; whether it is complete depends on its
meaning in Emacs.  Most single characters are complete Emacs commands.
@kbd{C-c}, @kbd{C-h}, @kbd{C-x} and @key{ESC} are the only ones that are not complete.

@cindex prefix key
  A sequence of characters that is not enough to specify an Emacs command
is called a @dfn{prefix key}.  A prefix key is the beginning of a series of
longer sequences that are valid keys; adding any single character to the
end of the prefix gives a valid key, which could be defined as an Emacs
command.  For example, @kbd{C-x} is normally defined as a prefix, so
@kbd{C-x} and the next input character combine to make a two-character key.
There are 256 different two-character keys starting with @kbd{C-x}, one for
each possible second character.  Many of these two-character keys starting
with @kbd{C-x} are standardly defined as Emacs commands.  Notable examples
include @kbd{C-x C-f} and @kbd{C-x s} (@pxref{Files}).

  Adding one character to a prefix key does not have to form a complete
key.  It could make another, longer prefix.  For example, @kbd{C-x 4} is
itself a prefix that leads to 256 different three-character keys, including
@kbd{C-x 4 f}, @kbd{C-x 4 b} and so on.  It would be possible to define one
of those three-character sequences as a prefix, creating a series of
four-character keys, but we did not define any of them this way.

  All told, the prefix keys in Emacs are @kbd{C-c}, @kbd{C-x}, @kbd{C-h},
@kbd{C-x 4}, and @key{ESC}.  But this is not built in; it is just a matter
of Emacs's standard key bindings.  In customizing Emacs, you could make
new prefix keys, or eliminate these.  @xref{Key Bindings}.

@node Commands, Entering Emacs, Keys, Top
@section Keys and Commands

@cindex binding
@cindex customization
@cindex keymap
@cindex function
@cindex command
  This manual is full of passages that tell you what particular keys do.
But Emacs does not assign meanings to keys directly.  Instead, Emacs
assigns meanings to @dfn{functions}, and then gives keys their meanings by
@dfn{binding} them to functions. 

  A function is a Lisp object that can be executed as a program.
Usually it is a Lisp symbol which has been given a function definition;
every symbol has a name, usually made of a few English words separated by
dashes, such as @code{next-line} or @code{forward-word}.  It also has a
@dfn{definition} which is a Lisp program; this is what makes the function
do what it does.  Only some functions can be the bindings of keys; these
are functions whose definitions use @code{interactive} to specify how to
call them interactively.  Such functions are called @dfn{commands}, and
the name of a symbol that is a command is called a @dfn{command name}.
More information on this subject will appear in the @i{GNU Emacs Lisp
Manual} (which is not yet written).

  The bindings between keys and functions are recorded in various tables
called @dfn{keymaps}.  @xref{Keymaps}.

  When we say that ``@kbd{C-n} moves down vertically one line'' we are
glossing over a distinction that is irrelevant in ordinary use but is vital
in understanding how to customize Emacs.  It is the function
@code{next-line} that is programmed to move down vertically.  @kbd{C-n} has
this effect @i{because} it is bound to that function.  If you rebind
@kbd{C-n} to the function @code{forward-word} then @kbd{C-n} will move
forward by words instead.  Rebinding keys is a common method of
customization.@refill

  In the rest of this manual, we usually ignore this subtlety to keep
things simple.  To give the customizer the information he needs, we
state the name of the command which really does the work in parentheses
after mentioning the key that runs it.  For example, we will say that
``The command @kbd{C-n} (@code{next-line}) moves point vertically down,''
meaning that @code{next-line} is a command that moves vertically down
and @kbd{C-n} is a key that is standardly bound to it.

@cindex variables
  While we are on the subject of customization information which you should
not be frightened of, it's a good time to tell you about @dfn{variables}.
Often the description of a command will say, ``To change this, set the
variable @code{mumble-foo}.''  A variable is a name used to remember a
value.  Most of the variables documented in this manual exist just to
permit customization: the variable's value is examined by some command,
and changing the value makes the command behave differently.  Until you
are interested in customizing,  you can ignore this information.  When you
are ready to be interested, read the basic information on variables, and
then the information on individual variables will make sense.
@xref{Variables}.

@node Entering Emacs, Exiting, Commands, Top
@chapter Entering and Exiting Emacs
@cindex entering Emacs
@cindex arguments (from shell)

  The simplest way to invoke Emacs is just to type @kbd{emacs @key{RET}}
at the shell.

  It is also possible to specify files to be visited, Lisp files to be
loaded, and functions to be called, by giving Emacs arguments in the
shell command line.  The command arguments are processed in the order they
appear in the command argument list; however, certain arguments must be at
the front of the list (@samp{-t} or @samp{-batch}) if they are used.

  Here are the arguments allowed:

@table @samp
@item @var{file}
Visit @var{file} using @code{find-file}.  @xref{Visiting}.

@item +@var{linenum} @var{file}
Visit @var{file} using @code{find-file}, then go to line number
@var{linenum} in it.

@item -l @var{file}
Load a file @var{file} of Lisp code with @code{load}.  @xref{Lisp
Libraries}.

@item -f @var{function}
Call Lisp function @var{function} with no arguments.

@item -kill
Exit from Emacs without asking for confirmation.
@end table

  The remaining switches are recognized only at the beginning of the
command line.  If more than one of them appears, they must appear in the
order that they appear in this table.

@table @samp
@item -t @var{device}
Use @var{device} as the terminal for editing input and output.

@cindex batch mode
@item -batch
Run Emacs in @dfn{batch mode}, which means that the text being edited is
not displayed and the standard Unix interrupt characters such as @kbd{C-z}
and @kbd{C-c} continue to have their normal effect.  Emacs in batch mode
outputs to @code{stdout} only what would normally be printed in the echo
area under program control.

Batch mode is used for running programs written in Emacs Lisp from
shell scripts, makefiles, and so on.  Normally the @samp{-l} switch
or @samp{-f} switch will be used as well, to invoke a Lisp program
to do the batch processing.

@samp{-batch} implies @samp{-q} (do not load an init file).  It also causes
Emacs to kill itself after all command switches have been processed.  In
addition, auto-saving is not done except in buffers for which it has been
explicitly requested.

@item -q
Do not load your Emacs init file @file{~/.emacs}.

@item -u @var{user}
Load @var{user}'s Emacs init file @file{~@var{user}/.emacs} instead of
your own.
@end table

  One way to use command switches is to visit many files automatically:

@example
emacs *.c
@end example

@noindent
passes each @code{.c} file as a separate argument to Emacs, so that Emacs
visits each file (@pxref{Visiting}).
  
  Here is an advanced example that assumes you have a Lisp program
file called @file{hack-c-program.el} which, when loaded, performs some
useful operation on current buffer, expected to be a C program.

@example
emacs -batch foo.c -l hack-c-program -f save-buffer -kill > log
@end example

@noindent
Here Emacs is told to visit @file{foo.c}, load @file{hack-c-program.el}
(which makes changes in the visited file), save @file{foo.c} (note that
@code{save-buffer} is the function that @kbd{C-x C-s} is bound to), and
then exit to the shell that this command was done with.  @samp{-batch}
guarantees there will be no problem redirecting output to @file{log},
because Emacs will not assume that it has a display terminal to work with.

@node Exiting, Basic, Entering Emacs, Top
@section Exiting Emacs
@cindex exiting
@cindex killing Emacs
@cindex suspending

  There are two commands for exiting Emacs because there are two kinds of
exiting: @dfn{suspending} Emacs and @dfn{killing} Emacs.  @dfn{Suspending} means
stopping Emacs temporarily and returning control to its superior (usually
the shell), allowing you to resume editing later in the same Emacs job,
with the same files, same kill ring, same undo history, and so on.  This is
the usual way to exit.  @dfn{Killing} Emacs means destroying the Emacs job.
You can run Emacs again after killing it, but you will get a fresh Emacs;
there is no way to resume the same editing session after it has been
killed.

@kindex C-z
@findex suspend-emacs
  To suspend Emacs, type @kbd{C-z} (@code{suspend-emacs}).  On systems that do not
permit programs to be suspended, @kbd{C-z} runs an inferior shell that
communicates directly with the terminal, and Emacs waits until you exit
the subshell.  The only way on these systems to get back to the shell from
which Emacs was run (to log out, for example) is to kill Emacs.  @kbd{C-d} or
@code{exit} are typical commands to exit a subshell.  

@kindex C-x C-c
@findex save-buffers-kill-emacs
  To kill Emacs, type @kbd{C-x C-c} (@code{save-buffers-kill-emacs}).  A
two-character key is used for this to make it harder to type.  Unless a
numeric argument is used, this command first offers to save any modified
buffers.  If you do not save them all, it asks for reconfirmation with
`yes' before killing Emacs, since any changes not saved before that will be
lost forever.  Also, if any subprocesses are still running, @kbd{C-x C-c}
asks for confirmation about them, since killing Emacs will kill the
subprocesses immediately.

  In most programs running on Berkeley Unix, @b{but not in Emacs}, the
characters @kbd{C-z} and @kbd{C-c} instantly suspend or kill, respectively.
The meanings of @kbd{C-z} and @kbd{C-x C-c} as keys in Emacs were inspired
by the standard Unix meanings of @kbd{C-z} and @kbd{C-c}, but there is no
causal connection.  The standard Berkeley Unix handling of @kbd{C-z} and
@kbd{C-c} is turned off in Emacs.  You could customize these keys to do
anything (@pxref{Keymaps}).

@c??? What about system V here?

@node Basic, Undo, Exiting, Top
@chapter Basic Editing Commands

@kindex C-h t
@findex help-with-tutorial
  We now give the basics of how to enter text, make corrections, and
save the text in a file.  If this material is new to you, you might
learn it more easily by running the Emacs learn-by-doing tutorial.  To
do this, type @kbd{Control-h t} (@code{help-with-tutorial}).

@section Inserting Text

@cindex insertion
@cindex point
@cindex cursor
@cindex graphic characters
  To insert printing characters into the text you are editing, just
type them.  Except in special modes, Emacs defines each printing
character as a key to run the command @code{self-insert}, which inserts
the character that you typed to invoke it into the buffer at the
cursor (that is, at @dfn{point}; @pxref{Point}).  The cursor moves
forward.  Any characters after the cursor move forward too.  If the
text in the buffer is @samp{FOOBAR}, with the cursor before the
@samp{B}, then if you type @kbd{XX}, you get @samp{FOOXXBAR}, with the
cursor still before the @samp{B}.

@kindex DEL
@cindex deletion
@findex delete-backward-char
   To @dfn{delete} text you have just inserted, you can use @key{DEL}
(which runs the command named @code{delete-backward-char}).  @key{DEL}
deletes the character @var{before} the cursor (not the one that the cursor
is on top of or under; that is the character @var{after} the cursor).  The
cursor and all characters after it move backwards.  Therefore, if you type
a printing character and then type @key{DEL}, they cancel out.

@kindex RET
@findex newline
@cindex newline
   To end a line and start typing a new one, type @key{RET} (running the
command @code{newline}).  @key{RET} operates by inserting a newline
character in the buffer.  If point is in the middle of a line, @key{RET}
splits the line.  Typing @key{DEL} when the cursor is at the beginning of a
line rubs out the newline before the line, thus joining the line with the
preceding line.

@cindex quoting
@kindex C-q
@findex quoted-insert
  Direct insertion works for printing characters and @key{SPC}, but other
characters act as editing commands and do not insert themselves.  If you
need to insert a control character or a character whose code is above 200
octal, you must @dfn{quote} it by typing @kbd{Control-q} (@code{quoted-insert}) first.
There are two ways to use @kbd{C-q}:

@itemize @bullet
@item
@kbd{Control-q} followed by any non-graphic character (even @kbd{C-g})
inserts that character.
@item
@kbd{Control-q} followed by three octal digits inserts the character
with the specified character code.
@end itemize

@noindent
A numeric argument to @kbd{C-q} specifies how many copies of the
quoted character should be inserted (@pxref{Arguments}).

@section Changing the Location of Point

  To do more than insert characters, you have to know how to move
point (@pxref{Point}).  Here are a few of the commands for doing that.

@kindex C-a
@kindex C-e
@kindex C-f
@kindex C-b
@kindex C-n
@kindex C-p
@kindex C-l
@kindex C-t
@kindex M->
@kindex M-<
@findex beginning-of-line
@findex end-of-line
@findex forward-char
@findex backward-char
@findex next-line
@findex previous-line
@findex recenter
@findex transpose-chars
@findex beginning-of-buffer
@findex end-of-buffer
@findex goto-char
@table @kbd
@item C-a
Move to the beginning of the line (@code{beginning-of-line}).
@item C-e
Move to the end of the line (@code{end-of-line}).
@item C-f
Move forward one character (@code{forward-char}).
@item C-b
Move backward one character (@code{backward-char}).
@item C-n
Move down one line, vertically (@code{next-line}).  This command
attempts to keep the horizontal position unchanged, so if you start in
the middle of one line, you end in the middle of the next.  When on
the last line of text, @kbd{C-n} creates a new line and moves onto it.
@item C-p
Move up one line, vertically (@code{previous-line}).
@item C-l
Clear the screen and reprint everything (@code{recenter}).
@item C-t
Transpose two characters, the ones before and after the cursor
(@code{transpose-chars}).
@item M-<
Move to the top of the buffer (@code{beginning-of-buffer}).  With
numeric argument @var{n}, move to @var{n}/10 of the way from the top.
@xref{Arguments}, for more information on numeric arguments.@refill
@item M->
Move to the end of the buffer (@code{end-of-buffer}).
@item M-x goto-char
Read a number @var{n} and move cursor to character number @var{n}.
Position 1 is the beginning of the buffer.
@item M-x goto-line
Read a number @var{n} and move cursor to line number @var{n}.  Line 1
is the beginning of the buffer.
@item C-x C-n
Use the current column of point as the @dfn{semipermanent goal column} for
@kbd{C-n} and @kbd{C-p} (@code{set-goal-column}).  Henceforth, those
commands always move to this column in each line moved into, or as
close as possible given the contents of the line.  This goal column remains
in effect until canceled.
@item C-u C-x C-n
Cancel the goal column.  Henceforth, @kbd{C-n} and @kbd{C-p} once
again try to avoid changing the horizontal position, as usual.
@end table

@vindex track-eol
  If you set the variable @code{track-eol} to a non-@code{nil} value, then
@kbd{C-n} and @kbd{C-p} when at the end of the starting line move to the
end of the line.  Normally, @code{track-eol} is @code{nil}.

@section Erasing Text

@table @kbd
@item @key{DEL}
Delete the character before the cursor (@code{delete-backward-char}).
@item C-d
Delete the character after the cursor (@code{delete-char}).
@item C-k
Kill to the end of the line (@code{kill-line}).
@end table

  You already know about the @key{DEL} key which deletes the character
before the cursor.  Another key, @kbd{Control-d}, deletes the character
after the cursor, causing the rest of the text on the line to shift left.
If @kbd{Control-d} is typed at the end of a line, that line and the next
line are joined together.

  To erase a larger amount of text, use the @kbd{Control-k} key, which
kills a line at a time.  If @kbd{Control-k} is done at the beginning or
middle of a line, it kills all the text up to the end of the line.  If
@kbd{Control-k} is done at the end of a line, it joins that line and the
next line.

  @xref{Killing}, for more flexible ways of killing text.

@section Files

@cindex files
  The commands above are sufficient for creating and altering text in an
Emacs buffer; the more advanced Emacs commands just make things easier.
But to keep any text permanently you must put it in a @dfn{file}.  Files
are named units of text which are stored by the operating system for you to
retrieve later by name.  To look at or use the contents of a file in any
way, including editing the file with Emacs, you must specify the file name.

  Consider a file named @file{/usr/rms/foo.c}.  To edit this file in Emacs,
type

@example
C-x C-f /usr/rms/foo.c @key{RET}
@end example

@noindent
Here the file name is given as an @dfn{argument} to the command @kbd{C-x
C-f} (@code{find-file}).  @key{RET} is used to terminate the argument.
Emacs obeys the command by @dfn{visiting} the file: creating a buffer,
copying the contents of the file into the buffer, and then displaying the
buffer for you to edit.  You can make changes in it, and then @dfn{save}
the file by typing @kbd{C-x C-s} (@code{save-buffer}).  This makes the
changes permanent by copying the altered contents of the buffer back into
the file @file{/usr/rms/foo.c}.  Until then, the changes are only inside
your Emacs, and the file @file{foo.c} is not changed.@refill

  To create a file, just visit the file with @kbd{C-x C-f} as if it already
existed.  Emacs will make an empty buffer in which you can insert the text
you want to put in the file.  When you save your text with @kbd{C-x C-s},
the file will be created.

  Of course, there is a lot more to learn about using files.  @xref{Files}.

@section Help

  If you forget what a key does, you can find out with the Help character,
which is @kbd{C-h}.  Type @kbd{C-h k} followed by the key you want to know
about; for example, @kbd{C-h k C-n} tells you all about what @kbd{C-n}
does.  @kbd{C-h} is a prefix key; @kbd{C-h k} is just one of its
subcommands (the command @code{describe-key}).  The other subcommands of
@kbd{C-h} provide different kinds of help.  @xref{Help}.@refill

@menu
* Blank Lines::        Commands to make or delete blank lines.
* Continuation Lines:: Lines too wide for the screen.
* Position Info::      What page, line, row, or column is point on?
* Arguments::	       Numeric arguments for repeating a command.
@end menu

@node Blank Lines, Continuation Lines, Basic, Basic
@section Blank Lines

@c widecommands
@table @kbd
@item C-o
Insert one or more blank lines after the cursor (@code{open-line}).
@item C-x C-o
Delete all but one of many consecutive blank lines
(@code{delete-blank-lines}).
@end table

@kindex C-o
@kindex C-x C-o
@cindex blank lines
@findex open-line
@findex delete-blank-lines
  When you want to insert a new line of text before an existing line, you
can do it by typing the new line of text, followed by @key{RET}.  However,
it may be easier to see what you are doing if you first make a blank line
and then insert the desired text into it.  This is easy to do using the key
@kbd{C-o} (@code{open-line}), which inserts a newline after point but leaves
point in front of the newline.  After @kbd{C-o}, type the text for the new
line.  @kbd{C-o F O O} has the same effect as @kbd{F O O @key{RET}}, except for
the final location of point.

  You can make several blank lines by typing @kbd{C-o} several times, or by
giving it an argument to tell it how many blank lines to make.
@xref{Arguments}, for how.

  If you have many blank lines in a row and want to get rid of them, use
@kbd{C-x C-o} (@code{delete-blank-lines}).  When point is on a blank line which
is adjacent to at least one other blank line, @kbd{C-x C-o} deletes all but
one of the consecutive blank lines, leaving exactly one.  With point on a
blank line with no other blank line adjacent to it, the sole blank line is
deleted, leaving none.  When point is on a nonblank line, @kbd{C-x C-o}
deletes any blank lines following that nonblank line.

@node Continuation Lines, Position Info, Blank Lines, Basic
@section Continuation Lines

@cindex continuation line
  If you add too many characters to one line, without breaking it with a
@key{RET}, the line will grow to occupy two (or more) lines on the screen,
with a @samp{\} at the extreme right margin of all but the last of them.
The @samp{\} says that the following screen line is not really a distinct
line in the text, but just the @dfn{continuation} of a line too long to fit
the screen.  Sometimes it is nice to have Emacs insert newlines
automatically when a line gets too long; for this, use Auto Fill mode
(@pxref{Filling}).

@vindex truncate-lines
@vindex default-truncate-lines
@cindex truncation
  Continuation can be turned off for a particular buffer by setting the
variable @code{truncate-lines} to non-@code{nil} in that buffer.  Then,
lines are @dfn{truncated}: the text that goes past the right margin does
not appear at all.  @samp{$} is used in the last column instead of @samp{\}
when truncation is in effect.  Truncation instead of continuation also
happens whenever horizontal scrolling is in use, and optionally whenever
side-by-side windows are in use (@pxref{Windows}).  @code{truncate-lines}
is automatically local in all buffers.  When a buffer is created, its value
of @code{truncate-lines} is initialized from the value of @code{default-truncate-lines},
normally @code{nil}.

@node Position Info, Arguments, Continuation Lines, Basic
@section Cursor Position Information

  If you are accustomed to other display editors, you may be surprised that
Emacs does not always display the page number or line number of point in
the mode line.  This is because the text is stored in a way that makes it
difficult to compute this information.  Displaying them all the time would
be intolerably slow.  They are not needed very often in Emacs anyway,
but there are commands to print them.

@table @kbd
@item C-x =
Print character code of character after point, character position of
point, and column of point (@code{what-cursor-position}).
@item M-x what-page
Print page number of point, and line number within page.
@item M-x what-line
Print line number of point in the buffer.
@item M-=
Print number of lines in the current region (@code{count-lines-region}).
@end table

@kindex C-x =
@findex what-cursor-position
  The command @kbd{C-x =} (@code{what-cursor-position}) can be used to find out
the column that the cursor is in, and other miscellaneous information about
point.  It prints a line in the echo area that looks like this:

@example
Char: x (0170)  point=65986 of 563027(12%)  x=44
@end example

@noindent
(In fact, this is the output produced when point is before the @samp{x=44}
in the example.)

  The two values after @samp{Char:} describe the character following point,
first by showing it and second by giving its octal character code.

  @samp{point=} is followed by the position of point expressed as a character
count.  The front of the buffer counts as position 1, one character later
as 2, and so on.  The next, larger number is the total number of characters
in the buffer.  Afterward in parentheses comes the position expressed as a
percentage of the total size.

  @samp{x=} is followed by the horizontal position of point, in columns from the
left edge of the window.

  If the buffer has been narrowed, making some of the text at the beginning and
the end temporarily invisible, @kbd{C-x =} prints additional text describing the
current visible range.  For example, it might say

@smallexample
Char: x (0170)  point=65986 of 563025(12%) <65102 - 68533>  x=44
@end smallexample

@noindent
where the two extra numbers give the smallest and largest character position
that point is allowed to assume.  The characters between those two positions
are the visible ones.  @xref{Narrowing}.

  If point is at the end of the buffer (or the end of the visible part),
@kbd{C-x =} omits any description of the character after point.
The output looks like

@smallexample
point=563026 of 563025(100%)  x=0
@end smallexample

@noindent
Usually @samp{x=0} at the end, because the text usually ends with a newline.

@findex what-page
@findex what-line
  There are two commands for printing line numbers.  @kbd{M-x what-line}
counts lines from the beginning of the file and prints the line number
point is on.  The first line of the file is line number 1.  By contrast,
@kbd{M-x what-page} counts pages from the beginning of the file, and
counts lines within the page, printing both of them.  @xref{Pages}.

@kindex M-=
@findex count-lines-region
  While on this subject, we might as well mention @kbd{M-=} (@code{count-lines-region}),
which prints the number of lines in the region (@pxref{Mark}).
@xref{Pages}, for the command @kbd{C-x l} which counts the lines in the
current page.

@node Arguments,, Position Info, Basic
@section Numeric Arguments
@cindex numeric arguments

  Any Emacs command can be given a @dfn{numeric argument}.  Some commands
interpret the argument as a repetition count.  For example, giving an
argument of ten to the key @kbd{C-f} (the command @code{forward-char}, move
forward one character) moves forward ten characters.  With these commands,
no argument is equivalent to an argument of one.  Negative arguments are
allowed.  Often they tell a command to move or act backwards.

@kindex M-1
@kindex M--
@findex digit-argument
@findex negative-argument
  If your terminal keyboard has a @key{META} key, the easiest way to
specify a numeric argument is to type digits and/or a minus sign while
holding down the the @key{META} key.  For example,
@example
M-5 C-n
@end example
@noindent
would move down five lines.  The characters @kbd{Meta-1}, @kbd{Meta-2},
etc., and @kbd{Meta--}, do this because they are keys bound to commands
(@code{digit-argument} and @code{negative-argument}) that are defined to
contribute to an argument for the next command.

@kindex C-u
@findex universal-argument
  Another way of specifying an argument is to use the @kbd{C-u}
(@code{universal-argument}) command followed by the digits of the argument.
With @kbd{C-u}, you can type the argument digits without holding
down shift keys.  To type a negative argument, start with a minus sign.
Just a minus sign normally means -1.  @kbd{C-u} works on all terminals.

  @kbd{C-u} followed by a character which is neither a digit nor a minus
sign has the special meaning of ``multiply by four''.  It multiplies the
argument for the next command by four.  @kbd{C-u} twice multiplies it by
sixteen.  Thus, @kbd{C-u C-u C-f} moves forward sixteen characters.  This
is a good way to move forward ``fast'', since it moves about 1/5 of a line
in the usual size screen.  Other useful combinations are @kbd{C-u C-n},
@kbd{C-u C-u C-n} (move down a good fraction of a screen), @kbd{C-u C-u
C-o} (make ``a lot'' of blank lines), and @kbd{C-u C-k} (kill four
lines).@refill

  Some commands care only about whether there is an argument, and not about
its value.  For example, the command @kbd{M-q} (@code{fill-paragraph}) with
no argument fills text; with an argument, it justifies the text as well.
(@xref{Filling}, for more information on @kbd{M-q}.)  Just @kbd{C-u} is a
handy way of providing an argument for such commands.

  Some commands use the value of the argument as a repeat count, but do
something peculiar when there is no argument.  For example, the command
@kbd{C-k} (@code{kill-line}) with argument @var{n} kills @var{n} lines,
including their terminating newlines.  But @kbd{C-k} with no argument is
special: it kills the text up to the next newline, or, if point is right at
the end of the line, it kills the newline itself.  Thus, two @kbd{C-k}
commands with no arguments can kill a nonblank line, just like @kbd{C-k}
with an argument of one.  (@xref{Killing}, for more information on
@kbd{C-k}.)@refill

  A few commands treat a plain @kbd{C-u} differently from an ordinary
argument.  A few others may treat an argument of just a minus sign
differently from an argument of -1.  These unusual cases will be described
when they come up; they are always for reasons of convenience of use of the
individual command.

@c section Autoarg Mode
@ignore
@cindex autoarg mode
  Users of ASCII keyboards may prefer to use Autoarg mode.  Autoarg mode
means that you don't need to type C-U to specify a numeric argument.
Instead, you type just the digits.  Digits followed by an ordinary
inserting character are themselves inserted, but digits followed by an
Escape or Control character serve as an argument to it and are not
inserted.  A minus sign can also be part of an argument, but only at the
beginning.  If you type a minus sign following some digits, both the digits
and the minus sign are inserted.

  To use Autoarg mode, set the variable Autoarg Mode nonzero.
@xref{Variables}.

  Autoargument digits echo at the bottom of the screen; the first nondigit
causes them to be inserted or uses them as an argument.  To insert some
digits and nothing else, you must follow them with a Space and then rub it
out.  C-G cancels the digits, while Delete inserts them all and then rubs
out the last.
@end ignore

@node Undo, Minibuffer, Basic, Top
@chapter Undoing Changes
@cindex undo

  Emacs allows all changes made in the text of a buffer to be undone,
up to a certain amount of change (8000 characters).  Each buffer records
changes individually, and the undo command always applies to the
current buffer.  Usually each editing command makes a separate entry
in the undo records, but some commands such as @code{query-replace}
make many entries, and very simple commands such as self-inserting
characters are often grouped to make undoing less tedious.

@table @kbd
@item C-x u
Undo one batch of changes (usually, one command worth) (@code{undo}).
@item C-_
The same.
@end table

@kindex C-x u
@kindex C-_
@findex undo
  The command @kbd{C-x u} or @kbd{C-_} is how you undo.  The first
time you give this command, it undoes the last change.  Point moves to
the beginning of the text affected by the undo, so you can see what
was undone.@refill

  Consecutive repetitions of the @kbd{C-_} or @kbd{C-x u} commands undo
earlier and earlier changes, back to the limit of what has been recorded.
If all recorded changes have already been undone, the undo command gets an
error.

  Any command other than an undo command breaks the sequence of undo
commands.  Starting at this moment, the previous undo commands are
considered ordinary changes that can themselves be undone.  Thus, you can
redo changes you have undone by typing @kbd{C-@key{SPC}}, @kbd{C-f} or any
other command that will have no important effect, and then using more undo
commands.

  If you notice that a buffer has been modified accidentally, the easiest
way to recover is to type @kbd{C-_} repeatedly until the stars disappear
from the front of the mode line.  At this time, all the modifications you
made have been cancelled.  If you do not remember whether you changed the
buffer deliberately, type @kbd{C-_} once, and when you see the last change
you made undone, you will remember why you made it.  If it was an accident,
leave it undone.  If it was deliberate, redo the change as described in the
preceding paragraph.

  Not all buffers record undo information.  Buffers whose names start with
spaces don't; these buffers are used internally by Emacs and its extensions
to hold text that users don't normally look at or edit.  Also, minibuffers,
help buffers and documentation buffers don't record undo information.

  At most 8000 or so characters of deleted or modified text can be
remembered in any one buffer for reinsertion by the undo command.  Also,
there is a limit on the number of individual insert, delete or change
actions that can be remembered.

  The reason the @code{undo} command has two keys, @kbd{C-x u} and @kbd{C-_}, set
up to run it is that it is worthy of a single-character key, but the way to
type @kbd{C-_} on some keyboards is not obvious.  @kbd{C-x u} is an
alternative that requires no special knowledge of the terminal.

@node Minibuffer, M-x, Undo, Top
@chapter The Minibuffer
@cindex minibuffer

  The @dfn{minibuffer} is the facility used by Emacs commands to read
arguments more complicated than a single number.  Minibuffer arguments can
be file names, buffer names, Lisp function names, Emacs command names, Lisp
expressions, and many other things, depending on the command reading the
argument.  The usual Emacs editing commands can be used to edit in the
minibuffer also.

@cindex prompt
  When the minibuffer is in use, it appears in the echo area, and the
terminal's cursor moves there.  The beginning of the minibuffer line
displays a @dfn{prompt} which says what kind of input you should supply and
how it will be used.  Often this prompt is derived from the name of the
command that the argument is for.  The prompt normally ends with a colon.

@cindex default argument
  Sometimes a @dfn{default argument} appears in parentheses after the
colon; it too is part of the prompt.  The default will be used as the
argument value if you enter an empty argument (e.g., just type @key{RET}).
For example, commands that read buffer names always show a default, which
is the name of the buffer that will be used if you type just @key{RET}.

@kindex C-g
  The simplest way to give a minibuffer argument is to type the text you
want, terminated by @key{RET} which exits the minibuffer.  You can get out
of the minibuffer, canceling the command that it was for, by typing
@kbd{C-g}.

  Since the minibuffer uses the screen space of the echo area, it can
conflict with other ways Emacs customarily uses the echo area.  Here is how
Emacs handles such conflicts:

@itemize @bullet
@item
If a command gets an error while you are in the minibuffer, this does
not cancel the minibuffer.  However, the echo area is needed for the
error message and therefore the minibuffer itself is hidden for a
while.  It comes back after a few seconds.

@item
If in the minibuffer you use a command whose purpose is to print a
message in the echo area, such as @kbd{C-x =}, the message is printed
normally, and the minibuffer is hidden for a while.  It comes back
after a few seconds.

@item
Echoing of keystrokes does not take place while the minibuffer is in
use.
@end itemize

@menu
* File: Minibuffer File.  Entering file names with the minibuffer.
* Edit: Minibuffer Edit.  How to edit in the minibuffer.
* Completion::		  An abbreviation facility for minibuffer input.
* Repetition::		  Re-executing commands that used the minibuffer.
@end menu

@node Minibuffer File, Minibuffer Edit, Minibuffer, Minibuffer
@section Minibuffers for File Names

  Sometimes the minibuffer starts out with text in it.  For example, when
you are supposed to give a file name, the minibuffer starts out containing
the @dfn{default directory}, which ends with a slash.  This is to inform
you which directory the file will be found in if you do not specify a
directory.  For example, the minibuffer might start out with

@example
Find File: /u2/emacs/src/
@end example

@noindent
where @samp{Find File:@: } is the prompt.  Typing @kbd{buffer.c} specifies
the file @file{/u2/emacs/src/buffer.c}.  To find files in nearby
directories, use @kbd{..}; thus, if you type @kbd{../lisp/simple.el}, the
file that you visit will be the one named @file{/u2/emacs/lisp/simple.el}.
Alternatively, you can kill with @kbd{M-@key{DEL}} the directory names you
don't want (@pxref{Words}).@refill

  You can also type an absolute file name, one starting with a slash or a
tilde, ignoring the default directory.  For example, to find the file
@file{/etc/termcap}, just type the name, giving

@example
Find File: /u2/emacs/src//etc/termcap
@end example

@noindent
Two slashes in a row are not normally meaningful in Unix file names, but
they are allowed in GNU Emacs.  They mean, ``ignore everything before the
second slash in the pair.''  Thus, @samp{/u2/emacs/src/} is ignored, and
you get the file @file{/etc/termcap}.

@vindex insert-default-directory
  If you set @code{insert-default-directory} to @code{nil}, the default directory
is not inserted in the minibuffer.  This way, the minibuffer starts out
empty.  But the name you type, if relative, is still interpreted with
respect to the same default directory.

@node Minibuffer Edit, Completion, Minibuffer File, Minibuffer
@section Editing in the Minibuffer

  The minibuffer is an Emacs buffer (albeit a peculiar one), and the usual
Emacs commands are available for editing the text of an argument you are
entering.

  Since @key{RET} in the minibuffer is defined to exit the minibuffer,
inserting a newline into the minibuffer must be done with @kbd{C-o} or with
@kbd{C-q @key{LFD}}.  (Recall that a newline is really the @key{LFD}
character.)

  The minibuffer has its own window which always has space on the screen
but acts as if it were not there when the minibuffer is not in use.  When
the minibuffer is in use, its window is just like the others; you can
switch to another window with @kbd{C-x o}, edit text in other windows and
perhaps even visit more files, before returning to the minibuffer to submit
the argument.  You can kill text in another window, return to the
minibuffer window, and then yank the text to use it in the argument.
@xref{Windows}.

  There are some restrictions on the use of the minibuffer window, however.
You cannot switch buffers in it---the minibuffer and its window are
permanently attached.  Also, you cannot split the minibuffer window.

  Recursive use of the minibuffer is supported by Emacs.  However, it is
easy to do this by accident (because of autorepeating keyboards, for
example) and get confused.  Therefore, most Emacs commands that use the
minibuffer refuse to operate if the minibuffer window is selected.  If the
minibuffer is active but you have switched to a different window, recursive
use of the minibuffer is allowed---if you know enough to try to do this,
you probably will not get confused.

@vindex enable-recursive-minibuffers
  If you set the variable @code{enable-recursive-minibuffers} to be
non-@code{nil}, recursive use of the minibuffer is always allowed.

@node Completion, Repetition, Minibuffer Edit, Minibuffer
@section Completion
@cindex completion

  When appropriate, the minibuffer provides a @dfn{completion} facility.
This means that you type enough of the argument to determine the rest,
based on Emacs's knowledge of which arguments make sense, and Emacs visibly
fills in the rest, or as much as can be determined from the part you have
typed.

  When completion is available, certain keys---@key{TAB}, @key{RET}, and @key{SPC}---are
redefined to complete an abbreviation present in the minibuffer into a
longer string that it stands for, by matching it against a set of
@dfn{completion alternatives} provided by the command reading the argument.
@kbd{?} is defined to display a list of possible completions of what you
have inserted.

  For example, when the minibuffer is being used by @kbd{Meta-x} to read
the name of a command, it is given a list of all available Emacs command
names to complete against.  The completion keys match the text in the
minibuffer against all the command names, find any additional characters of
the name that are implied by the ones already present in the minibuffer,
and add those characters to the ones you have given.

@kindex TAB
@findex minibuffer-complete
  A concrete example may help here.  If you type @kbd{Meta-x au @key{TAB}}, the @key{TAB}
looks for alternatives (in this case, command names) that start with
@samp{au}.  In this case, there are only two: @code{auto-fill-mode} and
@code{auto-save-mode}.  These are the same as far as @code{auto-}, so the
@samp{au} in the minibuffer changes to @samp{auto-}.

  If you go on to type @kbd{f @key{TAB}}, this second @key{TAB} sees @samp{auto-f}.
The only command name starting this way is @code{auto-fill-mode}, so that
is the completion.  You have now have @samp{auto-fill-mode} in the
minibuffer after typing just @kbd{au @key{TAB} f @key{TAB}}.  Note that
@key{TAB} has this effect because in the minibuffer it is bound to the
function @code{minibuffer-complete} when completion is supposed to be done.

  Case is normally significant in completion, because it is significant in
most of the names that you can complete (buffer names, file names and
command names).  Thus, @samp{fo} will not complete to @samp{Foo}.  When you
are completing a name in which case does not matter, the program can request
that case be ignored for completion as well.

  Here is a list of all the completion commands, defined in the minibuffer
when completion is available.

@table @kbd
@item @key{TAB}
Complete the text in the minibuffer as much as possible @*
(@code{minibuffer-complete}).
@item @key{SPC}
Complete the text in the minibuffer but don't add or fill out more
than one word (@code{minibuffer-complete-word}).
@item @key{RET}
Submit the text in the minibuffer as the argument, possibly completing
first as described below (@code{minibuffer-complete-and-exit}).
@item ?
Print a list of all possible completions of the text in the minibuffer
(@code{minibuffer-list-completions}).
@end table

@kindex SPC
@findex minibuffer-complete-word
  @key{SPC} completes much like @key{TAB}, but never adds goes beyond the
next hyphen.  If you have @samp{auto-f} in the minibuffer and type
@key{SPC}, it finds that the completion is @samp{auto-fill-mode}, but it
stops completing after @samp{fill-}.  This gives @samp{auto-fill-}.
Another @key{SPC} at this point completes all the way to
@samp{auto-fill-mode}.  @key{SPC} in the minibuffer runs the function
@code{minibuffer-complete-word} when completion is available.

  There are three different ways that @key{RET} can work in completing
minibuffers, depending on how the argument will be used.

@itemize @bullet
@item
@dfn{Strict} completion is used when it is meaningless to give any
argument except one of the known alternatives.  For example, when
@kbd{C-x k} reads the name of a buffer to kill, it is meaningless to
give anything but the name of an existing buffer.  In strict
completion, @key{RET} refuses to exit if the text in the minibuffer
does not complete to an exact match.

@item
@dfn{Cautious} completion is similar to strict completion, except that
@key{RET} exits only if the text was an exact match already, not
needing completion.  If the text is not an exact match, @key{RET} does
not exit, but it does complete the text.  If it completes to an exact
match, a second @key{RET} will exit.

Cautious completion is used for reading file names for files that must
already exist.

@item
@dfn{Permissive} completion is used when any string whatever is
meaningful, and the list of completion alternatives is just a guide.
For example, when @kbd{C-x C-f} reads the name of a file to visit, any
file name is allowed, in case you want to create a file.  In
permissive completion, @key{RET} takes the text in the minibuffer
exactly as given, without completing it.
@end itemize

@vindex completion-ignored-extensions
  When completion is done on file names, certain file names are usually
ignored.  The variable @code{completion-ignored-extensions} contains a list
of strings; a file whose name ends in any of those strings is ignored as a
possible completion.  The standard value of this variable is @code{(".o"
".elc" "~")}, which is designed to allow @samp{foo} to complete to
@samp{foo.c} even though @samp{foo.o} exists as well.  If the only possible
completions are files that end in ``ignored'' strings, then they are not
ignored.

@node Repetition,, Completion, Minibuffer
@section Repeating Minibuffer Commands

  Every command that uses the minibuffer at least once is recorded on a
special history list, together with the values of the minibuffer arguments,
so that you can repeat the command easily.  In particular, every
use of @kbd{Meta-x} is recorded, since @kbd{M-x} uses the minibuffer to
read the command name.

@c widecommands
@table @kbd
@item C-x @key{ESC}
Re-execute a recent minibuffer command @*(@code{repeat-complex-command}).
@end table

@kindex C-x ESC
@findex repeat-complex-command
  @kbd{C-x @key{ESC}} is used to re-execute a recent minibuffer-using
command.  With no argument, it repeats the last such command.  A numeric
argument specifies which command to repeat; one means the last one, and
larger numbers specify earlier ones.

  @kbd{C-x @key{ESC}} works by turning the previous command into a Lisp
expression and then entering a minibuffer initialized with the text for
that expression.  If you type just @key{RET}, the command is repeated as
before.  You can also change the command by editing the Lisp expression.
Whatever expression you finally submit is what will be executed.  The
repeated command does not go on the command history itself; @kbd{C-x
@key{ESC}} does not alter the command history.

@kindex M-n
@kindex M-p
@findex next-complex-command
@findex previous-complex-command
  Once inside the minibuffer for @kbd{C-x @key{ESC}}, if the command shown
to you is not the one you want to repeat, you can move around the list of
previous commands using @kbd{M-n} and @kbd{M-p}.  @kbd{M-p} replaces the
contents of the minibuffer with the next earlier recorded command, and
@kbd{M-n} replaces them with the next later command.  After finding the
desired previous command, you can edit its expression as usual and then
resubmit it by typing @key{RET} as usual.  Any editing you have done on the
command to be repeated is lost if you use @kbd{M-n} or @kbd{M-p}.

  @kbd{M-p} is more useful than @kbd{M-n}, since more often you will
initially request to repeat the most recent command and then decide to
repeat an older one instead.  These keys are specially defined within
@kbd{C-x @key{ESC}} to run the commands @code{next-complex-command} and
@code{previous-complex-command}.

@vindex command-history
  The list of previous minibuffer-using commands is stored as a Lisp list
in the variable @code{command-history}.  Each element is a Lisp expression
which describes one command and its arguments.  The command can be
reexecuted by feeding the corresponding @code{command-history} element to
@code{eval}.

@node M-x, Help, Minibuffer, Top
@chapter Running Commands by Name

  The Emacs commands that are used often or that must be quick to type are
bound to keys---short sequences of characters---for convenient use.  Other
Emacs commands that do not need to be brief are not bound to keys; to run
them, you must refer to them by name.

  A command name is, by convention, made up of one or more words, separated
by hyphens; for example, @code{auto-fill-mode} or @code{manual-entry}.  The
use of English words makes the command name easier to remember than a key
made up of obscure characters, even though it is more characters to type.
Any command can be run by name, even if it is also runnable by keys.

@kindex M-x
@findex execute-extended-command
@cindex minibuffer
  The way to run a command by name is to start with @kbd{M-x}, type the
command name, and finish it with @key{RET}.  Actually, @kbd{M-x} (the command
@code{execute-extended-command}) is using the minibuffer to read the
command name.

  Emacs uses the minibuffer for reading input for many different purposes;
on this occasion, the string @samp{M-x} is displayed at the beginning of
the minibuffer as a @dfn{prompt} to remind you that your input should be
the name of a command to be run.  @xref{Minibuffer}, for full information
the features of the minibuffer.

  You can use completion to enter the command name.  For example, the
command @code{forward-char} can be invoked by name by typing

@example
M-x forward-char @key{RET}

@exdent or

M-x fo @key{TAB} c @key{RET}
@end example

@noindent
Note that @code{forward-char} is the same command that you invoke with
the key @kbd{C-f}.  Any command (interactively callable function) defined
in Emacs can be called by its name using @kbd{M-x} whether or not any
keys are bound to it.

  If you type @kbd{C-g} while the command name is being read, you cancel
the @kbd{M-x} command and get out of the minibuffer, ending up at top level.

  To pass a numeric argument to the command you are invoking with
@kbd{M-x}, specify the numeric argument before the @kbd{M-x}.  @kbd{M-x}
passes the argument along to the function which it calls.  The argument
value appears in the prompt while the command name is being read.

  Normally, when describing a command that is run by name, we omit the
@key{RET} that is needed to terminate the name.  Thus we might speak of
@kbd{M-x auto-fill-mode} rather than @kbd{M-x auto-fill-mode @key{RET}}.
We mention the @key{RET} only when there is a need to emphasize its
presence, such as when describing a sequence of input that contains a
command name and arguments that follow it.

@iftex
  In this manual, the convention for font usage is that Lisp objects,
including command names (which are Lisp symbols), appear in @code{this
font}, but keyboard input appears in @kbd{this font}.  This brings up
a problem with names of commands that are normally run by name: is the
name a piece of Lisp code, or is it a sequence of characters to type?
Unfortunately, it is both, but only one of the two fonts can be used.
I have chosen to use the Lisp object font when discussing the command,
as in @code{auto-fill-mode}, but to use the keyboard input font for
sequences of input, as in @kbd{M-x auto-fill-mode}.
@end iftex

@node Help, Mark, M-x, Top
@chapter Help
@kindex Help
@cindex help
@cindex self-documentation

  Emacs provides extensive help features which revolve around a single
character, @kbd{C-h}.  @kbd{C-h} is a prefix key that is used only for
documentation-printing commands.  The characters that you can type after
@kbd{C-h} are called @dfn{help options}.  One help option is @kbd{C-h};
that is how you ask for help about using @kbd{C-h}.

  @kbd{C-h C-h} prints a list of the possible help options, and then asks
you to go ahead and type the option.  It prompts with a string

@smallexample
A, C, F, I, K, L, M, N, S, T, V, W, C-c, C-d, C-w or C-h for more help: 
@end smallexample

@noindent
and you should type one of those characters.  Typing a third @kbd{C-h}
displays a description of what the options mean; it still waits for you to
type an option.  To cancel, type @kbd{C-g}.

  Here is a summary of the defined help commands.

@table @kbd
@item C-h a
Display list of commands whose names contain a specified string
(@code{command-apropos}).
@item C-h b
Display a table of all key bindings in effect now; local bindings of
the current major mode first, followed by all global bindings
(@code{describe-bindings}).
@item C-h c @var{key}
Print the name of the command that @var{key} runs (@code{describe-key-briefly}).
@kbd{c} is for `character'.
@item C-h f @var{function} @key{RET}
Display documentation on the Lisp function named @var{function}
(@code{describe-function}).  Note that commands are Lisp functions, so
a command name may be used.
@item C-h k @var{key}
Display name and documentation of the command @var{key} runs (@code{describe-key}).
@item C-h i
Run Info, the program for browsing documentation files (@code{info}).
@item C-h l
Display a description of the last 100 characters you typed
(@code{view-lossage}).
@item C-h m
Display documentation of the current major mode (@code{describe-mode}).
@item C-h n
Display documentation of Emacs changes, most recent first
(@code{view-emacs-news}).
@item C-h s
Display current contents of the syntax table, plus an explanation of
what they mean (@code{describe-syntax}).
@item C-h t
Display the Emacs tutorial (@code{help-with-tutorial}).
@item C-h v @var{var} @key{RET}
Display the documentation of the Lisp variable @var{var}
(@code{describe-variable}).
@item C-h w @var{command} @key{RET}
Print which keys run the command named @var{command} (@code{where-is}).
@end table

@section Documentation for a Key

@kindex C-h c
@findex describe-key-briefly
  The most basic @kbd{C-h} options are @kbd{C-h c}
(@code{describe-key-briefly}) and @kbd{C-h k} (@code{describe-key}).
@kbd{C-h c @var{key}} prints in the echo area the name of the command that
@var{key} is bound to.  For example, @kbd{C-h c C-f} prints
@samp{forward-char}.  Since command names are chosen to describe what the
command does, this is a good way to get a very brief description of what
@var{key} does.@refill

@kindex C-h k
@findex describe-key
  @kbd{C-h k @var{key}} is similar but gives more information.  It displays
the documentation string of the command @var{key} is bound to as well as
its name.  This is too big for the echo area, so a window is used for the
display.

@section Help by Command or Variable Name

@kindex C-h f
@findex describe-function
  @kbd{C-h f} (@code{describe-function}) reads the name of a Lisp function
using the minibuffer, then displays that function's documentation string
in a window.  Since commands are Lisp functions, you can use this to get
the documentation of a command that is known by name.  For example,

@example
C-h f auto-fill-mode @key{RET}
@end example

@noindent
displays the documentation of @code{auto-fill-mode}.  This is the only
way to see the documentation of a command that is not bound to any key
(one which you would normally call using @kbd{M-x}).

  @kbd{C-h f} is also useful for Lisp functions that you are planning to
use in a Lisp program.  For example, if you have just written the code
@code{(make-vector len)} and want to be sure that you are using
@code{make-vector} properly, type @kbd{C-h f make-vector @key{RET}}.  Because
@kbd{C-h f} allows all function names, not just command names, you may find
that some of your favorite abbreviations that work in @kbd{M-x} don't work
in @kbd{C-h f}.  An abbreviation may be unique among command names yet fail
to be unique when other function names are allowed.

  The function name for @kbd{C-h f} to describe has a default which is
used if you type @key{RET} leaving the minibuffer empty.  The default is
the function called by the innermost Lisp expression in the buffer around
point, @i{provided} that is a valid, defined Lisp function name.  For
example, if point is located following the text @samp{(make-vector (car
x)}, the innermost list containing point is the one that starts with
@samp{(make-vector}, so the default is to describe the function
@code{make-vector}.

  @kbd{C-h f} is often useful just to verify that you have the right
spelling for the function name.  If @kbd{C-h f} mentions a default in the
prompt, you have typed the name of a defined Lisp function.  If that tells
you what you want to know, just type @kbd{C-g} to cancel the @kbd{C-h f}
command and go on editing.

@kindex C-h w
@findex where-is
  @kbd{C-h w @var{command} @key{RET}} tells you what keys are bound to
@var{command}.  It prints a list of the keys in the echo area.
Alternatively, it says that the command is not on any keys, which implies
that you must use @kbd{M-x} to call it.@refill

@kindex C-h v
@findex describe-variable
  @kbd{C-h v} (@code{describe-variable}) is like @kbd{C-h f} but describes
Lisp variables instead of Lisp functions.  Its default is the Lisp symbol
around or before point, but only if that is the name of a known Lisp
variable.  @xref{Variables}.@refill

@section Apropos

@kindex C-h a
@findex command-apropos
@cindex apropos
  A more sophisticated sort of question to ask is, ``What are the commands
for working with files?''  For this, type @kbd{C-h a file @key{RET}}, which
displays a list of all command names that contain @samp{file}, such as
@code{copy-file}, @code{find-file}, and so on.  With each command name
appears a brief description of how to use the command, and what keys you
can currently invoke it with.  For example, it would say that you can
invoke @code{find-file} by typing @kbd{C-x C-f}.  The @kbd{a} in @kbd{C-h
a} stands for `Apropos'; @kbd{C-h a} runs the Lisp function
@code{command-apropos}.@refill

  Because @kbd{C-h a} looks only for functions whose names contain the
string which you specify, you must use ingenuity in choosing substrings.
If you are looking for commands for killing backwards and @kbd{C-h a
kill-backwards @key{RET}} doesn't reveal any, don't give up.  Try just
@kbd{kill}, or just @kbd{backwards}, or just @kbd{back}.  Be persistent.
Pretend you are playing Adventure.

  Here is a set of arguments to give to @kbd{C-h a} that covers many
classes of Emacs commands, since there are strong conventions for naming
the standard Emacs commands.  By giving you a feel for the naming
conventions, this set should also serve to aid you in developing a
technique for picking @code{apropos} strings.

@quotation
char, line, word, sentence, paragraph, region, page, sexp, list, defun,
buffer, screen, window, file, dir, register, mode,
beginning, end, forward, backward, next, previous, up, down, search, goto,
kill, delete, mark, insert, yank, fill, indent, case,
change, set, what, list, find, view, describe.
@end quotation

@findex apropos
  To list all Lisp symbols that contain a match for a regexp, not just
the ones that are defined as commands, use the command @kbd{M-x apropos}
instead of @kbd{C-h a}.

@section Other Help Commands

@kindex C-h l
@findex view-lossage
  If something surprising happens, and you are not sure what commands you
typed, use @kbd{C-h l} (@code{view-lossage}).  @kbd{C-h l} prints the last
100 command characters you typed in.  If you see commands that you don't
know, you can use @kbd{C-h c} to find out what they do.

@kindex C-h m
@findex describe-mode
  Emacs has several major modes, each of which redefines a few keys and
makes a few other changes in how editing works.  @kbd{C-h m} (@code{describe-mode})
prints documentation on the current major mode, which normally describes
all the commands that are changed in this mode.

@kindex C-h b
@findex describe-bindings
  @kbd{C-h b} (@code{describe-bindings}) and @kbd{C-h s}
(@code{describe-syntax}) present other information about the current
Emacs mode.  @kbd{C-h b} displays a list of all the key bindings now
in effect; the local bindings of the current major mode first,
followed by the global bindings (@pxref{Key Bindings}).  @kbd{C-h s}
displays the contents of the syntax table, with explanations of each
character's syntax (@pxref{Syntax}).@refill

@kindex C-h i
@findex info
@kindex C-h n
@findex view-emacs-news
@kindex C-h t
@findex help-with-tutorial
@kindex C-h C-c
@findex describe-copying
@kindex C-h C-d
@findex describe-distribution
@kindex C-h C-w
@findex describe-no-warranty
  The other @kbd{C-h} options display various files of useful information.
@kbd{C-h C-w} displays the full details on the complete absence of warranty
for GNU Emacs.  @kbd{C-h n} (@code{view-emacs-news}) displays the file
@file{emacs/etc/NEWS}, which contains documentation on Emacs changes
arranged chronologically.  @kbd{C-h t} (@code{help-with-tutorial}) displays
the learn-by-doing Emacs tutorial.  @kbd{C-h i} (@code{info}) runs the Info
program, which is used for browsing through structured documentation files.
@kbd{C-h C-c} (@code{describe-copying}) displays the file
@file{emacs/etc/COPYING}, which tells you the conditions you must obey in
distributing copies of Emacs.  @kbd{C-h C-d} (@code{describe-distribution})
displays the file @file{emacs/etc/DISTRIB}, which tells you how you can
order a copy of the latest version of Emacs.@refill

@node Mark, Killing, Help, Top
@chapter The Mark and the Region
@cindex mark
@cindex region

  There are many Emacs commands which operate on an arbitrary contiguous
part of the current buffer.  To specify the text for such a command to
operate on, you set @dfn{the mark} at one end of it, and move point to the
other end.  The text between point and the mark is called @dfn{the region}.
You can move point or the mark to adjust the boundaries of the region.  It
doesn't matter which one is set first chronologically, or which one comes
earlier in the text.

  Once the mark has been set, it remains until it is set again at another
place.  The mark remains fixed with respect to the preceding character if
text is inserted or deleted in the buffer.  Each Emacs buffer has its own
mark, so that when you return to a buffer that had been selected
previously, it has the same mark it had before.

  Many commands that insert text, such as @kbd{C-y} (@code{yank}) and
@kbd{M-x insert-buffer}, position the mark at one end of the inserted
text---the opposite end from where point is positioned, so that the region
contains the text just inserted.

@menu
* Mark Ring::

  Aside from delimiting the region, the mark is also useful for remembering
a spot that you may want to go back to.  To make this feature more useful,
Emacs remembers 16 previous locations of the mark, in the @code{mark ring}.
@end menu

  Here are some commands for setting the mark:

@c WideCommands
@table @kbd
@item C-@key{SPC}
Set the mark where point is (@code{set-mark-command}).
@item C-@@
The same.
@item C-x C-x
Interchange mark and point (@code{exchange-point-and-mark}).
@item M-@@
Set mark after end of next word (@code{mark-word}).  This command and
the following one do not move point.
@item C-M-@@
Set mark after end of next Lisp expression (@code{mark-sexp}).
@item M-h
Put region around current paragraph (@code{mark-paragraph}).
@item C-M-h
Put region around current Lisp defun (@code{mark-defun}).
@item C-x h
Put region around entire buffer (@code{mark-whole-buffer}).
@item C-x C-p
Put region around current page (@code{mark-page}).
@end table

  For example, if you wish to convert part of the buffer to all upper-case,
you can use the @kbd{C-x C-u} (@code{upcase-region}) command, which operates
on the text in the region.  You can first go to the beginning of the text
to be capitalized, type @kbd{C-@key{SPC}} to put the mark there, move to
the end, and then type @kbd{C-x C-u}.  Or, you can set the mark at the end
of the text, move to the beginning, and then type @kbd{C-x C-u}.  Most
commands that operate on the text in the region have the word @code{region}
in their names.

@kindex C-SPC
@findex set-mark-command
  The most common way to set the mark is with the @kbd{C-@key{SPC}} command
(@code{set-mark-command}).  This sets the mark where point is.  Then you
can move point away, leaving the mark behind.  It is actually incorrect to
speak of the character @kbd{C-@key{SPC}}; there is no such character.  When
you type @key{SPC} while holding down @key{CTRL}, what you get on most
terminals is the character @kbd{C-@@}.  This is the key actually bound to
@code{set-mark-command}.  But unless you are unlucky enough to have a
terminal where typing @kbd{C-@key{SPC}} does not produce @kbd{C-@@}, you
might as well think of this character as @kbd{C-@key{SPC}}.

@kindex C-x C-x
@findex exchange-point-and-mark
  Since terminals have only one cursor, there is no way for Emacs to show
you where the mark is located.  You have to remember.  The usual solution
to this problem is to set the mark and then use it soon, before you forget
where it is.  But you can see where the mark is with the command @kbd{C-x
C-x} (@code{exchange-point-and-mark}) which puts the mark where point was and
point where the mark was.  The extent of the region is unchanged, but the
cursor and point are now at the previous location of the mark.

  @kbd{C-x C-x} is also useful when you are satisfied with the location of
point but want to move the mark; do @kbd{C-x C-x} to put point there and
then you can move it.  A second use of @kbd{C-x C-x}, if necessary, puts
the mark at the new location with point back at its original location.

@section Operating on the Region

  Once you have created an active region, you can do many things to
the text in it:
@itemize @bullet
@item
Kill it with @kbd{C-w} (@pxref{Killing}).
@item
Save it in a register with @kbd{C-x x} (@pxref{Registers}).
@item
Save it in a buffer or a file (@pxref{Accumulating Text}).
@item
Convert case with @kbd{C-x C-l} or @kbd{C-x C-u} @*(@pxref{Case}).
@item
Evaluate it as Lisp code with @kbd{M-x eval-region} (@pxref{Lisp Eval}).
@item
Fill it as text with @kbd{M-g} (@pxref{Filling}).
@item
Print hardcopy with @kbd{M-x print-region} (@pxref{Hardcopy}).
@item
Indent it with @kbd{C-x @key{TAB}} or @kbd{C-M-\} (@pxref{Indentation}).
@end itemize

@section Commands to Mark Textual Objects

@kindex M-@@
@kindex C-M-@@
@findex mark-word
@findex mark-sexp
  There are commands for placing the mark on the other side of a certain
object such as a word or a list, without having to move there first.
@kbd{M-@@} (@code{mark-word}) puts the mark at the end of the next word,
while @kbd{C-M-@@} (@code{mark-sexp}) puts it at the end of the next Lisp
expression.  These characters allow you to save a little typing or
redisplay, sometimes.

@kindex M-h
@kindex C-M-h
@kindex C-x C-p
@kindex C-x h
@findex mark-paragraph
@findex mark-defun
@findex mark-page
@findex mark-whole-buffer
   Other commands set both point and mark, to delimit an object in the
buffer.  @kbd{M-h} (@code{mark-paragraph}) moves point to the beginning of
the paragraph that surrounds or follows point, and puts the mark at the end
of that paragraph (@pxref{Paragraphs}).  @kbd{M-h} does all that's
necessary if you wish to indent, case-convert, or kill a whole paragraph.
@kbd{C-M-h} (@code{mark-defun}) similarly puts point before and the mark
after the current or following defun (@pxref{Defuns}).  @kbd{C-x C-p}
(@code{mark-page}) puts point before the current page (or the next or
previous, according to the argument), and mark at the end (@pxref{Pages}).
The mark goes after the terminating page delimiter (to include it), while
point goes after the preceding page delimiter (to exclude it).  Finally,
@kbd{C-x h} (@code{mark-whole}) sets up the entire buffer as the region, by
putting point at the beginning and the mark at the end.

@node Mark Ring,, Mark, Mark
@section The Mark Ring

@kindex C-u C-SPC
@cindex mark ring
@kindex C-u C-@@
  Aside from delimiting the region, the mark is also useful for remembering
a spot that you may want to go back to.  To make this feature more useful,
Emacs remembers 16 previous locations of the mark, in the @code{mark ring}.
Most commands that set the mark push the old mark onto this ring.  To
return to a marked location, use @kbd{C-u C-@@} (or @kbd{C-u C-@key{SPC}});
this is the command @code{set-mark-command} given a numeric argument.  This
moves point to where the mark was, and restores the mark from the ring of
former marks.  So repeated use of this command moves point to all of the
old marks on the ring, one by one.  Enough uses of @kbd{C-u C-@@} bring
point back to where it was originally.

  Each buffer has its own mark ring.  All editing commands that use the
mark ring use the current buffer's mark ring.  In particular, @kbd{C-u
C-@key{SPC}} always stays in the same buffer.

  Many commands that can move long distances, such as @kbd{M-<}
(@code{beginning-of-buffer}), start by setting the mark and saving the old
mark on the mark ring, just as a way of making it possible for you to move
to where point was before the command.  This is to make it easier for you
to move back later.  Searches do this except when they do not actually move
point.  You can tell when a command sets the mark because @samp{Mark Set}
is printed in the echo area.

@vindex mark-ring-max
  The variable @code{mark-ring-max} is the maximum number of entries to
keep in the mark ring.  If that many entries exist and another one is
pushed, the last one in the list is discarded.  Repeating @kbd{C-u
C-@key{SPC}} circulates through the limited number of entries that are
currently in the ring.

@vindex mark-ring
  The variable @code{mark-ring} holds the mark ring itself, as a list of
marker objects in the order most recent first.

@iftex
@chapter Killing and Moving Text

  @dfn{Killing} means erasing text and copying it into the @dfn{kill ring},
from which it can be retrieved by @dfn{yanking} it.

  The commonest way of moving or copying text with Emacs is to kill it and
later yank it in one or more places.  This is very safe because all the
text killed recently is remembered, and it is versatile, because the many
commands for killing syntactic units can also be used for moving those
units.  There are also other ways of copying text for special purposes.

  Emacs has only one kill ring, so you can kill text in one buffer and yank
it in another buffer.

@end iftex

@node Killing, Yanking, Mark, Top
@section Deletion and Killing
@findex delete-char
@c ??? Should be backward-delete-char
@findex delete-backward-char

@cindex killing
@cindex deletion
@kindex C-d
@kindex DEL
  Most commands which erase text from the buffer save it so that you can
get it back if you change your mind, or move or copy it to other parts of
the buffer.  These commands are known as @dfn{kill} commands.  The rest of
the commands that erase text do not save it; they are known as @dfn{delete}
commands.  (This distinction is made only for erasure of text in the
buffer.)

  The delete commands include @kbd{C-d} (@code{delete-char}) and
@key{DEL} (@code{delete-backward-char}), which delete only one character at
a time, and those commands that delete only spaces or newlines.  Commands
that can destroy significant amounts of nontrivial data generally kill.
The commands' names and individual descriptions use the words @samp{kill}
and @samp{delete} to say which they do.  If you do a kill or delete command
by mistake, you can use the @kbd{C-x u} (@code{undo}) command to undo it
(@pxref{Undo}).@refill

@subsection Deletion

@table @kbd
@item C-d
Delete next character (@code{delete-char}).
@item @key{DEL}
Delete previous character (@code{delete-backward-char}).
@item M-\
Delete spaces and tabs around point (@code{delete-horizontal-space}).
@item M-@key{SPC}
Delete spaces and tabs around point, leaving one space
(@code{just-one-space}).
@item C-x C-o
Delete blank lines around the current line (@code{delete-blank-lines}).
@item M-^
Join two lines by deleting the intervening newline, and any indentation
following it (@code{delete-indentation}).
@end table

  The most basic delete commands are @kbd{C-d} (@code{delete-char}) and
@key{DEL} (@code{delete-backward-char}).  @kbd{C-d} deletes the character
after point, the one the cursor is ``on top of''.  Point doesn't move.
@key{DEL} deletes the character before the cursor, and moves point back.
Newlines can be deleted like any other characters in the buffer; deleting a
newline joins two lines.  Actually, @kbd{C-d} and @key{DEL} aren't always
delete commands; if given an argument, they kill instead, since they can
erase more than one character this way.

@kindex M-\
@findex delete-horizontal-space
@kindex M-SPC
@findex just-one-space
@kindex C-x C-o
@findex delete-blank-lines
@kindex M-^
@findex delete-indentation
  The other delete commands are those which delete only formatting
characters: spaces, tabs and newlines.  @kbd{M-\} (@code{delete-horizontal-space})
deletes all the spaces and tab characters before and after point.
@kbd{M-@key{SPC}} (@code{just-one-space}) does likewise but leaves a single
space after point, regardless of the number of spaces that existed
previously (even zero).

  @kbd{C-x C-o} (@code{delete-blank-lines}) deletes all blank lines after
the current line, and if the current line is blank deletes all blank lines
preceding the current line as well (leaving one blank line, the current
line).  @kbd{M-^} (@code{delete-indentation}) joins the current line and
the previous line, or the current line and the next line if given an
argument, by deleting a newline and all surrounding spaces, possibly
leaving a single space.  @xref{Indentation,M-^}.

@subsection Killing by Lines

@table @kbd
@item C-k
Kill rest of line or one or more lines (@code{kill-line}).
@end table

@kindex C-k
@findex kill-line
  The simplest kill command is @kbd{C-k}.  If given at the beginning of a
line, it kills all the text on the line, leaving it blank.  If given on a
blank line, the blank line disappears.  As a consequence, if you go to the
front of a non-blank line and type @kbd{C-k} twice, the line disappears
completely.

  More generally, @kbd{C-k} kills from point up to the end of the line,
unless it is at the end of a line.  In that case it kills the newline
following the line, thus merging the next line into the current one.
Invisible spaces and tabs at the end of the line are ignored when deciding
which case applies, so if point appears to be at the end of the line, you
can be sure the newline will be killed.

  If @kbd{C-k} is given a positive argument, it kills that many lines and
the newlines that follow them (however, text on the current line before
point is spared).  With a negative argument, it kills back to a number of
line beginnings.  An argument of -2 means kill back to the second line
beginning.  If point is at the beginning of a line, that line beginning
doesn't count, so @kbd{C-u - 2 C-k} with point at the front of a line kills
the two previous lines.

  @kbd{C-k} with an argument of zero kills all the text before point on the
current line.

@subsection Other Kill Commands
@findex kill-line
@findex kill-region
@findex kill-word
@findex backward-kill-word
@findex kill-sexp
@findex kill-sentence
@findex backward-kill-sentence
@kindex M-d
@kindex M-DEL
@kindex C-M-k
@kindex C-x DEL
@kindex M-k
@kindex C-k
@kindex C-w

@c DoubleWideCommands
@table @kbd
@item C-w
Kill region (from point to the mark) (@code{kill-region}).
@item M-d
Kill word (@code{kill-word}).
@item M-@key{DEL}
Kill word backwards (@code{backward-kill-word}).
@item C-x @key{DEL}
Kill back to beginning of sentence (@code{backward-kill-sentence}).
@xref{Sentences}.
@item M-k
Kill to end of sentence (@code{kill-sentence}).
@item C-M-k
Kill sexp (@pxref{Lists}) (@code{kill-sexp}).
@item M-z @var{char}
Kill up to next occurrence of @var{char} (@code{zap-to-char}).
@end table

  A kill command which is very general is @kbd{C-w} (@code{kill-region}),
which kills everything between point and the mark.  With this command, you
can kill any contiguous sequence of characters, if you first set the mark
at one end of them and go to the other end.

@kindex M-z
@findex zap-to-char
  A convenient way of killing is combined with searching: @kbd{M-z}
(@code{zap-to-char}) reads a character and kills from point up to (but not
including) the next occurrence of that character in the buffer.  If there
is no next occurrence, killing goes to the end of the buffer.  A numeric
argument acts as a repeat count.  A negative argument means to search
backward and kill text before point.

  Other syntactic units can be killed: words, with @kbd{M-@key{DEL}} and
@kbd{M-d} (@pxref{Words}); sexps, with @kbd{C-M-k} (@pxref{Lists}); and
sentences, with @kbd{C-x @key{DEL}} and @kbd{M-k}
(@pxref{Sentences}).@refill

@node Yanking, Accumulating Text, Killing, Top
@section Yanking
@cindex moving text
@cindex kill ring
@cindex yanking

  @dfn{Yanking} is getting back text which was killed.  The usual way to
move or copy text is to kill it and then yank it one or more times.

@table @kbd
@item C-y
Yank last killed text (@code{yank}).
@item M-y
Replace re-inserted killed text with the previously killed text
(@code{yank-pop}).
@item M-w
Save region as last killed text without actually killing it
(@code{copy-region-as-kill}).
@item C-M-w
Append next kill to last batch of killed text (@code{append-next-kill}).
@end table

@kindex C-y
@findex Yank
  All killed text is recorded in the @dfn{kill ring}, a list of blocks of
text that have been killed.  There is only one kill ring, used in all
buffers, so you can kill text in one buffer and yank it in another buffer.
This is the usual way to move text from one file to another.
(@xref{Accumulating Text}, for some other ways.)

  The command @kbd{C-y} (@code{yank}) reinserts the text of the most recent
kill.  It leaves the cursor at the end of the text.  It sets the mark at
the beginning of the text.  @xref{Mark}.

  @kbd{C-u C-y} leaves the cursor in front of the text, and sets the mark
after it.  This is only if the argument is specified with just a @kbd{C-u},
precisely.  Any other sort of argument, including @kbd{C-u} and digits, has
an effect described below (under ``Yanking Earlier Kills'').

@kindex M-w
@findex copy-region-as-kill
  If you wish to copy a block of text, you might want to use @kbd{M-w}
(@code{copy-region-as-kill}), which copies the region into the kill ring
without removing it from the buffer.  This is approximately equivalent to
@kbd{C-w} followed by @kbd{C-y}, except that @kbd{M-w} does not mark the
buffer as ``modified'' and does not temporarily change the screen.

@subsection Appending Kills

  Normally, each kill command pushes a new block onto the kill ring.
However, two or more kill commands in a row combine their text into a
single entry, so that a single @kbd{C-y} gets it all back as it was before
it was killed.  This means that you don't have to kill all the text in one
command; you can keep killing line after line, or word after word, until
you have killed it all, and you can still get it all back at once.  (Thus
we join television in leading people to kill thoughtlessly.)

  Commands that kill forward from point add onto the end of the previous
killed text.  Commands that kill backward from point add onto the
beginning.  This way, any sequence of mixed forward and backward kill
commands puts all the killed text into one entry without rearrangement.
Numeric arguments do not break the sequence of appending kills.  For
example, suppose the buffer contains

@example
This is the first
line of sample text
and here is the third.
@end example

@noindent
with point at the beginning of the second line.  If you type @kbd{C-k C-u 2
M-@key{DEL} C-k}, the first @kbd{C-k} kills the text @samp{line of sample
text}, @kbd{C-u 2 M-@key{DEL}} kills @samp{the first} with the newline that
followed it, and the second @kbd{C-k} kills the newline after the second
line.  The result is that the buffer contains @samp{This is and here is the
third.} and a single kill entry contains @samp{the first@key{RET}line of
sample text@key{RET}}---all the killed text, in its original order.

@kindex C-M-w
@findex append-next-kill
  If a kill command is separated from the last kill command by other
commands (not just numeric arguments), it starts a new entry on the kill
ring.  But you can force it to append by first typing the command
@kbd{C-M-w} (@code{append-next-kill}) in front of it.  The @kbd{C-M-w}
tells the following command, if it is a kill command, to append the text it
kills to the last killed text, instead of starting a new entry.  With
@kbd{C-M-w}, you can kill several separated pieces of text and accumulate
them to be yanked back in one place.@refill

@subsection Yanking Earlier Kills

@kindex M-y
@findex yank-pop
  To recover killed text that is no longer the most recent kill, you need
the @kbd{Meta-y} (@code{yank-pop}) command.  @kbd{M-y} can be used only
after a @kbd{C-y} or another @kbd{M-y}.  It takes the text previously
yanked and replaces it with the text from an earlier kill.  So, to recover
the text of the next-to-the-last kill, you first use @kbd{C-y} to recover
the last kill, and then use @kbd{M-y} to replace it with the previous
kill.@refill

  You can think in terms of a ``last yank'' pointer which points at an item
in the kill ring.  Each time you kill, the ``last yank'' pointer moves to
the newly made item at the front of the ring.  @kbd{C-y} yanks the item
which the ``last yank'' pointer points to.  @kbd{M-y} moves the ``last
yank'' pointer to a different item, and the text in the buffer changes to
match.  Enough @kbd{M-y} commands can move the pointer to any item in the
ring, so you can get any item into the buffer.  Eventually the pointer
reaches the end of the ring; the next @kbd{M-y} moves it to the first item
again.

  @kbd{M-y} can take a numeric argument, which tells it how many items to
advance the ``last yank'' pointer by.  A negative argument moves the
pointer toward the front of the ring; from the front of the ring, it moves
to the last entry and starts moving forward from there.

  Once the text you are looking for is brought into the buffer, you can
stop doing @kbd{M-y} commands and it will stay there.  It's just a copy of
the kill ring item, so editing it in the buffer does not change what's in
the ring.  As long as no new killing is done, the ``last yank'' pointer
remains at the same place in the kill ring, so repeating @kbd{C-y} will
yank another copy of the same old kill.

  If you know how many @kbd{M-y} commands it would take to find the
text you want, you can yank that text in one step using @kbd{C-y} with
a numeric argument.  @kbd{C-y} with an argument greater than one
restores the text the specified number of entries back in the kill
ring.  Thus, @kbd{C-u 2 C-y} gets the next to the last block of killed
text.  It is equivalent to @kbd{C-y M-y}.  @kbd{C-y} with a numeric
argument starts counting from the ``last yank'' pointer, and sets the
``last yank'' pointer to the entry that it yanks.

@vindex kill-ring-max
  The length of the kill ring is controlled by the variable
@code{kill-ring-max}; no more than that many blocks of killed text are
saved.

@node Accumulating Text, Rectangles, Yanking, Top
@section Accumulating Text
@kindex C-x a
@findex append-to-buffer
@findex prepend-to-buffer
@findex copy-to-buffer
@findex append-to-file

  Usually we copy or move text by killing it and yanking it, but there are
other ways that are useful for copying one block of text in many places, or
for copying many scattered blocks of text into one place.

  You can accumulate blocks of text from scattered locations either into a
buffer or into a file if you like.  These commands are described here.  You
can also use Emacs registers for storing and accumulating text.
@xref{Registers}.

@table @kbd
@item C-x a
Append region to contents of specified buffer (@code{append-to-buffer}).
@item M-x prepend-to-buffer
Prepend region to contents of specified buffer.
@item M-x copy-to-buffer
Copy region into specified buffer, deleting that buffer's old contents.
@item M-x insert-buffer
Insert contents of specified buffer into current buffer at point.
@item M-x append-to-file
Append region to contents of specified file, at the end.
@end table

  To accumulate text into a buffer, use the command @kbd{C-x a @var{buffername}}
(@code{append-to-buffer}), which inserts a copy of the region into the
buffer @var{buffername}, at the location of point in that buffer.  If there
is no buffer with that name, one is created.  If you append text into a
buffer which has been used for editing, the copied text goes into the
middle of the text of the buffer, wherever point happens to be in it.

  Point in that buffer is left at the end of the copied text, so successive
uses of @kbd{C-x a} accumulate the text in the specified buffer in the same
order as they were copied.  Strictly speaking, @kbd{C-x a} does not always
append to the text already in the buffer; but if @kbd{C-x a} is the only
command used to alter a buffer, it does always append to the existing text
because point is always at the end.

  @kbd{M-x prepend-to-buffer} is just like @kbd{C-x a} except that point in
the other buffer is left before the copied text, so successive prependings
add text in reverse order.  @kbd{M-x copy-to-buffer} is similar except that
any existing text in the other buffer is deleted, so the buffer is left
containing just the text newly copied into it.

  You can retrieve the accumulated text from that buffer with @kbd{M-x
insert-buffer}; this too takes @var{buffername} as an argument.  It inserts
a copy of the text in buffer @var{buffername} into the selected buffer.
You could alternatively select the other buffer for editing, perhaps moving
text from it by killing or with @kbd{C-x a}.  @xref{Buffers}, for
background information on buffers.

  Instead of accumulating text within Emacs, in a buffer, you can append
text directly into a file with @kbd{M-x append-to-file}, which takes
@var{file-name} as an argument.  It adds the text of the region to the end
of the specified file.  The file is changed immediately on disk. This
command is normally used with files that are @i{not} being visited in
Emacs.  Using it on a file that Emacs is visiting can produce confusing
results, because the text inside Emacs for that file will not change
while the file itself changes.

@node Rectangles, Registers, Accumulating Text, Top
@section Rectangles
@cindex rectangles

  The rectangle commands affect rectangular areas of the text: all the
characters between a certain pair of columns, in a certain range of lines.
Commands are provided to kill rectangles, yank killed rectangles, clear
them out, or delete them.  Rectangle commands are useful with text in
multicolumnar formats, such as perhaps code with comments at the right,
or for changing text into or out of such formats.

  When you must specify a rectangle for a command to work on, you do
it by putting the mark at one corner and point at the opposite corner.
The rectangle thus specified is called the @dfn{region-rectangle}
because it is controlled about the same way the region is controlled.
But remember that a given combination of point and mark values can be
interpreted either as specifying a region or as specifying a
rectangle; it is up to the command that uses them to choose the
interpretation.

@table @kbd
@item M-x delete-rectangle
Delete the text of the region-rectangle, moving any following text on
each line leftward to the left edge of the region-rectangle.
@item M-x kill-rectangle
Similar, but also save the contents of the region-rectangle as the
``last killed rectangle''.
@item M-x yank-rectangle
Yank the last killed rectangle with its upper left corner at point.
@item M-x open-rectangle
Insert blank space to fill the space of the region-rectangle.
The previous contents of the region-rectangle are pushed rightward.
@item M-x clear-rectangle
Clear the region-rectangle by replacing its contents with spaces.
@end table

  The rectangle operations fall into two classes: commands deleting and
moving rectangles, and commands for blank rectangles.

@findex delete-rectangle
@findex kill-rectangle
  There are two ways to delete a rectangle: you can discard its contents,
or save them as the ``last killed'' rectangle.  The commands for these
two ways are @kbd{M-x delete-rectangle} and @kbd{M-x kill-rectangle}.  In
any case, the portion of each line that falls inside the rectangle's
boundaries is deleted, causing following text (if any) on the line to move
left.

  Note that ``killing'' a rectangle is not killing in the usual sense; the
rectangle is not stored in the kill ring, but in a special place that
can only record the most recent rectangle killed.  This is because yanking
a rectangle is so different from yanking linear text that different yank
commands have to be used and yank-popping is hard to make sense of.

  Inserting a rectangle is the opposite of deleting one.  All you need to
specify is where to put the upper left corner; that is done by putting
point there.  The rectangle's first line is inserted there, the rectangle's
second line is inserted at a point one line vertically down, and so on.
The number of lines affected is determined by the height of the saved
rectangle.

@findex yank-rectangle
  To insert the last killed rectangle, type @kbd{M-x yank-rectangle}.

@findex open-rectangle
@findex clear-rectangle
  There are two commands for working with blank rectangles: @kbd{M-x
clear-rectangle} to blank out existing text, and @kbd{M-x open-rectangle}
to insert a blank rectangle.  Clearing a rectangle is equivalent to
deleting it and then inserting as blank rectangle of the same size.

  Rectangles can also be copied into and out of registers.
@xref{RegRect,,Rectangle Registers}.

@node Registers, Display, Rectangles, Top
@chapter Registers
@cindex registers

  Emacs @dfn{registers} are places you can save text or positions for
later use.  Text saved in a register can be copied into the buffer
once or many times; a position saved in a register is used by moving
point to that position.  Rectangles can also be copied into and out of
registers (@pxref{Rectangles}).

  Each register has a name, which is a single character.  It can store
either a piece of text or a position or a rectangle; only one of the three
at any given time.  Whatever you store in a register remains there until
you store something else in that register.

@menu
* RegPos::    Saving positions in registers.
* RegText::   Saving text in registers.
* RegRect::   Saving rectangles in registers.
@end menu

@table @kbd
@item M-x view-register @key{RET} @var{r}
Display a description of what register @var{r} contains.
@end table

@findex view-register
  @kbd{M-x view-register} reads a register name as an argument and then
displays the contents of the specified register.

@node RegPos, RegText, Registers, Registers
@section Saving Positions in Registers

  Saving a position records a spot in a buffer so that you can move
back there later.  Moving to a saved position reselects the buffer
and moves point to the spot.

@table @kbd
@item C-x / @var{r}
Save location of point in register @var{r} (@code{point-to-register}).
@item C-x j @var{r}
Jump to the location saved in register @var{r} (@code{register-to-point}).
@end table

@kindex C-x /
@findex point-to-register
  To save the current location of point in a register, choose a name
@var{r} and type @kbd{C-x / @var{r}}.  The register @var{r} retains
the location thus saved until you store something else in that
register.@refill

@kindex C-x j
@findex register-to-point
  The command @kbd{C-x j @var{r}} moves point to the location recorded
in register @var{r}.  The register is not affected; it continues to
record the same location.  You can jump to the same position using the
same register any number of times.

@node RegText, RegRect, RegPos, Registers
@section Saving Text in Registers

  When you want to insert a copy of the same piece of text frequently, it
may be impractical to use the kill ring, since each subsequent kill moves
the piece of text farther down on the ring.  It becomes hard to keep track
of what argument is needed to retrieve the same text with @kbd{C-y}.  An
alternative is to store the text in a register with @kbd{C-x x}
(@code{copy-to-register}) and then retrieve it with @kbd{C-x g}
(@code{insert-register}).

@table @kbd
@item C-x x @var{r}
Copy region into register @var{r} (@code{copy-to-register}).
@item C-x g @var{r}
Insert text contents of register @var{r} (@code{insert-register}).
@end table

@kindex C-x x
@kindex C-x g
@findex copy-to-register
@findex insert-register
  @kbd{C-x x @var{r}} stores a copy of the text of the region into the
register named @var{r}.  Given a numeric argument, @kbd{C-x x} deletes the
text from the buffer as well.

  @kbd{C-x g @var{r}} inserts in the buffer the text from register @var{r}.
Normally it leaves point before the text and places the mark after, but
with a numeric argument it puts point after the text and the mark before.

@node RegRect,, RegText, Registers
@section Saving Rectangles in Registers
@cindex rectangle

  A register can contain a rectangle instead of linear text.  The rectangle
is represented as a list of strings.  @xref{Rectangles}, for basic
information on rectangles and how rectangles in the buffer are specified.

@table @kbd
@item C-x r @var{r}
Copy the region-rectangle into register @var{r} @*(@code{copy-region-to-rectangle}).
With numeric argument, delete it as well.
@item C-x g @var{r}
Insert the rectangle stored in register @var{r} (if it contains a
rectangle) (@code{insert-register}).
@end table

  The @kbd{C-x g} command inserts linear text if the register contains
that, or inserts a rectangle if the register contains one.

@node Display, Search, Registers, Top
@chapter Controlling the Display
@cindex scrolling

  Since only part of a large buffer fits in the window, Emacs tries to show
the part that is likely to be interesting.  The display control commands
allow you to ask to see a different part of the text.  This is also known
as @dfn{scrolling}.

  If a buffer contains text that is too large to fit entirely within a
window that is displaying the buffer, Emacs shows a contiguous section of
the text.  The section shown always contains point.  As you change the
text, Emacs always tries to keep the same position in the text at the top
of the window.  A new position moves to the top of the window only if this
is necessary to keep point visible, or if you request it explicitly with a
display control command.

@table @kbd
@item C-l
Clear screen and redisplay, scrolling the selected window to center
point vertically within it (@code{recenter}).
@item C-v
Scroll forward (a windowful or a specified number of lines) (@code{scroll-up}).
@item M-v
Scroll backward (@code{scroll-down}).
@item C-x <
Scroll text in current window to the left (@code{scroll-left}).
@item C-x >
Scroll to the right (@code{scroll-right}).
@item M-r
Move point to the text at a given vertical position within the window
(@code{move-to-window-line}).
@item C-x $
Make deeply indented lines invisible (@code{set-selective-display}).
@end table

@kindex C-l
@findex recenter
  The basic display control command is @kbd{C-l} (@code{recenter}).  In its
simplest form, with no argument, it clears the entire screen and redisplays
all windows, scrolling the selected window so that point is halfway down
from the top of the window.  Other windows are cleared and redisplayed, but
not scrolled.

  @kbd{C-l} with a numeric argument does not clear the screen; it does
nothing except scroll the selected window as specified by the argument.
With a positive argument @var{n}, it repositions text to put point @var{n}
lines down from the top.  An argument of zero puts point on the very top
line.  Point does not move with respect to the text; rather, the text and
point move rigidly on the screen.  @kbd{C-l} with a negative argument puts
point that many lines from the bottom of the window.  For example, @kbd{C-u
- 1 C-l} puts point on the bottom line, and @kbd{C-u - 5 C-l} puts it five
lines from the bottom.

@kindex C-v
@kindex M-v
@findex scroll-up
@findex scroll-down
  The @dfn{scrolling} commands @kbd{C-v} and @kbd{M-v} let you move the
whole display up or down a few lines.  @kbd{C-v} (@code{scroll-up}) with an
argument shows you that many more lines at the bottom of the window, moving
the text and point up together as @kbd{C-l} might.  @kbd{C-v} with a
negative argument shows you more lines at the top of the window.
@kbd{Meta-v} (@code{scroll-down}) is like @kbd{C-v}, but moves in the
opposite direction.@refill

@vindex next-screen-context-lines
  To read the buffer a windowful at a time, use @kbd{C-v} with no argument.
It takes the last two lines at the bottom of the window and puts them at
the top, followed by nearly a whole windowful of lines not previously
visible.  If point was in the text scrolled off the top, it moves to the
new top of the window.  @kbd{M-v} with no argument moves backward with
overlap similarly.  The number of lines of overlap across a @kbd{C-v} or
@kbd{M-v} is controlled by the variable @code{next-screen-context-lines}; by
default, it is two.

@vindex scroll-step
  Scrolling happens automatically if point has moved out of the visible
portion of the text when it is time to display.  Usually the scrolling is
done so as to put point vertically centered within the window.  However, if
the variable @code{scroll-step} has a nonzero value, an attempt is made to
scroll the buffer by that many lines; if that is enough to bring point back
into visibility, that is what is done.

@kindex C-x <
@kindex C-x >
@findex scroll-left
@findex scroll-right
@cindex horizontal scrolling
  The text in a window can also be scrolled horizontally.  This means that
each line of text is shifted sideways in the window, and one or more
characters at the beginning of each line are not displayed at all.  When a
window has been scrolled horizontally in this way, text lines are truncated
rather than continued (@pxref{Continuation Lines}), with a @samp{$} appearing
in the first column when there is text truncated to the left, and in the
last column when there is text truncated to the right.

  The command @kbd{C-x <} (@code{scroll-left}) scrolls the selected window
to the left by @var{n} columns with argument @var{n}.  With no argument, it scrolls
by almost the full width of the window (two columns less, to be precise).
@kbd{C-x >} (@code{scroll-right}) scrolls similarly to the right.
The window cannot be scrolled any farther to the right once it is
displaying normally (with each line starting at the window's left margin);
attempting to do so has no effect.

@kindex M-r
@findex move-to-window-line
  The commands described above all change the position of point on the
screen, carrying the text with it.  Another command moves point the same
way but leaves the text fixed.  It is @kbd{Meta-r} (@code{move-to-window-line}).
With no argument, it puts point at the beginning of the line at the center
of the window.  An argument is used to specify the line to put point on,
counting from the top if the argument is positive, or from the bottom if it
is negative.  Thus, @kbd{M-0 M-r} moves point to the text at the top of the
window.  @kbd{Meta-r} never causes any text to move on the screen; it
causes point to move with respect to the screen and the text.

@menu
* Selective Display::  Hiding lines with lots of indentation.
* Display Vars::       Information on variables for customizing display.
@end menu

@node Selective Display, Display Vars, Display, Display
@section Selective Display
@findex set-selective-display
@kindex C-x $

  Emacs has the ability to hide lines indented more than a certain number
of columns (you specify how many columns).  You can use this to get an
overview of a part of a program.

  To hide lines, type @kbd{C-x $} (@code{set-selective-display}) with a
numeric argument @var{n}.  (@xref{Arguments}, for how to give the
argument.)  Then lines with at least @var{n} columns of indentation
disappear from the screen.  The only indication of their presence is that
three dots (@samp{@dots{}}) appear at the end of each visible line that is
followed by one or more invisible ones.@refill

  The invisible lines are still present in the buffer, and most editing
commands see them as usual, so it is very easy to put point in the middle
of invisible text.  When this happens, the cursor appears at the end of the
previous line, after the three dots.  If point is at the end of the visible
line, before the newline that ends it, the cursor appears before the three
dots.

  The commands @kbd{C-n} and @kbd{C-p} move across the invisible lines as if they
were not there.

  To make everything visible again, type @kbd{C-x $} with no argument.

@node Display Vars,, Selective Display, Display
@section Variables Controlling Display
 
  This section contains information for customization only.  Beginning
users should skip it.

@vindex mode-line-inverse-video
  The variable @code{mode-line-inverse-video} controls whether the mode
line is displayed in inverse video (assuming the terminal supports it);
@code{nil} means don't do so.  @xref{Mode Line}.

@vindex inverse-video
  If the variable @code{inverse-video} is non-@code{nil}, Emacs attempts
to invert all the lines of the display from what they normally are.

@vindex visible-bell
If the variable @code{visible-bell} is non-@code{nil}, Emacs attempts
to make the whole screen blink when it would normally make an audible bell
sound.  This variable has no effect if your terminal does not have a way
to make the screen blink.@refill

@vindex echo-keystrokes
The variable @code{echo-keystrokes} controls the echoing of multi-character
keys; its value is the number of seconds of pause required to cause echoing
to start, or zero meaning don't echo at all.  @xref{Echo Area}.

@vindex ctl-arrow
@vindex default-ctl-arrow
If the variable @code{ctl-arrow} is @code{nil}, control characters in the buffer
are displayed with octal escape sequences, all except newline and tab.
This variable has a separate value in each buffer; in new buffers, its
value is initialized from the variable @code{default-ctl-arrow}.

@vindex tab-width
@vindex default-tab-width
  Normally, a tab character in the buffer is displayed as whitespace which
extends to the next display tab stop position, and display tab stops come
at intervals equal to eight spaces.  The number of spaces per tab is
controlled by the variable @code{tab-width}, which is local to every
buffer just like @code{ctl-arrow} and gets its value in a new buffer from
@code{default-tab-width}.  Note that how the tab character in the buffer is
displayed has nothing to do with the definition of @key{TAB} as a command.

@node Search, Fixit, Display, Top
@chapter Searching and Replacement
@cindex searching

  Like other editors, Emacs has commands for searching for occurrences of
a string.  The principal search command is unusual in that it is
@dfn{incremental}; it begins to search before you have finished typing the
search string.  There are also nonincremental search commands more like
those of other editors.

  Besides the usual @code{replace-string} command that finds all
occurrences of one string and replaces them with another, Emacs has a fancy
replacement command called @code{query-replace} which asks interactively
which occurrences to replace.

@menu
* Incremental Search::     Search happens as you type the string.
* Nonincremental Search::  Specify entire string and then search.
* Word Search::            Search for sequence of words.
* Regexp Search::          Search for match for a regexp.
* Regexps::                Syntax of regular expressions.
* Search Case::            To ignore case while searching, or not.
* Replace::                Search, and replace some or all matches.
* Other Repeating Search:: Operating on all matches for some regexp.
@end menu

@node Incremental Search, Nonincremental Search, Search, Search
@section Incremental Search

  An incremental search begins searching as soon as you type the first
character of the search string.  As you type in the search string, Emacs
shows you where the string (as you have typed it so far) would be found.
When you have typed enough characters to identify the place you want, you
can stop.  Depending on what you will do next, you may or may not need to
terminate the search explicitly with an @key{ESC} first.

@c WideCommands
@table @kbd
@item C-s
Incremental search forward (@code{isearch-forward}).
@item C-r
Incremental search backward (@code{isearch-backward}).
@end table

@kindex C-s
@kindex C-r
@findex isearch-forward
@findex isearch-backward
  @kbd{C-s} starts an incremental search.  @kbd{C-s} reads characters from
the keyboard and positions the cursor at the first occurrence of the
characters that you have typed.  If you type @kbd{C-s} and then @kbd{F},
the cursor moves right after the first @samp{F}.  Type an @kbd{O}, and see
the cursor move to after the first @samp{FO}.  After another @kbd{O}, the
cursor is after the first @samp{FOO} after the place where you started the
search.  Meanwhile, the search string @samp{FOO} has been echoed in the
echo area.@refill

  The echo area display ends with three dots when actual searching is going
on.  When search is waiting for more input, the three dots are removed.
(On slow terminals, the three dots are not displayed.)

  If you make a mistake in typing the search string, you can erase
characters with @key{DEL}.  Each @key{DEL} cancels the last character of
search string.  This does not happen until Emacs is ready to read another
input character; first it must either find, or fail to find, the character
you want to erase.  If you do not want to wait for this to happen, use
@kbd{C-g} as described below.@refill

  When you are satisfied with the place you have reached, you can type
@key{ESC}, which stops searching, leaving the cursor where the search
brought it.  Also, any command not specially meaningful in searches stops
the searching and is then executed.  Thus, typing @kbd{C-a} would exit the
search and then move to the beginning of the line.  @key{ESC} is necessary
only if the next command you want to type is a printing character,
@key{DEL}, @key{ESC}, or another control character that is special within
searches (@kbd{C-q}, @kbd{C-w}, @kbd{C-r}, @kbd{C-s} or @kbd{C-k}).

  Sometimes you search for @samp{FOO} and find it, but not the one you
expected to find.  There was a second @samp{FOO} that you forgot about,
before the one you were looking for.  In this event, type another @kbd{C-s}
to move to the next occurrence of the search string.  This can be done any
number of times.  If you overshoot, you can cancel some @kbd{C-s}
characters with @key{DEL}.

  After you exit a search, you can search for the same string again by
typing just @kbd{C-s C-s}: the first @kbd{C-s} is the key that invokes
incremental search, and the second @kbd{C-s} means ``search again''.

  If your string is not found at all, the echo area says @samp{Failing
I-Search}.  The cursor is after the place where Emacs found as much of your
string as it could.  Thus, if you search for @samp{FOOT}, and there is no
@samp{FOOT}, you might see the cursor after the @samp{FOO} in @samp{FOOL}.
At this point there are several things you can do.  If your string was
mistyped, you can rub some of it out and correct it.  If you like the place
you have found, you can type @key{ESC} or some other Emacs command to
``accept what the search offered''.  Or you can type @kbd{C-g}, which
removes from the search string the characters that could not be found (the
@samp{T} in @samp{FOOT}), leaving those that were found (the @samp{FOO} in
@samp{FOOT}).  A second @kbd{C-g} at that point cancels the search
entirely, returning point to where it was when the search started.

@cindex quitting (in search)
  The @kbd{C-g} ``quit'' character does special things during searches;
just what it does depends on the status of the search.  If the search has
found what you specified and is waiting for input, @kbd{C-g} cancels the
entire search.  The cursor moves back to where you started the search.  If
@kbd{C-g} is typed when there are characters in the search string that have
not been found---because Emacs is still searching for them, or because it
has failed to find them---then the search string characters which have not
been found are discarded from the search string.  With them gone, the
search is now successful and waiting for more input, so a second @kbd{C-g}
will cancel the entire search.

  To search for a control character such as @kbd{C-s} or @key{DEL} or @key{ESC},
you must quote it by typing @kbd{C-q} first.  This function of @kbd{C-q} is
analogous to its meaning as an Emacs command: it causes the following
character to be treated the way a graphic character would normally be
treated in the same context.

  You can change to searching backwards with @kbd{C-r}.  If a search fails
because the place you started was too late in the file, you should do this.
Repeated @kbd{C-r} keeps looking for more occurrences backwards.  A
@kbd{C-s} starts going forwards again.  @kbd{C-r} in a search can be cancelled
with @key{DEL}.

  If you know initially that you want to search backwards, you can
use @kbd{C-r} instead of @kbd{C-s} to start the search, because @kbd{C-r}
is also a key running a command (@code{isearch-reverse}) to search
backward.

  The characters @kbd{C-y} and @kbd{C-w} can be used in incremental search
to grab text from the buffer into the search string.  This makes it
convenient to search for another occurrence of text at point.  @kbd{C-w}
copies the word after point as part of the search string, advancing
point over that word.  Another @kbd{C-s} to repeat the search will then
search for a string including that word.  @kbd{C-y} is similar to @kbd{C-w}
but copies all the rest of the current line into the search string.

  All the characters special in incremental search can be changed by setting
the following variables:

@vindex search-delete-char
@vindex search-exit-char
@vindex search-quote-char
@vindex search-repeat-char
@vindex search-reverse-char
@vindex search-yank-line-char
@vindex search-yank-word-char
@table @code
@item search-delete-char
Character to delete from incremental search string (normally @key{DEL}).
@item search-exit-char
Character to exit incremental search (normally @key{ESC}).
@item search-quote-char
Character to quote special characters for incremental search (normally
@kbd{C-q}).
@item search-repeat-char
Character to repeat incremental search forwards (normally @kbd{C-s}).
@item search-reverse-char
Character to repeat incremental search backwards (normally @kbd{C-r}).
@item search-yank-line-char
Character to pull rest of line from buffer into search string
(normally @kbd{C-y}).
@item search-yank-word-char
Character to pull next word from buffer into search string (normally
@kbd{C-w}).
@end table

@subsection Slow Terminal Incremental Search

  Incremental search on a slow terminal uses a modified style of display
that is designed to take less time.  Instead of redisplaying the buffer at
each place the search gets to, it creates a new single-line window and uses
that to display the line that the search has found.  The single-line window
comes into play as soon as point gets outside of the text that is already
on the screen.

  When the search is terminated, the single-line window is removed.  Only
at this time is the window in which the search was done redisplayed to show
its new value of point.

  The three dots at the end of the search string, normally used to indicate
that searching is going on, are not displayed in slow style display.

@vindex isearch-slow-speed
  The slow terminal style of display is used when the terminal baud rate is
less than or equal to the value of the variable @code{isearch-slow-speed},
initially 1200.

@node Nonincremental Search, Word Search, Incremental Search, Search
@section Nonincremental Search
@cindex nonincremental search

  Emacs also has conventional nonincremental search commands, which require
you to type the entire search string before searching begins.

@table @kbd
@item C-s @key{ESC} @var{string} @key{RET}
Search for @var{string}.
@item C-r @key{ESC} @var{string} @key{RET}
Search backward for @var{string}.
@end table

  To do a nonincremental search, first type @kbd{C-s @key{ESC}}.  This
enters the minibuffer to read the search string; terminate the string with
@key{RET}, and then the search is done.  If the string is not found the
search command gets an error.

  The way @kbd{C-s @key{ESC}} works is that the @kbd{C-s} invokes
incremental search, which is specially programmed to invoke nonincremental
search if the argument you give it is empty.  (Such an empty argument would
otherwise be useless.)  @kbd{C-r @key{ESC}} also works this way.

@findex search-forward
@findex search-backward
  Forward and backward nonincremental searches are implemented by the
commands @code{search-forward} and @code{search-backward}.  These commands
may be bound to keys in the usual manner.  The reason that they are reached
by special-case code in incremental search is because @kbd{C-s @key{ESC}}
is the traditional sequence of characters used in Emacs to invoke
nonincremental search.

  However, nonincremental searches performed using @kbd{C-s @key{ESC}} do
not call @code{search-forward} right away.  The first thing done is to see
if the next character is @kbd{C-w}, which requests a word search.
@ifinfo
@xref{Word Search}.
@end ifinfo

@node Word Search, Regexp Search, Nonincremental Search, Search
@section Word Search
@cindex word search

  Word search searches for a sequence of words without regard to how the
words are separated.  More precisely, you type a string of many words,
using single spaces to separate them, and the string can be found even if
there are multiple spaces, newlines or other punctuation between the words.

  Word search is useful in editing documents formatted by text formatters.
If you edit while looking at the printed, formatted version, you can't tell
where the line breaks are in the source file.  With word search, you can
search without having to know them.

@table @kbd
@item C-s @key{ESC} C-w @var{words} @key{RET}
Search for @var{words}, ignoring differences in punctuation.
@item C-r @key{ESC} C-w @var{words} @key{RET}
Search backward for @var{words}, ignoring differences in punctuation.
@end table

  Word search is a special case of nonincremental search and is invoked
with @kbd{C-s @key{ESC} C-w}.  This is followed by the search string, which
must always be terminated with @key{RET}.  Being nonincremental, this
search does not start until the argument is terminated.  It works by
constructing a regular expression and searching for that.  @xref{Regexp
Search}.

  A backward word search can be done by @kbd{C-r @key{ESC} C-w}.

@findex word-search-forward
@findex word-search-backward
  Forward and backward word searches are implemented by the commands
@code{word-search-forward} and @code{word-search-backward}.  These commands
may be bound to keys in the usual manner.  The reason that they are reached
by special-case code in incremental and nonincremental search is because
@kbd{C-s @key{ESC} C-w} is the traditional Emacs sequence of keys to use to
do a word search.

@node Regexp Search, Regexps, Word Search, Search
@section Regular Expression Search
@cindex regular expression
@cindex regexp

  A @dfn{regular expression} (@dfn{regexp}, for short) is a pattern that
denotes a set of strings, possibly an infinite set.  Searching for matches
for a regexp is a very powerful operation that editors on Unix systems have
traditionally offered.  In GNU Emacs, you can search for the next match for
a regexp either incrementally or not.

@kindex C-M-s
@findex isearch-forward-regexp
  Incremental search for a regexp is done by typing @kbd{C-M-s}
(@code{isearch-forward-regexp}).  This command reads a search string
incrementally just like @kbd{C-s}, but it treats the search string as a
regexp rather than looking for an exact match against the text in the
buffer.  Each time you add text to the search string, you make the regexp
longer, and the new regexp is searched for.

  Note that adding characters to the regexp in an incremental regexp search
does not make the cursor move back and start again.  Perhaps it ought to; I
am not sure.  As it stands, if you have searched for @samp{foo} and you
add @samp{\|bar}, the search will not check for a @samp{bar} in the
buffer before the @samp{foo}.

@findex re-search-forward
@findex re-search-backward
  Nonincremental search for a regexp is done by the functions
@code{re-search-forward} and @code{re-search-backward}.  You can invoke
these with @kbd{M-x}, or bind them to keys.  Also, you can call
@code{re-search-forward} by way of incremental regexp search with
@kbd{C-M-s @key{ESC}}.

@node Regexps, Search Case, Regexp Search, Search
@section Syntax of Regular Expressions

Regular expressions have a syntax in which a few characters are special
constructs and the rest are @dfn{ordinary}.  An ordinary character is a
simple regular expression which matches that character and nothing else.
The special characters are @samp{$}, @samp{^}, @samp{.}, @samp{*},
@samp{+}, @samp{?}, @samp{[}, @samp{]} and @samp{\}.  Any other character
appearing in a regular expression is ordinary, unless a @samp{\} precedes
it.@refill

No new special characters will ever be defined.  All extensions to the
regular expression syntax are made by defining new two-character
constructs that begin with @samp{\}.

For example, @samp{f} is not a special character, so it is ordinary, and
therefore @samp{f} is a regular expression that matches the string @samp{f}
and no other string.  (It does @i{not} match the string @samp{ff}.)  Likewise,
@samp{o} is a regular expression that matches only @samp{o}.@refill

Any two regular expressions @var{a} and @var{b} can be concatenated.  The
result is a regular expression which matches a string if @var{a} matches
some amount of the beginning of that string and @var{b} matches the rest of
the string.@refill

As a simple example, we can concatenate the regular expressions @samp{f}
and @samp{o} to get the regular expression @samp{fo}, which matches only
the string @samp{fo}.  Still trivial.  To do something nontrivial, you
need to use one of the special characters.  Here is a list of them.

@table @kbd
@item .
is a special character that matches anything except a newline.  Using
concatenation, we can make regular expressions like @samp{a.b} which
matches any three-character string which begins with @samp{a} and ends
with @samp{b}.@refill

@item *
is not a construct by itself; it is a suffix, which means the
preceding regular expression is to be repeated as many times as
possible.  In @samp{fo*}, the @samp{*} applies to the @samp{o}, so
@samp{fo*} matches @samp{f} followed by any number of @samp{o}s.  The
case of zero @samp{o}s is allowed: @samp{fo*} does match @samp{f}.@refill

@samp{*} always applies to the @i{smallest} possible preceding expression.
Thus, @samp{fo*} has a repeating @samp{o}, not a repeating @samp{fo}.@refill

The matcher processes a @samp{*} construct by matching, immediately,
as many repetitions as can be found.  Then it continues with the rest
of the pattern.  If that fails, backtracking occurs, discarding some
of the matches of the @samp{*}-modified construct in case that makes
it possible to match the rest of the pattern.  For example, matching
@samp{c[ad]*ar} against the string @samp{caddaar}, the @samp{[ad]*}
first matches @samp{addaa}, but this does not allow the next @samp{a}
in the pattern to match.  So the last of the matches of @samp{[ad]} is
undone and the following @samp{a} is tried again.  Now it
succeeds.@refill

@item +
Is a suffix character similar to @samp{*} except that it requires that
the preceding expression be matched at least once.  So, for example,
@samp{c[ad]+r} will match the strings @samp{car} and @samp{caaadar}
but not the string @samp{cr}, whereas @samp{c[ad]*r} would match all
three strings.@refill

@item ?
Is a suffix character similar to @samp{*} except that it can match the
preceding expression either once or not at all.  For example,
@samp{c[ad]?r} will match @samp{car}, @samp{cdr} or @samp{cr}; nothing else.

@item [ ... ]
@samp{[} begins a @dfn{character set}, which is terminated by a
@samp{]}.  In the simplest case, the characters between the two form
the set.  Thus, @samp{[ad]} matches either @samp{a} or @samp{d}, and
@samp{[ad]*} matches any string of @samp{a} and @samp{d} (including
the empty string), from which it follows that @samp{c[ad]*r} matches
@samp{car}, etc.@refill

Character ranges can also be included in a character set, by writing
two characters with a @samp{-} between them.  Thus, @samp{[a-z]}
matches any lower-case letter.  Ranges may be intermixed freely with
individual characters, as in @samp{[a-z$%.]}, which matches any lower
case letter or @samp{$}, @samp{%} or period.@refill

Note that the usual special characters are not special any more inside
a character set.  A completely different set of special characters
exists inside character sets: @samp{]}, @samp{-} and @samp{^}.@refill

To include a @samp{]} in a character set, you must make it the first
character.  For example, @samp{[]a]} matches @samp{]} or @samp{a}.  To
include a @samp{-}, write @samp{---}, which is a range containing only
@samp{-}.  To include @samp{^}, make it other than the first character
in the set.@refill

@item [^ ... ]
@samp{[^} begins a @dfn{complement character set}, which matches any
character except the ones specified.  Thus, @samp{[^a-z0-9A-Z]}
matches all characters @i{except} letters and digits.@refill

@samp{^} is not special in a character set unless it is the first
character.  The character following the @samp{^} is treated as if it
were first (it may be a @samp{-} or a @samp{]}).

Note that a complement character set can match a newline, unless
newline is mentioned as one of the characters not to match.

@item ^
is a special character that matches the empty string, but only if at
the beginning of a line in the text being matched.  Otherwise it fails
to match anything.  Thus, @samp{^foo} matches a @samp{foo} which occurs
at the beginning of a line.

@item $
is similar to @samp{^} but matches only at the end of a line.  Thus,
@samp{xx*$} matches a string of one @samp{x} or more at the end of a line.

@item \
has two functions: it quotes the special characters (including
@samp{\}), and it introduces additional special constructs.

Because @samp{\} quotes special characters, @samp{\$} is a regular
expression which matches only @samp{$}, and @samp{\[} is a regular
expression which matches only @samp{[}, and so on.@refill
@end table

Note: for historical compatibility, special characters are treated as
ordinary ones if they are in contexts where their special meanings make no
sense.  For example, @samp{*foo} treats @samp{*} as ordinary since there is
no preceding expression on which the @samp{*} can act.  It is poor practice
to depend on this behavior; better to quote the special character anyway,
regardless of where is appears.@refill

For the most part, @samp{\} followed by any character matches only
that character.  However, there are several exceptions: characters
which, when preceded by @samp{\}, are special constructs.  Such
characters are always ordinary when encountered on their own.

@table @kbd
@item \|
specifies an alternative.
Two regular expressions @var{a} and @var{b} with @samp{\|} in
between form an expression that matches anything that either @var{a} or
@var{b} will match.@refill

Thus, @samp{foo\|bar} matches either @samp{foo} or @samp{bar}
but no other string.@refill

@samp{\|} applies to the largest possible surrounding expressions.  Only a
surrounding @samp{\( ... \)} grouping can limit the grouping power of
@samp{\|}.@refill

Full backtracking capability exists to handle multiple uses of @samp{\|}.

@item \( ... \)
is a grouping construct that serves three purposes:

@enumerate
@item
To enclose a set of @samp{\|} alternatives for other operations.
Thus, @samp{\(foo\|bar\)x} matches either @samp{foox} or @samp{barx}.

@item
To enclose a complicated expression for the postfix @samp{*} to operate on.
Thus, @samp{ba\(na\)*} matches @samp{bananana}, etc., with any (zero or
more) number of @samp{na} strings.@refill

@item
To mark a matched substring for future reference.

@end enumerate

This last application is not a consequence of the idea of a
parenthetical grouping; it is a separate feature which happens to be
assigned as a second meaning to the same @samp{\( ... \)} construct
because there is no conflict in practice between the two meanings.
Here is an explanation of this feature:

@item \@var{digit}
after the end of a @samp{\( ... \)} construct, the matcher remembers the
beginning and end of the text matched by that construct.  Then, later on
in the regular expression, you can use @samp{\} followed by @var{digit}
to mean ``match the same text matched the @var{digit}'th time by the
@samp{\( ... \)} construct.''@refill

The strings matching the first nine @samp{\( ... \)} constructs
appearing in a regular expression are assigned numbers 1 through 9 in
order of their beginnings.  @samp{\1} through @samp{\9} may be used to
refer to the text matched by the corresponding @samp{\( ... \)}
construct.

For example, @samp{\(.*\)\1} matches any newline-free string that is
composed of two identical halves.  The @samp{\(.*\)} matches the first
half, which may be anything, but the @samp{\1} that follows must match
the same exact text.

@item \`
matches the empty string, but only if it is at the beginning
of the buffer.

@item \'
matches the empty string, but only if it is at the end of
the buffer.

@item \b
matches the empty string, but only if it is at the beginning or
end of a word.  Thus, @samp{\bfoo\b} matches any occurrence of
@samp{foo} as a separate word.  @samp{\bballs?\b} matches
@samp{ball} or @samp{balls} as a separate word.@refill

@item \B
matches the empty string, provided it is @i{not} at the beginning or
end of a word.

@item \<
matches the empty string, provided it is at the beginning of a word.

@item \>
matches the empty string, provided it is at the end of a word.

@item \w
matches any word-constituent character.  The editor syntax table
determines which characters these are.

@item \W
matches any character that is not a word-constituent.

@item \s@var{code}
matches any character whose syntax is @var{code}.  @var{code} is a
character which represents a syntax code: thus, @samp{w} for word
constituent, @samp{-} for whitespace, @samp{(} for open-parenthesis,
etc.  @xref{Syntax}.@refill

@item \S@var{code}
matches any character whose syntax is not @var{code}.
@end table

@node Search Case, Replace, Regexps, Search
@section Searching and Case

@vindex case-fold-search
@vindex default-case-fold-search
  All sorts of searches in Emacs normally ignore the case of the text they
are searching through; if you specify searching for @samp{FOO}, then
@samp{Foo} and @samp{foo} are also considered a match.  Regexps, and in
particular character sets, are included: @samp{[aB]} would match @samp{a}
or @samp{A} or @samp{b} or @samp{B}.@refill

  If you do not want this feature, set the variable @code{case-fold-search}
to @code{nil}.  Then all letters must match exactly, including case.  This
variable has separate values in all individual buffers; in a new buffer,
its value is initialized from @code{default-case-fold-search}.
@xref{Variables}.

@node Replace, Other Repeating Search, Search Case, Search
@section Replacement Commands
@cindex replacement
@cindex string substitution
@cindex global substitution

  Global search-and-replace operations are not needed as often in Emacs as
they are in other editors, but they are available.  In addition to the
simple @code{replace-string} command which is like that found in most
editors, there is a @code{query-replace} command which asks you, for each
occurrence of the pattern, whether to replace it.

  The replace commands all replace one string (or regexp) with one
replacement string.  It is possible to perform several replacements in
parallel using the command @code{expand-region-abbrevs}.  @xref{Expanding
Abbrevs}.

@menu
* Unconditional Replace::  Everything about replacement except for querying.
* Query Replace::          How to use querying.
@end menu

@node Unconditional Replace, Query Replace, Replace, Replace
@subsection Unconditional Replacement
@findex replace-string
@findex replace-regexp

@table @kbd
@item M-x replace-string @key{RET} @var{string} @key{RET} @var{newstring} @key{RET}
Replace every occurrence of @var{string} with @var{newstring}.
@item M-x replace-regexp @key{RET} @var{regexp} @key{RET} @var{newstring} @key{RET}
Replace every match for @var{regexp} with @var{newstring}.
@end table

  To replace every instance of @samp{foo} after point with @samp{bar}, use
the command @kbd{M-x replace-string} with the two arguments @samp{foo} and
@samp{bar}.  Replacement occurs only after point, so if you want to cover
the whole buffer you must go to the beginning first.  All occurrences up to
the end of the buffer are replaced; to limit replacement to part of the
buffer, narrow to that part of the buffer before doing the replacement.

  When @code{replace-string} exits, point is left at the last occurrence
replaced.  The value of point when the @code{replace-string} command was
issued is remembered on the mark ring; @kbd{C-u C-@key{SPC}} moves back
there.

  @code{replace-string} replaces exact matches for a single string.  The
similar command @code{replace-regexp} replaces any match for a specified
pattern.

  In @code{replace-regexp}, the @var{newstring} need not be constant.  It
can refer to all or part of what is matched by the @var{regexp}.  @samp{\&}
in @var{newstring} is replaced by the entire text being replaced.
@samp{\@var{d}} in @var{newstring}, where @var{d} is a digit, is replaced
by whatever matched the @var{d}'th parenthesized grouping in @var{regexp}.
For example,@refill

@example
M-x replace-regexp @key{RET} c[ad]+r @key{RET} \&-safe @key{RET}
@end example

@noindent
would replace (for example) @samp{cadr} with @samp{cadr-safe} and @samp{cddr}
with @samp{cddr-safe}.

@example
M-x replace-regexp @key{RET} \(c[ad]+r\)-safe @key{RET} \1 @key{RET}
@end example

@noindent
would perform exactly the opposite replacements.  To include a @samp{\}
in the text to replace with, you must give @samp{\\}.

  A numeric argument to either of the @code{replace-} commands restricts
replacement to matches that are surrounded by word boundaries.

@vindex case-replace
@vindex case-fold-search
  If the arguments to @code{replace-string} are in lower case, it preserves
case when it makes a replacement.  Thus, the command

@example
M-x replace-string @key{RET} foo @key{RET} bar @key{RET}
@end example

@noindent
replaces a lower case @samp{foo} with a lower case @samp{bar}, @samp{FOO}
with @samp{BAR}, and @samp{Foo} with @samp{Bar}.  If upper case letters are
used in the second argument, they remain upper case every time that
argument is inserted.  If upper case letters are used in the first
argument, the second argument is always substituted exactly as given, with
no case conversion.  Likewise, if the variable @code{case-replace} is set
to @code{nil}, replacement is done without case conversion.  If
@code{case-fold-search} is set to @code{nil}, case is significant in
matching occurrences of @samp{foo} to replace; also, case conversion of the
replacement string is not done.

@node Query Replace,, Unconditional Replace, Replace
@subsection Query Replace
@cindex Query Replace

@table @kbd
@item M-% @key{RET} @var{string} @key{RET} @var{newstring} @key{RET}
@itemx M-x query-replace @key{RET} @var{string} @key{RET} @var{newstring} @key{RET}
Replace some occurrences of @var{string} with @var{newstring}.
@item M-x query-replace-regexp @key{RET} @var{regexp} @key{RET} @var{newstring} @key{RET}
Replace some matches for @var{string} with @var{newstring}.
@end table

@kindex M-%
@findex query-replace
  If you want to change only some of the occurrences of @samp{foo} to
@samp{bar}, not all of them, then you cannot use an ordinary
@code{replace-string}.  Instead, use @kbd{M-%} (@code{query-replace}).
This command finds occurrences of @samp{foo} one by one, displays each
occurrence and asks you whether to replace it.  A numeric argument to
@code{query-replace} tells it to consider only occurrences that are bounded
by word-delimiter characters.@refill

  Aside from querying, @code{query-replace} works just like
@code{replace-string}, and @code{query-replace-regexp} works
just like @code{replace-regexp}.@refill

  The things you can type when you are shown an occurrence of @var{string}
or a match for @var{regexp} are:

@kindex SPC (query-replace)
@kindex DEL (query-replace)
@kindex Comma (query-replace)
@kindex ESC (query-replace)
@kindex . (query-replace)
@kindex ! (query-replace)
@kindex ^ (query-replace)
@kindex C-r (query-replace)
@kindex C-w (query-replace)
@kindex C-l (query-replace)

@c WideCommands
@table @kbd
@item @key{SPC}
to replace the occurrence with @var{newstring}.  This preserves case, just
like @code{replace-string}, provided @code{case-replace} is non-@code{nil},
as it normally is.@refill

@item @key{DEL}
to skip to the next occurrence without replacing this one.

@item ,
to replace this occurrence and display the result.  You are then asked
for another input character, except that since the replacement has
already been made, @key{DEL} and @key{SPC} are equivalent.  You could
type @kbd{C-r} at this point (see below) to alter the replaced text.  You
could also type @kbd{C-x u} to undo the replacement; this exits the
@code{query-replace}, so if you want to do further replacement you must use
@kbd{C-x ESC} to restart (@pxref{Repetition}).

@item @key{ESC}
to exit without doing any more replacements.

@item .
to replace this occurrence and then exit.

@item !
to replace all remaining occurrences without asking again.

@item ^
to go back to the location of the previous occurrence (or what used to
be an occurrence), in case changed it by mistake.  This works by
popping the mark ring.  Only one @kbd{^} is allowed, because only one
previous replacement location is kept during @code{query-replace}.

@item C-r
to enter a recursive editing level, in case the occurrence needs to be
edited rather than just replaced with @var{newstring}.  When you are
done, exit the recursive editing level with @kbd{C-M-c} and the next
occurrence will be displayed.  @xref{Recursive Edit}.

@item C-w
to delete the occurrence, and then enter a recursive editing level as
in @kbd{C-r}.  Use the recursive edit to insert text to replace the
deleted occurrence of @var{string}.  When done, exit the recursive
editing level with @kbd{C-M-c} and the next occurrence will be
displayed.

@item C-l
to redisplay the screen and then give another answer.

@item C-h
to display a message summarizing these options, then give another
answer.
@end table

  If you type any other character, the @code{query-replace} is exited, and
the character executed as a command.  To restart the @code{query-replace},
use @kbd{C-x @key{ESC}}, which repeats the @code{query-replace} because it
used the minibuffer to read its arguments.  @xref{Repetition, C-x ESC}.
  
@node Other Repeating Search,, Replace, Search
@section Other Search-and-Loop Commands

  Here are some other commands that find matches for a regular expression.
They all operate from point to the end of the buffer.

@findex list-matching-lines
@findex count-matches
@findex delete-non-matching-lines
@findex delete-matching-lines
@c grosscommands
@table @kbd
@item M-x list-matching-lines
Print each line that follows point and contains a match for the
specified regexp.  A numeric argument specifies the number of context
lines to print before and after each matching line; the default is
none.

@item M-x count-matches
Print the number of matches following point for the specified regexp.

@item M-x delete-non-matching-lines
Delete each line that follows point and does not contain a match for
the specified regexp.

@item M-x delete-matching-lines
Delete each line that follows point and contains a match for the
specified regexp.
@end table

@node Fixit, Files, Search, Top
@chapter Commands for Fixing Typos
@cindex typos

  In this chapter we describe the commands that are especially useful for
the times when you catch a mistake in your text just after you have made
it, or change your mind while composing text on line.

@menu
* Kill Errors:: Commands to kill a batch of recently entered text.
* Transpose::   Exchanging two characters, words, lines, lists...
* Fixing Case:: Correcting case of last word entered.
* Spelling::    Apply spelling checker to a word, or a whole file.
@end menu

@node Kill Errors, Transpose, Fixit, Fixit
@section Killing Your Mistakes

@table @kbd
@item @key{DEL}
Delete last character (@code{delete-backward-char}).
@item M-@key{DEL}
Kill last word (@code{backward-kill-word}).
@item C-x @key{DEL}
Kill to beginning of sentence (@code{backward-kill-sentence}).
@end table

@kindex DEL
@findex delete-backward-char
  The @key{DEL} character (@code{delete-backward-char}) is the most
important correction command.  When used among graphic (self-inserting)
characters, it can be thought of as canceling the last character typed.

@kindex M-DEL
@kindex C-x DEL
@findex backward-kill-word
@findex backward-kill-sentence
  When your mistake is longer than a couple of characters, it might be more
convenient to use @kbd{M-@key{DEL}} or @kbd{C-x @key{DEL}}.
@kbd{M-@key{DEL}} kills back to the start of the last word, and @kbd{C-x
@key{DEL}} kills back to the start of the last sentence.  @kbd{C-x
@key{DEL}} is particularly useful when you are thinking of what to write as
you type it, in case you change your mind about phrasing.
@kbd{M-@key{DEL}} and @kbd{C-x @key{DEL}} save the killed text for
@kbd{C-y} and @kbd{M-y} to retrieve.  @xref{Yanking}.@refill

  @kbd{M-@key{DEL}} is often useful even when you have typed only a few
characters wrong, if you know you are confused in your typing and aren't
sure exactly what you typed.  At such a time, you cannot correct with
@key{DEL} except by looking at the screen to see what you did.  It requires
less thought to kill the whole word and start over again.

@node Transpose, Fixing Case, Kill Errors, Fixit
@section Transposing Text

@table @kbd
@item C-t
Transpose two characters (@code{transpose-chars}).
@item M-t
Transpose two words (@code{transpose-words}).
@item C-M-t
Transpose two balanced expressions (@code{transpose-sexps}).
@item C-x C-t
Transpose two lines (@code{transpose-lines}).
@end table

@cindex transposition
@kindex C-t
@findex transpose-chars
  The common error of transposing two characters can be fixed, when they
are adjacent, with the @kbd{C-t} command (@code{transpose-chars}).  Normally,
@kbd{C-t} transposes the two characters on either side of point.  When
given at the end of a line, rather than transposing the last character of
the line with the newline, which would be useless, @kbd{C-t} transposes the
last two characters on the line.  So, if you catch your transposition error
right away, you can fix it with just a @kbd{C-t}.  If you don't catch it so
fast, you must move the cursor back to between the two transposed
characters.  If you transposed a space with the last character of the word
before it, the word motion commands are a good way of getting there.
Otherwise, a reverse search (@kbd{C-r}) is often the best way.
@xref{Search}.


@kindex C-x C-t
@findex transpose-lines
@kindex M-t
@findex transpose-words
@kindex C-M-t
@findex transpose-sexps
  @kbd{Meta-t} (@code{transpose-words}) transposes the word before point
with the word after point.  It moves point forward over a word, dragging
the word preceding or containing point forward as well.  The punctuation
characters between the words do not move.  For example, @w{@samp{FOO, BAR}}
transposes into @w{@samp{BAR, FOO}} rather than @samp{@w{BAR FOO,}}.

  @kbd{C-M-t} (@code{transpose-sexps}) is a similar command for transposing
two expressions (@pxref{Lists}), and @kbd {C-x C-t} (@code{transpose-lines})
exchanges lines.  They work like @kbd{M-t} except in determining the
division of the text into syntactic units.

  A numeric argument to a transpose command serves as a repeat count: it
tells the transpose command to move the character (word, sexp, line) before
or containing point across several other characters (words, sexps, lines).
For example, @kbd{C-u 3 C-t} moves the character before point forward
across three other characters.  This is equivalent to repeating @kbd{C-t}
three times.  @kbd{C-u - 4 M-t} moves the word before point backward across
four words.  @kbd{C-u - C-M-t} would cancel the effect of plain
@kbd{C-M-t}.@refill

  A numeric argument of zero is assigned a special meaning (because
otherwise a command with a repeat count of zero would do nothing): to
transpose the character (word, sexp, line) ending after point with the
one ending after the mark.

@node Fixing Case, Spelling, Transpose, Fixit
@section Case Conversion

@table @kbd
@item M-- M-l
Convert last word to lower case.  @kbd{Meta--} is Meta-minus!
@item M-- M-u
Convert last word to all upper case.
@item M-- M-c
Convert last word to lower case with capital initial.
@end table

@findex downcase-word
@findex upcase-word
@findex capitalize-word
@kindex M-- M-l
@kindex M-- M-u
@kindex M-- M-c
@cindex case conversion
@cindex words
  A very common error is to type words in the wrong case.  Because of this,
the word case-conversion commands @kbd{M-l}, @kbd{M-u} and @kbd{M-c} have a
special feature when used with a negative argument: they do not move the
cursor.  As soon as you see you have mistyped the last word, you can simply
case-convert it and go on typing.  @xref{Case}.@refill

@node Spelling,, Fixing Case, Fixit
@section Checking and Correcting Spelling
@cindex spelling

@c doublewidecommands
@table @kbd
@item M-$
Check and correct spelling of word (@code{spell-word}).
@item M-x spell-buffer
Check and correct spelling of each word in the buffer.
@item M-x spell-region
Check and correct spelling of each word in the region.
@item M-x spell-string
Check spelling of specified word.
@end table

@kindex M-$
@findex spell-word
  To check the spelling of the word before point, and optionally correct it
as well, use the command @kbd{M-$} (@code{spell-word}).  This command runs
an inferior process containing the @code{spell} program to see whether the
word is correct English.  If it is not, it asks you to edit the word (in
the minibuffer) into a corrected spelling, and then does a @code{query-replace}
to substitute the corrected spelling for the old one throughout the buffer.

  If you exit the minibuffer without altering the original spelling, it
means you do not want to do anything to that word.  Then the @code{query-replace}
is not done.

@findex spell-buffer
  @kbd{M-x spell-buffer} checks each word in the buffer the same way that
@code{spell-word} does, doing a @code{query-replace} if appropriate for
every incorrect word.@refill

@findex spell-region
  @kbd{M-x spell-region} is similar but operates only on the region, not
the entire buffer.

@findex spell-string
  @kbd{M-x spell-string} reads a string as an argument and checks whether
that is a correctly spelled English word.  It prints in the echo area a
message giving the answer.

@node Files, Buffers, Fixit, Top
@chapter File Handling
@cindex files

  The basic unit of stored data in Unix is the @dfn{file}.  To edit a file,
you must tell Emacs to examine the file and prepare a buffer containing a
copy of the file's text.  This is called @dfn{visiting} the file.  Editing
commands apply directly to text in the buffer; that is, to the copy inside
Emacs.  Your changes appear in the file itself only when you @dfn{save} the
buffer back into the file.

  In addition to visiting and saving files, Emacs can delete, copy, rename,
and append to files, and operate on file directories.

@menu
* File Names::   How to type and edit file name arguments.
* Visiting::     Visiting a file prepares Emacs to edit the file.
* Saving::       Saving makes your changes permanent.
* Reverting::    Reverting cancels all the changes not saved.
* Auto Save::    Auto Save periodically protects against loss of data.
* ListDir::      Listing the contents of a file directory.
* Dired::        ``Editing'' a directory to delete, rename, etc.
                  the files in it.
* Misc File Ops:: Other things you can do on files.
@end menu

@node File Names,, Files, Files
@section File Names
@cindex file names

  Most Emacs commands that operate on a file require you to specify the
file name.  (Saving and reverting are exceptions; the buffer knows which
file name to use for them.)  File names are specified using the minibuffer
(@pxref{Minibuffer}).  @dfn{Completion} is available, to make it easier to
specify long file names.  @xref{Completion}.

  There is always a @dfn{default file name} which will be used if you type
just @key{RET}, entering an empty argument.  Normally the default file name
is the name of the file visited in the current buffer; this makes it easy
to operate on that file with any of the Emacs file commands.

@vindex default-directory
  Each buffer has a default directory, normally the same as the directory
of the file visited in that buffer.  When Emacs reads a file name, if you
do not specify a directory, the default directory is used.  If you specify
a directory in a relative fashion, with a name that does not start with a
slash, it is interpreted with respect to the default directory.  The
default directory is kept in the variable @code{default-directory}, which
has a separate value in every buffer.

  For example, if the default file name is @file{/u/rms/gnu/gnu.tasks} then
the default directory is @file{/u/rms/gnu/}.  If you type just @samp{foo},
which does not specify a directory, it is short for @file{/u/rms/gnu/foo}.
@samp{../.login} would stand for @file{/u/rms/.login}.  @samp{new/foo}
would stand for the filename @file{/u/rms/gnu/new/foo}.

  The default directory actually appears initially in the minibuffer when
the file name is read.  This serves two purposes: it shows you what the
default is, so that you can type a relative file name and know with
certainty what it will mean, and it allows you to edit the default to
specify a different directory.

  Note that it is legitimate to type an absolute file name after you enter
the minibuffer, ignoring the presence of the default directory name as part
of the text.  The final minibuffer contents may look invalid, but that is
not so.  @xref{Minibuffer File}.

  The command @kbd{M-x pwd} prints the current buffer's default directory,
and the command @kbd{M-x cd} sets it (to a value read using the
minibuffer).  A buffer's default directory changes only when the @code{cd}
command is used.  A file-visiting buffer's default directory is initialized
to the directory of the file that is visited there.  If a buffer is made
randomly with @kbd{C-x b}, its default directory is copied from that of the
buffer that was current at the time.

@node Visiting, Saving, File Names, Files
@section Visiting Files
@cindex visiting files

@c WideCommands
@table @kbd
@item C-x C-f
Visit a file (@code{find-file}).
@item C-x C-v
Visit a different file instead of the one visited last
(@code{find-alternate-file}).
@item C-x 4 C-f
Visit a file, in another window (@code{find-file-other-window}).  Don't
change this window.
@end table

@cindex files
@cindex visiting
@cindex saving
@vindex ask-about-buffer-names
  @dfn{Visiting} a file means copying its contents into Emacs where you can
edit them.  Emacs makes a new buffer for each file that you visit.  We say
that the buffer is visiting the file that it was created to hold.  Emacs
constructs the buffer name from the file name by throwing away the
directory, keeping just the name proper.  For example, a file named
@file{/usr/rms/emacs.tex} would get a buffer named @samp{emacs.tex}.  If
there is already a buffer with that name, a unique name is constructed by
appending @samp{<2>}, @samp{<3>}, or so on, using the lowest number that
makes a name that is not already in use.  If the variable
@code{ask-about-buffer-names} is non-@code{nil}, the user is asked what
buffer name to use; this takes the place of automatic uniquization.

  Each window's mode line shows the name of the buffer that is being displayed
in that window, so you can always tell what buffer you are editing.

  The changes you make with Emacs are made in the Emacs buffer.  They do
not take effect in the file that you visited, or any place permanent, until
you @dfn{save} the buffer.  Saving the buffer means that Emacs writes the
current contents of the buffer into its visited file.  @xref{Saving}.

@cindex modified (buffer)
  If a buffer contains changes that have not been saved, the buffer is said
to be @dfn{modified}.  This is important because it implies that some
changes will be lost if the buffer is not saved.  The mode line displays
two stars near the left margin if the buffer is modified.

@kindex C-x C-f
@findex find-file
  To visit a file, use the command @kbd{C-x C-f} (@code{find-file}).  Follow
the command with the name of the file you wish to visit, terminated by a
@key{RET}.

  The file name is read using the minibuffer (@pxref{Minibuffer}), with
defaulting and completion in the standard manner (@pxref{File Names}).
While in the minibuffer, you can abort @kbd{C-x C-f} by typing @kbd{C-g}.

  Your confirmation that @kbd{C-x C-f} has completed successfully is the
appearance of new text on the screen and a new buffer name in the mode
line.  If the specified file does not exist and could not be created, or
cannot be read, then an error results.  The error message is printed in the
echo area, and includes the file name which Emacs was trying to visit.

  If you visit a file that is already in Emacs, @kbd{C-x C-f} does not make
another copy.  It selects the existing buffer containing that file.
However, before doing so, it checks that the file itself has not changed
since you visited or saved it last.  If the file has changed, a warning
message is printed.  @xref{Interlocking,,Simultaneous Editing}.

@cindex creating files
  What if you want to create a file?  Just visit it.  Emacs prints
@samp{(New File)} in the echo area, but in other respects behaves as if you
had visited an existing empty file.  If you make any changes and save them,
the file is created.

@kindex C-x C-v
@findex find-alternate-file
  If you visit a nonexistent file unintentionally (because you typed the
wrong file name), use the @kbd{C-x C-v} (@code{find-alternate-file})
command to visit the file you wanted.  @kbd{C-x C-v} is similar to @kbd{C-x
C-f}, but it kills the current buffer (after first offering to save it if
it is modified).@refill

@vindex find-file-run-dired
  If the file you specify is actually a directory, Dired is called on that
directory (@pxref{Dired}).  This can be inhibited by setting the variable
@code{find-file-run-dired} to @code{nil}; then it is an error to try to
visit a directory.

@kindex C-x 4 f
@findex find-file-other-window
  @kbd{C-x 4 f} (@code{find-file-other-window}) is like @kbd{C-x C-f}
except that the buffer containing the specified file is selected in another
window.  The window that was selected before @kbd{C-x 4 f} continues to
show the same buffer it was already showing.  If this command is used when
only one window is being displayed, that window is split in two, with one
window showing the same before as before, and the other one showing the
newly requested buffer.

@node Saving, Reverting, Visiting, Files
@section Saving Files

  @dfn{Saving} a buffer in Emacs means writing its contents back into the file
that was visited in the buffer.

@table @kbd
@item C-x C-s
Save the current buffer in its visited file (@code{save-buffer}).
@item C-x s
Save any or all buffers in their visited files (@code{save-some-buffers}).
@item M-~
Forget that the current buffer has been changed (@code{not-modified}).
@item C-x C-w
Save the current buffer in a specified file, and record that file as
the one visited in the buffer (@code{write-file}).
@item M-x set-visited-file-name
Mark the current buffer as visiting a specified file.
@end table

@kindex C-x C-s
@findex save-buffer
  When you wish to save the file and make your changes permanent, type
@kbd{C-x C-s} (@code{save-buffer}).  After saving is finished, @kbd{C-x C-s}
prints a message such as

@example
Wrote /u/rms/gnu/gnu.tasks
@end example

@noindent
If the selected buffer is not modified (no changes have been made in it
since the buffer was created or last saved), saving is not really done,
because it would be redundant.  Instead, @kbd{C-x C-s} prints a message in
the echo area saying

@example
(No changes need to be written)
@end example

@kindex C-x s
@findex save-some-buffers
  The command @kbd{C-x s} (@code{save-some-buffers}) can save any or all modified
buffers.  First it asks, for each modified buffer, whether to save it.
These questions appear as typeout, overlying the buffer text, and should
be answered with @kbd{y} or @kbd{n}.  After all questions have been asked,
the buffers you have approved are all saved.

@kindex M-~
@findex not-modified
  If you have changed a buffer and do not want the changes to be saved, you
should take some action to prevent it.  Otherwise, each time you use
@code{save-some-buffers} you are liable to save it by mistake.  One thing
you can do is type @kbd{M-~} (@code{not-modified}), which clears out the
indication that the buffer is modified.  If you do this, none of the save
commands will believe that the buffer needs to be saved.  (If we take
@samp{~} to mean `not', then @kbd{Meta-~} is `not', metafied.)  You could
also use @code{set-visited-file-name} (see below) to mark the buffer as
visiting a different file name, one which is not in use for anything
important.  Alternatively, you can undo all the changes made since the file
was visited or saved, by reading the text from the file again.  This is
called @dfn{reverting}.  @xref{Reverting}.  You could also undo all the
changes by repeating the undo command @kbd{C-x u} until you have undone all
the changes; but this only works if you have not made more changes than the
undo mechanism can remember.

@findex set-visited-file-name
  @kbd{M-x set-visited-file-name} alters the name of the file that the
current buffer is visiting.  It reads the new file name using the
minibuffer.  It can be used on a buffer that is not visiting a file, too.
The buffer's name is changed to correspond to the file it is now visiting
in the usual fashion (unless the new name is in use already for some other
buffer; in that case, the buffer name is not changed).
@code{set-visited-file-name} does not save the buffer in the newly visited
file; it just alters the records inside Emacs so that, if you save the
buffer, it will be saved in that file.  It also marks the buffer as
``modified'' so that @kbd{C-x C-s} @i{will} save.

@kindex C-x C-w
@findex write-file
  If you wish to mark the buffer as visiting different file and save it
right away, use @kbd{C-x C-w} (@code{write-file}).  It is precisely
equivalent to @code{set-visited-file-name} followed by @kbd{C-x C-s}.
@kbd{C-x C-s} used on a buffer that is not visiting with a file has the
same effect as @kbd{C-x C-w}; that is, it reads a file name, marks the
buffer as visiting that file, and saves it there.  The default file name in
a buffer that is not visiting a file is made by combining the buffer name
with the buffer's default directory.

  If Emacs is about to save a file and sees that the date of the latest
version on disk does not match what Emacs last read or wrote, Emacs
notifies you of this fact, because it probably indicates a problem caused
by simultaneous editing and requires your immediate attention.
@xref{Interlocking,, Simultaneous Editing}.

@vindex require-final-newline
  If the variable @code{require-final-newline} is non-@code{nil}, Emacs
puts a newline at the end of any file that doesn't already end in one,
every time a file is saved or written.

@menu
* Backup::       How Emacs saves the old version of your file.
* Interlocking:: How Emacs protects against simultaneous editing
                  of one file by two users.
@end menu

@node Backup, Interlocking, Saving Saving
@subsection Backup Files
@cindex backup file

  Because Unix does not provide version numbers in file names, rewriting a
file in Unix automatically destroys all record of what the file used to
contain.  Thus, saving a file from Emacs throws away the old contents of
the file---or it would, except that Emacs carefully copies the old contents
to another file, called the @dfn{backup} file, before actually saving.
At your option, Emacs can keep either a single backup file or a series of
numbered backup files for each file that you edit.

  Emacs makes a backup for a file only the first time the file is saved
from one buffer.  No matter how many times you save a file, its backup file
continues to contain the contents from before the file was visited.
Normally this means that the backup file contains the contents from before
the current editing session; however, if you kill the buffer and then visit
the file again, a new backup file will be made by the next save.

  If you choose to have a single backup file (this is the default),
the backup file's name is constructed by appending @samp{~} to the
file name being edited; thus, the backup file for @file{eval.c} would
be @file{eval.c~}.

  If you choose to have a series of numbered backup files, backup file
names are made by appending @samp{.~}, the number, and another @samp{~} to
the original file name.  Thus, the backup files of @file{eval.c} would be
called @file{eval.c.~1~}, @file{eval.c.~2~}, and so on, through names
like @file{eval.c.~259~} and beyond.

@vindex version-control
  The choice of single backup or numbered backups is controlled by the
variable @code{version-control}.  Its possible values are

@table @code
@item t
Make numbered backups.
@item nil
Make numbered backups for files that have numbered backups already.
@item never
Do not in any case make numbered backups.
@end table

@noindent
@code{version-control} may be set locally in an individual buffer to
control the making of backups for that buffer's file.  For example,
Rmail mode locally sets @code{version-control} to @code{never} to make sure
that there is only one backup for an Rmail file.  @xref{Locals}.

@vindex make-backup-files
  If the variable @code{make-backup-files} is set to @code{nil}, backup
files are not written at all.

@subsubsection Automatic Deletion of Backups

@vindex kept-old-versions
@vindex kept-new-versions
  To prevent unlimited consumption of disk space, Emacs can delete numbered
backup versions automatically.  Generally Emacs keeps the first few backups
and the latest few backups, deleting any in between.  This happens every
time a new backup is made.  The two variables that control the deletion are
@code{kept-old-versions} and @code{kept-new-versions}.  Their values are, respectively
the number of oldest (lowest-numbered) backups to keep and the number of
newest (highest-numbered) ones to keep, each time a new backup is made.
Recall that these values are used just after a new backup version is made;
that newly made backup is included in the count in @code{kept-new-versions}.
By default, both variables are 2.

@vindex trim-versions-without-asking
  If @code{trim-versions-without-asking} is non-@code{nil}, the excess
middle versions are deleted without a murmur.  If it is @code{nil}, the
default, then you are asked whether the excess middle versions should
really be deleted.

  Dired's @kbd{.} command can also be used to delete old versions;
@xref{Dired}.

@subsubsection Copying vs.@: Renaming

  Backup files can be made by copying the old file or by renaming it.  This
makes a difference when the old file has multiple names.  If the old file
is renamed into the backup file, then the alternate names become names for
the backup file.  If the old file is copied instead, then the alternate
names remain names for the file that you are editing, and the contents
accessed by those names will be the new contents.

@vindex backup-by-copying
@vindex backup-by-copying-when-linked
  The choice of renaming or copying is controlled by two variables.
Normally, renaming is done.  If the variable @code{backup-by-copying} is
non-@code{nil}, copying is used.  If the variable @code{backup-by-copying-when-linked}
is non-@code{nil}, then copying is done for files that have multiple names,
but renaming is done when the file being edited has only one name.  (For
files with only one name, the major difference between renaming and copying
is that renaming is faster.)

@node Interlocking,,Backup,Saving
@subsection Protection against Simultaneous Editing

@cindex file dates
@cindex simultaneous editing
  Simultaneous editing occurs when two users visit the same file, both make
changes, and then both save them.  If nobody were informed that this was
happening, whichever user saved first would later find that his changes
were lost.  On some systems, Emacs notices immediately when the second user
starts to change the file, and issues an immediate warning.  When this is
not possible, or if the second user has gone on to change the file despite
the warning, Emacs checks later when the file is saved, and issues a second
warning when a user is about to overwrite a file containing another user's
changes.  If the editing user takes the proper corrective action at this
point, he can prevent actual loss of work.

@findex ask-user-about-lock
  When you make the first modification in an Emacs buffer that is visiting
a file, Emacs records that you have locked the file.  (It does this by
writing another file in a directory reserved for this purpose).  The lock
is removed when you save the changes.  The idea is that the file is locked
whenever the buffer is modified.  If you begin to modify the buffer while
the visited file is locked by someone else, this constitutes a collision,
and Emacs asks you what to do.  It does this by calling the Lisp function
@code{ask-user-about-lock}, which you can redefine for the sake of
customization.  The standard definition of this function asks you a
question and accepts three possible answers:

@table @kbd
@item s
Steal the lock.  Whoever was already changing the file loses the lock,
and you gain the lock.
@item p
Proceed.  Go ahead and edit the file despite its being locked by someone else.
@item q
Quit.  This causes an error (@code{file-locked}) and the modification you
were trying to make in the buffer does not actually take place.
@end table

  Note that locking works on the basis of a file name; if a file has
multiple names, Emacs does not realize that the two names are the same file
and cannot prevent two user from editing it simultaneously under different
names.  However, basing locking on names means that Emacs can interlock the
editing of new files that will not really exist until they are saved.

  Some systems are not configured to allow Emacs to make locks.  On
these systems, Emacs cannot detect trouble in advance, but it still can
detect it in time to prevent you from overwriting someone else's changes.

  Every time Emacs saves a buffer, it first checks the last-modification
date of the existing file on disk to see that it has not changed since the
file was last visited or saved.  If the date does not match, it implies
that changes were made in the file in some other way, and these changes are
about to be lost if Emacs actually does save.  To prevent this, Emacs
prints a warning message and asks for confirmation before saving.
Occasionally you will know why the file was changed and know that it does
not matter; then you can answer `yes' and proceed.  Otherwise, you should
cancel the save with @kbd{C-g} and investigate the situation.

  The first thing you should do when notified that simultaneous editing has
already taken place is to list the directory with @kbd{C-u C-x C-d}
(@pxref{ListDir,,Directory Listing}).  This will show the file's current
author.  You should attempt to contact him to warn him not to continue
editing.  Often the next step is to save the contents of your Emacs buffer
under a different name, and use @code{diff} to compare the two
files.@refill

  Simultaneous editing checks are also made when you visit with @kbd{C-x
C-f} a file that is already visited.  This is not strictly necessary, but
it can cause you to find out about the problem earlier, when perhaps
correction takes less work.

@node Reverting, Auto Save, Saving, Files
@section Reverting a Buffer
@findex revert-buffer
@cindex drastic changes

  If you have made extensive changes to a file and then change your mind
about them, you can get rid of them by reading in the previous version of
the file.  To do this, use @kbd{M-x revert-buffer}, which operates on the
current buffer.  Since this is a very dangerous thing to do, you must
confirm it with `yes'.

  If the current buffer has been auto-saved more recently than it has been
saved for real, @code{revert-buffer} offers to read the auto save file
instead of the visited file.  This question comes before the usual request
for confirmation, and demands @kbd{y} or @kbd{n} as an answer.  If you have
started to type @kbd{yes} for confirmation without realizing that the other
question was going to be asked, the @kbd{y} will answer that question, but
the @kbd{es} will not be valid confirmation.  So you will have a chance to
cancel the operation with @kbd{C-g} and try it again with the answers that
you really intend.

  @code{revert-buffer} keeps point at the same distance (measured in
characters) from the beginning of the file.  If the file was edited only
slightly, you will be at approximately the same piece of text after
reverting as before.  If you have made drastic changes, the same value of
point in the old file may address a totally different piece of text.

  A buffer reverted from its visited file is marked ``not modified'' until
another change is made.

  Some kinds of buffers whose contents reflect data bases other than files,
such as Dired buffers, can also be reverted.  For them, reverting means
recalculating their contents from the appropriate data base.  Buffers
created randomly with @kbd{C-x b} cannot be reverted; @code{revert-buffer}
reports an error when asked to do so.

@node Auto Save, ListDir, Reverting, Files
@section Auto Saving: Protection Against Disasters
@cindex Auto Save mode
@cindex crashes

  Emacs saves all the visited files from time to time (based on counting
your keystrokes) without being asked.  This is called @dfn{auto-saving}.
It prevents you from losing more than a limited amount of work if the
system crashes.

@vindex auto-save-visited-file-name
  Auto-saving does not normally save in the files that you visited, because
it can be very undesirable to save a program that is in an inconsistent
state because you have made half of a planned change.  Instead, auto-saving
is done in a different file called the @dfn{auto-save file}, and the
visited file is changed only when you request saving explicitly (such as
with @kbd{C-x C-s}).  If you want auto-saving to be done in the visited
file, set the variable @code{auto-save-visited-file-name} to be non-@code{nil}.
The file name to be used for auto-saving in a buffer is calculated when
auto-saving is turned on in that buffer, based on the variable values in
effect at that time.

  Normally, the auto-save file name is made by appending @samp{#} to the
front of the visited file name.  Thus, a buffer visiting file @file{foo.c}
would be auto-saved in a file @file{#foo.c}.  Most buffers that are not
visiting files are auto-saved only if you request it explicitly; when they
are auto-saved, the auto-save file name is made by appending @samp{#%} to
the buffer name.  For example, the @samp{*mail*} buffer in which you
compose messages to be sent is auto-saved in a file named @file{#%*mail*}.
Auto-save file names are made this way unless you reprogram parts of Emacs
to do something different.

@vindex auto-save-default
@findex auto-save-mode
  Each time you visit a file, auto saving is turned on for that file's
buffer if the variable @code{auto-save-default} is non-@code{nil} (but not
in batch mode; @pxref{Entering Emacs}).  The default for this variable is
@code{t}, so auto-saving is the usual practice for file-visiting buffers.
Auto-saving can be turned on or off for any existing buffer with the
command @kbd{M-x auto-save-mode}.  Like other minor mode commands, @kbd{M-x
auto-save-mode} turns auto-saving on with a positive argument, off with a
zero or negative argument; with no argument, it toggles.

@vindex auto-save-interval
@findex do-auto-save
  Emacs does auto-saving every so often, based on counting how many
characters you have typed since the last time auto-saving was done.  The
variable @code{auto-save-interval} specifies how many characters there are
between auto-saves.  By default, it is 300.  Emacs also auto-saves whenever
you call the function @code{do-auto-save}.

  Emacs also does auto-saving whenever it gets a fatal error.  This
includes killing the Emacs job with a shell command such as @code{kill
%emacs}, or disconnecting a phone line or network connection.

  When Emacs determines that it is time for auto-saving, each buffer is
considered, and is auto-saved if auto-saving is turned on for it and it has
been changed since the last time it was auto-saved.  If any auto-saving is
done, the message @samp{Auto-saving...} is displayed in the echo area until
auto-saving is finished.  Errors occurring during auto-saving are trapped
so that they do not interfere with the execution of commands you have been
typing.

@vindex delete-auto-save-files
  A buffer's auto-save file is deleted when you save the buffer in its
visited file.  To inhibit this, set the variable @code{delete-auto-save-files}
to @code{nil}.

@findex recover-file
  The way to use the contents of an auto save file to recover from a loss
of data is with the command @kbd{M-x recover-file @key{RET} @var{file}
@key{RET}}.  This visits @var{file} and then (after your confirmation) it
from its auto-save file @file{#@var{file}}.  You can then save with
@kbd{C-x C-s} to put the recovered text into @var{file} itself.  For
example, to recover file @file{foo.c} from its auto-save file
@file{#foo.c}, do:@refill

@example
M-x recover-file @key{RET} foo.c @key{RET}
C-x C-s
@end example

@node ListDir, Dired, Auto Save, Files
@section Listing a File Directory

@cindex file directory
@cindex directory listing
  Files are classified by Unix into @dfn{directories}.  A @dfn{directory
listing} is a list of all the files in a directory.  Emacs provides
directory listings in brief format (file names only) and verbose format
(sizes, dates, and authors included).

@table @kbd
@item C-x C-d @var{dir-or-pattern}
Print a brief directory listing (@code{list-directory}).
@item C-u C-x C-d @var{dir-or-pattern}
Print a verbose directory listing.
@end table

@findex list-directory
@kindex C-x C-d
  The command to print a directory listing is @kbd{C-x C-d} (@code{list-directory}).
It reads using the minibuffer a file name which is either a directory to be
listed or a wildcard-containing pattern for the files to be listed.  For
example,

@example
C-x C-d /u2/emacs/etc @key{RET}
@end example

@noindent
lists all the files in directory @file{/u2/emacs/etc}.  An example of
specifying a file name pattern is

@example
C-x C-d /u2/emacs/src/*.c @key{RET}
@end example

  Normally, @kbd{C-x C-d} prints a brief directory listing containing just
file names.  A numeric argument (regardless of value) tells it to print a
verbose listing (like @code{ls -l}).

@vindex list-directory-brief-switches
@vindex list-directory-verbose-switches
  The text of a directory listing is obtained by running @code{ls} in an
inferior process.  Two Emacs variables control the switches passed to
@code{ls}: @code{list-directory-brief-switches} is a string giving the
switches to use in brief listings (@code{"-CF"} by default), and
@code{list-directory-verbose-switches} is a string giving the switches to
use in a verbose listing (@code{"-l"} by default).

@node Dired, Misc File Ops, ListDir, Files
@section Dired, the Directory Editor
@cindex Dired
@cindex deletion (of files)

  Dired makes it easy to delete or visit many of the files in a single
directory at once.  It makes an Emacs buffer containing a listing of the
directory.  You can use the normal Emacs commands to move around in this
buffer, and special Dired commands to operate on the files.

@findex dired
@kindex C-x d
@vindex dired-listing-switches
  To invoke dired, do @kbd{C-x d} or @kbd{M-x dired}.  The command reads a
directory name or wildcard file name pattern as a minibuffer argument just
like the @code{list-directory} command, @kbd{C-x C-d}.  Where @code{dired}
differs from @code{list-directory} is in naming the buffer after the
directory name or the wildcard pattern used for the listing, and putting
the buffer into Dired mode so that the special commands of Dired are
available in it.  The variable @code{dired-listing-switches} is a string
used as an argument to @code{ls} in making the directory; this string
@i{must} contain @samp{-l}.

@findex dired-other-window
@kindex C-x 4 d
  To display the Dired buffer in another window rather than in the selected
window, use @kbd{C-x 4 d} (@code{dired-other-window)} instead of @kbd{C-x d}.

  Once the Dired buffer exists, you can switch freely between it and other
Emacs buffers.  Whenever the Dired buffer is selected, certain special
commands are provided that operate on files that are listed.  The Dired
buffer is ``read-only'', and inserting text in it is not useful, so
ordinary printing characters such as @kbd{d} and @kbd{x} are used for Dired
commands.  Most Dired commands operate on the file described by the line
that point is on.  Some commands perform operations immediately; others
``flag'' the file to be operated on later.

  Most Dired commands that operate on the current line's file also treat a
numeric argument a repeat count, meaning to apply to the files of the next
few lines.  A negative argument means to operate on the files of the
preceding lines, and leave point on the first of those lines.

  All the usual Emacs cursor motion commands are available in Dired
buffers.  Some special purpose commands are also provided.  The keys
@kbd{C-n} and @kbd{C-p} are redefined so that they try to position
the cursor at the beginning of the filename on the line, rather than
at the beginning of the line.

  For extra convenience, @key{SPC} and @kbd{n} in Dired are equivalent to
@kbd{C-n}.  @kbd{p} is equivalent to @kbd{C-p}.  Moving by lines is done so
often in Dired that it deserves to be easy to type.  @key{DEL} (move up and
unflag) is often useful simply for moving up.@refill

@section Deleting Files with Dired

  The primary use of Dired is to flag files for deletion and then delete
them.

@table @kbd
@item d
Flag this file for deletion.
@item u
Remove deletion-flag on this line.
@item @key{DEL}
Remove deletion-flag on previous line, moving point to that line.
@item x
Delete the files that are flagged for deletion.
@item #
Flag all auto-save files (files whose names start with @samp{#}) for
deletion (@pxref{Auto Save}).
@item ~
Flag all backup files (files whose names end with @samp{~}) for deletion
(@pxref{Backup}).
@item .
Flag excess numeric backup files for deletion.  The oldest and newest
few backup files of any one file are exempt; the middle ones are flagged.
@end table

  You can flag a file for deletion by moving to the line describing the
file and typing @kbd{d} or @kbd{C-d}.  The deletion flag is visible as a
@samp{D} at the beginning of the line.  Point is moved to the beginning of
the next line, so that repeated @kbd{d} commands flag successive files.

  The files are flagged for deletion rather than deleted immediately to
avoid the danger of deleting a file accidentally.  Until you direct Dired
to delete the flagged files, you can remove deletion flags using the
commands @kbd{u} and @key{DEL}.  @kbd{u} works just like @kbd{d}, but
removes flags rather than making flags.  @key{DEL} moves upward, removing
flags; it is like @kbd{u} with numeric argument automatically negated.

  To delete the flagged files, type @kbd{x}.  This command first displays a
list of all the file names flagged for deletion, and requests confirmation
with `yes'.  Once you confirm, all the flagged files are deleted, and their
lines are deleted from the text of the Dired buffer.  The shortened Dired
buffer remains selected.  If you answer `no' or quit with @kbd{C-g}, you
return immediately to Dired, with the deletion flags still present and no
files actually deleted.

  The @kbd{#}, @kbd{~} and @kbd{.} commands flags many files for
deletion, based on their names.  These commands are useful precisely
because they do not actually delete any files; you can remove the
deletion flags from any flagged files that you really wish to keep.@refill

  @kbd{#} flags for deletion all files that appear to have been made
by auto-saving (that is, files whose names begin with @samp{#}).
@kbd{~} flags for deletion all files that appear to have been made as
backups for files that were edited (that is, files whose names end
with @samp{~}).

  @kbd{.} flags just some of the backup files for deletion: only
numeric backups that are not among the oldest few nor the newest few
backups of any one file.  Normally @code{dired-kept-versions}
specifies the number of newest versions of each file to keep, and
@code{kept-old-versions} specifies the number of oldest versions to
keep.  A positive numeric argument to @kbd{.} specifies the number of
newest versions to keep, overriding @code{dired-kept-versions}.  A
negative numeric argument overrides @code{kept-old-versions}, using
minus the value of the argument to specify the number of oldest
versions of each file to keep.@refill


@section Immediate File Operations in Dired

  Some file operations in Dired take place immediately when they are
requested.

@table @kbd
@item c
Copies the file described on the current line.  You must supply a file name
to copy to, using the minibuffer.
@item f
Visits the file described on the current line.  It is just like typing
@kbd{C-x C-f} and supplying that file name.  If the file on this line is a
subdirectory, @kbd{f} actually causes Dired to be invoked on that
subdirectory.  @xref{Visiting}.
@item o
Like @kbd{f}, but uses another window to display the file's buffer.  The
Dired buffer remains visible in the first window.  This is like using
@kbd{C-x 4 C-f} to visit the file.  @xref{Windows}.
@item r
Renames the file described on the current line.  You must supply a file
name to rename to, using the minibuffer.
@item v
Views the file described on this line using @kbd{M-x view-file}.  Viewing a
file is like visiting it, but is slanted toward moving around in the file
conveniently and does not allow changing the file.  @xref{Misc File
Ops,View File}.  Viewing a file that is a directory runs Dired on that
directory.@refill
@end table

@node Misc File Ops,, Dired, Files
@section Miscellaneous File Operations

  Emacs has commands for performing many other operations on files.

@findex view-file
@cindex viewing
  @kbd{M-x view-file} allows you to scan or read a file by sequential
screenfuls.  It reads a file name argument using the minibuffer.  After
reading the file into an Emacs buffer, @code{view-file} reads and displays
one windowful.  You can then type @key{SPC} to scroll forward one windowful,
or @key{DEL} to scroll backward.  Various other commands are provided for
moving around in the file, but none for changing it; type @kbd{C-h} while
viewing for a list of them.  They are mostly the same as normal Emacs
cursor motion commands.  To exit from viewing, type @kbd{C-c}.

@findex insert-file
  @kbd{M-x insert-file} inserts the contents of the specified file into the
current buffer at point, leaving point unchanged before the contents and
the mark after them.  @xref{Mark}.

@findex write-region
@findex append-to-file
  @kbd{M-x write-region} is the inverse of @kbd{M-x insert-file}; it copies
the contents of the region into the specified file.  @kbd{M-x append-to-file}
adds the text of the region to the end of the specified file.

@findex delete-file
@cindex deletion (of files)
  @kbd{M-x delete-file} deletes the specified file, like the @code{rm}
command in the shell.  If you are deleting many files in one directory, it
may be more convenient to use Dired (@pxref{Dired}).

@findex rename-file
  @kbd{M-x rename-file} reads two file names @var{old} and @var{new} using
the minibuffer, then renames file @var{old} as @var{new}.  If a file named
@var{new} already exists, you must confirm with `yes' or renaming is not
done; this is because renaming causes the old meaning of the name @var{new}
to be lost.  If @var{old} and @var{new} are on different file systems, the
file @var{old} is copied and deleted.

@findex add-name-to-file
  The similar command @kbd{M-x add-name-to-file} is used to add an
additional name to an existing file without removing its old name.
The new name must belong on the same file system that the file is on.

@findex copy-file
  @kbd{M-x copy-file} reads the file @var{old} and writes a new file named
@var{new} with the same contents.  Confirmation is required if a file named
@var{new} already exists, because copying has the consequence of overwriting
the old contents of the file @var{new}.

@findex make-symbolic-link
  @kbd{M-x make-symbolic-link} reads two file names @var{old} and @var{linkname},
and then creates a symbolic link named @var{linkname} and pointing at @var{old}.
The effect is that future attempts to open file @var{linkname} will refer
to whatever file is named @var{old} at the time the opening is done, or
will get an error if the name @var{old} is not in use at that time.
Confirmation is required when creating the link if @var{linkname} is in
use.  Note that not all systems support symbolic links.

@node Buffers, Windows, Files, Top
@chapter Using Multiple Buffers

@cindex buffers
  The text you are editing in Emacs resides in an object called a
@dfn{buffer}.  Each time you visit a file, a buffer is created to hold the
file's text.  Each time you invoke Dired, a buffer is created to hold the
directory listing.  If you send a message with @kbd{C-x m}, a buffer named
@samp{*mail*} is used to hold the text of the message.  When you ask for a
command's documentation, that appears in a buffer called @samp{*Help*}.

@cindex selected buffer
@cindex current buffer
  At any time, one and only one buffer is @dfn{selected}.  It is also
called the @dfn{current buffer}.  Often we say that a command operates on
``the buffer'' as if there were only one; but really this means that the
command operates on the selected buffer (most commands do).

  When Emacs makes multiple windows, each window has a chosen buffer which
is displayed there, but at any time only one of the windows is selected and
its chosen buffer is the selected buffer.  Each window's mode line displays
the name of the buffer that the window is displaying (@pxref{Windows}).

  Each buffer has a name, which can be of any length, and you can select
any buffer by giving its name.  Most buffers are made by visiting files,
and their names are derived from the files' names.  But you can also create
an empty buffer with any name you want.  A newly started Emacs has a buffer
named @samp{*scratch*} which can be used for evaluating Lisp expressions in
Emacs.  The distinction between upper and lower case matters in buffer
names.

  Each buffer records individually what file it is visiting, whether it is
modified, and what major mode and minor modes are in effect in it
(@pxref{Major Modes}).  Any Emacs variable can be made @dfn{local to} a
particular buffer, meaning its value in that buffer can be different from
the value in other buffers.  @xref{Locals}.

@menu
* Select Buffer::   Creating a new buffer or reselecting an old one.
* List Buffers::    Getting a list of buffers that exist.
* Misc Buffer::     Renaming; changing read-onliness; copying text.
* Kill Buffer::     Killing buffers you no longer need.
* Several Buffers:: How to go through the list of all buffers
                     and operate variously on several of them.
@end menu

@node Select Buffer, List Buffers, Buffers, Buffers
@section Creating and Selecting Buffers

@table @kbd
@item C-x b @var{buffer} @key{RET}
Select or create a buffer named @var{buffer} (@code{switch-to-buffer}).
@item C-x 4 b @var{buffer} @key{RET}
Similar but select a buffer named @var{buffer} in another window
(@code{switch-to-buffer-other-window}).
@end table

@kindex C-x 4 b
@findex switch-to-buffer-other-window
@kindex C-x b
@findex switch-to-buffer
  To select the buffer named @var{bufname}, type @kbd{C-x b @var{bufname}
@key{RET}}.  This is the command @code{switch-to-buffer} with argument
@var{bufname}.  Because completion is provided for buffer names, you can
abbreviate the buffer name (@pxref{Completion}).  An empty argument to
@kbd{C-x b} specifies the most recently selected buffer that is not
displayed in any window.@refill

  Most buffers are created by visiting files, or by Emacs commands that
want to display some text, but you can also create a buffer explicitly by
typing @kbd{C-x b @var{bufname} @key{RET}}.  This makes a new, empty buffer which
is not visiting any file, and selects it for editing.  Such buffers are
used for making notes to yourself.  If you try to save one, you are asked
for the file name to use.  The new buffer's major mode is determined by the
value of @code{default-major-mode} (@pxref{Major Modes}).

  Note that @kbd{C-x C-f}, and any other command for visiting a file, can
also be used to switch buffers.  @xref{Visiting}.

@node List Buffers, Misc Buffer, Select Buffer, Buffers
@section Listing Existing Buffers

@table @kbd
@item C-x C-b
List the existing buffers (@code{list-buffers}).
@end table

@kindex C-x C-b
@findex list-buffers
  To print a list of all the buffers that exist, type @kbd{C-x C-b}.
Each line in the list shows one buffer's name, major mode and visited file.
@samp{*} at the beginning of a line indicates the buffer is ``modified''.
If several buffers are modified, it may be time to save some with @kbd{C-x
s} (@pxref{Saving}).  @samp{%} indicates a read-only buffer.  @samp{.}
marks the selected buffer.  Here is an example of a buffer list:@refill

@smallexample
 MR Buffer         Size  Mode           File
 -- ------         ----  ----           ----
.*  gmacs.tex      421336 Text          /u2/emacs/man/gmacs.tex
    *Help*         1287  Fundamental	
    files.el       23076 Emacs-Lisp     /u2/emacs/lisp/files.el
  % RMAIL          64042 RMAIL          /u/rms/RMAIL
    emacs.tex      383402 Text          /u2/emacs/man/emacs.tex
 *% man            747   Dired		
    net.emacs      343885 Fundamental   /u/rms/net.emacs
    fileio.c       27691 C              /u2/emacs/src/fileio.c
    NEWS           67340 Text           /u2/emacs/etc/NEWS
@end smallexample

@noindent
Note that the buffer @samp{*Help*} was made by a help request; it is not
visiting any file.  The buffer @code{man} was made by Dired on the
directory @file{/u2/emacs/man}.

@node Misc Buffer, Kill Buffer, List Buffers, Buffers
@section Miscellaneous Buffer Operations

@table @kbd
@item C-x C-q
Toggle read-only status of buffer (@code{toggle-read-only}).
@item M-x rename-buffer
Change the name of the current buffer.
@item M-x view-buffer
Scroll through a buffer.
@end table

@cindex read-only buffer
@kindex C-x C-q
@findex toggle-read-only
@vindex buffer-read-only
  A buffer can be @dfn{read-only}, which means that commands to change its
text are not allowed.  Normally, read-only buffers are made by subsystems
such as Dired and Rmail that have special commands to operate on the text;
a read-only buffer is also made if you visit a file that is protected so
you cannot write it.  If you wish to make changes in a read-only buffer,
use the command @kbd{C-x C-q} (@code{toggle-read-only}).  It makes a
read-only buffer writable, and makes a writable buffer read-only.  This
works by setting the variable @code{buffer-read-only}, which has a local
value in each buffer and makes the buffer read-only if its value is
non-@code{nil}.

@findex rename-buffer
  @kbd{M-x rename-buffer} changes the name of the current buffer.  Specify
the new name as a minibuffer argument.  There is no default.  If you
specify a name that is in use for some other buffer, an error happens and
no renaming is done.

@findex view-buffer
  @kbd{M-x view-buffer} is much like @kbd{M-x view-file} (@pxref{Misc File Ops})
except that it examines an already existing Emacs buffer.  View mode
provides commands for scrolling through the buffer conveniently but not
for changing it. When you exit View mode, the value of point that resulted
from your perusal remains in effect.

  The commands @kbd{C-x a} (@code{append-to-buffer}) and @kbd{M-x
insert-buffer} can be used to copy text from one buffer to another.
@xref{Accumulating Text}.@refill

@node Kill Buffer, Several Buffers, Misc Buffer, Buffers
@section Killing Buffers

  After you use Emacs for a while, you may accumulate a large number of
buffers.  You may then find it convenient to eliminate the ones you no
longer need.  There are several commands provided for doing this.

@c WideCommands
@table @kbd
@item C-x k
Kill a buffer, specified by name (@code{kill-buffer}).
@item M-x kill-some-buffers
Offer to kill each buffer, one by one.
@end table

@findex kill-buffer
@findex kill-some-buffers
@kindex C-x k

  @kbd{C-x k} (@code{kill-buffer}) kills one buffer, whose name you specify
in the minibuffer.  The default, used if you type just @key{RET} in the
minibuffer, is to kill the current buffer.  If the current buffer is
killed, another buffer is selected; a buffer that has been selected
recently but does not appear in any window now is chosen to be selected.
If the buffer being killed is modified (has unsaved editing) then you are
asked to confirm with `yes' before the buffer is killed.

  The command @kbd{M-x kill-some-buffers} asks about each buffer, one by
one.  An answer of @kbd{y} means to kill the buffer.  Killing the current
buffer or a buffer containing unsaved changes selects a new buffer or asks
for confirmation just like @code{kill-buffer}.

@node Several Buffers,, Kill Buffer, Buffers
@section Operating on Several Buffers
@cindex buffer menu

  The @dfn{buffer-menu} facility is like a ``Dired for buffers''; it allows
you to request operations on various Emacs buffers by editing an Emacs
buffer containing a list of them.

@table @kbd
@item M-x buffer-menu
Begin editing a buffer listing all Emacs buffers.
@end table

@findex buffer-menu
  The command @code{buffer-menu} writes a list of all Emacs buffers into
the buffer @samp{*Buffer List*}, and selects that buffer in Buffer Menu
mode.  The buffer is read-only, and can only be changed through the special
commands described in this section.  Most of these commands are graphic
characters.  The usual Emacs cursor motion commands can be used in the
@samp{*Buffer List*} buffer.  The following special commands apply to the
buffer described on the current line.

@table @kbd
@item k
Request to kill the buffer.  The request shows as a @samp{K} on the
line, before the buffer name.  Requested kills take place when the
@kbd{x} command is used.
@item s
Request to save the buffer.  The request shows as an @samp{S} on the
line.  Requested saves take place when the @kbd{x} command is used.
You may request both saving and killing for one buffer.
@item ~
Mark buffer ``unmodified''.  The command @kbd{~} does this,
immediately when typed.
@item x
Perform previously requested kills and saves.
@item u
Remove any request made for the current line.
@item @key{DEL}
Move to previous line and remove any request made for that line.
@end table

  There are also special commands to use the buffer list to select another
buffer, and to specify one or more other buffers for display in additional
windows.

@table @kbd
@item 1
Select the buffer in a full-screen window.  This command takes effect
immediately.
@item 2
Set up two windows, with this buffer in one, and the previously
selected buffer (aside from the buffer @samp{*Buffer List*}) in the
other.
@item q
Select this buffer, and also display in other windows any buffers
previously flagged with the @kbd{m} command.  If there are no such
buffers, this command is equivalent to @kbd{1}.
@item m
Flag this buffer to be displayed in another window if the @kbd{q}
command is used.  The request shows as a @samp{>} at the beginning of
the line.  The same buffer may not have both a kill request and a
display request.
@end table

  All the commands that put in flags to request operations later also move
down a line, and accept a numeric argument as a repeat count.

  The command @kbd{u} cancels any request flagged for the current line, and
moves down; @key{DEL} does so for the previous line, and moves up to it.

  All that @code{buffer-menu} does directly is create and select a suitable
buffer, and turn on Buffer Menu mode.  Everything else described above is
implemented by the special commands provided in Buffer Menu mode.  One
consequence of this is that you can switch from the @samp{*Buffer List*}
buffer to another Emacs buffer, and edit there.  You can reselect the
@code{buffer-menu} buffer later, to perform the operations already
requested, or you can kill it, or pay no further attention to it.

  The only difference between @code{buffer-menu} and @code{list-buffers} is
that @code{buffer-menu} selects the @samp{*Buffer List*} buffer and
@code{list-buffers} does not.  If you run @code{list-buffers} (that is,
type @kbd{C-x C-b}) and select the buffer list manually, you can use all of
the commands described here.

@node Windows, Major Modes, Buffers, Top
@chapter Multiple Windows
@cindex windows

  Emacs can split the screen into two or many windows, which can display
parts of different buffers, or different parts of one buffer.

  When multiple windows are being displayed, each window has an Emacs
buffer designated for display in it.  The same buffer may appear in more
than one window; if it does, any changes in its text are displayed in all
the windows where it appears.  But the windows showing the same buffer can
show different parts of it, because each window has its own value of point.

@cindex selected window
  At any time, one of the windows is the @dfn{selected window}; the buffer
this window is displaying is the current buffer.  The terminal's cursor
shows the location of point in this window.  Each other window has a
location of point as well, but since the terminal has only one cursor there
is no way to show where those locations are.

  Commands to move point affect the value of point for the selected Emacs
window only.  They do not change the value of point in any other Emacs
window, even one showing the same buffer.  The same is true for commands
such as @kbd{C-x b} to change the selected buffer in the selected window;
they do not affect other windows at all.  However, there are other commands
such as @kbd{C-x 4 b} that select a different window and switch buffers in
it.  Also, all commands that display information in a window, including
(for example) @kbd{C-h f} (@code{describe-function}) and @kbd{C-x C-b}
(@code{list-buffers}), work by switching buffers in a nonselected window
without affecting the selected window.

  Each window has its own mode line, which displays the buffer name,
modification status and major and minor modes of the buffer that is
displayed in the window.  @xref{Mode Line}, for full details on the mode
line.

@c WideCommands
@table @kbd
@item C-x 2
Split the selected window in two, one window above the other
(@code{split-window-vertically}).
@item C-x 5
Split the selected window into two windows side by side
(@code{split-window-horizontally}).
@item C-x o
Select another window (@code{other-window}).  That is @kbd{o}, not zero.
@item C-x 0
Get rid of the selected window (@code{kill-window}).  That is a zero.
@item C-x 1
Get rid of all windows except the selected one (@code{delete-other-windows}).
@item C-x 4
Prefix key for commands to select a buffer in various ways ``in
another window''.
@item C-x ^
Make the selected window taller, at the expense of the other(s)
(@code{enlarge-window}).
@item C-x @}
Make the selected window wider (@code{enlarge-window-horizontally}).
@item C-M-v
Scroll the next window (@code{scroll-other-window}).
@item M-x compare-windows
Find next place where the text in the selected window does not match
the text in the next window.
@end table

@kindex C-x 2
@findex split-window-vertically
  The command @kbd{C-x 2} (@code{split-window-vertically}) breaks the
selected window into two windows, one above the other.  Both windows start
out displaying the same buffer, with the same value of point.  By default
the two windows each get half the height of the window that was split; a
numeric argument specifies how many lines to give to the top window.

@kindex C-x 5
@findex split-window-horizontally
  @kbd{C-x 5} (@code{split-window-horizontally}) breaks the selected
window into two side-by-side windows.  A numeric argument specifies
how many columns to give the one on the left.  A line of vertical bars
separates the two windows.  Windows that are not the full width of the
screen have mode lines, but they are truncated; also, they do not
always appear in inverse video, because, the Emacs display routines
have not been taught how to display a region of inverse video that is
only part of a line on the screen.

@vindex truncate-partial-width-windows
  When a window is less than the full width, text lines too long to fit are
frequent.  Continuing all those lines might be confusing.  The variable
@code{truncate-partial-width-windows} can be set non-@code{nil} to force
truncation in all windows less than the full width of the screen,
independent of the buffer being displayed and its value for
@code{truncate-lines}.  @xref{Continuation Lines}.@refill

  Horizontal scrolling is often used in side-by-side windows.
@xref{Display}.

@kindex C-x o
@findex other-window
  To select a different window, use @kbd{C-x o} (@code{other-window}).
That is an @kbd{o}, for `other', not a zero.  When there are more than two
windows, this command moves through all the windows in a cyclic order,
generally top to bottom and left to right.  From the rightmost and
bottommost window, it goes back to the one at the upper left corner.  A
numeric argument means to move several steps in the cyclic order of
windows.  A negative argument moves around the cycle in the opposite order.
When the minibuffer is active, the minibuffer is the last window in the
cycle; you can switch from the minibuffer window to one of the other
windows, and later switch back and finish supplying the minibuffer argument
that is requested.  @xref{Minibuffer Edit}.

@kindex C-M-v
@findex scroll-other-window
  The usual scrolling commands (@pxref{Display}) apply to the selected
window only, but there is one command to scroll the next window.
@kbd{C-M-v} (@code{scroll-other-window}) scrolls the window that @kbd{C-x o}
would select.  The kind of scrolling done is the same as for @kbd{C-v}.

@kindex C-x 0
@findex delete-window
  To delete a window, type @kbd{C-x 0} (@code{delete-window}).  The space
it used to occupy is distributed among the other active windows (but not
the minibuffer window, even if that is active at the time).  Once a window
is deleted, everything about it is forgotten; there is no automatic way to
make another window showing the same contents.

@kindex C-x 1
@findex delete-other-windows
  @kbd{C-x 1} (@code{delete-other-windows}) is more powerful than @kbd{C-x 0};
it deletes all the windows except the selected one (and the minibuffer);
the selected window expands to use the whole screen except for the echo
area.

@kindex C-x ^
@findex enlarge-window
@kindex C-x @}
@findex enlarge-window-horizontally
@vindex window-min-height
@vindex window-min-width
  To readjust the division of space among existing windows, use @kbd{C-x ^}
(@code{enlarge-window}).  It makes the currently selected window get one
line bigger, or as many lines as is specified with a numeric argument.
With a negative argument, it makes the selected window smaller.  @kbd{C-x
@}} (@code{enlarge-window-horizontally}) makes the selected window wider
by the specified number of columns.  The extra screen space given to a
window comes from one of its neighbors, if that is possible; otherwise, all
the competing windows are shrunk in the same proportion.  If this makes any
windows too small, those windows are deleted and their space is divided up.
The minimum size is specified by the variables @code{window-min-height} and
@code{window-min-width}.

@findex compare-windows
  The command @kbd{M-x compare-windows} compares the text in the current
window with that in the next window (the one @kbd{C-M-v} would scroll).
Comparison starts at point in each window.  Point moves forward in each
window, a character at a time in each window, until the next characters
in the two windows are different.  Then the command is finished.

@kindex C-x 4
  @kbd{C-x 4} is a prefix key for commands that select another window
(splitting the window if there is only one) and select a buffer in that
window.  Different @kbd{C-x 4} commands have different ways of finding the
buffer to select.

@findex switch-to-buffer-other-window
@findex find-file-other-window
@findex find-tag-other-window
@findex dired-other-window
@findex mail-other-window
@table @kbd
@item C-x 4 b @var{bufname} @key{RET}
Select buffer @var{bufname} in another window.  This runs
@code{switch-to-buffer-other-window}.
@item C-x 4 f @var{filename} @key{RET}
Visit file @var{filename} and select its buffer in another window.  This
runs @code{find-file-other-window}.  @xref{Visiting}.
@item C-x 4 d @var{directory} @key{RET}
Select a Dired buffer for directory @var{directory} in another window.
This runs @code{dired-other-window}.  @xref{Dired}.
@item C-x 4 m
Start composing a mail message in another window.  This runs
@code{mail-other-window}, and its same-window version is @kbd{C-x m}
(@pxref{Sending Mail}).
@item C-x 4 .
Find a tag in the current tag table in another window.  This runs
@code{find-tag-other-window}, the multiple-window variant of @kbd{M-.}
(@pxref{Tags}).
@end table

@node Major Modes, Indentation, Windows, Top
@chapter Major Modes
@cindex major modes
@kindex TAB
@kindex DEL
@kindex LFD

  Emacs has many different @dfn{major modes}, each of which customizes
Emacs for editing text of a particular sort.  The major modes are mutually
exclusive, and each buffer has one major mode at any time.  The mode line
normally contains the name of the current major mode, in parentheses.
@xref{Mode Line}.

  The least specialized major mode is called @dfn{Fundamental mode}.  This
mode has no mode-specific redefinitions or variable settings, so that each
Emacs command behaves in its most general manner, and each option is in its
default state.  For editing any specific type of text, such as Lisp code or
English text, you should switch to the appropriate major mode, such as Lisp
mode or Text mode.

  Selecting a major mode changes the meanings of a few keys to become more
specifically adapted to the language being edited.  The ones which are
changed frequently are @key{TAB}, @key{DEL}, and @key{LFD}.  In addition,
the commands which handle comments use the mode to determine how comments
are to be delimited.  Many major modes redefine the syntactical properties
of characters appearing in the buffer.  @xref{Syntax}.

  The major modes fall into three major groups.  Lisp mode (which has
several variants), C mode and Muddle mode are for specific programming
languages.  Text mode, Nroff mode, @TeX{} mode and outline mode are for
editing English text.  The remaining major modes are not intended for use
on user's files; they are used in buffers created for specific purposes by
Emacs, such as Dired mode for buffers made by Dired (@pxref{Dired}), and
Mail mode for buffers made by @kbd{C-x m} (@pxref{Sending Mail}), and Shell
mode for buffers used for communicating with an inferior shell process
(@pxref{Shell}).

  Selecting a new major mode is done with an @kbd{M-x} command.  From the
name of a major mode, add @code{-mode} to get the name of a command
function to select that mode.  Thus, you can enter Lisp mode by executing
@kbd{M-x lisp-mode}.

@vindex auto-mode-alist
  When you visit a file, Emacs usually chooses the right major mode based
on the file's name.  For example, files whose names end in @code{.c} are
edited in C mode.  The correspondence between file names and major mode is
controlled by the variable @code{auto-mode-alist}.  Its value is a list in
which each element has the form

@example
(@var{regexp} . @var{mode-function})
@end example

@noindent
For example, one element normally found in the list has the form
@code{(@t{"\\.c$"} . c-mode)}, and it is responsible for selecting C mode
for files whose names end in @code{.c}.  (Note that @samp{\\} is needed in
Lisp syntax to include a @samp{\} in the string.)

  You can specify which major mode should be used for editing a certain
file by a special sort of text in the first nonblank line of the file.  The
mode name should appear in this line both preceded and followed by
@samp{-*-}.  Other text may appear on the line as well.  For example,

@example
;-*-Lisp-*-
@end example

@noindent
tells Emacs to use Lisp mode.  Note how the semicolon is used to make Lisp
treat this line as a comment.  Such an explicit specification overrides any
defaulting based on the file name.

  Another format of mode specification is

@example
-*-Mode: @var{modename};-*-
@end example

@noindent
which allows other things besides the major mode name to be specified.
However, Emacs does not look for anything except the mode name.

@vindex default-major-mode
  When a file is visited that does not specify a major mode to use, or when
a new buffer is created with @kbd{C-x b}, the major mode used is that
specified by the variable @code{default-major-mode}.  Normally this value
is the symbol @code{fundamental-mode}, which specifies Fundamental mode.
If @code{default-major-mode} is @code{nil}, the major mode is taken from
the previously selected buffer.

  Most programming language major modes specify that only blank lines
separate paragraphs.  This is so that the paragraph commands remain useful.
@xref{Paragraphs}.  They also cause Auto Fill mode to use the definition of
@key{TAB} to indent the new lines it creates.  This is because most lines
in a program are usually indented.  @xref{Indentation}.

@node Indentation, Text, Major Modes, Top
@chapter Indentation
@cindex indentation

@c WideCommands
@table @kbd
@item @key{TAB}
Indent current line ``appropriately'' in a mode-dependent fashion.
@item @key{LFD}
Perform @key{RET} followed by @key{TAB} (@code{newline-and-indent}).
@item M-^
Merge two lines (@code{delete-indentation}).  This would cancel out
the effect of @key{LFD}.
@item C-M-o
Split line at point; text on the line after point becomes a new line
indented to the same column that it now starts in (@code{split-line}).
@item M-m
Move (forward or back) to the first nonblank character on the current
line (@code{back-to-indentation}).
@item C-M-\
Indent several lines to same column (@code{indent-region}).
@item C-x @key{TAB}
Shift block of lines rigidly right or left (@code{indent-rigidly}).
@item M-i
Indent from point to the next prespecified tab stop column
(@code{tab-to-tab-stop}).
@item M-x indent-relative
Indent from point to under an indentation point in the previous line.
@end table

@kindex TAB
@cindex indentation
  Most programming languages have some indentation convention.  For Lisp
code, lines are indented according to their nesting in parentheses.  The
same general idea is used for C code, though many details are different.

  Whatever the language, to indent a line, use the @key{TAB} command.  Each
major mode defines this command to perform the sort of indentation
appropriate for the particular language.  In Lisp mode, @key{TAB} aligns
the line according to its depth in parentheses.  No matter where in the
line you are when you type @key{TAB}, it aligns the line as a whole.  In C
mode, @key{TAB} implements a subtle and sophisticated indentation style that
knows about many aspects of C syntax.

@kindex TAB
@kindex LFD
@findex indent-new-line
  In Text mode, @key{TAB} runs the command @code{tab-to-tab-stop}, which
indents to the next tab stop column.  You can set the tab stops with
@kbd{M-x edit-tab-stops}.

@menu
* Indentation Commands:: Various commands and techniques for indentation.
* Tab Stops::            You can set arbitrary "tab stops" and then
                         indent to the next tab stop when you want to.
* Just Spaces::          You can request indentation using just spaces.
@end menu

@node Indentation Commands,, Indentation, Indentation
@section Indentation Commands and Techniques
@c ??? Explain what Emacs has instead of space-indent-flag.

  If you just want to insert a tab character in the buffer, you can type
@kbd{C-q @key{TAB}}.

@kindex M-m
@findex back-to-indentation
  To move over the indentation on a line, do @kbd{Meta-m}
(@code{back-to-indentation}).  This command, given anywhere on a line,
positions point at the first nonblank character on the line.

  To insert an indented line before the current line, do @kbd{C-a C-o
@key{TAB}}.  To make an indented line after the current line, use @kbd{C-e
@key{LFD}}.

@kindex C-M-o
@findex split-line
  @kbd{C-M-o} (@code{split-line}) moves the text from point to the end of
the line vertically down, so that the current line becomes two lines.
@kbd{C-M-o} first moves point forward over any spaces and tabs.  Then it
inserts after point a newline and enough indentation to reach the same
column point is on.  Point remains before the inserted newline; in this
regard, @kbd{C-M-o} resembles @kbd{C-o}.

@kindex M-\
@kindex M-^
@findex delete-horizontal-space
@findex delete-indentation
  To join two lines cleanly, use the @kbd{Meta-^} (@code{delete-indentation})
command to delete the indentation at the front of the current line, and the
line boundary as well.  They are replaced by a single space, or by no space
if at the beginning of a line or before a @samp{)} or after a @samp{(}.  To
delete just the indentation of a line, go to the beginning of the line and
use @kbd{Meta-\} (@code{delete-horizontal-space}), which deletes all spaces
and tabs around the cursor.

@kindex C-M-\
@kindex C-x TAB
@findex indent-region
@findex indent-rigidly
  There are also commands for changing the indentation of several lines at
once.  @kbd{Control-Meta-\} (@code{indent-region}) gives each line which
begins in the region the ``usual'' indentation by invoking @key{TAB} at the
beginning of the line.  A numeric argument specifies the column to indent
to, and each line is shifted left or right so that its first nonblank
character appears in that column.  @kbd{C-x @key{TAB}}
(@code{indent-rigidly}) moves all of the lines in the region right by its
argument (left, for negative arguments).  The whole group of lines moves
rigidly sideways, which is how the command gets its name.@refill

@findex indent-relative
  @kbd{M-x indent-relative} indents at point based on the previous line
(actually, the last nonempty line.)  It inserts whitespace at point, moving
point, until it is underneath an indentation point in the previous line.
An indentation point is the end of a sequence of whitespace or the end of
the line.  If point is farther right than any indentation point in the
previous line, the whitespace before point is deleted and the first
indentation point then applicable is used.  If no indentation point is
applicable even then, @code{tab-to-tab-stop} is run.

  @code{indent-relative} is the definition of @key{TAB} in Indented Text
mode.  @xref{Text}.

@node Tab Stops, Just Spaces, Indentation Commands, Indentation
@section Tab Stops

@kindex M-i
@findex tab-to-tab-stop
  For typing in tables, you can use Text mode's definition of @key{TAB},
@code{tab-to-tab-stop}.  This command inserts indentation before point,
enough to reach the next tab stop column.  If you are not in Text mode,
this function can be found on @kbd{M-i} anyway.

@findex edit-tab-stops
@findex edit-tab-stops-note-changes
@kindex C-x C-s
@vindex tab-stop-list
  The tab stops used by @kbd{M-i} can be set arbitrarily by the user.
They are stored in a variable called @code{tab-stop-list}, as a list of
column-numbers in increasing order.

  The convenient way to set the tab stops using @kbd{M-x edit-tab-stops},
which creates and selects a buffer containing a description of the tab stop
settings.  You can edit this buffer to specify different tab stops, and
then type @kbd{C-x C-s} to make those new tab stops take effect.  In the
tab stop buffer, @kbd{C-x C-s} runs the function
@code{edit-tab-stops-note-changes} rather than its usual definition
@code{save-buffer}.  @code{edit-tab-stops} records which buffer was current
when you invoked it, and stores the tab stops back in that buffer; normally
all buffers share the same tab stops and changing them in one buffer
affects all, but if you happen to make @code{tab-stop-list} local in one
buffer then @code{edit-tab-stops} in that buffer will edit the local
settings.

  Here is what the text representing the tab stops looks like for ordinary
tab stops every eight columns.

@example
        :       :       :       :       :       :
0         1         2         3         4        
0123456789012345678901234567890123456789012345678
To install changes, type C-x C-s
@end example

  The first line contains a colon at each tab stop.  The remaining lines
are present just to help you see where the colons are and know what to do.

  Note that the tab stops that control @code{tab-to-tab-stop} have nothing
to do with displaying tab characters in the buffer.  @xref{Display Vars},
for more information on that.

@node Just Spaces,, Tab Stops, Indentation
@section Tabs vs. Spaces

@vindex indent-tabs-mode
  Emacs normally uses both tabs and spaces to indent lines.  If you prefer,
all indentation can be made from spaces only.  To request this, set
@code{indent-tabs-mode} to @code{nil}.

@findex tabify
@findex untabify
  There are also commands to convert tabs to spaces or vice versa, always
preserving the columns of all nonblank text.  @kbd{M-x tabify} scans the
region for sequences of spaces, and converts sequences of at least three
spaces to tabs if that can be done without changing indentation.  @kbd{M-x
untabify} changes all tabs in the region to appropriate numbers of spaces.

@node Text, Programs, Indentation, Top
@chapter Commands for Human Languages
@cindex text

  The term @dfn{text} has two widespread meanings in our area of the
computer field.  One is data that is a sequence of characters.  Any file
that you edit with Emacs is text, in this sense of the word.  The other
meaning is more restrictive; it is, a sequence of characters in a human
language for humans to read (possibly after processing by a text
formatter), as opposed to a program or commands for a program.

  Human languages have syntactic/stylistic conventions that can be
supported or used to advantage by editor commands: conventions involving
words, sentences, paragraphs, and capital letters.  This chapter describes
Emacs commands for all of these things.  There are also commands for
@dfn{filling}, or rearranging paragraphs into lines of approximately equal
length.

  The commands for moving over and killing words (@pxref{Words}), sentences
(@pxref{Sentences}) and paragraphs (@pxref{Paragraphs}) are are primarily
intended for human-language text, but are very often useful in editing
programs also.

  Emacs has several major modes for editing human language text.
If the file contains text pure and simple, use Text mode, which customizes
Emacs in small ways for the syntactic conventions of text.  For text which
contains embedded commands for text formatters, Emacs has other major modes,
each for a particular text formatter.  Thus, for input to @TeX{}, you would
use @TeX{} mode; for input to nroff, Nroff mode.

@menu
* Text Mode::   The major modes for editing text files.
* Nroff Mode::  The major mode for editing input to the formatter nroff.
* TeX Mode::    The major mode for editing input to the formatter TeX.
* Outline Mode::The major mode for editing outlines.
* Words::       Moving over and killing words.
* Sentences::   Moving over and killing sentences.
* Paragraphs::	Moving over paragraphs.
* Pages::	Moving over pages.
* Filling::     Filling or justifying text
* Case::        Changing the case of text
@end menu

@node Text Mode, Words, Text, Text
@section Text Mode

@findex tab-to-tab-stop
@findex edit-tab-stops
@cindex Text mode
@kindex TAB
@findex text-mode
  Editing files of text in a human language ought to be done using Text
mode rather than Lisp or Fundamental mode.  Invoke @kbd{M-x text-mode} to
enter Text mode.  In Text mode, @key{TAB} runs the function
@code{tab-to-tab-stop}, which allows you to use arbitrary tab stops set
with @kbd{M-x edit-tab-stops} (@pxref{Tab Stops}).  Features concerned with
comments in programs are turned off except when explicitly invoked.  The
syntax table is changed so that periods are not considered part of a word,
while apostrophes, backspaces and underlines are.

@findex indented-text-mode
  A similar variant mode is Indented Text mode, intended for editing text
in which most lines are indented.  This mode defines @key{TAB} to run
@code{indent-relative} (@pxref{Indentation}), and makes Auto Fill indent
the lines it creates.  The result is that normally a line made by Auto
Filling, or by @key{LFD}, is indented just like the previous line.  Use
@kbd{M-x indented-text-mode} to select this mode.

@vindex text-mode-hook
  Entering Text mode or Indented Text mode calls with no arguments the
value of the variable @code{text-mode-hook}, if that value exists and is
not @code{nil}.  This value is also called when modes related to Text mode
are entered; this includes Nroff mode, @TeX{} mode, Outline mode and Mail
mode.  Your hook can look at the value of @code{major-mode} to see which of
these modes is actually being entered.

@menu
  Two modes similar to Text mode are of use for editing text that is to
be passed through a text formatter before achieving the form in which
humans are to read it.

* Nroff Mode::  The major mode for editing input to the formatter nroff.
* TeX Mode::    The major mode for editing input to the formatter TeX.

  Another similar mode is used for editing outlines.  It allows you
to view the text at various levels of detail.  You can view either
the outline headings alone or both headings and text; you can also
hide some of the headings at lower levels from view to make the high
level structure more visible.

* Outline Mode::The major mode for editing outlines.
@end menu

@node Nroff Mode, TeX Mode, Text Mode, Text Mode
@subsection Nroff Mode

@findex nroff-mode
  Nroff mode is a mode like Text mode but modified to handle nroff commands
present in the text.  Invoke @kbd{M-x nroff-mode} to enter this mode.  It
differs from Text mode in only a few ways.  All nroff command lines are
considered paragraph separators, so that filling will never garble the
nroff commands.  Pages are separated by @samp{.bp} commands.  Also, three
special commands are provided that are not in Text mode:

@findex forward-text-line
@findex backward-text-line
@findex count-text-lines
@kindex M-n
@kindex M-p
@kindex M-?
@table @kbd
@item M-n
Move to the beginning of the next line that isn't an nroff command
(@code{forward-text-line}).  An argument is a repeat count.
@item M-p
Like @kbd{M-n} but move up (@code{backward-text-line}).
@item M-?
Prints in the echo area the number of text lines (lines that are not
nroff commands) in the region (@code{count-text-lines}).
@end table

@findex electric-nroff-mode
  The other feature of Nroff mode is that you can turn on Electric
Nroff newline mode.  This is a minor mode that you can turn on or off
with @kbd{M-x electric-nroff-mode} (@pxref{Minor Modes}).  When the
mode is on, each time you use @key{RET} to end a line that contains
an nroff command that opens a kind of grouping, the matching
nroff command to close that grouping is automatically inserted on
the following line.  For example, if you are at the beginning of
a line and type @kbd{.@: ( b @key{RET}}, the matching command
@samp{.)b} will be inserted on a new line following point.

@vindex nroff-mode-hook
  Entering Nroff mode calls with no arguments the value of the variable
@code{text-mode-hook}, if that value exists and is not @code{nil}; then it
does the same with the variable @code{nroff-mode-hook}.

@node TeX Mode, Outline Mode, Nroff Mode, Text Mode
@subsection @TeX{} Mode
@cindex TeX
@cindex LaTeX
@findex TeX-mode
@findex tex-mode

  @TeX{} is an extremely powerful text formatter written by Donald Knuth;
it is also free, like GNU Emacs.  La@TeX{} is a simplified input format for
@TeX{}, implemented by @TeX{} macros.  It comes with @TeX{}.@refill

  @TeX{} mode is designed for editing files of input for plain @TeX{} or La@TeX{}.
It provides facilities for checking the balance of delimiters and for
invoking @TeX{} on all or part of the file.  Type @kbd{M-x tex-mode} to
enter @TeX{} mode.

@table @kbd
@item "
Insert @samp{@`@`}, @samp{"} or @samp{@'@'} according to context (@code{TeX-insert-quote}).
@item @key{LFD}
Insert a paragraph break (two newlines) and check the previous
paragraph for unbalanced braces or dollar signs
(@code{TeX-terminate-paragraph}).
@item M-x validate-TeX-buffer
Check each paragraph in the buffer for unbalanced braces or dollar signs.
@item M-@{
Insert @samp{@{@}} and position point between them (@code{TeX-insert-braces}).
@item M-@}
Move forward past the next unmatched close brace (@code{up-list}).
@item C-c C-r
Invoke @TeX{} on the current region, plus the buffer's header
(@code{TeX-region}).
@item C-c C-b
Invoke @TeX{} on the entire current buffer (@code{TeX-buffer}).
@item C-c C-p
Print the output from the last @kbd{C-c C-r} or @kbd{C-c C-b} command
(@code{TeX-print}).
@end table

@findex TeX-insert-quote
@kindex " (TeX mode)
  In @TeX{}, the character @samp{"} is not normally used; one uses @samp{``}
to start a quotation and @samp{''} to end one.  @TeX{} mode defines the key
@kbd{"} to insert @samp{``} after whitespace or an open brace, @samp{"}
after a backslash, or @samp{''} otherwise.  This is done by the command
@code{TeX-insert-quote}.  If you need the character @samp{"} itself in
unusual contexts, use @kbd{C-q} to insert it.  Also, @kbd{"} with a
numeric argument always inserts that number of @samp{"} characters.

  In @TeX{} mode, @samp{$} has a special syntax code which attempts to
understand the way @TeX{} math mode delimiters match.  When you insert a
@samp{$} that is meant to exit math mode, the position of the matching
@samp{$} that entered math mode is displayed for a second.  This is the
same feature that displays the open brace that matches a close brace that
is inserted.  However, there is no way to tell whether a @samp{$} enters
math mode or leaves it; so when you insert a @samp{$} that enters math
mode, the previous @samp{$} position is shown as if it were a match, even
though they are actually unrelated.

@findex TeX-insert-braces
@kindex M-@{
@findex up-list
@kindex M-@}
  If you prefer to keep braces balanced at all times, you can use @kbd{M-@{}
(@code{TeX-insert-braces}) to insert a pair of braces.  It leaves point
between the two braces so you can insert the text that belongs inside.
Afterward, use the command @kbd{M-@}} (@code{up-list}) to move forward
past the close brace.

@findex validate-TeX-buffer
@findex TeX-terminate-paragraph
@kindex LFD (TeX mode)
  There are two commands for checking the matching of braces.  @key{LFD}
(@code{TeX-terminate-paragraph}) checks the paragraph before point, and
inserts two newlines to start a new paragraph.  It prints a message in the
echo area if any mismatch is found.  @kbd{M-x validate-TeX-buffer} checks
the entire buffer, paragraph by paragraph.  When it finds a paragraph that
contains a mismatch, it displays point at the beginning of the paragraph
for a few seconds and pushes a mark at that spot.  Scanning continues
until the whole buffer has been checked or until you type another key.
The positions of the last several paragraphs with mismatches can be
found in the mark ring (@pxref{Mark Ring}).
  
Note that square brackets and parentheses are matched in @TeX{} mode, not
just braces.  This is wrong for the purpose of checking @TeX{} syntax.
However, parentheses and square brackets are likely to be used in text as
matching delimiters and it is useful for the various motion commands and
automatic match display to work with them.

@findex TeX-buffer
@kindex C-c C-b
@findex TeX-print
@kindex C-c C-p
  You can pass the current buffer through an inferior @TeX{} by means of
@kbd{C-c C-b} (@code{TeX-buffer}).  The error messages appear in a buffer
called @samp{*TeX-shell*}.  The formatted output appears in a file in
@file{/tmp}; to print it, type @kbd{C-c C-p} (@code{TeX-print}).@refill

@findex TeX-region
@kindex C-c C-r
  You can also pass an arbitrary region through an inferior @TeX{} by typing
@kbd{C-c C-r} (@code{TeX-region}).  This is tricky, however, because most files
of @TeX{} input contain commands at the beginning to set paramaters and
define macros, without which no later part of the file will format
correctly.  To solve this problem, @kbd{C-c C-r} allows you to designate a
part of the file as containing essential commands; it is included before
the specified region as part of the input to @TeX{}.  The designated part
of the file is called the @dfn{header}.

@cindex header (TeX mode)
  To indicate the bounds of the header, insert two special strings in the
file.  Insert @samp{%**start of header} before the header, and @samp{%**end of
header} after it.  Each string must appear entirely on one line, but there
may be other text on the line before or after.  The lines containing the
two strings are not included in the header.

  If @samp{%**start of header} does not appear within the first ten lines of
the text in the buffer, @kbd{C-c C-r} assumes that there is no header.

@vindex TeX-mode-hook
  Entering @TeX{} mode calls with no arguments the value of the variable
@code{text-mode-hook}, if that value exists and is not @code{nil}; then it
does the same with the variable @code{TeX-mode-hook}.

  @TeX{} for Berkeley Unix can be obtained on a 1600bpi tar tape for a $75
distribution fee from

@display
Pierre MacKay
Department of Computer Science, FR-35
University of Washington
Seattle, WA 98195
@end display

@noindent
It would work on system V as well if that version of Unix had a reasonable
Pascal compiler.  Outside the U.S., add $10 to cover extra costs.

@node Outline Mode,, TeX Mode, Text Mode
@subsection Outline Mode
@cindex outlines
@cindex selective display
@cindex invisible lines

  Outline mode is a major mode much like Text mode but intended for editing
outlines.  It allows you to make parts of the text temporarily invisible
to that you can see just the overall structure of the outline.  Type
@kbd{M-x outline-mode} to turn on Outline mode in the current buffer.

  When a line is invisible in outline mode, it does not appear on the
screen in any form.  The screen appears exactly as if the invisible line
were deleted.  All editing commands treat the text of the invisible line as
part of the previous visible line.  For example, @kbd{C-n} moves onto the
next visible line.  Killing an entire visible line, including its
terminating newline, really kills all the following invisible lines along
with it; yanking it all back yanks the invisible lines and they remain
invisible.

@cindex heading lines (Outline mode)
@cindex body lines (Outline mode)
  Outline mode assumes that the lines in the buffer are of two types:
@dfn{heading lines} and @dfn{body lines}.  A heading line represents a topic in the
outline.  Heading lines start with one or more stars; the number of stars
determines the depth of the heading in the outline structure.  Thus, a
heading line with one star is a major topic; all the heading lines with
two stars between it and the next one-star heading are its subtopics; and
so on.  Any line that is not a heading line is a body line.  Body lines
belong to the preceding heading line.  Here is an example:

@example
* Food

This is the body,
which says something about the topic of food.

** Delicious Food

This is the body of the second-level header.

** Distasteful Food

This could have
a body too, with
several lines.

*** Dormitory Food

* Shelter

A second first-level topic with its header line.
@end example


  A heading line together with all following body lines are called
collectively an @dfn{entry}.  A heading line together with all following
deeper heading lines and their body lines is called a @dfn{subtree}.

@table @kbd
@item M-@}
Move point to the next visible heading line (@code{next-visible-heading}).
@item M-@{
Move point to the previous visible heading line
(@code{previous-visible-heading}).
@item M-x hide-body
Make all body lines in the buffer invisible.
@item M-x show-all
Make all lines in the buffer visible.
@item C-c C-h
Make everything under this heading invisible, but not this heading itself
(@code{hide-subtree}).
@item C-c C-s
Make everything under this heading visible, including body, subheadings,
and their bodies (@code{show-subtree}).
@item C-c C-i
Make immediate subheadings (one level down) of this heading line visible
(@code{show-children}).
@item M-x hide-entry
Make this heading line's body invisible.
@item M-x show-entry
Make this heading line's body visible.
@item M-x hide-leaves
Make the body of this heading line, and of all its subheadings, invisible.
@item M-x show-branches
Make all subheadings of this heading line, at all levels, visible.
@end table

@findex next-visible-heading
@findex previous-visible-heading
@kindex M-@{ (Outline mode)
@kindex M-@} (Outline mode)
  There are two special motion commands in Outline mode.  @kbd{M-@}}
(@code{next-visible-heading}) moves down to the next heading line.
@kbd{M-@{} (@code{previous-visible-heading}) moves similarly backward.
Both accept numeric arguments as repeat counts.  The names emphasize that
invisible headings are skipped, but this is not really a special feature.
All editing commands that look for lines ignore the invisible lines
automatically.@refill

  The other special commands of outline mode are used to make lines visible
or invisible.  Their names all start with @code{hide} or @code{show}.
Most of them fall into pairs of opposites.  They are not undoable; instead,
you can undo right past them.  Making lines visible or invisible is simply
not recorded by the undo mechanism.

@findex hide-entry
@findex show-entry
  Two commands that are exact opposites are @kbd{M-x hide-entry} and
@kbd{M-x show-entry}.  They are used with point on a heading line, and
apply only to the body lines of that heading.  The subtopics and their
bodies are not affected.

@findex hide-subtree
@findex show-subtree
@kindex C-c C-s (Outline mode)
@kindex C-c C-h (Outline mode)
@cindex subtree (Outline mode)
  Two more powerful opposites are @kbd{C-c C-h} (@code{hide-subtree}) and
@kbd{C-c C-s} (@code{show-subtree}).  Both expect to be used when point is
on a heading line, and both apply to all the lines of that heading's
@dfn{subtree}: its body, all its subheadings, both direct and indirect, and
all of their bodies.  In other words, the subtree contains everything
following this heading line, up to and not including the next heading of
the same or higher rank.@refill

@findex hide-leaves
@findex show-branches
  Intermediate between a visible subtree and an invisible one is having
all the subheadings visible but none of the body.  There are two commands
for doing this, depending on whether you want to hide the bodies or
make the subheadings visible.  They are @kbd{M-x hide-leaves} and
@kbd{M-x show-branches}.

@kindex C-c C-i
@findex show-children
  A little weaker than @code{show-branches} is @kbd{C-c C-i}
(@code{show-children}).  It makes just the direct subheadings
visible---those one level down.  Deeper subheadings remain invisible, if
they were invisible.@refill

@findex hide-body
@findex show-all
  Two commands have a blanket effect on the whole file.  @kbd{M-x hide-body}
makes all body lines invisible, so that you see just the outline structure.
@kbd{M-x show-all} makes all lines visible.  These commands can be thought
of as a pair of opposites even though @kbd{M-x show-all} applies to more
than just body lines.

  Outline mode makes a line invisible by changing the newline before it
into an ASCII Control-M (code 015).  Most editing commands that work on
lines treat an invisible line as part of the previous line because,
strictly speaking, it @i{is} part of that line, since there is no longer a
newline in between.  When you save the file in Outline mode, Control-M
characters are saved as newlines, so the invisible lines become ordinary
lines in the file.  But saving does not change the visibility status of a
line inside Emacs.

@vindex outline-mode-hook
  Entering Outline mode calls with no arguments the value of the variable
@code{text-mode-hook}, if that value exists and is not @code{nil}; then it
does the same with the variable @code{outline-mode-hook}.

@node Words, Sentences, Text Mode, Text
@section Words
@cindex words
@cindex Meta

  Emacs has commands for moving over or operating on words.  By convention,
the keys for them are all @kbd{Meta-} characters.

@c widecommands
@table @kbd
@item M-f
Move forward over a word (@code{forward-word}).
@item M-b
Move backward over a word (@code{backward-word}).
@item M-d
Kill up to the end of a word (@code{kill-word}).
@item M-@key{DEL}
Kill back to the beginning of a word (@code{backward-kill-word}).
@item M-@@
Mark the end of the next word (@code{mark-word}).
@item M-t
Transpose two words;  drag a word forward
or backward across other words (@code{transpose-words}).
@end table

  Notice how these keys form a series that parallels the
character-based @kbd{C-f}, @kbd{C-b}, @kbd{C-d}, @kbd{C-t} and
@key{DEL}.  @kbd{M-@@} is related to @kbd{C-@@}, which is an alias for
@kbd{C-@key{SPC}}.@refill

@kindex M-f
@kindex M-b
@findex forward-word
@findex backward-word
  The commands @kbd{Meta-f} (@code{forward-word}) and @kbd{Meta-b}
(@code{backward-word}) move forward and backward over words.  They are thus
analogous to @kbd{Control-f} and @kbd{Control-b}, which move over single
characters.  Like their @kbd{Control-} analogues, @kbd{Meta-f} and
@kbd{Meta-b} move several words if given an argument.  @kbd{Meta-f} with a
negative argument moves backward, and @kbd{Meta-b} with a negative argument
moves forward.  Forward motion stops right after the last letter of the
word, while backward motion stops right before the first letter.@refill

@kindex M-d
@findex kill-word
  @kbd{Meta-d} (@code{kill-word}) kills the word after point.  To be
precise, it kills everything from point to the place @kbd{Meta-f} would
move to.  Thus, if point is in the middle of a word, @kbd{Meta-d} kills
just the part after point.  If some punctuation comes between point and the
next word, it is killed along with the word.  If you wish to kill only the
next word but not the punctuation before it, simply do @kbd{Meta-f} to get
the end, and kill the word backwards with @kbd{Meta-@key{DEL}}.
@kbd{Meta-d} takes arguments just like @kbd{Meta-f}.

@findex backward-kill-word
@kindex M-DEL
  @kbd{Meta-@key{DEL}} (@code{backward-kill-word}) kills the word before
point.  It kills everything from point back to where @kbd{Meta-b} would
move to.  If point is after the space in @w{@samp{FOO, BAR}}, then
@w{@samp{FOO, }} is killed.  If you wish to kill just @samp{FOO}, do
@kbd{Meta-b Meta-d} instead of @kbd{Meta-@key{DEL}}.

@cindex transposition
@kindex M-t
@findex transpose-words
  @kbd{Meta-t} (@code{transpose-words}) exchanges the words before or
containing point with the following word.  The delimiter characters between
the words do not move.  For example, @w{@samp{FOO, BAR}} transposes into
@w{@samp{BAR, FOO}} rather than @samp{@w{BAR FOO,}}.  @xref{Transpose}, for
more on transposition and on arguments to transposition commands.

@kindex M-@@
@findex mark-word
  To operate on the next @var{n} words with an operation which applies
between point and mark, you can either set the mark at point and then move
over the words, or you can use the command @kbd{Meta-@@} (@code{mark-word})
which does not move point, but sets the mark where @kbd{Meta-f} would move
to.  It can be given arguments just like @kbd{Meta-f}.

@cindex syntax table
  The word commands' understanding of syntax is completely controlled by
the syntax table.  Any character can, for example, be declared to be a word
delimiter.  @xref{Syntax}.

@node Sentences, Paragraphs, Words, Text
@section Sentences
@cindex sentences

  The Emacs commands for manipulating sentences and paragraphs are mostly
on @kbd{Meta-} keys, so as to be like the word-handling commands.

@table @kbd
@item M-a
Move back to the beginning of the sentence (@code{backward-sentence}).
@item M-e
Move forward to the end of the sentence (@code{forward-sentence}).
@item M-k
Kill forward to the end of the sentence (@code{kill-sentence}).
@item C-x @key{DEL}
Kill back to the beginning of the sentence @*(@code{backward-kill-sentence}).
@end table

@kindex M-a
@kindex M-e
@findex backward-sentence
@findex forward-sentence
  The commands @kbd{Meta-a} and @kbd{Meta-e} (@code{backward-sentence} and
@code{forward-sentence}) move to the beginning and end of the current
sentence, respectively.  They were chosen to resemble @kbd{Control-a} and
@kbd{Control-e}, which move to the beginning and end of a line.  Unlike
them, @kbd{Meta-a} and @kbd{Meta-e} if repeated or given numeric arguments
move over successive sentences.  Emacs considers a sentence to end wherever
there is a @samp{.}, @samp{?} or @samp{!} followed by the end of a line or
two spaces, with any number of @samp{)}, @samp{]}, @samp{'}, or @samp{"}
characters allowed in between.  A sentence also begins or ends wherever a
paragraph begins or ends.@refill

  Neither @kbd{M-a} nor @kbd{M-e} moves past the newline or spaces beyond
the sentence edge at which it is stopping.

@kindex M-k
@kindex C-x DEL
@findex kill-sentence
@findex backward-kill-sentence
  Just as @kbd{C-a} and @kbd{C-e} have a kill command, @kbd{C-k}, to go
with them, so @kbd{M-a} and @kbd{M-e} have a corresponding kill command
@kbd{M-k} (@code{kill-sentence}) which kills from point to the end of the
sentence.  With minus one as an argument it kills back to the beginning of
the sentence.  Larger arguments serve as a repeat count.@refill

  There is a special command, @kbd{C-x @key{DEL}}
(@code{backward-kill-sentence}) for killing back to the beginning of a
sentence, because this is useful when you change your mind in the middle of
composing text.@refill

@vindex sentence-end
  The variable @code{sentence-end} controls recognition of the end of a
sentence.  It is a regexp that matches the last few characters of a
sentence, together with the whitespace following the sentence.  Its
normal value is

@example
"[.?!][]\")]*\\($\\|\t\\|  \\)[ \t\n]*"
@end example

@noindent
(Note that @samp{\\} is needed in Lisp syntax to include a @samp{\} in the
string.)

@node Paragraphs, Pages, Sentences, Text
@section Paragraphs
@cindex paragraphs
@kindex M-[
@kindex M-]
@findex backward-paragraph
@findex forward-paragraph

  The Emacs commands for manipulating paragraphs are also @kbd{Meta-}
keys.

@table @kbd
@item M-[
Move back to previous paragraph beginning @*(@code{backward-paragraph}).
@item M-]
Move forward to next paragraph end (@code{forward-paragraph}).
@item M-h
Put point and mark around this or next paragraph (@code{mark-paragraph}).
@end table

  @kbd{Meta-[} moves to the beginning of the current or previous paragraph,
while @kbd{Meta-]} moves to the end of the current or next paragraph.
Blank lines and text formatter command lines separate paragraphs and are
not part of any paragraph.  Also, an indented line starts a new
paragraph.

  In major modes for programs (as opposed to Text mode), paragraphs begin
and end only at blank lines.  This makes the paragraph commands continue to
be useful even though there are no paragraphs per se.

  When there is a fill prefix, then paragraphs are delimited by all lines
which don't start with the fill prefix.  @xref{Filling}.

@kindex M-h
@findex mark-paragraph
  When you wish to operate on a paragraph, you can use the command
@kbd{Meta-h} (@code{mark-paragraph}) to set the region around it.  This
command puts point at the beginning and mark at the end of the paragraph
point was in.  If point is between paragraphs (in a run of blank lines, or
at a boundary), the paragraph following point is surrounded by point and
mark.  If there are blank lines preceding the first line of the paragraph,
one of these blank lines is included in the region.  Thus, for example,
@kbd{M-h C-w} kills the paragraph around or after point.

@vindex paragraph-start
@vindex paragraph-separate
  The precise definition of a paragraph boundary is controlled by the
variables @code{paragraph-separate} and @code{paragraph-start}.  The value
of @code{paragraph-start} is a regexp that should match any line that
either starts or separates paragraphs.  The value of
@code{paragraph-separate} is another regexp that should match only lines
that separate paragraphs without being part of any paragraph.  For example,
normally @code{paragraph-start} is @code{"^[ @t{\}t@t{\}n@t{\}f]"} and
@code{paragraph-separate} is @code{"^[ @t{\}t@t{\}f]*$"}.@refill

  Normally it is desirable for page boundaries to separate paragraphs.
The default values of these variables recognize the usual separator for
pages.

@node Pages, Filling, Paragraphs, Text
@section Pages

@cindex pages
@cindex formfeed
  Files are often thought of as divided into @dfn{pages} by the
@dfn{formfeed} character (ASCII Control-L, octal code 014).  For example,
if a file is printed on a line printer, each page of the file, in this
sense, will start on a new page of paper.  Emacs treats a page-separator
character just like any other character.  It can be inserted with @kbd{C-q
C-l}, or deleted with @key{DEL}.  Thus, you are free to paginate your file,
or not.  However, since pages are often meaningful divisions of the file,
commands are provided to move over them and operate on them.

@c WideCommands
@table @kbd
@item C-x C-p
Put point and mark around this page (or another page) (@code{mark-page}).
@item C-x [
Move point to previous page boundary (@code{backward-page}).
@item C-x ]
Move point to next page boundary (@code{forward-page}).
@item C-x l
Count the lines in this page (@code{count-lines-page}).
@end table

@kindex C-x [
@kindex C-x ]
@findex forward-page
@findex backward-page
  The @kbd{C-x [} (@code{backward-page}) command moves point to immediately
after the previous page delimiter.  If point is already right after a page
delimiter, it skips that one and stops at the previous one.  A numeric
argument serves as a repeat count.  The @kbd{C-x ]} (@code{forward-page})
command moves forward past the next page delimiter.

@kindex C-x C-p
@findex mark-page
  The @kbd{C-x C-p} command (@code{mark-page}) puts point at the beginning
of the current page and the mark at the end.  The page delimiter at the end
is included (the mark follows it).  The page delimiter at the front is
excluded (point follows it).  This command can be followed by @kbd{C-w} to
kill a page which is to be moved elsewhere.  If it is inserted after a page
delimiter, at a place where @kbd{C-x ]} or @kbd{C-x [} would take you, then
the page will be properly delimited before and after once again.

  A numeric argument to @kbd{C-x C-p} is used to specify which page to go
to, relative to the current one.  Zero means the current page.  One means
the next page, and -1 means the previous one.

@kindex C-x l
@findex count-lines-page
  The @kbd{C-x l} command (@code{count-lines-page}) is good for deciding
where to break a page in two.  It prints in the echo area the total number
of lines in the current page, and then divides it up into those preceding
the current line and those following, as in

@example
Page has 96 (72+25) lines
@end example

@noindent
  Notice that the sum is off by one; this is correct if point is not at the
front of a line.

@vindex page-delimiter
  The variable @code{page-delimiter} should have as its value a regexp that
matches the beginning of a line that separates pages.  This is what defines
where pages begin.  The normal value of this variable is @code{"^@t{\}f"},
which matches a formfeed character at the beginning of a line.

@node Filling, Case, Pages, Text
@section Filling Text
@cindex filling

@cindex Auto Fill mode
  With Auto Fill mode, text can be @dfn{filled} (broken up into lines that
fit in a specified width) as you insert it.  If you alter existing text it
may no longer be properly filled; then explicit commands for filling can be
used.

@table @kbd
@item M-x auto-fill-mode
Enable or disable Auto Fill mode.
@item @key{SPC}
@itemx @key{RET}
In Auto Fill mode, break lines when appropriate.
@item M-q
Fill current paragraph (@code{fill-paragraph}).
@item M-g
Fill each paragraph in the region (@code{fill-region}).
@item M-x fill-region-as-paragraph.
Fill the region, considering it as one paragraph.
@item M-x fill-individual-paragraphs
Fill the region, considering each change of indentation as starting a
new paragraph.
@item M-s
Center a line.
@end table

@findex auto-fill-mode
  @kbd{M-x auto-fill-mode} turns Auto Fill mode on if it was off, or off if
it was on.  With a positive numeric argument it always turns Auto Fill mode
on, and with a negative argument always turns it off.  You can see when
Auto Fill mode is in effect by the presence of the word @samp{Fill} in the
mode line, inside the parentheses.  Auto Fill mode is a minor mode, turned
on or off for each buffer individually.  @xref{Minor Modes}.

  In Auto Fill mode, lines are broken automatically at spaces when they get
longer than the desired width.  Line breaking and rearrangement takes place
only when you type @key{SPC} or @key{RET}.  If you wish to insert a space
or newline without permitting line-breaking, type @kbd{C-q @key{SPC}} or
@kbd{C-q @key{LFD}} (recall that a newline is really a linefeed).  Also,
@kbd{C-o} inserts a newline without line breaking.

  Auto Fill mode works well with Lisp mode, because when it makes a new
line in Lisp mode it indents that line with @key{TAB}.  If a line ending in
a comment gets too long, the text of the comment is split into two
comments.

@kindex M-q
@findex fill-paragraph
  Auto Fill mode does not refill entire paragraphs.  It can break lines but
cannot merge lines.  So editing in the middle of a paragraph can result in
a paragraph that is not correctly filled.  To refill a paragraph, use the
command @kbd{Meta-q} (@code{fill-paragraph}).  It causes the paragraph that
point is inside, or the one after point if point is between paragraphs, to
be refilled.  All the line-breaks are removed, and then new ones are
inserted where necessary.  @kbd{M-q} can be undone with @kbd{C-_}.
@xref{Undo}.

@kindex M-g
@findex fill-region
  To refill many paragraphs, use @kbd{M-g} (@code{fill-region}), which
divides the region into paragraphs and fills each of them.

@findex fill-region-as-paragraph
  @kbd{Meta-q} and @kbd{Meta-g} use the same criteria as @kbd{Meta-h} for
finding paragraph boundaries (@pxref{Paragraphs}).  For more control, you
can use @kbd{M-x fill-region-as-paragraph}, which refills everything
between point and mark.  This command recognizes only blank lines as
paragraph separators.@refill

@cindex justification
  A numeric argument to @kbd{M-g} or @kbd{M-q} causes it to @dfn{justify}
the text as well as filling it.  This means that extra spaces are inserted
to make the right margin line up exactly at the fill column.  To remove the
extra spaces, use @kbd{M-q} or @kbd{M-g} with no argument.@refill

@kindex M-s
@cindex centering
@findex center-line
  The command @kbd{Meta-s} (@code{center-line}) centers the current line
within the current fill column.  With an argument, it centers several lines
individually and moves past them.

@vindex fill-column
@vindex default-fill-column
  The maximum line width for filling is in the variable @code{fill-column}.
This variable has a separate value in each buffer; setting it in one buffer
has no effect on any other buffer.  The initial value in a new buffer is
taken from the variable @code{default-fill-column}.

@kindex C-x f
@findex set-fill-column
  The easiest way to set @code{fill-column} is to use the command @kbd{C-x
f} (@code{set-fill-column}).  With no argument, it sets @code{fill-column}
to the current horizontal position of point.  With a numeric argument, it
uses that as the new fill column.

@cindex fill prefix
@kindex C-x .
@findex set-fill-prefix
@vindex fill-prefix
  To fill a paragraph in which each line starts with a special marker
(which might be a few spaces, giving an indented paragraph), use the
@dfn{fill prefix} feature.  The fill prefix is a string which Emacs expects
every line to start with, and which is not included in filling.  It is
stored in the variable @code{fill-prefix}.

  To specify a fill prefix, move to a line that starts with the desired
prefix, put point at the end of the prefix, and give the command
@w{@kbd{C-x .}}@: (@code{set-fill-prefix}).  That's a period after the
@kbd{C-x}.  To turn off the fill prefix, specify an empty prefix: type
@w{@kbd{C-x .}}@: with point at the beginning of a line.@refill

  When a fill prefix is in effect, the fill commands remove the fill prefix
from each line before filling and insert it on each line after filling.  In
Auto Fill mode, @key{SPC} also inserts the fill prefix on any new line.
Lines that do not start with the fill prefix are considered to start
paragraphs, both in @kbd{M-q} and the paragraph commands; this is just
right if you are using paragraphs with hanging indentation (every line
indented except the first one).  Lines which are blank or indented once the
prefix is removed also separate or start paragraphs; this is what you want
if you are writing multi-paragraph comments with a comment delimiter on
each line.

@findex fill-individual-paragraphs
  Another way to use fill prefixes is through @kbd{M-x
fill-individual-paragraphs}.  This function divides the region into groups
of consecutive lines with the same amount and kind of indentation and fills
each group as a paragraph using its indentation as a fill prefix.

  Many users like Auto Fill mode and want to use it in all text files.
Execute the following Lisp expression, perhaps in your init file, to cause
Auto Fill mode to be turned on whenever Text mode is entered:

@lisp
(setq text-mode-hook 'turn-on-auto-fill)
@end lisp

@node Case,, Filling, Text
@section Case Conversion Commands
@cindex case conversion

  Emacs has commands for converting either a single word or any arbitrary
range of text to upper case or to lower case.

@c WideCommands
@table @kbd
@item M-l
Convert following word to lower case (@code{downcase-word}).
@item M-u
Convert following word to upper case (@code{upcase-word}).
@item M-c
Capitalize the following word (@code{capitalize-word}).
@item C-x C-l
Convert region to lower case (@code{downcase-region}).
@item C-x C-u
Convert region to upper case (@code{upcase-region}).
@end table

@kindex M-l
@kindex M-u
@kindex M-c
@cindex words
@findex downcase-word
@findex upcase-word
@findex capitalize-word
  The word conversion commands are the most useful.  @kbd{Meta-l}
(@code{downcase-word}) converts the word after point to lower case, moving
past it.  Thus, repeating @kbd{Meta-l} converts successive words.
@kbd{Meta-u} (@code{upcase-word}) converts to all capitals instead, while
@kbd{Meta-c} (@code{capitalize-word}) puts the first letter of the word
into upper case and the rest into lower case.  All these commands convert
several words at once if given an argument.  They are especially convenient
for converting a large amount of text from all upper case to mixed case,
because you can move through the text using @kbd{M-l}, @kbd{M-u} or
@kbd{M-c} on each word as appropriate, occasionally using @kbd{M-f} instead
to skip a word.

  When given a negative argument, the word case conversion commands apply
to the appropriate number of words before point, but do not move point.
This is convenient when you have just typed a word in the wrong case: you
can give the case conversion command and continue typing.

  If a word case conversion command is given in the middle of a word, it
applies only to the part of the word which follows point.  This is just
like what @kbd{Meta-d} (@code{kill-word}) does.  With a negative argument,
case conversion applies only to the part of the word before point.

@kindex C-x C-l
@kindex C-x C-u
@cindex region
@findex downcase-region
@findex upcase-region
  The other basic case conversion commands are @kbd{C-x C-u}
(@code{upcase-region}) and @kbd{C-x C-l} (@code{downcase-region}), which
convert everything between point and mark to the specified case.  Point and
mark do not move.@refill

@node Programs, Running, Text, Top
@chapter Editing Programs
@cindex Lisp
@cindex C

  Emacs has many commands designed to understand the syntax of programming
languages such as Lisp and C.  These commands can

@itemize @bullet
@item
Move over or kill balanced expressions or @dfn{sexps} (@pxref{Lists}).
@item
Move over or mark top-level balanced expressions (@dfn{defuns}, in Lisp;
functions, in C).
@item
Show how parentheses balance (@pxref{Matching}).
@item
Insert, kill or align comments (@pxref{Comments}).
@item
Follow the usual indentation conventions of the language
(@pxref{Grinding}).
@end itemize

  The commands for words, sentences and paragraphs are very useful in
editing code even though their canonical application is for editing human
language text.  Most symbols contain words (@pxref{Words}); sentences can
be found in strings and comments (@pxref{Sentences}).  Paragraphs per se
are not present in code, but the paragraph commands are useful anyway,
because Lisp mode and C mode define paragraphs to begin and end at blank
lines (@pxref{Paragraphs}).  Judicious use of blank lines to make the
program clearer will also provide interesting chunks of text for the
paragraph commands to work on.

  The selective display feature is useful for looking at the overall
structure of a function (@pxref{Selective Display}).  This feature causes
only the lines that are indented less than a specified amount to appear
on the screen.

@menu
* Program Modes::       Major modes for editing programs.
* Lists::               Expressions with balanced parentheses.
                         There are editing commands to operate on them.
* Defuns::              Each program is made up of separate functions.
                         There are editing commands to operate on them.
* Grinding::            Adjusting indentation to show the nesting.
* Matching::            Insertion of a close-delimiter flashes matching open.
* Comments::            Inserting, illing and aligning comments.
* Balanced Editing::    Inserting two matching parentheses at once, etc.
* Documentation::       Getting documentation of functions you plan to call.
* Change Log::          Maintaining a change history for your program.
* Tags::                Go direct to any function in your program in one
                         command.  Tags remembers which file it is in.
@end menu

@node Program Modes, Lists, Programs, Programs
@section Major Modes for Programming Languages

@cindex Lisp mode
@cindex C mode
@cindex Scheme mode
  Emacs also has major modes for the programming languages Lisp, Scheme (a
variant of Lisp), C and Muddle.  Ideally, a major mode should be
implemented for each programming language that you might want to edit with
Emacs; but often the mode for one language can serve for other
syntactically similar languages.  The language modes that exist are those
that someone decided to take the trouble to write.

  There are several forms of Lisp mode, which differ in the way they
interface to Lisp execution.  @xref{Lisp Modes}.

  Each of the programming language modes defines the @key{TAB} key to run
an indentation function that knows the indentation conventions of that
language and updates the current line's indentation accordingly.  For
example, in C mode @key{TAB} is bound to @code{c-indent-line}.  @key{LFD}
is normally defined to do @key{RET} followed by @key{TAB}; thus, it too
indents in a mode-specific fashion.

@kindex DEL
@findex backward-delete-char-untabify
  In most programming languages, indentation is likely to vary from line to
line.  So the major modes for those languages rebind @key{DEL} to treat a
tab as if it were the equivalent number of spaces (using the command
@code{backward-delete-char-untabify}).  This makes it possible to rub out
indentation one column at a time without worrying whether it is made up of
spaces or tabs.  Use @kbd{C-b C-d} to delete a tab character before point,
in these modes.

  Programming language modes define paragraphs to be separated only by
blank lines, so that the paragraph commands remain useful.  Auto Fill mode,
if enabled in a programming language major mode, indents the new lines
which it creates.

@cindex mode hook
@vindex c-mode-hook
@vindex lisp-mode-hook
@vindex emacs-lisp-mode-hook
@vindex lisp-interaction-mode-hook
@vindex scheme-mode-hook
@vindex muddle-mode-hook
  Turning on a major mode calls a user-supplied function called the
@dfn{mode hook}, which is the value of a Lisp variable.  For example,
turning on C mode calls the value of the variable @code{c-mode-hook} if
that value exists and is non-@code{nil}.  Mode hook variables for other
programming language modes include @code{lisp-mode-hook},
@code{emacs-lisp-mode-hook}, @code{lisp-interaction-mode-hook},
@code{scheme-mode-hook} and @code{muddle-mode-hook}.  The mode hook
function receives no arguments.@refill

@node Lists, Defuns, Program Modes, Programs
@section Lists and Sexps

@c doublewidecommands
@table @kbd
@item C-M-f
Move forward over a sexp (@code{forward-sexp}).
@item C-M-b
Move backward over a sexp (@code{backward-sexp}).
@item C-M-k
Kill sexp forward (@code{kill-sexp}).
@item C-M-u
Move up and backward in list structure (@code{backward-up-list}).
@item C-M-d
Move down and forward in list structure (@code{down-list}).
@item C-M-n
Move forward over a list (@code{forward-list}).
@item C-M-p
Move backward over a list (@code{backward-list}).
@item C-M-t
Transpose expressions (@code{transpose-sexps}).
@item C-M-@@
Put mark after following expression (@code{mark-sexp}).
@end table

@cindex Control-Meta
  By convention, Emacs keys for dealing with balanced expressions are
usually @kbd{Control-Meta-} characters.  They tend to be analogous in
function to their @kbd{Control-} and @kbd{Meta-} equivalents.  These commands
are usually thought of as pertaining to expressions in programming
languages, but can be useful with any language in which some sort of
parentheses exist (including English).

@cindex list
@cindex sexp
@cindex expression
  These commands fall into two classes.  Some deal only with @dfn{lists}
(parenthetical groupings).  They see nothing except parentheses, brackets,
braces, and escape characters that might be used to quote those.  The other
commands deal with expressions or @dfn{sexps} (short for @dfn{s-expression}, the
ancient term for a balanced expression in Lisp).  A parenthetical grouping
is one kind of sexp, but a symbol name is also a sexp, and so is a string.
Numbers and character constants can also be sexps.  The idea is to define
the major mode for a language so that the expressions of that language
count as sexps, as much as possible.

  Except in Lisp-like languages, not all expressions can be sexps.  For
example, C mode does not recognize @samp{foo + bar} as a sexp, even though
it @i{is} a C expression; it recognizes @samp{foo} as one sexp and @samp{bar} as
another, with the @samp{+} as punctuation between them.  This is a
fundamental ambiguity: both @samp{foo + bar} and @samp{foo} are legitimate
choices for the sexp to move over if point is at the @samp{f}.  Note that
@samp{(foo + bar)} is a sexp in C mode.

  Some languages have obscure forms of syntax for expressions that nobody
has bothered to make Emacs understand properly.

@kindex C-M-f
@kindex C-M-b
@findex forward-sexp
@findex backward-sexp
  To move forward over a sexp, use @kbd{C-M-f} (@code{forward-sexp}).  If
the first significant character after point is an opening delimiter
(@samp{(} in Lisp; @samp{(}, @samp{[} or @samp{@{} in C), @kbd{C-M-f}
moves past the matching closing delimiter.  If the character begins a
symbol, string, or number, @kbd{C-M-f} moves over that.  If the character
after point is a closing delimiter, @kbd{C-M-f} just moves past it.  (This
last is not really moving across a sexp; it is an exception which is
included in the definition of @kbd{C-M-f} because it is as useful a
behavior as anyone can think of for that situation.)@refill

  The command @kbd{C-M-b} (@code{backward-sexp}) moves backward over a
sexp.  The detailed rules are like those above for @kbd{C-M-f}, but with
directions reversed.  If there are any prefix characters (singlequote,
backquote and comma, in Lisp) preceding the sexp, @kbd{C-M-b} moves back
over them as well.

  @kbd{C-M-f} or @kbd{C-M-b} with an argument repeats that operation the
specified number of times; with a negative argument, it moves in the
opposite direction.

  The sexp commands move across comments as if they were whitespace, in
languages such as C where the comment-terminator can be recognized.  In
Lisp, and other languages where comments run until the end of a line, it is
very difficult to ignore comments when parsing backwards; therefore, in
such languages the sexp commands treat the text of comments as if it were
code.

@kindex C-M-k
@findex kill-sexp
  Killing a sexp at a time can be done with @kbd{C-M-k} (@code{kill-sexp}).
@kbd{C-M-k} kills the characters that @kbd{C-M-f} would move over.

@kindex C-M-n
@kindex C-M-p
@findex forward-list
@findex backward-list
  The @dfn{list commands} move over lists like the sexp commands but skip
blithely over any number of other kinds of sexps (symbols, strings, etc).
They are @kbd{C-M-n} (@code{forward-list}) and @kbd{C-M-p}
(@code{backward-list}).  The main reason they are useful is that they
usually ignore comments (since the comments usually do not contain any
lists).@refill

@kindex C-M-u
@kindex C-M-d
@findex backward-up-list
@findex down-list
  @kbd{C-M-n} and @kbd{C-M-p} stay at the same level in parentheses, when
that's possible.  To move @i{up} one (or @var{n}) levels, use @kbd{C-M-u}
(@code{backward-up-list}).
@kbd{C-M-u} moves backward up past one unmatched opening delimiter.  A
positive argument serves as a repeat count; a negative argument reverses
direction of motion and also requests repetition, so it moves forward and
up one or more levels.@refill

  To move @i{down} in list structure, use @kbd{C-M-d} (@code{down-list}).  In Lisp mode,
where @samp{(} is the only opening delimiter, this is nearly the same as
searching for a @samp{(}.

@cindex transposition
@kindex C-M-t
@findex transpose-sexps
  A somewhat random-sounding command which is nevertheless easy to use is
@kbd{C-M-t} (@code{transpose-sexps}), which drags the previous sexp across
the next one.  An argument serves as a repeat count, and a negative
argument drags backwards (thus canceling out the effect of @kbd{C-M-t} with
a positive argument).  An argument of zero, rather than doing nothing,
transposes the sexps ending after point and the mark.

@kindex C-M-@@
@findex mark-sexp
  To make the region be the next sexp in the buffer, use @kbd{C-M-@@}
(@code{mark-sexp}) which sets mark at the same place that @kbd{C-M-f} would
move to.  @kbd{C-M-@@} takes arguments like @kbd{C-M-f}.  In particular, a
negative argument is useful for putting the mark at the beginning of the
previous sexp.

  The list and sexp commands' understanding of syntax is completely
controlled by the syntax table.  Any character can, for example, be
declared to be an opening delimiter and act like an open parenthesis.
@xref{Syntax}.

@node Defuns, Grinding, Lists, Programs
@section Defuns
@cindex defuns

  In Emacs, a list at the top level in the buffer is called a @dfn{defun}.
The name derives from the fact that most top level lists in a Lisp file are
instances of the special form @code{defun}, but any top level list counts
as a defun in Emacs parlance regardless of what its contents are, and
regardless of the programming language in use.  For example, in C, the body
of a function definition is a defun.

@c doublewidecommands
@table @kbd
@item C-M-a
Move to beginning of current or preceding defun
(@code{beginning-of-defun}).
@item C-M-e
Move to end of current or following defun (@code{end-of-defun}).
@item C-M-h
Put region around whole current or following defun (@code{mark-defun}).
@end table

@kindex C-M-a
@kindex C-M-e
@kindex C-M-h
@findex beginning-of-defun
@findex end-of-defun
@findex mark-defun
  The commands to move to the beginning and end of the current defun are
@kbd{C-M-a} (@code{beginning-of-defun}) and @kbd{C-M-e} (@code{end-of-defun}).

  If you wish to operate on the current defun, use @kbd{C-M-h}
(@code{mark-defun}) which puts point at the beginning and mark at the end
of the current or next defun.  For example, this is the easiest way to get
ready to move the defun to a different place in the text.  In C mode,
@kbd{C-M-h} runs the function @code{mark-c-function}, which is almost the
same as @code{mark-defun}; the difference is that it backs up over the
argument declarations, function name and returned data type so that the
entire C function is inside the region.

  Emacs assumes that any open-parenthesis found in the leftmost column is
the start of a defun.  Therefore, @b{never put an open-parenthesis at the
left margin in a Lisp file unless it is the start of a top level list.
Never put an open-brace or other opening delimiter at the beginning of a
line of C code unless it starts the body of a function.}  The most likely
problem case is when you want an opening delimiter at the start of a line
inside a string.  To avoid trouble, put an escape character (@samp{\}, in C
and Emacs Lisp, @samp{/} in some other Lisp dialects) before the opening
delimiter.  It will not affect the contents of the string.

  In the remotest past, the original Emacs found defuns by moving upward a
level of parentheses until there were no more levels to go up.  This always
required scanning all the way back to the beginning of the buffer, even for
a small function.  To speed up the operation, Emacs was changed to assume
that any @samp{(} (or other character assigned the syntactic class of
opening-delimiter) at the left margin is the start of a defun.  This
heuristic was nearly always right and avoided the costly scan; however,
it mandated the convention described above.

@node Grinding, Matching, Defuns, Programs
@section Indentation for Programs
@cindex indentation
@cindex grinding

  The best way to keep a program properly indented (``ground'') is to use
Emacs to re-indent it as you change it.  Emacs has commands to indent
properly either a single line, a specified number of lines, or all of the
lines inside a single parenthetical grouping.

@c WideCommands
@table @kbd
@item @key{TAB}
Adjust indentation of current line.
@item @key{LFD}
Equivalent to @key{RET} followed by @key{TAB} (@code{newline-and-indent}).
@item C-M-q
Re-indent all the lines within one list (@code{indent-sexp}).
@item C-u @key{TAB}
Shift an entire list rigidly sideways so that its first line
is properly indented.
@item C-M-\
Re-indent all lines in the region (@code{indent-region}).
@end table

@kindex TAB
@findex c-indent-line
@findex lisp-indent-line
  The basic indentation command is @key{TAB}, which gives the current line
the correct indentation as determined from the previous lines.  The
function that @key{TAB} runs depends on the major mode; it is @code{lisp-indent-line}
in Lisp mode, @code{c-indent-line} in C mode, etc.  These functions
understand different syntaxes for different languages, but they all do
about the same thing.  @key{TAB} in any programming language major mode
inserts or deletes whitespace at the beginning of the current line,
independent of where point is in the line.  If point is inside the
whitespace at the beginning of the line, @key{TAB} leaves it at the end of
that whitespace; otherwise, @key{TAB} leaves point fixed with respect to
the characters around it.

  Use @kbd{C-q @key{TAB}} to insert a tab at point.

@kindex LFD
@findex newline-and-indent
  When entering a large amount of new code, use @key{LFD} (@code{newline-and-indent}),
which is equivalent to a @key{RET} followed by a @key{TAB}.  @key{LFD} creates
a blank line, and then gives it the appropriate indentation.

  @key{TAB} indents the second and following lines of the body of an
parenthetical grouping each under the preceding one; therefore, if you
alter one line's indentation to be nonstandard, the lines below will tend
to follow it.  This is the right behavior in cases where the standard
result of @key{TAB} is unaesthetic.

  Remember that an open-parenthesis, open-brace or other opening delimiter
at the left margin is assumed by Emacs (including the indentation routines)
to be the start of a function.  Therefore, you must never have an opening
delimiter in column zero that is not the beginning of a function; not even
inside a string.  This restriction is vital for making the indentation
commands fast; you must simply accept it.  @xref{Defuns}, for more
information on this.

@subsection Indenting Several Lines

  When you wish to re-indent code which has been altered or moved to a
different level in the list structure, you have several commands available.

@kindex C-M-q
@findex indent-sexp
@findex indent-c-exp
  You can re-indent the contents of a single list by positioning point
before the beginning of it and typing @kbd{C-M-q} (@code{indent-sexp} in
Lisp mode, @code{indent-c-exp} in C mode; also bound to other suitable
functions in other modes).  The indentation of the line the sexp starts on
is not changed; therefore, only the relative indentation within the list,
and not its position, is changed.  To correct the position as well, type a
@key{TAB} before the @kbd{C-M-q}.

@kindex C-u TAB
  If the relative indentation within a list is correct but the indentation
of its beginning is not, go to the line the list begins on and type
@kbd{C-u @key{TAB}}.  When @key{TAB} is given a numeric argument, it moves all the
lines in the grouping starting on the current line sideways the same amount
that the current line moves.  It is clever, though, and does not move lines
that start inside strings, or C preprocessor lines when in C mode.

@kindex C-M-\
@findex indent-region
  Another way to specify the range to be re-indented is with point and
mark.  The command @kbd{C-M-\} (@code{indent-region}) applies @key{TAB} to every line
whose first character is between point and mark.
 
@subsection Customizing Lisp Indentation
@cindex customization

  The indentation pattern for a Lisp expression can depend on the function
called by the expression.  For each Lisp function, you can choose among
several predefined patterns of indentation, or define an arbitrary one with
a Lisp program.

  The standard pattern of indentation is as follows: the second line of the
expression is indented under the first argument, if that is on the same
line as the beginning of the expression; otherwise, the second line is
indented underneath the function name.  Each following line is indented
under the previous line whose nesting depth is the same.

@vindex lisp-indent-offset
  If the variable @code{lisp-indent-offset} is non-@code{nil}, it overrides
the usual indentation pattern for the second line of an expression, so that
such lines are always indented @code{lisp-indent-offset} more columns than
the containing list.

@vindex lisp-body-indention
  The standard pattern is overridded for certain functions.  Functions
whose names start with @code{def} always indent the second line by
@code{lisp-body-indention} extra columns beyond the open-parenthesis
starting the expression.

  The standard pattern can be overridden in various ways for individual
functions, according to the @code{lisp-indent-hook} property of the
function name.  There are four possibilities for this property:

@table @asis
@item @code{nil}
This is the same as no property; the standard indentation pattern is used.
@item @code{defun}
The pattern used for function names that start with @code{def} is used for
this function also.
@item a number, @var{number}
The first @var{number} arguments of the function are
@dfn{distinguished} arguments; the rest are considered the @dfn{body}
of the expression.  A line in the expression is indented according to
whether the first argument on it is distinguished or not.  If the
argument is part of the body, the line is indented @code{lisp-body-indent}
more columns than the open-parenthesis starting the containing
expression.  If the argument is distinguished and is either the first
or second argument, it is indented @i{twice} that many extra columns.
If the argument is distinguished and not the first or second argument,
the standard pattern is followed for that line.
@item a symbol, @var{symbol}
@var{symbol} should be a function name; that function is called to
calculate the indentation of a line within this expression.  The
function receives two arguments:
@table @asis
@item @var{state}
The value returned by @code{parse-partial-sexp} (a Lisp primitive for
indentation and nesting computation) when it parses up to the
beginning of this line.
@item @var{pos}
The position at which the line being indented begins.
@end table
@noindent
It should return either a number, which is the number of columns of
indentation for that line, or a list whose car is such a number.  The
difference between returning a number and returning a list is that a
number says that all following lines at the same nesting level should
be indented just like this one; a list says that following lines might
call for different indentations.  This makes a difference when the
indentation is being computed by @kbd{C-M-q}; if the value is a
number, @kbd{C-M-q} need not recalculate indentation for the following
lines until the end of the list.
@end table

@subsection Customizing C Indentation

  C does not have anything analogous to particular function names for which
special forms of indentation are desirable.  However, it has a different
need for customization facilities: many different styles of C indentation
are in common use.  There are six variables you can set to control the
style that Emacs C mode will use.

@vindex c-indent-level
  The variable @code{c-indent-level} controls the indentation for C
statements with respect to the surrounding block.  In the example

@example
    @{
      foo ();
@end example

@noindent
the difference in indentation between the lines is @code{c-indent-level}.
Its standard value is 2.

If the open-brace beginning the compound statement is not at the beginning
of its line, the @code{c-indent-level} is added to the indentation of the
line, not the column of the open-brace.  For example,

@example
if (losing) @{
  do_this ();
@end example

@noindent
One popular indentation style is that which results from setting
@code{c-indent-level} to 8 and putting open-braces at the end of a line in
this way.

@vindex c-brace-imaginary-offset
  In fact, the value of the variable @code{c-brace-imaginary-offset} is
also added to the indentation of such a statement.  Normally this variable
is zero.  Think of this variable as the imaginary position of the open
brace, relative to the first nonblank character on the line.  By setting
this variable to 4 and @code{c-indent-level} to 0, you can get this style:

@example
if (x == y) @{
    do_it ();
    @}
@end example

  When @code{c-indent-level} is zero, the statements inside most braces
will line up right under the open brace.  But there is an exception made
for braces in column zero, such as surrounding a function's body.  The
statements just inside it do not go at column zero.  Instead,
@code{c-brace-offset} and @code{c-continued-statement-offset} (see below)
are added to produce a typical offset between brace levels, and the
statements are indented that far.

@vindex c-continued-statement-offset
  @code{c-continued-statement-offset} controls the extra indentation for a
line that starts within a statement (but not within parentheses or
brackets).  These lines are usually statements that are within other
statements, such as the then-clauses of @code{if} statements and the bodies
of @code{while} statements.  This parameter is the difference in
indentation between the two lines in

@example
if (x == y)
  do_it ();
@end example

@noindent
Its standard value is 2.  Some popular indentation styles correspond to a
value of zero for @code{c-continued-statement-offset}.

@vindex c-brace-offset
  @code{c-brace-offset} is the extra indentation given to a line that
starts with an open-brace.  Its standard value is zero;
compare

@example
if (x == y)
  @{
@end example

@noindent
with

@example
if (x == y)
  do_it ();
@end example

@noindent
if @code{c-brace-offset} were set to 4, the first example would become

@example
if (x == y)
      @{
@end example

@vindex c-argdecl-indent
  @code{c-argdecl-indent} controls the indentation of declarations of the
arguments of a C function.  It is absolute: argument declarations receive
exactly @code{c-argdecl-indent} spaces.  The standard value is 5, resulting
in code like this:

@example
char *
index (string, char)
     char *string;
     int char;
@end example

@vindex c-label-offset
  @code{c-label-offset} is the extra indentation given to a line that
contains a label, a case statement, or a default statement.  Its standard
value is -2, resulting in code like this

@example
switch (c)
  @{
  case 'x':
@end example

@noindent
If @code{c-label-offset} were zero, the same code would be indented as

@example
switch (c)
  @{
    case 'x':
@end example

@noindent
This example assumes that the other variables above also have their
standard values.

  I strongly recommend that you try out the indentation style produced by
the standard settings of these variables, together with putting open braces
on separate lines.  You can see how it looks in all the C source files of
GNU Emacs.

@vindex c-auto-newline
  One other variable, @code{c-auto-newline}, does not affect the style of
indentation that is used, but makes insertion of certain characters insert
newlines automatically.  When this variable is non-@code{nil}, newlines are
inserted both before and after braces that you insert, and after colons and
semicolons.  Correct C indentation is done on all the lines that are made
this way.

@node Matching, Comments, Grinding, Programs
@section Automatic Display Of Matching Parentheses
@cindex matching parentheses
@cindex parentheses

  The Emacs parenthesis-matching feature is designed to show automatically
how parentheses match in the text.  When ever a self-inserting character
that is a closing delimiter is typed, the cursor moves momentarily to the
location of the matching opening delimiter, provided that is on the screen.
If it is not on the screen, some text starting with that opening delimiter
is displayed in the echo area.  Either way, you can tell what grouping is
being closed off.

  In Lisp, automatic matching applies only to parentheses.  In C, it
applies to braces and brackets too.  Emacs knows which characters to regard
as matching delimiters based on the syntax table, which is set by the major
mode.  @xref{Syntax}.

  If the opening delimiter and closing delimiter are mismatched---such as,
in @samp{[x)}---a warning message is displayed in the echo area.  The
correct matches are specified in the syntax table.

@vindex blink-matching-paren
@vindex blink-matching-paren-distance
  Two variables control parenthesis match display.  @code{blink-matching-paren}
turns the feature on or off; @code{nil} turns it off, but the default is
@code{t} to turn match display on.  @code{blink-matching-paren-distance}
specifies how many characters back to search to find the matching opening
delimiter.  If the match is not found in that far, scanning stops, and
nothing is displayed.  This is to prevent scanning for the matching
delimiter from wasting lots of time when there is no match.

@node Comments, Balanced Editing, Matching, Programs
@section Manipulating Comments
@cindex comments
@kindex M-;
@cindex indentation
@findex indent-for-comment

  The comment commands insert, kill and align comments.  There are also
commands for moving through existing code and inserting comments.

@c WideCommands
@table @kbd
@item M-;
Insert or align comment (@code{indent-for-comment}).
@item C-x ;
Set comment column (@code{set-comment-column}).
@item C-u - C-x ;
Kill comment on current line (@code{kill-comment}).  With region, kill
comments in region.
@item M-@key{LFD}
Like @key{RET} followed by inserting and aligning the @code{comment-start}
string (@code{indent-new-comment-line}).
@end table

  The command that creates a comment is @kbd{Meta-;} (@code{indent-for-comment}).
If there is no comment already on the line, a new comment is created,
aligned at a specific column called the @dfn{comment column}.  The comment
is created by inserting the string Emacs thinks comments should start with
(the value of @code{comment-start}; see below).  Point is left after that
string.  If the text of the line extends past the comment column, then the
indentation is done to a suitable boundary (usually, at least one space is
inserted).  If the major mode has specified a string to terminate comments,
that is inserted after point, to keep the syntax valid.

  @kbd{Meta-;} can also be used to align an existing comment.  If a line
already contains the string that starts comments, then @kbd{M-;} just moves
point after it and re-indents it to the right column.  Exception: comments
starting in column 0 are not moved.  Also, in particular modes, there are
special rules for indenting certain kinds of comments in certain contexts.

  For example, in Lisp code, comments which start with two semicolons are
indented as if they were lines of code, instead of at the comment column.
Comments which start with three semicolons are supposed to start at the
left margin.  Emacs understands these conventions by indenting a
double-semicolon comment using @key{TAB}, and by not changing the
indentation of a triple-semicolon comment at all.

  In C code, a comment preceded on its line by nothing but whitespace
is indented like a line of code.

  Even when an existing comment is properly aligned, @kbd{M-;} is still
useful for moving directly to the start of the comment.

@kindex C-u - C-x ;
@findex kill-comment
  @kbd{C-u - C-x ;} (@code{kill-comment}) kills the comment on the current line,
if there is one.  The indentation before the start of the comment is killed
as well.  If there does not appear to be a comment in the line, nothing is
done.  To reinsert the comment on another line, move to the end of that
line, do @kbd{C-y}, and then do @kbd{M-;} to realign it.  Note that
@kbd{C-u - C-x ;} is not a distinct key; it is @kbd{C-x ;} (@code{set-comment-column})
with a negative argument.  That command is programmed so that when it
receives a negative argument it calls @code{kill-comment}.  However,
@code{kill-comment} is a valid command which you could bind directly to a
key if you wanted to.

@subsection Multiple Lines of Comments

  If you wish to align a large number of comments, give @kbd{Meta-;} an
argument, and it indents what comments exist on that many lines, creating
none.  Point is left after the last line processed (unlike the no-argument
case).

@kindex M-LFD
@cindex blank lines
@cindex Auto Fill mode
@findex indent-new-comment-line
  If you are typing a comment and find that you wish to continue it on
another line, you can use the command @kbd{Meta-@key{LFD}} (@code{indent-new-comment-line}),
which terminates the comment you are typing, creates a new blank line
afterward, and begins a new comment indented under the old one.  When Auto
Fill mode is on, going past the fill column while typing a comment causes
the comment to be continued in just this fashion.  If point is not at the
end of the line when @kbd{M-@key{LFD}} is typed, the text on the rest of
the line becomes part of the new comment line.

@subsection Options Controlling Comments

@vindex comment-column
@kindex C-x ;
@findex set-comment-column
  The comment column is stored in the variable @code{comment-column}.  You
can set it to a number explicitly.  Alternatively, the command @kbd{C-x ;}
(@code{set-comment-column}) sets the comment column to the column point is
at.  @kbd{C-u C-x ;} sets the comment column to match the last comment
before point in the buffer, and then does a @kbd{Meta-;} to align the
current line's comment under the previous one.  Note that @kbd{C-u - C-x ;}
runs the function @code{kill-comment} as described above.

@cindex major modes
  Many major modes supply default local values for the comment column.
@xref{Locals}.

@vindex comment-start-skip
  The comment commands recognize comments based on the regular expression
that is the value of the variable @code{comment-start-skip}.  This regexp
should not match the null string.  It may match more than the comment
starting delimiter in the strictest sense of the word; for example, in C
mode the value of the variable is @code{@t{"/\\*+ *"}}, which matches extra
stars and spaces after the @samp{/*} itself.  (Note that @samp{\\} is
needed in Lisp syntax to include a @samp{\} in the string, which is needed
to deny the first star its special meaning in regexp syntax; @pxref{Regexps})

@vindex comment-start
@vindex comment-end
  When a comment command makes a new comment, it inserts the value of
@code{comment-start} to begin it.  The value of @code{comment-end} is
inserted after point, so that it will follow the text that you will insert
into the comment.  In C mode, @code{comment-start} has the value
@code{"/* "} and @code{comment-end} has the value @code{" */"}.

@vindex comment-multi-line
  @code{comment-multi-line} controls how @kbd{M-@key{LFD}} (@code{indent-new-comment-line})
behaves when used inside a comment.  If @code{comment-multi-line} is
@code{nil}, as it normally is, then the comment on the starting line is
terminated and a new comment is started on the new following line.  If
@code{comment-multi-line} is not @code{nil}, then the new following line is
set up as part of the same comment that was found on the starting line.
This is done by not inserting a terminator on the old line, and not
inserting a starter on the new line.  In languages where multi-line comments
work, the choice of value for this variable is a matter of taste.

@vindex comment-indent-hook
  The variable @code{comment-indent-hook} should contain a function that
will be called to compute the indentation for a newly inserted comment or
for aligning an existing comment.  It is set differently by various major
modes.  The function is called with no arguments, but with point at the
beginning of the comment, or at the end of a line if a new comment is to be
inserted.  It should return the column in which the comment ought to start.
For example, in Lisp mode, the indent hook function bases its decision
on how many semicolons begin an existing comment, and on the code in the
preceding lines.

@node Balanced Editing, Documentation, Comments, Programs
@section Editing Without Unbalanced Parentheses

@table @kbd
@item M-(
Put parentheses around next sexp(s) (@code{insert-parentheses}).
@item M-)
Move past next close parenthesis and re-indent
(@code{move-over-close-and-reindent}).
@end table

@kindex M-(
@kindex M-)
@findex insert-parentheses
@findex move-over-close-and-reindent
  The commands @kbd{M-(} (@code{insert-parentheses}) and @kbd{M-)}
(@code{move-over-close-and-reindent}) are designed to facilitate a style of
editing which keeps parentheses balanced at all times.  @kbd{M-(} inserts a
pair of parentheses, either together as in @samp{()}, or, if given an
argument, around the next several sexps, and leaves point after the open
parenthesis.  Instead of typing @kbd{( F O O )}, you can type @kbd{M-( F O
O}, which has the same effect except for leaving the cursor before the
close parenthesis.  Then you would type @kbd{M-)}, which moves past the
close parenthesis, deleting any indentation preceding it (in this example
there is none), and indenting with @key{LFD} after it.

@node Documentation, Change Log, Balanced Editing, Programs
@section Documentation Commands

@kindex C-h f
@findex describe-function
@kindex C-h v
@findex describe-variable
  As you edit Lisp code to be run in Emacs, the commands @kbd{C-h f}
(@code{describe-function}) and @kbd{C-h v} (@code{describe-variable}) can
be used to print documentation of functions and variables that you want to
call.  These commands use the minibuffer to read the name of a function or
variable to document, and display the documentation in a window.

  For extra convenience, these commands provide default arguments based on
the code in the neighborhood of point.  @kbd{C-h f} sets the default to the
function called in the innermost list containing point.  @kbd{C-h v} uses
the symbol name around or adjacent to point as its default.

@findex manual-entry
  Documentation on Unix commands, system calls and libraries can be
obtained with the @kbd{M-x manual-entry} command.  This reads a topic as an
argument, and displays the text on that topic from the Unix manual.
@code{manual-entry} always searches all 8 sections of the manual, and
concatenates all the entries that are found.  For example, the topic
@samp{termcap} finds the description of the termcap library from section 3,
followed by the description of the termcap data base from section 5.

@node Change Log, Tags, Documentation, Programs
@section Change Logs

@cindex change log
@findex add-change-log-entry
  The Emacs command @kbd{M-x add-change-log-entry} helps you keep a record
of when and why you have changed a program.  It assumes that you have a
file in which you write a chronological sequence of entries describing
individual changes.  The default is to store the change entries in a file
called @file{ChangeLog} in the same directory as the file you are editing.
The same @file{ChangeLog} file therefore records changes for all the files
in the directory.

  A change log entry starts with a header line that contains your name and
the current date.  Aside from these header lines, every line in the change
log starts with a tab.  One entry can describe several changes; each change
starts with a line starting with a tab and a star.  @kbd{M-x add-change-log-entry}
visits the change log file and creates a new entry unless the most recent
entry is for today's date and your name.  In either case, it adds a new
line to start the description of another change just after the header line
of the entry.  When @kbd{M-x add-change-log-entry} is finished, all is
prepared for you to edit in the description of what you changed and how.
You must then save the change log file yourself.

  The change log file is always visited in Indented Text mode, which means
that @key{LFD} and auto-filling indent each new line like the previous
line.  This is convenient for entering the contents of an entry, which must
all be indented.  @xref{Text Mode}.

  Here is an example of the formatting conventions used in the change log
for Emacs:

@smallexample
Wed Jun 26 19:29:32 1985  Richard M. Stallman  (rms at mit-prep)

        * xdisp.c (try_window_id):
        If C-k is done at end of next-to-last line,
        this fn updates window_end_vpos and cannot leave
        window_end_pos nonnegative (it is zero, in fact).
        If display is preempted before lines are output,
        this is inconsistent.  Fix by setting
        blank_end_of_window to nonzero.

Tue Jun 25 05:25:33 1985  Richard M. Stallman  (rms at mit-prep)

        * cmds.c (Fnewline):
        Call the auto fill hook if appropriate.

        * xdisp.c (try_window_id):
        If point is found by compute_motion after xp, record that
        permanently.  If display_text_line sets point position wrong
        (case where line is killed, point is at eob and that line is
        not displayed), detect and set it again in final compute_motion.
@end smallexample

@node Tags,, Change Log, Programs
@section Tag Tables
@cindex tag table

  A @dfn{tag table} is a description of how a multi-file program is broken
up into files.  It lists the names of the component files and the names and
positions of the functions in each file.  Grouping the related files makes
it possible to search or replace through all the files with one command.
Recording the function names and positions makes possible the @kbd{Meta-.}
command which you can use to find the definition of a function without
having to know which of the files it is in.

  Tag tables are stored in files called @dfn{tag table files}.  The
conventional name for a tag table file is @code{TAGS}.

  Each entry in the tag table records the name of one tag, the name of the
file that the tag is defined in (implicitly), and the position in that file
of the tag's definition.

  Just what names from the described files are recorded in the tag table
depends on the programming language of the described file.  They normally
include all functions and subroutines, and may also include global
variables, data types, and anything else convenient.  In any case, each
name recorded is called a @dfn{tag}.

  In Lisp code, any function defined with @code{defun}, any variable
defined with @code{defvar} or @code{defconst}, and in general the first
argument of any expression that starts with @samp{(def} in column zero, is
a tag.

  In C code, any C function is a tag, and so is any typedef if @code{-t} is
specified when the tag table is constructed.

  In Fortran code, functions and subroutines are tags.

@subsection Creating Tag Tables
@cindex etags program

  The @code{etags} program is used to create a tag table file.  It knows
the syntax of C, Fortran and Emacs Lisp.  To use @code{etags}, type

@example
etags @var{inputfiles}@dots{}
@end example

@noindent
as a shell command.  It reads the specified files and writes a tag table
named @code{TAGS} in the current working directory.  @code{etags}
recognizes the language used in an input file based on its file name and
contents; there are no switches for specifying the language.  The @code{-t}
switch tells @code{etags} to record typedefs in C code as tags.

  If the tag table data become outdated due to changes in the files
described in the table, the way to update the tag table is the same way it
was made in the first place.  It is not necessary to do this often.

  If the tag table fails to record a tag, or records it for the wrong file,
then Emacs cannot possibly find its definition.  However, if the position
recorded in the tag table becomes a little bit wrong (due to some editing
in the file that the tag definition is in), the only consequence is to slow
down finding the tag slightly.  Even if the stored position is very wrong,
Emacs will still find the tag, but it must search the entire file for it.

  So you should update a tag table when you define new tags that you want
to have listed, or when you move tag definitions from one file to another,
or when changes become substantial.  Normally there is no need to update
the tag table after each edit, or even every day.

@subsection Selecting a Tag Table

@vindex tags-file-name
@findex visit-tags-table
  Emacs has at any time one @dfn{selected} tag table, and all the commands
for working with tag tables use the selected one.  To select a tag table,
type @kbd{M-x visit-tags-table}, which reads the tag table file name as an
argument.  The name @code{TAGS} in the default directory is used as the
default file name.

  All this command does is store the file name in the variable
@code{tags-file-name}.  Emacs does not actually read in the tag table
contents until you try to use them.  Setting this variable yourself is just
as good as using @code{visit-tags-table}.  The variable's initial value is
@code{nil}; this value tells all the commands for working with tag tables
that they must ask for a tag table file name to use.

@subsection Finding a Tag

  The most important thing that a tag table enables you to do is to find
the definition of a specific tag.

@table @kbd
@item M-.@: @var{tag}
Find first definition of @var{tag} (@code{find-tag}).
@item C-u M-.
Find next alternate definition of last tag specified.
@item C-x 4 .@: @var{tag}
Find first definition of @var{tag}, but display it in another window
(@code{find-tag-other-window}).
@end table

@kindex M-.
@findex find-tag
  @kbd{M-.}@: (@code{find-tag}) is the command to find the definition of a
specified tag.  It searches through the tag table for that tag, as a
string, and then uses the tag table info to determine the file that the
definition is in and the approximate character position in the file of the
definition.  Then @code{find-tag} visits that file, moves point to the
approximate character position, and starts searching ever-increasing
distances away for the the text that should appear at the beginning of the
definition.

  If an empty argument is given (just type @key{RET}), the sexp in the
buffer before or around point is used as the name of the tag to find.
@xref{Lists}, for info on sexps.

  The argument to @code{find-tag} need not be the whole tag name; it can be
a substring of a tag name.  However, there can be many tag names containing
the substring you specify.  Since @code{find-tag} works by searching the
text of the tag table, it finds the first tag in the table that the
specified substring appears in.  The way to find other tags that match the
substring is to give @code{find-tag} a numeric argument, as in @kbd{M-0 M-.};
this does not read a tag name, but continues searching the tag table's text
for another tag containing the same substring last used.

@kindex C-x 4 .
@findex find-tag-other-window
  Like most commands that can switch buffers, @code{find-tag} has another
similar command that displays the new buffer in another window.  @kbd{C-x 4
.}@: invokes the function @code{find-tag-other-window}.

@subsection Searching and Replacing with Tag Tables

  The commands in this section visit and search all the files listed in the
selected tag table, one by one.

@table @kbd
@item M-x tags-search
Search for the specified regexp through the files in the selected tag
table.
@item M-x tags-query-replace
Perform a @code{query-replace} on each file in the selected tag table.
@item M-,
Restart one of the commands above, from the current location of point
(@code{tags-loop-continue}).
@end table

@findex tags-search
  @kbd{M-x tags-search} reads a regexp using the minibuffer, then visits
the files of the selected tag table one by one, and searches through each
one for that regexp.  As soon as an occurrence is found, @code{tags-search}
returns.

@kindex M-,
@findex tags-loop-continue
  Having found one match, you probably want to find all the rest.  To find
one more match, type @kbd{M-,} (@code{tags-loop-continue}) to resume the
@code{tags-search}.  This searches the rest of the current buffer, followed
by the remaining files of the tag table.

@findex tags-query-replace
  @kbd{M-x tags-query-replace} performs a single @code{query-replace}
through all the files in the tag table.  It reads a string to search for
and a string to replace with, just like ordinary @kbd{M-x query-replace}.
It searches much like @kbd{M-x tags-search} but repeatedly, processing
matches according to your input.  @xref{Replace}, for more information on
@code{query-replace}.@refill

  It is possible to get through all the files in the tag table with a
single invocation of @kbd{M-x tags-query-replace}.  But since any
unrecognized character causes the command to exit, you may need to continue
where you left off.  @kbd{M-,} can be used for this.  It resumes the last
tags search or replace command that you did.

  It may have struck you that @code{tags-search} is a lot like @code{grep}.
You can also run @code{grep} itself as an inferior of Emacs and have Emacs
show you the matching lines one by one.  This works mostly the same as
running a compilation and having Emacs show you where the errors were.
@xref{Compilation}.

@subsection Stepping Through a Tag Table
@findex next-file

  If you wish to process all the files in the selected tag table, but
@kbd{M-x tags-search} and @kbd{M-x tags-query-replace} in particular are not what
you want, you can use @kbd{M-x next-file}.

@table @kbd
@item C-u M-x next-file
With a numeric argument, regardless of its value, visit the first
file in the tag table, and prepare to advance sequentially by files.
@item M-x next-file
Visit the next file in the selected tag table.
@end table

@subsection Tag Table Inquiries

@table @kbd
@item M-x list-tags
Display a list of the tags defined in a specific program file.
@item M-x tags-apropos
Display a list of all tags matching a specified regexp.
@end table

@findex list-tags
  @kbd{M-x list-tags} reads the name of one of the files described by the
selected tag table, and displays a list of all the tags defined in that
file.  The ``file name'' argument is really just a string to compare
against the names recorded in the tag table; it is read as a string rather
than as a file name.  Therefore, completion and defaulting are not
available, and you must enter the string the same way it appears in the tag
table.  Do not include a directory as part of the file name unless the file
name recorded in the tag table includes a directory.

@findex tags-apropos
  @kbd{M-x tags-apropos} is like @code{apropos} for tags.  It reads a regexp,
then finds all the tags in the selected tag table whose entries match that
regexp, and displays the tag names found.

@node Running, Abbrevs, Programs, Top
@chapter Compiling and Testing Programs

  The previous chapter discusses the Emacs commands that are useful for
making changes in programs.  This chapter deals with commands that assist
in the larger process of developing and maintaining programs.

@menu
* Compilation::        Compiling programs in languages other than Lisp
                        (C, Pascal, etc.)
* Modes: Lisp Modes.   Various modes for editing Lisp programs, with
                       different facilities for running the Lisp programs.
* Libraries: Lisp Libraries.      Creating Lisp programs to run in Emacs.
* Interaction: Lisp Interaction.  Executing Lisp in an Emacs buffer.
* Eval: Lisp Eval.     Executing a single Lisp expression in Emacs.
* Debug: Lisp Debug.   Debugging Lisp programs running in Emacs.
* External Lisp::      Communicating through Emacs with a separate Lisp.
@end menu

@node Compilation, Lisp Modes, Running, Running
@section Running `make', or Compilers Generally
@cindex inferior process
@cindex make
@cindex compilation errors
@cindex error log

  Emacs can run compilers for noninteractive languages such as C and
Fortran as inferior processes, feeding the error log into an Emacs buffer.
It can also parse the error messages and visit the files in which errors
are found, moving point right to the line where the error occurred.

@table @kbd
@item M-x compile
Run a compiler asynchronously under Emacs, with error messages to
@samp{*compilation*} buffer.
@item M-x grep
Run @code{grep} asynchronously under Emacs, with matching lines
listed in the @samp{*compilation*} buffer.
@item M-x kill-compiler
@itemx M-x kill-grep
Kill the running compilation or @code{grep} subprocess.
@item C-x `
Visit the locus of the next compiler error message or @code{grep} match.
@end table

@findex compile
  To run @code{make} or another compiler, do @kbd{M-x compile}.  This command
reads a shell command line using the minibuffer, and then executes the
specified command line in an inferior shell with output going to the buffer
named @samp{*compilation*}.  The current buffer's default directory is used
as the working directory for the execution of the command; normally,
therefore, the makefile comes from this directory.

@vindex compile-command
  When the shell command line is read, the minibuffer appears containing a
default command line, which is the command you used the last time you did
@kbd{M-x compile}.  If you type just @key{RET}, the same command line is used
again.  The first @kbd{M-x compile} provides @code{make -k} as the default.
The default is taken from the variable @code{compile-command}; if the
appropriate compilation command for a file is something other than
@code{make -k}, it can be useful to have the file specify a local value for
@code{compile-command} (@pxref{File Variables}).

  Starting a compilation causes the buffer @samp{*compilation*} to be
displayed in another window but not selected.  Its mode line tells you
whether compilation is finished, with the word @samp{run} or @samp{exit} inside
the parentheses.  You do not have to keep this buffer visible; compilation
continues in any case.

@findex kill-compilation
  To kill the compilation process, do @kbd{M-x kill-compilation}.  You will
see that the mode line of the @samp{*compilation*} buffer changes to say
@samp{signal} instead of @samp{run}.  Starting a new compilation also kills
any running compilation, as only one can exist at any time.  However, this
requires confirmation before actually killing a compilation that is
running.@refill

@kindex C-x `
@findex next-error
  To parse the compiler error messages, type @kbd{C-x `} (@code{next-error}).  The
character following the @kbd{C-x} is the grave accent, not the single
quote.  This command displays the buffer @samp{*compilation*} in one window
and the buffer in which the next error occurred in another window.  Point
in that buffer is moved to the line where the error was found.  The
corresponding error message is scrolled to the top of the window in which
@samp{*compilation*} is displayed.

  The first time @kbd{C-x `} is used, after the start of a compilation, it
parses all the error messages, visits all the files that have error
messages, and makes markers pointing at the lines that the error messages
refer to.  Then it moves to the first error message location.  Subsequent
uses of @kbd{C-x `} advance down the data set up by the first use.  When
the preparsed error messages are exhausted, the next @kbd{C-x `} checks for
any more error messages that have come in; this is useful if you start
editing the compiler errors while the compilation is still going on.  If no
more error messages have come in, @kbd{C-x `} reports an error.

  @kbd{C-u C-x `} discards the preparsed error message data and parses the
@samp{*compilation*} buffer over again, then displaying the first error.
This way, you can process the same set of errors again.

  Instead of running a compiler, you can run @code{grep} and see the lines
on which matches were found.  To do this, type @kbd{M-x grep} with an argument
line that contains the same arguments you would give @code{grep} when running
it normally: a @code{grep}-style regexp (usually in doublequotes to quote
the shell's special characters) followed by filenames which may use wildcards.
The output from @code{grep} goes in the @samp{*compilation*} buffer and the
lines that matched can be found with @kbd{C-x `} as if they were compilation
errors.

  Note: a shell is used to run the compile command, but the shell is told
that it should be noninteractive.  This means in particular that the shell
starts up with no prompt.  If you find your usual shell prompt making an
unsightly appearance in the @samp{*compilation*} buffer, it means you have
made a mistake in your shell's init file (@file{.cshrc} or @file{.shrc} or
@dots{}) by setting the prompt unconditionally.  The shell init file should
set the prompt only if there already is a prompt.  In @code{csh}, here is
how to do it:

@example
if ($?prompt) set prompt = ... 
@end example

@node Lisp Modes, Lisp Libraries, Compilation, Running
@section Major Modes for Lisp

  Emacs has four different major modes for Lisp.  They are the same in
terms of editing commands, but differ in the commands for executing Lisp
expressions.

@table @asis
@item Emacs-Lisp mode
The mode for editing source files of programs to run in Emacs Lisp.
This mode defines @kbd{C-M-x} to evaluate the current defun.
@xref{Lisp Libraries}.
@item Lisp Interaction mode
The mode for an interactive session with Emacs Lisp.  It defines
@key{LFD} to evaluate the sexp before point and insert its value in the
buffer.  @xref{Lisp Interaction}.
@item Lisp mode
The mode for editing source files of programs that run in Lisps other
than Emacs Lisp.  This mode defines @kbd{C-M-x} to send the current defun
to an inferior Lisp process.  @xref{External Lisp}.
@item Inferior Lisp mode
The mode for an interactive session with an inferior Lisp process.
This mode combines the special features of Lisp mode and Shell mode
(@pxref{Shell}).
@end table

@node Lisp Libraries, Lisp Eval, Lisp Modes, Running
@section Libraries of Lisp Code for Emacs
@cindex libraries
@cindex loading Lisp code

  Lisp code for Emacs editing commands is stored in files whose names
conventionally end in @file{.el}.  This ending tells Emacs to edit them in
Emacs-Lisp mode, so you can use the @kbd{C-M-x} command described in the
following section to install changed functions.

@findex load
  Only the maintainers of such a file will want to edit its contents or
evaluate text from it, but every user must be able to load the file.  This
is done with @kbd{M-x load}.

  @kbd{M-x load} reads a file name using the minibuffer and executes the
specified file as Lisp code.  But it has an important difference from all
other Emacs commands that read file names: it searches a sequence of
directories, and tries three file names in each directory.

  Because normally one does not want the argument to @code{load} to contain
an explicit directory name, the usual mechanism for reading file names
cannot be used, and therefore file name completion is not available.
(Which directory would it complete in, anyway?)

  The argument you give to @kbd{M-x load} is usually not the full file
name.  Usually you omit the @file{.el} that the file name ends in.
@kbd{M-x load} tries three file names in each directory: first, the name
you specified; second, that name with @file{.elc} appended; third, that
name with @file{.el} appended.  A @file{.elc} file would be the result of
compiling the Lisp file into byte code; it is loaded if possible in
preference to the Lisp file itself because the compiled file will load and
run faster.

@vindex load-path
  The sequence of directories searched by @kbd{M-x load} is specified by
the variable @code{load-path}, a list of strings that are directory names.
Normally the first element of this list is @code{nil}, which means to
search the current default directory at that point; the remaining elements
are the names of the directories in which the Lisp code of Emacs itself is
stored.  Therefore, you can load an installed Emacs library without having
to specify a directory name.

  Often you do not have to run the @code{load} command yourself, because
the commands in a library have permanent definitions to @dfn{autoload}
that library.  Running any of those commands causes @code{load} to be
called to load the library; this replaces the autoload definitions with
the real ones from the library.

  If autoloading a file does not finish, either because of an error or
because of a @kbd{C-g} quit, all function definitions made by the file are
undone automatically.  So are any calls to @code{provide}.  As a consequence,
if you use one of the autoloadable commands again, the entire file will be
loaded a second time.  This prevents problems where the command is no
longer autoloading but it works wrong because not all the file was loaded.
Function definitions are undone only for autoloading; explicit calls to
@code{load} do not undo anything if loading is not completed.

@findex byte-compile-file
  The way to make a byte-code compiled file from an Emacs-Lisp source file
is with @kbd{M-x byte-compile-file}.  The default argument for this
function is the file visited in the current buffer.  It reads the specified
file, compiles it into byte code, and writes an output file whose name is
made by appending @file{c} to the input file name.  Thus, the file
@file{rmail.el} would be compiled into @file{rmail.elc}.

@findex byte-recompile-directory
  To recompile the changed Lisp files in a directory, use @kbd{M-x
byte-recompile-directory}.  Specify just the directory name as an argument.
Each @file{.el} file that has been byte-compiled before is byte-compiled
again if it has changed since the previous compilation.  A numeric argument
to this command tells it to offer to compile each @file{.el} file that has
not already been compiled.  You must answer @kbd{y} or @kbd{n} to each
offer.

@findex batch-byte-compile
  Emacs can be invoked noninteractively from the shell to do byte compilation
with the aid of the function @code{batch-byte-compile}.  In this case,
the files to be compiled are specified with command-line arguments.
Use a shell command of the form

@example
emacs -batch -f batch-byte-compile @var{files}...
@end example

  Directory names may also be given as arguments;
@code{byte-recompile-directory} is invoked (in effect) on each such directory.
@code{batch-byte-compile} uses all the remaining command-line arguments as
file or directory names, then kills the Emacs process.

@cindex mocklisp
  GNU Emacs can run Mocklisp files by converting them to Emacs Lisp first.
To convert a Mocklisp file, visit it and then type @kbd{M-x
convert-mocklisp-buffer}.  Then save the resulting buffer of Lisp file in a
file whose name ends in @file{.el} and use the new file as a Lisp library.

@node Lisp Eval, Lisp Debug, Lisp Libraries, Running
@section Evaluating Emacs-Lisp Expressions
@cindex Emacs-Lisp mode

@findex emacs-lisp-mode
  Lisp programs intended to be run in Emacs should be edited in Emacs-Lisp
mode; normally this will happen based on the file name that ends in
@file{.el}.  By contrast, Lisp mode itself is used for editing Lisp programs
intended for other Lisp systems.  Emacs-Lisp mode can be selected with the
command @kbd{M-x emacs-lisp-mode}.

  For testing of Lisp programs to run in Emacs, it is useful to be able to
evaluate part of the program as it is found in the Emacs buffer.  For
example, after changing the text of a Lisp function definition, evaluating
the definition installs the change for future calls to the function.
Evaluation of Lisp expressions is also useful in any kind of editing task
for invoking noninteractive functions (functions that are not commands).

@table @kbd
@item M-@key{ESC}
Read a Lisp expression in the minibuffer, evaluate it, and print the
value in the minibuffer (@code{eval-expression}).
@item C-x C-e
Evaluate the Lisp expression before point, and print the value in the
minibuffer (@code{eval-last-sexp}).
@item C-M-x
Evaluate the defun containing or after point, and print the value in
the minibuffer (@code{eval-defun}).
@item M-x eval-region
Evaluate all the Lisp expressions in the region.
@item M-x eval-current-buffer
Evaluate all the Lisp expressions in the buffer.
@end table

@kindex M-ESC
@findex eval-expression
  @kbd{M-@key{ESC}} (@code{eval-expression}) is the most basic command for evaluating
a Lisp expression interactively.  It reads the expression using the
minibuffer, so you can execute any expression on a buffer regardless of
what the buffer contains.  When the expression is evaluated, the current
buffer is once again the buffer that was current when @kbd{M-@key{ESC}} was
typed.

  @kbd{M-@key{ESC}} can easily confuse users who do not understand it,
especially on keyboards with autorepeat where it can result from holding
down the @key{ESC} key for too long.  Therefore, @code{eval-expression} is
normally a disabled command.  Attempting to use this command asks for
confirmation and gives you the option of enabling it; once you enable the
command, confirmation will no longer be required for it.
@xref{Disabling}.@refill

@kindex C-M-x
@findex eval-defun
  In Emacs-Lisp mode, the key @kbd{C-M-x} is bound to the function @code{eval-defun},
which parses the defun containing or following point as a Lisp expression
and evaluates it.  The value is printed in the echo area.  This command is
convenient for installing in the Lisp environment changes that you have
just made in the text of a function definition.

@kindex C-x C-e
@findex eval-last-sexp
  The command @kbd{C-x C-e} (@code{eval-last-sexp}) performs a similar job
but is available in all major modes, not just Emacs-Lisp mode.  It finds
the sexp before point, reads it as a Lisp expression, evaluates it, and
prints the value in the echo area.  It is sometimes useful to type in an
expression and then, with point still after it, type @kbd{C-x C-e}.

  If @kbd{C-M-x} or @kbd{C-x C-e} is given a numeric argument, it prints the value
by insertion into the current buffer at point, rather than in the echo
area.  The argument value does not matter.

@findex eval-region
@findex eval-current-buffer
  The most general command for evaluating Lisp expressions from a buffer is
@code{eval-region}.  @kbd{M-x eval-region} parses the text of the region as one or
more Lisp expressions, evaluating them one by one.  @kbd{M-x eval-current-buffer}
is similar but evaluates the entire buffer.  This is a reasonable way to
install the contents of a file of Lisp code that you are just ready to
test.  After finding and fixing a bug, use @kbd{C-M-x} on each function
that you change, to keep the Lisp world in step with the source file.

@node Lisp Debug, Lisp Interaction, Lisp Eval, Running
@section The Lisp Debugger
@cindex debugger

@vindex debug-on-error
@vindex debug-on-quit
  GNU Emacs contains a debugger for Lisp programs executing inside it.
This debugger is normally not used; many commands frequently get Lisp
errors when invoked in inappropriate contexts (such as @kbd{C-f} at the end
of the buffer) and it would be very unpleasant for that to enter a special
debugging mode.  When you want to make Lisp errors invoke the debugger, you
must set the variable @code{debug-on-error} to non-@code{nil}.  Quitting
with @kbd{C-g} is not considered an error, and @code{debug-on-error} has no
effect on the handling of @kbd{C-g}.  However, if you set
@code{debug-on-quit} non-@code{nil}, @kbd{C-g} will invoke the debugger.
This can be useful for debugging an infinite loop; type @kbd{C-g} once the
loop has had time to reach its steady state.  @code{debug-on-quit} has no
effect on errors.@refill

@findex debug-on-entry
@findex cancel-debug-on-entry
@findex debug
  You can also cause the debugger to be entered when a specified function
is called, or at a particular place in Lisp code.  Use @kbd{M-x
debug-on-entry} with argument @var{fun-name} to cause function
@var{fun-name} to enter the debugger as soon as it is called.  Use
@kbd{M-x cancel-debug-on-entry} to make the function stop entering the
debugger when called.  (Redefining the function also does this.)  To enter
the debugger from some other place in Lisp code, you must insert the
expression @code{(debug)} there and install the changed code with
@kbd{C-M-x}.  @xref{Lisp Eval}.@refill

  When the debugger is entered, it displays the previously selected buffer
in one window and a buffer named @samp{*Backtrace*} in another window.  The
backtrace buffer contains one line for each level of Lisp function
execution currently going on.  At the beginning of this buffer is a message
describing the reason that the debugger was invoked (such as, what error
message if it was invoked due to an error).

  The backtrace buffer is read-only, and is in a special major mode,
Backtrace mode, in which letters are defined as debugger commands.  The
usual Emacs editing commands are available; you can switch windows to
examine the buffer that was being edited at the time of the error, and you
can also switch buffers, visit files, and do any other sort of editing.
However, the debugger is a recursive editing level (@pxref{Recursive Edit})
and it is wise to go back to the backtrace buffer and exit the debugger
officially when you don't want to use it any more.  Exiting the debugger
kills the backtrace buffer.

@cindex current stack frame
  The contents of the backtrace buffer show you the functions that are
executing and the arguments that were given to them.  It has the additional
purpose of allowing you to specify a stack frame by moving point to the line
describing that frame.  The frame whose line point is on is considered the
@dfn{current frame}.  Some of the debugger commands operate on the current
frame.  Debugger commands are mainly used for stepping through code an
expression at a time.  Here is a list of them.

@table @kbd
@item c
Exit the debugger and continue execution.  In most cases, execution of the
program continues as if the debugger had never been entered (aside from the
effect of any variables or data structures you may have changed while
inside the debugger).  This includes entry to the debugger due to function
entry or exit, explicit invocation, quitting or certain errors.  Most
errors cannot be continued; trying to continue one of them causes the same
error to occur again.
@item d
Continue execution, but enter the debugger the next time a Lisp
function is called.  This allows you to step through the
subexpressions of an expression, seeing what values the subexpressions
compute and what else they do.

The stack frame made for the function call which enters the debugger
in this way will be flagged automatically for the debugger to be called
when the frame is exited.  You can use the @kbd{u} command to cancel
this flag.
@item b
Set up to enter the debugger when the current frame is exited.  Frames
that will invoke the debugger on exit are flagged with stars.
@item u
Don't enter the debugger when the current frame is exited.  This
cancels a @kbd{b} command on that frame.
@item e
Read a Lisp expression in the minibuffer, evaluate it, and print the
value in the echo area.  The same as the command @kbd{M-@key{ESC}},
except that @kbd{e} is not normally disabled like @kbd{M-@key{ESC}}.
@item q
Terminate the program being debugged; return to top level Emacs
command execution.

If the debugger was entered due to a @kbd{C-g} but you really want
to quit, not to debug, use the @kbd{q} command.
@item r
Return a value from the debugger.  The value is computed by reading an
expression with the minibuffer and evaluating it.

The value returned by the debugger makes a difference when the debugger
was invoked due to exit from a Lisp call frame (as requested with @kbd{b});
then the value specified in the @kbd{r} command is used as the value of
that frame.

The debugger's return value also matters with many errors.  For example,
@code{wrong-type-argument} errors will use the debugger's return value
instead of the invalid argument; @code{no-catch} errors will use the
debugger value as a throw tag instead of the tag that was not found.
If an error was signaled by calling the Lisp function @code{signal},
the debugger's return value is returned as the value of @code{signal}.
@end table

@node Lisp Interaction, External Lisp, Lisp Debug, Running
@section Lisp Interaction Buffers

  The buffer @samp{*scratch*} which is selected when Emacs starts up is
provided for evaluating Lisp expressions interactively inside Emacs.  Both
the expressions you evaluate and their output goes in the buffer.

  The @samp{*scratch*} buffer's major mode is Lisp Interaction mode, which
is the same as Emacs-Lisp mode except for one command, @key{LFD}.  In
Emacs-Lisp mode, @key{LFD} is an indentation command, as usual.  In Lisp
Interaction mode, @key{LFD} is bound to @code{eval-print-last-sexp}.  This
function reads the Lisp expression before point, evaluates it, and inserts
the value in printed representation before point.

  Thus, the way to use the @samp{*scratch*} buffer is to insert Lisp expressions
at the end, ending each one with @key{LFD} so that it will be evaluated.
The result is a complete typescript of the expressions you have evaluated
and their values.

@findex lisp-interaction-mode
  The rationale for this feature is that Emacs must have a buffer when it
starts up, but that buffer is not useful for editing files since a new
buffer is made for every file that you visit.  The Lisp interpreter
typescript is the most useful thing I can think of for the initial buffer
to do.  @kbd{M-x lisp-interaction-mode} will put any buffer in Lisp
Interaction mode.

@node External Lisp,, Lisp Interaction, Running
@section Running an External Lisp

  Emacs has facilities for running programs in other Lisp systems.  You can
run a Lisp process as an inferior of Emacs, and pass expressions to it to
be evaluated.  You can also pass changed function definitions directly from
the Emacs buffers in which you edit the Lisp programs to the inferior Lisp
process.

@findex run-lisp
  To run an inferior Lisp process, type @kbd{M-x run-lisp}.  This runs the
program named @code{lisp}, the same program you would run by typing
@code{lisp} as a shell command, with both input and output going through an
Emacs buffer named @samp{*lisp*}.  That is to say, any ``terminal output''
from Lisp will go into the buffer, advancing point, and any ``terminal
input'' for Lisp comes from text in the buffer.  To give input to Lisp, go
to the end of the buffer and type the input, terminated by @key{RET}.  The
@samp{*lisp*} buffer is in Inferior Lisp mode, a mode which has all the
special characteristics of Lisp mode and Shell mode (@pxref{Shell}).

@findex lisp-mode
  For the source files of programs to run in external Lisps, use Lisp mode.
This mode can be selected with @kbd{M-x lisp-mode}, and is used automatically
for files whose names end in @file{.l} or @file{.lisp}, as most Lisp
systems usually expect.

@kindex C-M-x
@findex lisp-send-defun
  When you edit a function in a Lisp program you are running, the easiest
way to send the changed definition to the inferior Lisp process is the key
@kbd{C-M-x}.  In Lisp mode, this runs the function @code{lisp-send-defun},
which finds the defun around or following point and sends it as input to
the Lisp process.  (Emacs can send input to any inferior process regardless
of what buffer is current.)

  Contrast the meanings of @kbd{C-M-x} in Lisp mode (for editing programs
to be run in another Lisp system) and Emacs-Lisp mode (for editing Lisp
programs to be run in Emacs): in both modes it has the effect of installing
the function definition that point is in, but the way of doing so is
different according to where the relevant Lisp environment is found.
@xref{Lisp Modes}.

@node Abbrevs, Picture, Running, Top
@chapter Abbrevs
@cindex abbrevs
@cindex expansion (of abbrevs)

  An @dfn{abbrev} is a word which @dfn{expands}, if you insert it, into some
different text.  Abbrevs are defined by the user to expand in specific
ways.  For example, you might define @samp{foo} as an abbrev expanding to
@samp{find outer otter}.  With this abbrev defined, you would be able to
get @samp{find outer otter } into the buffer by typing @kbd{f o o @key{SPC}}.

@findex abbrev-mode
@vindex abbrev-mode
  Abbrevs expand only when Abbrev mode (a minor mode) is enabled.
Disabling Abbrev mode does not cause abbrev definitions to be forgotten,
but they do not expand until Abbrev mode is enabled again.  The command
@kbd{M-x abbrev-mode} toggles Abbrev mode; with a numeric argument, it
turns Abbrev mode on if the argument is positive, off otherwise.
@xref{Minor Modes}.  @code{abbrev-mode} is also a variable, local to each
buffer; Abbrev mode is on when the variable is non-@code{nil}.

  Abbrev definitions can be @dfn{mode-specific}---active only in one major
mode.  Abbrevs can also have @dfn{global} definitions that are active in
all major modes.  The same abbrev can have a global definition and various
mode-specific definitions for different major modes.  A mode specific
definition for the current major mode overrides a global definition.

  Abbrevs can be defined interactively during the editing session.  Lists
of abbrev definitions can also be saved in files and reloaded in later
sessions.  Some users keep extensive lists of abbrevs that they load in
every session.

@menu
* Defining Abbrevs::  Defining an abbrev, so it will expand when typed.
* Expanding Abbrevs:: Controlling expansion: prefixes, canceling expansion.
* Editing Abbrevs::   Viewing or editing the entire list of defined abbrevs.
* Saving Abbrevs::    Saving the entire list of abbrevs for another session.
@end menu

@node Defining Abbrevs, Expanding Abbrevs, Abbrevs, Abbrevs
@section Defining Abbrevs

@table @kbd
@item C-x +
Define an abbrev to expand into some text before point
(@code{add-global-abbrev}).
@item C-x C-a
Similar, but define an abbrev available only in the current major mode
(@code{add-mode-abbrev}).
@item C-x -
Define a word in the buffer as an abbrev (@code{inverse-add-global-abbrev}).
@item C-x C-h
Define a word in the buffer as a mode-specific abbrev
(@code{inverse-add-mode-abbrev}).
@item M-x kill-all-abbrevs
After this command, there are no abbrev definitions in effect.
@end table

@kindex C-x +
@findex add-global-abbrev
  The usual way to define an abbrev is to enter the text you want the
abbrev to expand to, position point after it, and type @kbd{C-x +}
(@code{add-global-abbrev}).  This reads the abbrev itself using the
minibuffer, and then defines it as an abbrev for one or more words before
point.  Use a numeric argument to say how many words before point should be
taken as the expansion.  For example, to define the abbrev @samp{foo} as
mentioned above, insert the text @samp{find outer otter} and then type
@kbd{C-u 3 C-x + f o o @key{RET}}.

  An argument of zero to @kbd{C-x +} means to use the contents of the
region as the expansion of the abbrev being defined.

@kindex C-x C-a
@findex add-mode-abbrev
  The command @kbd{C-x C-a} (@code{add-mode-abbrev}) is similar, but
defines a mode-specific abbrev.  Mode specific abbrevs are active only in a
particular major mode.  @kbd{C-x C-a} defines an abbrev for the major mode
in effect at the time @kbd{C-x C-a} is typed.  The arguments work the same
as for @kbd{C-x +}.

@kindex C-x -
@findex inverse-add-global-abbrev
@kindex C-x C-h
@findex inverse-add-mode-abbrev
  If the text of the abbrev you want is already in the buffer instead of
the expansion, use command @kbd{C-x -} (@code{inverse-add-global-abbrev})
instead of @kbd{C-x +}, or use @kbd{C-x C-h}
(@code{inverse-add-mode-abbrev}) instead of @kbd{C-x C-a}.  These commands
are called ``inverse'' because they invert the meaning of the argument
found in the buffer and the argument read using the minibuffer.@refill

  To change the definition of an abbrev, just add the new definition.  You
will be asked to confirm if the abbrev has a prior definition.  To remove
an abbrev definition, give a negative argument to @kbd{C-x +} or @kbd{C-x
C-a}.  You must choose the command to specify whether to kill a global
definition or a mode-specific definition for the current mode, since those
two definitions are independent for one abbrev.

@findex kill-all-abbrevs
  @kbd{M-x kill-all-abbrevs} removes all the abbrev definitions there are.

@node Expanding Abbrevs, Editing Abbrevs, Defining Abbrevs, Abbrevs
@section Controlling Abbrev Expansion

  An abbrev expands whenever it is present in the buffer just before point
and a self-inserting punctuation character (@key{SPC}, comma, etc.@:) is
typed.  Most often the way an abbrev is used is to insert the abbrev
followed by punctuation.

@vindex abbrev-all-caps
  Abbrev expansion preserves case; thus, @samp{foo} expands into @samp{find
outer otter}; @samp{Foo} into @samp{Find outer otter}, and @samp{FOO} into
@samp{FIND OUTER OTTER} or @samp{Find Outer Otter} according to the
variable @code{abbrev-all-caps} (a non-@code{nil} value chooses the first
of the two expansions).@refill

  These two commands are used to control abbrev expansion:

@table @kbd
@item M-'
Separate a prefix from a following abbrev to be expanded
(@code{abbrev-prefix-mark}).
@item M-x unexpand-abbrev
Undo last abbrev expansion.
@item M-x expand-region-abbrevs
Expand some or all abbrevs found in the region.
@end table

@kindex M-'
@findex abbrev-prefix-mark
  You may wish to expand an abbrev with a prefix attached; for example, if
@samp{cnst} expands into @samp{construction}, you might want to use it to
enter @samp{reconstruction}.  It does not work to type @kbd{recnst},
because that is not necessarily a defined abbrev.  What does work is to use
the command @kbd{M-'} (@code{abbrev-prefix-mark}) in between the prefix
@samp{re} and the abbrev @samp{cnst}.  First, insert @samp{re}.  Then type
@kbd{M-'}; this inserts a minus sign in the buffer to indicate that it has
done its work.  Then insert the abbrev @samp{cnst}; the buffer now contains
@samp{re-cnst}.  Now insert a punctuation character to expand the abbrev
@samp{cnst} into @samp{construction}.  The minus sign is deleted at this
point, because @kbd{M-'} left word for this to be done.  The resulting text
is the desired @samp{reconstruction}.@refill

  If you actually want the text of the abbrev in the buffer, rather than
its expansion, you can accomplish this by inserting the following
punctuation with @kbd{C-q}.  Thus, @kbd{foo C-q -} leaves @samp{foo-} in the
buffer.

@findex unexpand-abbrev
  If you expand an abbrev by mistake, you can undo the expansion (replace
the expansion by the original abbrev text) with @kbd{M-x unexpand-abbrev}.
@kbd{C-_} (@code{undo}) can also be used to undo the expansion; but first
it will undo the insertion of the following punctuation character!

@findex expand-region-abbrevs
  @kbd{M-x expand-region-abbrevs} searches through the region for defined
abbrevs, and for each one found offers to replace it with its expansion.
This command is useful if you have typed in text using abbrevs but forgot
to turn on Abbrev mode first.  It may also be useful together with a
special set of abbrev definitions for making several global replacements at
once.

@node Editing Abbrevs, Saving Abbrevs, Expanding Abbrevs, Abbrevs
@section Examining and Editing Abbrevs

@table @kbd
@item M-x list-abbrevs
Print a list of all abbrev definitions.
@item M-x edit-abbrevs
Edit a list of abbrevs; you can add, alter or remove definitions.
@end table

@findex list-abbrevs
  The output from @kbd{M-x list-abbrevs} looks like this:

@example
(lisp-mode-abbrev-table)
"dk"	       0    "define-key"
(global-abbrev-table)
"dfn"	       0    "definition"
@end example

@noindent
(Some blank lines of no semantic significance, and some other abbrev
tables, have been omitted.)

  A line containing a name in parentheses is the header for abbrevs in a
particular abbrev table; @code{global-abbrev-table} contains all the global
abbrevs, and the other abbrev tables that are named after major modes
contain the mode-specific abbrevs.

  Within each abbrev table, each nonblank line defines one abbrev.  The
word at the beginning is the abbrev.  The number that appears is the number
of times the abbrev has been expanded.  Emacs keeps track of this to help
you see which abbrevs you actually use, in case you decide to eliminate
those that you don't use often.  The string at the end of the line is the
expansion.

@findex edit-abbrevs
@kindex C-x C-s
@findex edit-abbrevs-redefine
  @kbd{M-x edit-abbrevs} allows you to add, change or kill abbrev
definitions by editing a list of them in an Emacs buffer.  The list has the
same format described above.  The buffer of abbrevs is called @samp{*Abbrevs*},
and is in Edit-Abbrevs mode.  This mode redefines the key @kbd{C-x C-s} to
install the abbrev definitions as specified in the buffer.  The command
that does this is @code{edit-abbrevs-redefine}.  Any abbrevs not described
in the buffer are eliminated when this is done.

  @code{edit-abbrevs} is actually the same as @code{list-abbrevs} except
that it selects the buffer @samp{*Abbrevs*} whereas @code{list-abbrevs}
merely displays it in another window.

@node Saving Abbrevs,, Editing Abbrevs, Abbrevs
@section Saving Abbrevs

  These commands allow you to keep abbrev definitions between editing
sessions.

@table @kbd
@item M-x write-abbrev-file
Write a file describing all defined abbrevs.
@item M-x read-abbrev-file
Read such a file and define abbrevs as specified there.
@item M-x quietly-read-abbrev-file
Similar but do not display a message about what is going on.
@item M-x define-abbrevs
Define abbrevs from buffer.
@item M-x insert-abbrevs
Insert all abbrevs and their expansions into the buffer.
@end table

@findex write-abbrev-file
  @kbd{M-x write-abbrev-file} reads a file name using the minibuffer and
writes a description of all current abbrev definitions into that file.  The
text stored in the file looks like the output of @kbd{M-x list-abbrevs}.
This is used to save abbrev definitions for use in a later session.

@findex read-abbrev-file
@findex quietly-read-abbrev-file
@vindex abbrev-file-name
  @kbd{M-x read-abbrev-file} reads a file name using the minibuffer and
reads the file, defining abbrevs according to the contents of the file.
@kbd{M-x quietly-read-abbrev-file} is the same except that it does not
display a message in the echo area saying that it is doing its work; it
is actually useful primarily in the @file{.emacs} file.  If an empty
argument is given to either of these functions, the file name used is the
value of the variable @code{abbrev-file-name}, which is by default
@code{"~/.abbrev_defs"}.

@vindex save-abbrevs
  Emacs will offer to save abbrevs automatically if you have changed any of
them, whenever it offers to save all files (for @kbd{C-x s} or @kbd{C-x
C-c}).  This feature can be inhibited by setting the variable
@code{save-abbrevs} to @code{nil}.

@findex insert-abbrevs
@findex define-abbrevs
  The commands @kbd{M-x insert-abbrevs} and @kbd{M-x define-abbrevs} are
similar to the previous commands but work on text in an Emacs buffer.
@kbd{M-x insert-abbrevs} inserts text into the current buffer before point,
describing all current abbrev definitions; @kbd{M-x define-abbrevs} parses
the entire current buffer and defines abbrevs accordingly.@refill

@node Picture, Sending Mail, Abbrevs, Top
@chapter Editing Pictures
@cindex pictures
@findex edit-picture

  If you want to create a picture made out of text characters (for example,
a picture of the division of a register into fields, as a comment in a
program), use the command @code{edit-picture} to enter Picture mode.

  In Picture mode, editing is based on the @dfn{quarter-plane} model of
text, according to which the text characters lie studded on an area that
stretches infinitely far to the left and downward.  The concept of the end
of a line does not exist in this model; the most you can say is where the
last nonblank character on the line is found.

  Of course, Emacs really always considers text as a sequence of
characters, and lines really do have ends.  But in Picture mode most
frequently-used keys are rebound to commands that simulate the
quarter-plane model of text.  They do this by inserting spaces or by
converting tabs to spaces.

  Most of the basic editing commands of Emacs are redefined by Picture mode
to do essentially the same thing but in a quarter-plane way.  In addition,
Picture mode defines various keys starting with the @kbd{C-c} prefix to
run special picture editing commands.

  One of these keys, @kbd{C-c C-c}, is pretty important.  Often a picture
is part of a larger file that is usually edited in some other major mode.
@kbd{M-x edit-picture} records the name of the previous major mode, and
then you can use the @kbd{C-c C-c} command (@code{Picture-mode-exit}) to
restore that mode.  @kbd{C-c C-c} also deletes spaces from the ends of
lines, unless given a numeric argument.

  The commands used in Picture mode all work in other modes (provided the
@file{picture} library is loaded), but are not bound to keys except in
Picture mode.  Note that the descriptions below talk of moving ``one
column'' and so on, but all the picture mode commands handle numeric
arguments as their normal equivalents do.

@vindex picture-mode-hook
  Turning on Picture mode calls the value of the variable @code{picture-mode-hook}
as a function, with no arguments, if that value exists and is non-@code{nil}.

@menu
* Basic Picture::         Basic concepts and simple commands of Picture Mode.
* Insert in Picture::     Controlling direction of cursor motion
                           after "self-inserting" characters.
* Tabs in Picture::       Various features for tab stops and indentation.
* Rectangles in Picture:: Clearing and superimposing rectangles.
@end menu

@node Basic Picture, Insert in Picture, Picture Mode, Picture
@section Basic Editing in Picture Mode

@findex Picture-forward-column
@findex Picture-backward-column
@findex Picture-move-down
@findex Picture-move-up
  Most keys do the same thing in Picture mode that they usually do, but do
it in a quarter-plane style.  For example, @kbd{C-f} is rebound to run
@code{Picture-forward-column}, which is defined to move point one column to
the right, by inserting a space if necessary, so that the actual end of the
line makes no difference.  @kbd{C-b} is rebound to run
@code{Picture-backward-column}, which always moves point left one column,
converting a tab to multiple spaces if necessary.  @kbd{C-n} and @kbd{C-p}
are rebound to run @code{Picture-move-down} and @code{Picture-move-up},
which can either insert spaces or convert tabs as necessary to make sure
that point stays in exactly the same column.  @kbd{C-e} runs
@code{Picture-end-of-line}, which moves to after the last nonblank
character on the line.  There is no need to change @kbd{C-a}, as the choice
of screen model does not affect beginnings of lines.@refill

@findex Picture-newline
  Insertion of text is adapted to the quarter-plane screen model through
the use of Overwrite mode (@pxref{Minor Modes}).  Self-inserting characters
replace existing text, column by column, rather than pushing existing text
to the right.  @key{RET} runs @code{Picture-newline}, which just moves to
the beginning of the following line so that new text will replace that
line.

@findex Picture-backward-clear-column
@findex Picture-clear-column
@findex Picture-clear-line
  Deletion and killing of text are replaced with erasure.  @key{DEL}
(@code{Picture-backward-clear-column}) replaces the preceding character
with a space rather than removing it.  @kbd{C-d}
(@code{Picture-clear-column}) does the same thing in a forward direction.
@kbd{C-k} (@code{Picture-clear-line}) really kills the contents of lines,
but does not ever remove the newlines from the buffer.@refill

@findex Picture-open-line
  To do actual insertion, you must use special commands.  @kbd{C-o}
(@code{Picture-open-line}) still creates a blank line, but does so after
the current line; it never splits a line.  @kbd{C-M-o}, @code{split-line},
makes sense in Picture mode, so it is not changed.  @key{LFD}
(@code{Picture-duplicate-line}) inserts below the current line another line
with the same contents.@refill

@kindex C-c C-d
@findex delete-char
  Real deletion can be done with @kbd{C-w}, or with @kbd{C-c C-d} (which is
defined as @code{delete-char}, as @kbd{C-d} is in other modes), or with one
of the picture rectangle commands (@pxref{Rectangles in Picture}).

@node Insert in Picture, Tabs in Picture, Basic Picture, Picture
@section Controlling Motion after Insert

@findex Picture-movement-up
@findex Picture-movement-down
@findex Picture-movement-left
@findex Picture-movement-right
@findex Picture-movement-nw
@findex Picture-movement-ne
@findex Picture-movement-sw
@findex Picture-movement-se
@kindex M-`
@kindex M-'
@kindex M--
@kindex M-=
@kindex C-c `
@kindex C-c '
@kindex C-c /
@kindex C-c \
  Since ``self-inserting'' characters in Picture mode just overwrite and
move point, there is no essential restriction on how point should be moved.
Normally point moves right, but you can specify any of the eight orthogonal
or diagonal directions for motion after a ``self-inserting'' character.
This is useful for drawing lines in the buffer.

@table @kbd
@item M-`
Move left after insertion (@code{Picture-movement-left}).
@item M-'
Move right after insertion (@code{Picture-movement-right}).
@item M--
Move up after insertion (@code{Picture-movement-up}).
@item M-=
Move down after insertion (@code{Picture-movement-down}).
@item C-c `
Move up and left (``northwest'') after insertion @*(@code{Picture-movement-nw}).
@item C-c '
Move up and right (``northeast'') after insertion @*
(@code{Picture-movement-ne}).
@item C-c /
Move down and left (``southwest'') after insertion
@*(@code{Picture-movement-sw}).
@item C-c \
Move down and right (``southeast'') after insertion
@*(@code{Picture-movement-se}).
@end table

@kindex C-c C-f
@kindex C-c C-b
@findex Picture-motion
@findex Picture-motion-reverse
  Two motion commands move based on the current Picture insertion
direction.  @kbd{C-c C-f} (@code{Picture-motion}) moves in the same
direction as motion after ``insertion'' currently does, while @kbd{C-c C-b}
(@code{Picture-motion-reverse}) moves in the opposite direction.

@node Tabs in Picture, Rectangles in Picture, Insert in Picture, Picture
@section Picture Mode Tabs

@kindex M-TAB
@findex Picture-tab-search
@vindex picture-tab-chars
  Two kinds of tab-like action are provided in Picture mode.
Context-based tabbing is done with @kbd{M-@key{TAB}}
(@code{Picture-tab-search}).  With no argument, it moves to a point
underneath the next ``interesting'' character that follows whitespace in
the previous nonblank line.  ``Next'' here means ``appearing at a
horizontal position greater than the one point starts out at''.  With an
argument, as in @kbd{C-u M-@key{TAB}}, this command moves to the next such
interesting character in the current line.  @kbd{M-@key{TAB}} does not
change the text; it only moves point.  ``Interesting'' characters are
defined by the variable @code{picture-tab-chars}, which contains a string
whose characters are all considered interesting.  Its default value is
@code{"!-~"}.@refill

@findex Picture-tab
  @key{TAB} itself runs @code{Picture-tab}, which operates based on the
current tab stop settings; it is the Picture mode equivalent of
@code{tab-to-tab-stop}.  Normally it just moves point, but with a numeric
argument it clears the text that it moves over.

@kindex C-c TAB
@findex Picture-set-tab-stops
  The context-based and tab-stop-based forms of tabbing are brought
together by the command @kbd{C-c @key{TAB}}, @code{Picture-set-tab-stops}.
This command sets the tab stops to the positions which @kbd{M-@key{TAB}}
would consider significant in the current line.  The use of this command,
together with @key{TAB}, can get the effect of context-based tabbing.  But
@kbd{M-@key{TAB}} is more convenient in the cases where it is sufficient.

@node Rectangles in Picture,, Tabs in Picture, Picture
@section Picture Mode Rectangle Commands
@cindex rectangle

  Picture mode defines commands for working on rectangular pieces of the
text in ways that fit with the quarter-plane model.  The standard rectangle
commands may also be useful (@pxref{Rectangles}).

@table @kbd
@item C-c C-k
Clear out the region-rectangle (@code{Picture-clear-rectangle}).  With
argument, kill it.
@item C-c C-w @var{r}
Similar but save rectangle contents in register @var{r} first
(@code{Picture-clear-rectangle-to--register}).
@item C-c C-y
Overwrite last killed rectangle into the buffer, with upper left corner at
point (@code{Picture-yank-rectangle}).  With argument, insert instead.
@item C-c C-x @var{r}
Similar, but take the rectangle from register @var{r}
(@code{Picture-yank-rectangle-from-register}).
@end table

@kindex C-c C-k
@kindex C-c C-w
@findex Picture-clear-rectangle
@findex Picture-clear-rectangle-to-register
  The picture rectangle commands @kbd{C-c C-k}
(@code{Picture-clear-rectangle}) and @kbd{C-c C-w}
(@code{Picture-clear-rectangle-to-register}) differ from the standard
rectangle commands in that they normally clear the rectangle instead of
deleting it; this is analogous with the way @kbd{C-d} is changed in Picture
mode.@refill

  However, deletion of rectangles can be useful in Picture mode, so these
commands delete the rectangle if given a numeric argument.

@kindex C-c C-y
@kindex C-c C-x
@findex Picture-yank-rectangle
@findex Picture-yank-rectangle-from-register
  The Picture mode commands for yanking rectangles differ from the standard
ones in overwriting instead of inserting.  This is the same way that
Picture mode insertion of other text is different from other modes.
@kbd{C-c C-y} (@code{Picture-yank-rectangle}) inserts (by overwriting) the
rectangle that was most recently killed, while @kbd{C-c C-x}
(@code{Picture-yank-rectangle-from-register}) does likewise for the
rectangle found in a specified register.

@node Sending Mail, Rmail, Picture, Top
@chapter Sending Mail
@cindex mail
@cindex message

  To send a message in Emacs, you start by typing a command (@kbd{C-x m})
to select and initialize the @samp{*mail*} buffer.  Then you edit the text
and headers of the message in this buffer, and type another command
(@kbd{C-c C-c}) to send the message.

@table @kbd
@item C-x m
Begin composing a message to send (@code{mail}).
@item C-x 4 m
Likewise, but display the message in another window
(@code{mail-other-window}).
@end table

@kindex C-x m
@findex mail
@kindex C-x 4 m
@findex mail-other-window
  The command @kbd{C-x m} (@code{mail}) selects a buffer named
@samp{*mail*} and initializes it with the skeleton of an outgoing message.
@kbd{C-x 4 m} (@code{mail-other-window}) selects the @samp{*mail*} buffer
in a different window, leaving the previous current buffer visible.@refill

@cindex headers (of message)
  In addition to the @dfn{text} or contents, a message has @dfn{header
fields} which say who sent it, when, to whom, why, and so on.  Some header
fields such as the date and sender are created automatically after the
message is sent.  Others, such as the recipient names, must be specified by
you in order to send the message properly.

  The line in the buffer that says

@example
--text follows this line--
@end example

@vindex mail-header-separator
@noindent
is a special delimiter that separates the headers you have specified from
the text.  Whatever follows this line is the text of the message; the
headers precede it.  The delimiter line itself does not appear in the
message actually sent.  The text used for the delimiter line is controlled
by the variable @code{mail-header-separator}.

Here is an example of what the headers and text in the @samp{*mail*} buffer
might look like.

@example
To: rms@@mc
CC: mly@@mc, rg@@oz
Subject: The Emacs Manual
--Text follows this line--
Please ignore this message.
@end example

  Because the mail composition buffer is an ordinary Emacs buffer, you can
switch to other buffers while in the middle of composing mail, and switch
back later (or never).  If you use the @kbd{C-x m} command again when you
have been composing another message but have not sent it, you are asked to
confirm before the old message is erased.  If you answer @kbd{n}, the
@samp{*mail*} buffer is left selected with its old contents, so you can
finish the old message and send it.  @kbd{C-u C-x m} is another way to do
this.  Sending the message marks the @samp{*mail*} buffer ``unmodified'',
which avoids the need for confirmation when @kbd{C-x m} is next used.

@section Mail Header Fields

  There are several header fields you can use in the @samp{*mail*} buffer.
Each header field starts with a field name at the beginning of a line,
terminated by a colon.  It does not matter whether you use upper or lower
case in the field name.  After the colon and optional whitespace comes the
contents of the field.

@table @samp
@item To
This field contains the mailing addresses to which the message is
addressed.

@item Subject
The contents of the @samp{Subject} field should be a piece of text
that says what the message is about.  The reason @samp{Subject} fields
are useful is that most mail-reading programs can provide a summary of
messages, listing the subject of each message but not its text.

@item CC
This field contains additional mailing addresses to send the message
to, but whose readers should not regard the message as addressed to
them.

@item BCC
This field contains additional mailing addresses to send the message
to, but which should not appear in the header of the message actually
sent.

@item FCC
This field contains the name of one file (in Unix mail file format) to
which a copy of the message should be appended when the message is
sent.

@item From
Use the @samp{From} field to say who you are, when the account you are
using to send the mail is not your own.  The contents of the
@samp{From} field should be a valid mailing address, since replies
will normally go there.

@item Reply-To
Use the @samp{Reply-to} field to direct replies to a different
address, not your own.  There is no difference between @samp{From} and
@samp{Reply-to} in their effect on where replies go, but they convey a
different meaning to the human who reads the message.

@item In-Reply-To
This field contains a piece of text describing a message you are
replying to.  Some mail systems can use this information to correlate
related pieces of mail.  Normally this field is filled in by Rmail
when you are replying to a message in Rmail, and you never need to
think about it.
@end table

@noindent
The @samp{To}, @samp{CC}, @samp{BCC} and @samp{FCC} fields can appear
any number of times, to specify many places to send the message.

@noindent
The @samp{To}, @samp{CC}, and @samp{BCC} fields can have continuation
lines.  All the lines starting with whitespace, following the line on
which the field starts, are considered part of the field.  For
example,@refill

@group
@example
To: foo@@bar, this@@that,
  me@@here
@end example
@end group

@noindent
If you have a @file{~/.mailrc} file, Rmail will scan it for mail aliases
the first time you try to send mail in an Rmail session.  Aliases found
in the @samp{To}, @samp{CC}, and @samp{BCC} fields will be expanded where
appropriate.

@vindex mail-archive-file-name
  If the variable @code{mail-archive-file-name} is non-@code{nil}, it should be a
string, naming a file; every time you start to edit a message to sent,
an @samp{FCC} field will be put in for that file.  Unless you remove the
@samp{FCC} field, every message will be written into that file when it is
sent.

@section Mail Mode

  The major mode used in the @samp{*mail*} buffer is Mail mode, which is
much like Text mode except that various special commands are provided on
the @kbd{C-c} prefix.  These commands all have to do specifically with
editing or sending the message.

@table @kbd
@item C-c C-s
Send the message, and leave the @samp{*mail*} buffer selected
(@code{mail-send}).
@item C-c C-c
Send the message, and select some other buffer (@code{mail-send-and-exit}).
@item C-c t
Move to the @samp{To} header field, creating one if there is none
(@code{mail-to}).
@item C-c s
Move to the @samp{Subject} header field, creating one if there is
none (@code{mail-subject}).
@item C-c c
Move to the @samp{CC} header field, creating one if there is none
(@code{mail-cc}).
@item C-c w
Insert the file @file{~/.signature} at the end of the message text
(@code{mail-signature}).
@item C-c y
Yank the selected message from Rmail (@code{mail-yank-original}).
This command does nothing unless your command to start sending a
message was issued with Rmail.
@item C-c q
Fill all paragraphs of yanked old messages, each individually
(@code{mail-fill-yanked-message}).
@end table

@kindex C-c C-s
@kindex C-c C-c
@findex mail-send
@findex mail-send-and-exit
  There are two ways to send the message.  @kbd{C-c C-s} (@code{mail-send})
sends the message and marks the @samp{*mail*} buffer unmodified, but leaves
that buffer selected so that you can modify the message (perhaps with new
recipients) and send it again.  @kbd{C-c C-c} (@code{mail-send-and-exit})
sends and then deletes the window (if there is another window) or switches
to another buffer.  It puts the @samp{*mail*} buffer at the lowest priority
for automatic reselection, since you are finished with using it.  This is
the usual way to send the message.

@kindex C-c t
@findex mail-to
@kindex C-c s
@findex mail-subject
@kindex C-c c
@findex mail-cc
  Mail mode provides some other special commands that are useful for
editing the headers and text of the message before you send it.  There are
four commands defined to move point to particular header fields: @kbd{C-c
t} (@code{mail-to}) to move to the @samp{To} field, @kbd{C-c s}
(@code{mail-subject}) for the @samp{Subject} field, and @kbd{C-c c}
(@code{mail-cc}) for the @samp{CC} field.@refill

@kindex C-c w
@findex mail-signature
  @kbd{C-c w} (@code{mail-signature}) adds a standard piece text at the end of the
message to say more about who you are.  The text comes from the file
@file{.signature} in your home directory.

@kindex C-c y
@findex mail-yank-original
  When mail sending is invoked from the Rmail mail reader using an Rmail
command, @kbd{C-c y} can be used inside the @samp{*mail*} buffer to insert
the text of the message you are replying to.  Normally it indents each line
of that message four spaces and eliminates most header fields.  A numeric
argument specifies the number of spaces to indent.  An argument of just
@kbd{C-u} says not to indent at all and not to eliminate anything.
@kbd{C-c y} always uses the current message from the @samp{rmail} buffer,
so you can insert several old messages by selecting one in @samp{rmail},
switching to @samp{*mail*} and yanking it, then switching back to
@samp{rmail} to select another.@refill

@kindex C-c q
@findex mail-fill-yanked-message
  After using @kbd{C-c y}, the command @kbd{C-c q} (@code{mail-fill-yanked-message}) can
be used to fill the paragraphs of the yanked old message or messages.  One
use of @kbd{C-c q} fills all such paragraphs, each one separately.

@vindex mail-mode-hook
  Turning on Mail mode (which @kbd{C-x m} does automatically) calls the
value of @code{text-mode-hook}, if it is not void or @code{nil}, and then calls
the value of @code{mail-mode-hook} if that is not void or @code{nil}.

@node Rmail, Recursive Edit, Sending Mail, Top
@chapter Reading Mail with Rmail
@cindex Rmail
@cindex message

  Rmail is an Emacs subsystem for reading and disposing of mail that you
receive.  Rmail stores mail messages in files called Rmail files.  Reading
the message in an Rmail file is done in a special major mode, Rmail mode,
which redefines most letters to run commands for managing mail.

@cindex primary mail file
  Using Rmail in the simplest fashion, you have one Rmail file @file{~/RMAIL}
in which all of your mail is saved.  It is called your @dfn{primary mail
file}.  In more sophisticated usage, you can copy messages into other Rmail
files and then edit those files with Rmail.

  Rmail displays only one message at a time.  It is called the @dfn{current
message}.  Rmail mode's special commands can do such things as move to
another message, delete the message, copy the message into another file, or
send a reply.

@cindex message number
  Within the Rmail file, messages are arranged sequentially in order
of receipt.  They are also assigned consecutive integers as their
@dfn{message numbers}.  The number of the current message is displayed
in Rmail's mode line, followed by the total number of messages in the
file.  You can move to a message by specifying its message number
using the @kbd{j} key (@pxref{Rmail Motion}).

@kindex s (Rmail)
@findex rmail-save
  Following the usual conventions of Emacs, changes in an Rmail file become
permanent only when the file is saved.  You can do this with @kbd{s}
(@code{rmail-save}), which also expunges deleted messages from the file
first (@pxref{Rmail Deletion}).  To save the file without expunging, use
@kbd{C-x C-s}.  Rmail saves the Rmail file spontaneously when moving new
mail from an inbox file (@pxref{Rmail Inbox}).

@kindex q (Rmail)
@findex rmail-quit
  You can exit Rmail with @kbd{q} (@code{rmail-quit}); this expunges and saves the
Rmail file and then switches to another buffer.  But there is no need to
`exit' formally.  If you switch from Rmail to editing in other buffers, and
never happen to switch back, you have exited.  Just make sure to save the
Rmail file eventually (like any other file you have changed).  @kbd{C-x s}
is a good enough way to do this (@pxref{Saving}).

@menu
* Scroll: Rmail Scrolling.   Scrolling through a message.
* Motion: Rmail Motion.      Moving to another message.
* Deletion: Rmail Deletion.  Deleting and expunging messages.
* Inbox: Rmail Inbox.        How mail gets into the Rmail file.
* Files: Rmail Files.        Using multiple Rmail files.
* Labels: Rmail Labels.      Classifying messages by labeling them.
* Summary: Rmail Summary.    Summaries show brief info on many messages.
* Reply: Rmail Reply.        Sending replies to messages you are viewing.
* Editing: Rmail Editing.    Editing message text and headers in Rmail.
* Digest: Rmail Digest.      Extracting the messages from a digest message.
@end menu

@node Rmail Scrolling, Rmail Motion, Rmail, Rmail
@section Scrolling Within a Message

  When Rmail displays a message that does not fit on the screen, it is
necessary to scroll through it.  This could be done with @kbd{C-v}, @kbd{M-v}
and @kbd{M-<}, but in Rmail scrolling is so frequent that it deserves to be
easier to type.

@table @kbd
@item @key{SPC}
Scroll forward (@code{scroll-up}).
@item @key{DEL}
Scroll backward (@code{scroll-down}).
@item .
Scroll to start of message (@code{rmail-beginning-of-message}).
@end table

@kindex SPC (Rmail)
@kindex DEL (Rmail)
  Since the most common thing to do while reading a message is to scroll
through it by screenfuls, Rmail makes @key{SPC} and @key{DEL} synonyms of
@kbd{C-v} (@code{scroll-up}) and @kbd{M-v} (@code{scroll-down})

@kindex . (Rmail)
@findex rmail-beginning-of-message
  The command @kbd{.} (@code{rmail-beginning-of-message}) scrolls back to the
beginning of the selected message.  This is not quite the same as @kbd{M-<}:
for one thing, it does not set the mark; for another, it resets the buffer
boundaries to the current message if you have changed them.

@node Rmail Motion, Rmail Deletion, Rmail Scrolling, Rmail
@section Moving Among Messages

  The most basic thing to do with a message is to read it.  The way to do
this in Rmail is to make the message current.  You can make any message
current given its message number using the @kbd{j} command, but the usual
thing to do is to move sequentially through the file, since this is the
order of receipt of messages.  When you enter Rmail, you are positioned at
the first new message (new messages are those received since the previous
use of Rmail), or at the last message if there are no new messages this
time.  Move forward to see the other new messages; move backward to
reexamine old messages.

@table @kbd
@item n
Move to the next nondeleted message, skipping any intervening
deleted messages (@code{rmail-next-undeleted-message}).
@item p
Move to the previous nondeleted message @*
(@code{rmail-previous-undeleted-message}).
@item M-n
Move to the next message, including deleted messages
(@code{rmail-next-message}).
@item M-p
Move to the previous message, including deleted messages
(@code{rmail-previous-message}).
@item j
Move to the first message.  With argument @var{n}, move to
message number @var{n} (@code{rmail-show-message}).
@item >
Move to the last message (@code{rmail-last-message}).

@item M-s @var{regexp} @key{RET}
Move to the next message containing a match for @var{regexp}
(@code{rmail-search}).  If @var{regexp} is empty, the last regexp used is
used again.

@item - M-s @var{regexp} @key{RET}
Move to the previous message containing a match for @var{regexp}.
If @var{regexp} is empty, the last regexp used is used again.
@end table

@kindex n (Rmail)
@kindex p (Rmail)
@kindex M-n (Rmail)
@kindex M-p (Rmail)
@findex rmail-next-undeleted-message
@findex rmail-previous-undeleted-message
@findex rmail-next-message
@findex rmail-previous-message
  @kbd{n} and @kbd{p} are the usual way of moving among messages in Rmail.  They
move through the messages sequentially, but skipping over deleted messages,
which is usually what you want to do.  Their command definitions are named
@code{rmail-next-undeleted-message} and @code{rmail-previous-undeleted-message}.  If
you do not want to skip deleted messages---for example, if you want to move
to a message to undelete it---use the variants @kbd{M-n} and @kbd{M-p}
(@code{rmail-next-message} and @code{rmail-previous-message}).  A numeric
argument to any of these commands serves as a repeat count.@refill

@kindex M-s (Rmail)
@findex rmail-search
  The @kbd{M-s} (@code{rmail-search}) command is Rmail's version of search.  The
usual incremental search command @kbd{C-s} works in Rmail, but it searches
only within the current message.  The purpose of @kbd{M-s} is to search for
another message.  It reads a regular expression (@pxref{Regexps})
nonincrementally, then searches starting at the beginning of the following
message for a match.  The message containing the match is selected.

  To search backward in the file for another message, give @kbd{M-s} a
negative argument.  In Rmail this can be done with @kbd{- M-s}.

@kindex j (Rmail)
@kindex > (Rmail)
@findex rmail-show-message
@findex rmail-last-message
  To move to a message specified by absolute message number, use @kbd{j}
(@code{rmail-show-message}) with the message number as argument.  With no
argument, @kbd{j} selects the first message.  @kbd{>} (@code{rmail-last-message}) selects
the last message.

@node Rmail Deletion, Rmail Inbox, Rmail Motion, Rmail
@section Deleting Messages

@cindex deletion (Rmail)
  When you no longer need to keep a message, you can @dfn{delete} it.  This
flags it as ignorable, and some Rmail commands will pretend it is no longer
present; but it still has its place in the Rmail file, and still has its
message number.

@cindex expunging (Rmail)
  @dfn{Expunging} the Rmail file actually removes the deleted messages.
The remaining messages are renumbered consecutively.  Expunging is the only
action that changes the message number of any message, except for
undigestifying (@pxref{Rmail Digest}).

@table @kbd
@item d
Delete the current message, and move to the next nondeleted message
(@code{rmail-delete-forward}).
@item C-d
Delete the current message, and move to the previous nondeleted
message (@code{rmail-delete-backward}).
@item u
Move back to a deleted message and undelete it
(@code{rmail-undelete-previous-message}).
@item e
Expunge the Rmail file (@code{rmail-expunge}).
@end table

@kindex d (Rmail)
@kindex C-d (Rmail)
@findex rmail-delete-forward
@findex rmail-delete-backward
  There are two Rmail commands for deleting messages.  Both delete the
current message and select another message.  @kbd{d} (@code{rmail-delete-forward})
moves to the following message, skipping messages already deleted, while
@kbd{C-d} (@code{rmail-delete-backward}) moves to the previous nondeleted message.
If there is no nondeleted message to move to in the specified direction,
the message that was just deleted remains current.

@cindex undeletion (Rmail)
@kindex e (Rmail)
@findex rmail-expunge
  To make all the deleted messages finally vanish from the Rmail file,
type @kbd{e} (@code{rmail-expunge}).  Until you do this, you can still @dfn{undelete}
the deleted messages.

@kindex u (Rmail)
@findex rmail-undelete-previous-message
  To undelete, type
@kbd{u} (@code{rmail-undelete-previous-message}), which is designed to cancel the
effect of a @kbd{d} command (usually).  It undeletes the current message
if the current message is deleted.  Otherwise it moves backward to previous
messages until a deleted message is found, and undeletes that message.

  You can usually undo a @kbd{d} with a @kbd{u} because the @kbd{u} moves
back to and undeletes the message that the @kbd{d} deleted.  But this does
not work when the @kbd{d} skips a few already-deleted messages that follow
the message being deleted; then the @kbd{u} command will undelete the last
of the messages that were skipped.  There is no clean way to avoid this
problem.  However, by repeating the @kbd{u} command, you can eventually get
back to the message that you intended to undelete.@refill

  A deleted message has the @samp{deleted} attribute, and as a result
@samp{deleted} appears in the mode line when the current message is
deleted.  In fact, deleting or undeleting a message is nothing more than
adding or removing this attribute.  @xref{Rmail Labels}.

@node Rmail Inbox, Rmail Files, Rmail Deletion, Rmail
@section Rmail Files and Inboxes
@cindex inbox file

  Unix places incoming mail for you in a file that we call your @dfn{inbox}.
When you start up Rmail, it copies the new messages from your inbox into
your primary mail file, an Rmail file, which also contains other messages
saved from previous Rmail sessions.  It is in this file that you actually
read the mail with Rmail.  This operation is called @dfn{getting new mail}.
It can be repeated at any time using the @kbd{g} key in Rmail.

  There are two reason for having separate Rmail files and inboxes.

@enumerate
@item
The format in which Unix delivers the mail in the inbox is not
adequate for Rmail mail storage.  It has no way to record attributes
(such as @samp{deleted}) or user-specified labels; it has no way to record
old headers and reformatted headers; it has no way to record cached
summary line information.

@item
It is very cumbersome to access an inbox file without danger of losing
mail, because it is necessary to interlock with mail delivery.
Moreover, different Unix systems use different interlocking
techniques.  The strategy of moving mail out of the inbox once and for
all into a separate Rmail file avoids the need for interlocking in all
the rest of Rmail, since only Rmail operates on the Rmail file.
@end enumerate

  When getting new mail, Rmail first copies the new mail from the inbox
file to the Rmail file; then it saves the Rmail file; then it deletes the
inbox file.  This way, a system crash may cause duplication of mail between
the inbox and the Rmail file, but cannot lose mail.

@node Rmail Files, Rmail Labels, Rmail Inbox, Rmail
@section Multiple Mail Files

  Rmail operates by default on your @dfn{primary mail file}, which is named
@file{~/RMAIL} and receives your incoming mail from your system inbox file.
But you can also have other mail files and edit them with Rmail.  These
files can receive mail through their own inboxes, or you can move messages
into them by explicit command in Rmail.

@table @kbd
@item i @var{file} @key{RET}
Read @var{file} into Emacs and run Rmail on it (@code{rmail-input}).

@item M-x set-rmail-inbox-list @key{RET} @var{files} @key{RET}
Specify inbox file names of current Rmail file.

@item g
Merge new mail from current Rmail file's inboxes
(@code{rmail-get-new-mail}).

@item C-u g @var{file}
Merge new mail from inbox file @var{file}.

@item o @var{file} @key{RET}
Append a copy of the current message to the file @var{file},
writing it in Rmail file format (@code{rmail-output-to-rmail-file}).

@item C-o @var{file} @key{RET}
Append a copy of the current message to the file @var{file},
writing it in Unix mail file format (@code{rmail-output}).
@end table

@kindex i (Rmail)
@findex rmail-input
  To run Rmail on a file other than your primary mail file, you may use the
@kbd{i} (@code{rmail-input}) command in Rmail.  This visits the file, puts it in
Rmail mode, and then gets new mail from the file's inboxes if any.

  The file you read with @kbd{i} does not have to be in Rmail file format.
It could also be Unix mail format, or mmdf format; or it could be a mixture
of all three, as long as each message belongs to one of the three formats.
Rmail recognizes all three and converts all the messages to proper Rmail
format before showing you the file.

@findex set-rmail-inbox-list
  Each Rmail file can contain a list of inbox file names; you can specify
this list with @kbd{M-x set-rmail-inbox-list @key{RET} @var{files}
@key{RET}}.  The argument can contain any number of file names, separated
by commas.  It can also be empty, which specifies that this file should
have no inboxes.@refill

@kindex g (Rmail)
@findex rmail-get-new-mail
  If an Rmail file has inboxes, new mail is merged in from the inboxes when
the Rmail file is brought into Rmail, and when the @kbd{g} (@code{rmail-get-new-mail})
command is used.  If the Rmail file specifies no inboxes, then no new mail
is merged in at these times.  A special exception is made for your primary
mail file, in using the standard system inbox for it if it does not specify
any.

  Inboxes usually contain messages in Unix mail format, but they can just
as well contain Rmail or mmdf format messages.  Each message that is not in
Rmail format is converted, just as when an Rmail file is read in.

  To merge mail from a file that is not the usual inbox, give the @kbd{g}
key a numeric argument, as in @kbd{C-u g}.  Then it reads a file name and
merges mail from that file.  The inbox file is not deleted or changed in
any way when @kbd{g} with an argument is used.  This is, therefore, a
general way of merging one file of messages into another.

@kindex o (Rmail)
@findex rmail-output-to-rmail-file
@kindex C-o (Rmail)
@findex rmail-output
  If an Rmail file has no inboxes, how does it get anything in it?  By
explicit @kbd{o} or @kbd{C-o} commands in Rmail, or the like in other mail
processors.

  The @kbd{C-o} (@code{rmail-output}) command in Rmail writes a copy of the current
message into a specified file, in Unix mail file format.  This is useful
for moving messages into files to be read by other mail processors that do
not understand Rmail format.  @kbd{o} (@code{rmail-output-to-rmail-file}) is
another command that writes the message into a file in Rmail format.  This
is the best command to use to move messages between Rmail files.

  If you use @kbd{C-o} to move a message into an Rmail file, nothing bad
happens.  It's true that the Rmail file will contain a message in Unix
format, which is not strictly valid for an Rmail file; but next time Rmail
reads the mail file in, it will recognize the Unix format message and
convert it to Rmail format.  However, using @kbd{o} preserves any labels
the message has (@pxref{Rmail Labels}).

  Copying a message with @kbd{o} or @kbd{C-o} gives the original copy of the
message the @samp{filed} attribute, so that @samp{filed} appears in the mode
line when such a message is current.

@node Rmail Labels, Rmail Summary, Rmail Files, Rmail
@section Labels
@cindex label (Rmail)
@cindex attribute (Rmail)

  Each message can have various @dfn{labels} assigned to it as a means of
classification.  A label has a name; different names mean different labels.
Any given label is either present or absent on a particular message.  A few
label names have standard meanings and are given to messages automatically
by Rmail when appropriate; these special labels are called @dfn{attributes}.
All other labels are assigned by the user.

@table @kbd
@item a @var{label} @key{RET}
Assign the label @var{label} to the current message (@code{rmail-add-label}).
@item k @var{label} @key{RET}
Remove the label @var{label} to the current message (@code{rmail-kill-label}).
@item C-M-n @var{labels} @key{RET}
Move to the next message that has one of the labels @var{labels}
(@code{rmail-next-labeled-message}).
@item C-M-p @var{labels} @key{RET}
Move to the previous message that has one of the labels @var{labels}
(@code{rmail-previous-labeled-message}).
@item C-M-l @var{labels} @key{RET}
Make a summary of all messages containing any of the labels @var{labels}
(@code{rmail-summary-by-labels}).
@end table

@noindent
Specifying an empty string for one these commands means to use the last
label specified for any of these commands.

@kindex a (Rmail)
@kindex k (rmail)
@findex rmail-add-label
@findex rmail-kill-label
  The @kbd{a} (@code{rmail-add-label}) and @kbd{k} (@code{rmail-kill-label}) commands allow
you to assign or remove any label on the current message.  If the @var{label}
argument is empty, it means to assign or remove the same label most
recently assigned or removed.

  Once you have given messages labels to classify them as you wish, there
are two ways to use the labels: in moving, and in summaries.

@kindex C-M-n (Rmail)
@kindex C-M-p (Rmail)
@findex rmail-next-labeled-message
@findex rmail-previous-labeled-message
  The command @kbd{C-M-n @var{label} @key{RET}}
(@code{rmail-next-labeled-message}) moves to the next message that has one
of the labels @var{labels}.  @var{labels} is one or more label names,
separated by commas.  @kbd{C-M-p} (@code{rmail-previous-labeled-message})
is similar, but moves backwards to previous messages.  A preceding numeric
argument to either one serves as a repeat count.@refill

@kindex C-M-l (Rmail)
@findex rmail-summary-by-labels
  The command @kbd{C-M-l @var{labels} @key{RET}}
(@code{rmail-summary-by-labels}) displays a summary containing only the
messages that have at least one of a specified set of messages.  The
argument @var{labels} is one or more label names, separated by commas.
@xref{Rmail Summary}, for information on how summaries are used.@refill

  If the @var{labels} argument to @kbd{C-M-n}, @kbd{C-M-p} or @kbd{C-M-l} is empty, it means
to use the last set of labels specified for any of these commands.

  Some labels such as @samp{deleted} and @samp{filed} have built-in meanings and
are assigned to or removed from messages automatically at appropriate
times; these labels are called @dfn{attributes}.  Here is a list of Rmail
attributes:

@table @samp
@item unseen
Means the message has never been current.  Assigned to messages when
they come from an inbox file, and removed when a message is made
current.
@item deleted
Means the message is deleted.  Assigned by deletion commands and
removed by undeletion commands (@pxref{Rmail Deletion}).
@item filed
Means the message has been copied to some other file.  Assigned by the
file output commands (@pxref{Rmail Files}).
@item answered
Means you have mailed an answer to the message.  Assigned by the @kbd{r}
command (@code{rmail-reply}).  @xref{Rmail Reply}.
@item forwarded
Means you have forwarded the message to other users.  Assigned by the
@kbd{f} command (@code{rmail-forward}).  @xref{Rmail Reply}.
@end table

  All other labels are assigned or removed only by the user, and it is up
to the user to decide what they mean.

@node Rmail Summary, Rmail Reply, Rmail Labels, Rmail
@section Summaries
@cindex summary (Rmail)

  A @dfn{summary} is a buffer containing one line per message that Rmail
can make and display to give you an overview of the mail in an Rmail file.
Each line shows the message number, the sender, the labels, and the
subject.  When the summary buffer is selected, various commands can be used
to select messages by moving in the summary buffer, or delete or undelete
messages.

  A summary buffer applies to a single Rmail file only; if you are
editing multiple Rmail files, they have separate summary buffers.  The
summary buffer name is made by appending @samp{-summary} to the Rmail buffer's
name.  Only one summary buffer will be displayed at a time unless you make
several windows and select the summary buffers by hand.

@subsection Making Summaries

@table @kbd
@item h
@itemx C-M-h
Summarize all messages (@code{rmail-summary}).
@item l @var{labels} @key{RET}
@itemx C-M-l @var{labels} @key{RET}
Summarize message that have one or more of the specified labels
(@code{rmail-summary-by-labels}).
@item C-M-r @var{rcpts} @key{RET}
Summarize messages that have one or more of the specified recipients
(@code{rmail-summary-by-recipients})
@end table

@kindex h
@findex rmail-summary
  The @kbd{h} or @kbd{C-M-h} (@code{rmail-summary}) command fills the summary buffer
for the current Rmail file with a summary of all the messages in the file.
It then displays and selects the summary buffer in another window.

@kindex l
@kindex C-M-l
@findex rmail-summary-by-labels
  @kbd{C-M-l @var{labels} @key{RET}} (@code{rmail-summary-by-labels}) makes
a partial summary mentioning only the messages that have one or more of the
labels @var{labels}.  @var{labels} should contain label names separated by
commas.@refill

@kindex C-M-r
@findex rmail-summary-by-recipients
  @kbd{C-M-r @var{rcpts} @key{RET}} (@code{rmail-summary-by-recipients})
makes a partial summary mentioning only the messages that have one or more
of the recipients @var{rcpts}.  @var{rcpts} should contain mailing
addresses separated by commas.@refill

  Note that there is only one summary buffer for any Rmail file; making one
kind of summary discards any previously made summary.  Also, summary
buffers are not updated automatically when the Rmail buffer is changed.

@subsection Editing in Summaries

  Summary buffers are given the major mode Rmail Summary mode, which
provides the following special commands:

@table @kbd
@item j
Select the message described by the line that point is on
(@code{rmail-summary-goto-msg}).
@item C-n
Move to next line and select its message in Rmail
(@code{rmail-summary-next-all}).
@item C-p
Move to previous line and select its message
(@code{rmail-summary-previous-all}).
@item n
Move to next line, skipping lines saying `deleted', and select its
message (@code{rmail-summary-next-msg}).
@item p
Move to previous line, skipping lines saying `deleted', and select
its message (@code{rmail-summary-previous-msg}).
@item d
Delete the current line's message, then do like @kbd{n}
(@code{rmail-summary-delete-forward}).
@item u
Undelete and select this message or the previous deleted message in
the summary (@code{rmail-summary-undelete}).
@item @key{SPC}
Scroll the other window (presumably Rmail) forward
(@code{rmail-summary-scroll-msg-up}).
@item @key{DEL}
Scroll the other window backward (@code{rmail-summary-scroll-msg-down}).
@item x
Kill the summary window (@code{rmail-summary-exit}).
@item q
Exit Rmail (@code{rmail-summary-quit}).
@end table

@kindex C-n (Rmail summary)
@kindex C-p (Rmail summary)
@findex rmail-summary-next-all
@findex rmail-summary-previous-all
  The keys @kbd{C-n} and @kbd{C-p} are modified in Rmail Summary mode so that in
addition to moving point in the summary buffer they also cause the line's
message to become current in the associated Rmail buffer.  That buffer is
also made visible in another window if it is not already so.

@kindex n (Rmail summary)
@kindex p (Rmail summary)
@findex rmail-summary-next-msg
@findex rmail-summary-previous-msg
  @kbd{n} and @kbd{p} are similar to @kbd{C-n} and @kbd{C-p}, but skip
lines that say `message deleted'.  They are like the @kbd{n} and @kbd{p}
keys of Rmail itself.  Note, however, that in a partial summary these
commands move only among the message listed in the summary.@refill

@kindex j (Rmail summary)
@findex rmail-summary-goto-msg
  The other Emacs cursor motion commands are not changed in Rmail Summary
mode, so it is easy to get the point on a line whose message is not
selected in Rmail.  This can also happen if you switch to the Rmail window
and switch messages there.  To get the Rmail buffer back in sync with the
summary, use the @kbd{j} (@code{rmail-summary-goto-msg}) command, which selects
in Rmail the message of the current summary line.

@kindex d (Rmail summary)
@kindex u (Rmail summary)
@findex rmail-summary-delete-forward
@findex rmail-summary-undelete
  Deletion and undeletion can also be done from the summary buffer.  They
always work based on where point is located in the summary buffer, ignoring
which message is selected in Rmail.  @kbd{d} (@code{rmail-summary-delete-forward})
deletes the current line's message, then moves to the next line whose
message is not deleted and selects that message.  The inverse of this is
@kbd{u} (@code{rmail-summary-undelete}), which moves back (if necessary) to a line
whose message is deleted, undeletes that message, and selects it in Rmail.

@kindex SPC (Rmail summary)
@kindex DEL (Rmail summary)
@findex rmail-summary-scroll-down
@findex rmail-summary-scroll-up
  When moving through messages with the summary buffer, it is convenient to
be able to scroll the message while remaining in the summary window.
The commands @key{SPC} (@code{rmail-summary-scroll-up}) and @key{DEL}
(@code{rmail-summary-scroll-down}) do this.  They scroll the message just
as those same keys do when the Rmail buffer is selected.@refill

@kindex x (Rmail summary)
@findex rmail-summary-exit
  When you are finished using the summary, type @kbd{x} (@code{rmail-summary-exit})
to kill the summary buffer's window.

@kindex q (Rmail summary)
@findex rmail-summary-quit
  You can also exit Rmail while in the summary.  @kbd{q} (@code{rmail-summary-quit})
kills the summary window, then saves the Rmail file and switches to another
buffer.

@node Rmail Reply, Rmail Editing, Rmail Summary, Rmail
@section Sending Replies

  Rmail has several commands that use Mail mode to send outgoing mail.
@xref{Sending Mail}, for information on using Mail mode.  What are
documented here are the special commands of Rmail for entering Mail mode.
Note that the usual keys for sending mail, @kbd{C-x m} and @kbd{C-x 4 m},
are available in Rmail mode and work just as they usually do.@refill

@table @kbd
@item m
Send a message (@code{rmail-mail}).
@item c
Continue editing already started outgoing message @*(@code{rmail-continue}).
@item r
Send a reply to the current Rmail message (@code{rmail-reply}).
@item f
Forward current message to other users (@code{rmail-forward}).
@end table

@kindex r (Rmail)
@findex rmail-reply
@vindex rmail-dont-reply-to
@cindex reply to a message
  The most common reason to send a message while in Rmail is to reply to
the message you are reading.  To do this, type @kbd{r}
(@code{rmail-reply}).  This displays the @samp{*mail*} buffer in another
window, much like @kbd{C-x 4 m}, but preinitializes the @samp{Subject},
@samp{To}, @samp{CC} and @samp{In-reply-to} header fields based on the
message being replied to.  The @samp{To} field is given the sender of that
message, and the @samp{CC} gets all the recipients of that message (but
recipients that match elements of the list @code{rmail-dont-reply-to} are
omitted; by default, this list contains your own mailing address).@refill

  Once you have initialized the @samp{*mail*} buffer this way, sending the
mail goes as usual (@pxref{Sending Mail}).  You can edit the presupplied
header fields if they are not right for you.

@kindex C-c y
@findex mail-yank-original
  One additional Mail mode command is available when mailing is invoked
from Rmail: @kbd{C-c y} (@code{mail-yank-original}) inserts into the outgoing
message a copy of the current Rmail message; normally this is the message
you are replying to, but you can also switch to the Rmail buffer, select a
different message, switch back, and yank new current message.  Normally the
yanked message is indented four spaces and has most header fields deleted
from it; an argument to @kbd{C-c y} specifies the amount to indent, and
@kbd{C-u C-c y} does not indent at all and does not delete any header
fields.@refill

@kindex f (Rmail)
@findex rmail-forward
@cindex forward a message
  Another frequent reason to send mail in Rmail is to forward the current
message to other users.  @kbd{f} (@code{rmail-forward}) makes this easy by
preinitializing the @samp{*mail*} buffer with the current message as the
text, and a subject designating a forwarded message.  All you have to do is
fill in the recipients and send.@refill

@kindex m (Rmail)
@findex rmail-mail
  The @kbd{m} (@code{rmail-mail}) command is used to start editing an
outgoing message that is not a reply.  It leaves the header fields empty.
Its only difference from @kbd{C-x 4 m} is that it makes the Rmail buffer
accessible for @kbd{C-c y}, just as @kbd{r} does.  Thus, @kbd{m} can be
used to reply to or forward a message; it can do anything @kbd{r} or @kbd{f}
can do.@refill

@kindex c (Rmail)
@findex rmail-continue
  The @kbd{c} (@code{rmail-continue}) command resumes editing the
@samp{*mail*} buffer, to finish editing an outgoing message you were
already composing, or to alter a message you have sent.@refill

@node Rmail Editing, Rmail Digest, Rmail Reply, Rmail
@section Editing Within a Message

  Rmail mode provides a few special commands for moving within and editing
the current message.  In addition, the usual Emacs commands are available
(except for a few, such as @kbd{C-r} and @kbd{C-M-h}, that are redefined by Rmail for
other purposes).  However, the Rmail buffer is normally read-only, and to
alter it you must use the Rmail command @kbd{C-r} described below.

@table @kbd
@item t
Toggle display of original headers (@code{rmail-toggle-headers}).
@item C-r
Edit current message (@code{rmail-edit-current-message}).
@end table

@kindex t (Rmail)
@findex rmail-toggle-header
@vindex rmail-ignored-headers
  Rmail reformats the header of each message before displaying it.
Normally this involves deleting most header fields, on the grounds that
they are not interesting.  The variable @code{rmail-ignored-headers} should
contain a regexp that matches the header fields to discard in this way.
The original headers are saved permanently, and to see what they look like,
use the @kbd{t} (@code{rmail-toggle-headers}) command.  This discards the reformatted
headers of the current message and displays it with the original headers.
Repeating @kbd{t} reformats the message again.  Selecting the message again
also reformats.

@kindex C-r (Rmail)
@findex rmail-edit-current-message
  The Rmail buffer is normally read only, and most of the characters you
would type to modify it (including most letters) are redefined as Rmail
commands.  This is usually not a problem since it is rare to want to change
the text of a message.  When you do want to do this, the way is to type
@kbd{C-r} (@code{rmail-edit-current-message}), which changes from Rmail mode into
Rmail Edit mode, another major mode which is nearly the same as Text mode.
The mode line illustrates this change.

  In Rmail Edit mode, letters insert themselves as usual and the Rmail
commands are not available.  When you are finished editing the message and
are ready to go back to Rmail, type @kbd{C-c C-c}, which switches back to
Rmail mode.  Alternatively, you can return to Rmail mode but cancel all the
editing that you have done by typing @kbd{C-c C-]}.

@vindex rmail-edit-mode-hook
  Entering Rmail Edit mode calls with no arguments the value of the variable
@code{text-mode-hook}, if that value exists and is not @code{nil}; then it
does the same with the variable @code{rmail-edit-mode-hook}.

@node Rmail Digest,, Rmail Editing, Rmail
@section Digest Messages
@cindex digest message
@cindex undigestify

  A @dfn{digest message} is a message which exists to contain and carry
several other messages.  Digests are used on moderated mailing lists; all
the messages that arrive for the list during a period of time such as one
day are put inside a single digest which is then sent to the subscribers.
Transmitting the single digest uses much less computer time than
transmitting the individual messages even though the total size is the
same, because the per-message overhead in network mail transmission is
considerable.

@findex undigestify-rmail-message
  When you receive a digest message, the most convenient way to read it is
to @dfn{undigestify} it: to turn it back into many individual messages.
Then you can read and delete the individual messages as it suits you.

  To undigestify a message, select it and then type @kbd{M-x
undigestify-rmail-message}.  This copies each submessage as a separate
Rmail message and inserts them all following the digest.  The digest
message itself is flagged as deleted.

@iftex
@chapter Miscellaneous Commands

  This chapter contains several brief topics that do not fit anywhere else.

@end iftex
@node Recursive Edit, Narrowing, Rmail, Top
@section Recursive Editing Levels
@cindex recursive edit

  A @dfn{recursive edit} is a situation in which you are using Emacs
commands to perform arbitrary editing while in the middle of another Emacs
command.  For example, when you type @kbd{C-r} inside of a @code{query-replace},
you enter a recursive edit in which you can change the current buffer.  On
exiting from the recursive edit, you go back to the @code{query-replace}.

@kindex C-M-c
@findex exit-recursive-edit
@cindex exiting
  @dfn{Exiting} the recursive edit means returning to the unfinished
command, which continues execution.  For example, exiting the recursive
edit requested by @kbd{C-r} in @code{query-replace} causes query replacing
to resume.  Exiting is done with @kbd{C-M-c} (@code{exit-recursive-edit}).

@kindex C-]
@findex abort-recursive-edit
  You can also @dfn{abort} the recursive edit.  This is like exiting, but
also quits the unfinished command immediately.  Use the command @kbd{C-]}
(@code{abort-recursive-edit}) for this.  @xref{Quitting}.

  The mode line shows you when you are in a recursive edit, by displaying
square brackets around the parentheses that always surround the major and
minor mode names.  Every window's mode line shows this, in the same way,
since being in a recursive edit is true regardless of what buffer is
selected.

@findex top-level
  It is possible to be in recursive edits within recursive edits.  For
example, after typing @kbd{C-r} in a @code{query-replace}, you might type a
command that entered the debugger.  In such circumstances, two or more sets
of square brackets appear in the mode line.  Exiting the inner recursive
edit (such as, with the debugger @kbd{c} command) would resume the command
where it called the debugger.  After the end of this command, you would be
able to exit the first recursive edit.  Aborting also gets out of only one
level of recursive edit; it returns immediately to the command level of the
previous recursive edit.  So you could immediately abort that one too.

  Alternatively, the command @kbd{M-x top-level} aborts all levels of
recursive edits, returning immediately to the top level command reader.

  The text being edited inside the recursive edit need not be the same text
that you were editing at top level.  It depends on what the recursive edit
is for.  If the command that invokes the recursive edit selects a different
buffer first, that is the buffer you will edit recursively.  In any case,
you can switch buffers within the recursive edit in the normal manner (as
long as the buffer-switching keys have not been rebound).  You could
probably do all the rest of your editing inside the recursive edit,
visiting files and all.  But this could have surprising effects (such as
stack overflow) from time to time.  So remember to exit or abort the
recursive edit when you no longer need it.

  In general, GNU Emacs tries to avoid using recursive edits.  It is
usually preferable to allow the user to switch among the possible editing
modes in any order he likes.  With recursive edits, the only way to get to
another state is to go ``back'' to the state that the recursive edit was
invoked from.

@node Narrowing, Shell, Recursive Edit, Top
@section Narrowing
@cindex widening
@cindex restriction
@cindex narrowing

  @dfn{Narrowing} means focusing in on some portion of the buffer, making
the rest temporarily invisible and inaccessible.  Cancelling the narrowing,
and making the entire buffer once again visible, is called @dfn{widening}.
The amount of narrowing in effect in a buffer at any time is called the
buffer's @dfn{restriction}.

@c WideCommands
@table @kbd
@item C-x n
Narrow down to between point and mark (@code{narrow-to-region}).
@item C-x w
Widen to make the entire buffer visible again (@code{widen}).
@end table

  When you have narrowed down to a part of the buffer, that part appears to
be all there is.  You can't see the rest, you can't move into it (motion
commands won't go outside the visible part), you can't change it in any
way.  However, it is not gone, and if you save the file all the invisible
text will be saved.  In addition to sometimes making it easier to
concentrate on a single subroutine or paragraph by eliminating clutter,
narrowing can be used to restrict the range of operation of a replace
command or repeating keyboard macro.  The word @samp{Narrow} appears in the
mode line whenever narrowing is in effect.

@kindex C-x n
@findex narrow-to-region
  The primary narrowing command is @kbd{C-x n} (@code{narrow-to-region}).
It sets the current buffer's restrictions so that the text in the current
region remains visible but all text before the region or after the region
is invisible.  Point and mark do not change.

  Because narrowing can easily confuse users who do not understand it,
@code{narrow-to-region} is normally a disabled command.  Attempting to use
this command asks for confirmation and gives you the option of enabling it;
once you enable the command, confirmation will no longer be required for
it.  @xref{Disabling}.

@kindex C-x w
@findex widen
  The way to undo narrowing is to widen with @kbd{C-x w} (@code{widen}).
This makes all text in the buffer accessible again.

  You can get information on what part of the buffer you are narrowed down
to using the @code{C-x =} command.  @xref{Position Info}.

@node Shell, Hardcopy, Narrowing, Top
@section Running Shell Commands from Emacs
@cindex subshell
@cindex shell commands

  Emacs has commands for passing single command lines to inferior shell
processes; it can also run a shell interactively with input and output to
an Emacs buffer @samp{*shell*}.

@table @kbd
@item M-!
Run a specified shell command line and display the output
(@code{shell-command}).
@item M-|
Run a specified shell command line with region contents as input;
optionally replace the region with the output
(@code{shell-command-on-region}).
@item M-x shell
Run a subshell with input and output through an Emacs buffer.
You can then give commands interactively.
@end table

@subsection Single Shell Commands

@kindex M-!
@findex shell-command
  @kbd{M-!} (@code{shell-command}) reads a line of text using the
minibuffer and creates an inferior shell to execute the line as a command.
Standard input from the command comes from the null device.  If the shell
command produces any output, the output goes into an Emacs buffer named
@samp{*Shell Command Output*}, which is displayed in another window but not
selected.  A numeric argument, as in @kbd{M-1 M-!}, directs this command to
insert any output into the current buffer.  In that case, point is left
before the output and the mark is set after the output.

@kindex M-|
@findex shell-command-on-region
  @kbd{M-|} (@code{shell-command-on-region}) is like @kbd{M-!} but passes
the contents of the region as input to the shell command, instead of no
input.  If a numeric argument is used, meaning insert output in the current
buffer, then the old region is deleted first and the output replaces it as
the contents of the region.@refill

@vindex shell-file-name
@cindex environment
  Both @kbd{M-!} and @kbd{M-|} use @code{shell-file-name} to specify the
shell to use.  This variable is initialized based on your @code{SHELL}
environment variable when Emacs is started.  If the file name does not
specify a directory, the directories in the list @code{exec-path} are
searched; this list is initialized based on the environment variable
@code{PATH} when Emacs is started.  Your @file{.emacs} file can override
either or both of these default initializations.@refill

  With @kbd{M-!} and @kbd{M-|}, Emacs has to wait until the shell command
completes.  You can quit with @kbd{C-g}; that terminates the shell command.

@subsection Interactive Inferior Shell

@findex M-x shell
  To run a subshell interactively, putting its typescript in an Emacs
buffer, use @kbd{M-x shell}.  This creates (or reuses) a buffer named
@samp{*shell*} and runs a subshell with input coming from and output going
to that buffer.  That is to say, any ``terminal output'' from the subshell
will go into the buffer, advancing point, and any ``terminal input'' for
the subshell comes from text in the buffer.  To give input to the subshell,
go to the end of the buffer and type the input, terminated by @key{RET}.

  Emacs does not wait for the subshell to do anything.  You can switch
windows or buffers and edit them while the shell is waiting, or while it is
running a command.  Output from the subshell waits until Emacs has time to
process it; this happens whenever Emacs is waiting for keyboard input or
for time to elapse.

@vindex explicit-shell-file-name
  The file name used to load the subshell is the value of the variable
@code{explicit-shell-file-name}, if that is non-@code{nil}.  Otherwise, the
environment variable @code{ESHELL} is used, or the environment variable
@code{SHELL} if there is no @code{ESHELL}.  If no directory is specified,
the directories in the list @code{exec-path} are searched; see
above.@refill

  As soon as the subshell is started, it is sent as input the contents of
the file @file{~/.emacs_@var{shellname}}, if that file exists, where
@var{shellname} is the name of the file that the shell was loaded from.
For example, if you use @code{csh}, the file sent to it is
@file{~/.emacs_csh}.@refill

@cindex Shell mode
  The shell buffer uses Shell mode, which defines several special keys
attached to the @kbd{C-c} prefix.  They are chosen to resemble the usual
editing and job control characters present in shells that are not under
Emacs, except that you must type @kbd{C-c} first.  Here is a complete list
of the special key bindings of Shell mode:

@kindex RET
@kindex C-c C-d
@kindex C-c C-u
@kindex C-c C-w
@kindex C-c C-c
@kindex C-c C-z
@kindex C-c C-\
@kindex C-c C-o
@kindex C-c C-r
@kindex C-c C-y
@findex send-shell-input
@findex shell-send-eof
@findex interrupt-shell-subjob
@findex stop-shell-subjob
@findex quit-shell-subjob
@findex kill-output-from-shell
@findex show-output-from-shell
@findex copy-last-shell-input
@vindex shell-prompt-pattern
@table @kbd
@item @key{RET}
At end of buffer send line as input; otherwise, copy current line to end of
buffer and send it (@code{send-shell-input}).  When a line is copied, any
text at the beginning of the line that matches the variable
@code{shell-prompt-pattern} is left out; this variable's value should be a
regexp string that matches the prompts that you use in your subshell.
@item C-c C-d
Send end-of-file as input, probably causing the shell or its current
subjob to finish (@code{shell-send-eof}).
@item C-c C-u
Kill all text that has yet to be sent as input (@code{kill-shell-input}).
@item C-c C-w
Kill a word before point (@code{backward-kill-word}).
@item C-c C-c
Interrupt the shell or its current subjob if any
(@code{interrupt-shell-subjob}).
@item C-c C-z
Stop the shell or its current subjob if any (@code{stop-shell-subjob}).
@item C-c C-\
Send quit signal to the shell or its current subjob if any
(@code{quit-shell-subjob}).
@item C-c C-o
Delete last batch of output from shell (@code{kill-output-from-shell}).
@item C-c C-r
Scroll top of last batch of output to top of window
(@code{show-output-from-shell}).
@item C-c C-y
Copy the previous bunch of shell input, and insert it into the
buffer before point (@code{copy-last-shell-input}).  No final newline
is inserted, and the input copied is not resubmitted until you type
@key{RET}.
@end table

@vindex shell-pushd-regexp
@vindex shell-popd-regexp
@vindex shell-cd-regexp
  @code{cd}, @code{pushd} and @code{popd} commands given to the inferior
shell are watched by Emacs so it can keep the @samp{*shell*} buffer's
default directory the same as the shell's working directory.  These
commands are recognized syntactically by examining lines of input that are
sent.  If you use aliases for these commands, you can tell Emacs to
recognize them also.  For example, if the value of the variable
@code{shell-pushd-regexp} matches the beginning of a shell command line,
that line is regarded as a @code{pushd} command.  Change this variable when
you add aliases for @samp{pushd}.  Likewise, @code{shell-popd-regexp} and
@code{shell-cd-regexp} are used to recognize commands with the meaning of
@samp{popd} and @samp{cd}.  These commands are recognized only at the
beginning of a shell command line.@refill

@node Hardcopy, Dissociated Press, Shell, Top
@section Hardcopy Output
@cindex hardcopy

@table @kbd
@item M-x print-buffer
Print hardcopy of current buffer using Unix command @code{lpr -p}.
This makes page headings containing the file name and page number.
@item M-x lpr-buffer
Print hardcopy of current buffer using Unix command @code{lpr}.
This makes no page headings.
@item M-x print-region
Like @code{print-buffer} but prints only the current region.
@item M-x lpr-region
Like @code{lpr-buffer} but prints only the current region.
@end table

@findex print-buffer
@findex print-region
@findex lpr-buffer
@findex lpr-region
@vindex lpr-switches
  All the hardcopy commands pass extra switches to the @code{lpr} program
based on the value of the variable @code{lpr-switches}.  Its value should
be a list of strings, each string a switch starting with @samp{-}.

@node Dissociated Press, Amusements, Hardcopy, Top
@section Dissociated Press

@findex dissociated-press
  @kbd{M-x dissociated-press} is a command for scrambling a file of text
either word by word or character by character.  Starting from a buffer of
straight English, it produces extremely amusing output.  The input comes
from the current Emacs buffer.  Dissociated Press writes its output in a
buffer named @samp{*Dissociation*}, and redisplays that buffer after every
couple of lines (approximately) to facilitate reading it.

  @code{dissociated-press} asks every so often whether to continue
operating.  Answer @kbd{n} to stop it.  You can also stop at any time by
typing @kbd{C-g}.  The dissociation output remains in the @samp{*Dissociation*}
buffer for you to copy elsewhere if you wish.

@cindex presidentagon
  Dissociated Press operates by jumping at random from one point in the
buffer to another.  In order to produce plausible output rather than
gibberish, it insists on a certain amount of overlap between the end of one
run of consecutive words or characters and the start of the next.  That is,
if it has just printed out `president' and then decides to jump to a
different point in the file, it might spot the `ent' in `pentagon' and
continue from there, producing `presidentagon'.  Long sample texts produce
the best results.

@cindex againformation
  A positive argument to @kbd{M-x Dissociated Press} tells it to operate
character by character, and specifies the number of overlap characters.  A
negative argument tells it to operate word by word and specifies the number
of overlap words.  In this mode, whole words are treated as the elements to
be permuted, rather than characters.  No argument is equivalent to an
argument of two.  For your againformation, the output goes only into the
buffer @samp{*Dissociation*}.  The buffer you start with is not changed.

@cindex Markov chain
@cindex ignoriginal
@cindex techniquitous
  Dissociated Press produces nearly the same results as a Markov chain
based on a frequency table constructed from the sample text.  It is,
however, an independent, ignoriginal invention.  Dissociated Press
techniquitously copies several consecutive characters from the sample
between random choices, whereas a Markov chain would choose randomly for
each word or character.  This makes for more plausible sounding results,
and runs faster.

@cindex outragedy
@cindex buggestion
@cindex properbose
  It is a mustatement that too much use of Dissociated Press can be a
developediment to your real work.  Sometimes to the point of outragedy.
And keep dissociwords out of your documentation, if you want it to be well
userenced and properbose.  Have fun.  Your buggestions are welcome.

@node Amusements, Customization, Dissociated Press, Top
@section Other Amusements
@cindex boredom
@findex hanoi

  If you are a little bit bored, you can try @kbd{M-x hanoi}.  If you are
considerably bored, give it a numeric argument.  If you are very very
bored, try an argument of 9.

  When you are frustrated, try the famous Eliza program.  Just do
@kbd{M-x doctor}.

@node Customization, Quitting, Amusements, Top
@chapter Customization
@cindex customization

  This chapter talks about various topics relevant to adapting the
behavior of Emacs in minor ways.

  All kinds of customization affect only the particular Emacs job that you
do them in.  They are completely lost when you kill the Emacs job, and have
no effect on other Emacs jobs you may run at the same time or later.  The
only way an Emacs job can affect anything outside of it is by writing a
file; in particular, the only way to make a customization `permanent' is to
put something in your @file{.emacs} file or other appropriate file to do the
customization in each session.  @xref{Init File}.

@menu
* Minor Modes::     Each minor mode is one feature you can turn on
                     independently of any others.
* Variables::       Many Emacs commands examine Emacs variables
                     to decide what to do; by setting variables,
                     you can control their functioning.
* Keyboard Macros:: A keyboard macro records a sequence of keystrokes
                     to be replayed with a single command.
* Key Bindings::    The keymaps say what command each key runs.
                     By changing them, you can "redefine keys".
* Syntax::          The syntax table controls how words and expressions
                     are parsed.
* Init File::       How to write common customizations in the @file{.emacs} file.
@end menu

@node Minor Modes, Variables, Customization, Customization
@section Minor Modes
@cindex minor modes

@cindex mode line
  Minor modes are options which you can use or not.  For example, Auto Fill
mode is a minor mode in which @key{SPC} breaks lines between words as you
type.  All the minor modes are independent of each other and of the
selected major mode.  Most minor modes say in the mode line when they are
on; for example, @samp{Fill} in the mode line means that Auto Fill mode is
on.

  Append @code{-mode} to the name of a minor mode to get the name of a
command function that turns the mode on or off.  Thus, the command to
enable or disable Auto Fill mode is called @kbd{M-x auto-fill-mode}.  These
commands are usually invoked with @kbd{M-x}, but you can bind keys to them
if you wish.  With no argument, the function turns the mode on if it was
off and off if it was on.  This is known as @dfn{toggling}.  A positive
argument always turns the mode on, and an explicit zero argument or a
negative argument always turns it off.

@cindex Auto Fill mode
  Auto Fill mode allows you to enter filled text without breaking lines
explicitly.  Emacs inserts newlines as necessary to prevent lines from
becoming too long.  @xref{Filling}.

@cindex Overwrite mode
  Overwrite mode causes ordinary printing characters to replace existing
text instead of shoving it over.  For example, if the point is in front of
the @samp{B} in @samp{FOOBAR}, then in Overwrite mode typing a @kbd{G}
changes it to @samp{FOOGAR}, instead of making it @samp{FOOGBAR} as
usual.@refill

@cindex Abbrev mode
  Abbrev mode allows you to define abbreviations that automatically expand
as you type them.  For example, @samp{amd} might expand to @samp{abbrev
mode}.  @xref{Abbrevs}, for full information.

@node Variables, Keyboard Macros, Minor Modes, Customization
@section Variables
@cindex variable
@cindex option

  A @dfn{variable} is a Lisp symbol which has a value.  The symbol's name
is also called the name of the variable.  Variable names can contain any
characters, but conventionally they are chosen to be words separated by
hyphens.  A variable can have a documentation string which describes what
kind of value it should have and how the value will be used.

  Lisp allows any variable to have any kind of value, but most variables
that Emacs uses require a value of a certain type.  Often the value should
always be a string, or should always be a number.  Sometimes we say that a
certain feature is turned on if a variable is ``non-@code{nil},'' meaning
that if the variable's value is @code{nil}, the feature is off, but the
feature is on for @i{any} other value.  The conventional value to use to
turn on the feature---since you have to pick one particular value when you
set the variable---is @code{t}.

  Emacs uses many Lisp variables for internal recordkeeping, as any Lisp
program must, but the most interesting variables for you are the ones that
exist for the sake of customization.  Emacs does not (usually) change the
values of these variables; instead, you set the values, and thereby alter
and control the behavior of certain Emacs commands.  These variables are
called @dfn{options}.  Most options are documented in this manual, and
appear in the Variable Index (@pxref{Variable Index}).

  One example of a variable which is an option is @code{fill-column}, which
specifies the position of the right margin (as a number of characters from
the left margin) to be used by the fill commands (@pxref{Filling}).

@menu
* Examining::           Examining or setting one variable's value.
* Edit Options::        Examining or editing list of all variables' values.
* Locals::              Per-buffer values of variables.
* File Variables::      How files can specify variable values.
@end menu

@node Examining, Edit Options, Variables, Variables
@subsection Examining and Setting Variables

@table @kbd
@item C-h v
@itemx M-x describe-variable
Print the value and documentation of a variable.
@item M-x set-variable
Change the value of a variable.
@end table

@kindex C-h v
@findex describe-variable
  To examine the value of a single variable, use @kbd{C-h v}
(@code{describe-variable}), which reads a variable name using the
minibuffer, with completion.  It prints both the value and the
documentation of the variable.

@example
C-h v fill-column @key{RET}
@end example
@noindent
prints something like
@smallexample
fill-column's value is 75

Documentation:
*Column beyond which automatic line-wrapping should happen.
Separate value in each buffer.
@end smallexample

@cindex option
@noindent
The star at the beginning of the documentation indicates that this variable
is an option.  @kbd{C-h v} is not restricted to options; they allow any
variable name.

@findex set-variable
  If you know which option you want to set, you can set it using @kbd{M-x
set-variable}.  This reads the variable name with the minibuffer (with
completion), and then reads a Lisp expression for the new value using the
minibuffer a second time.  For example,

@example
M-x set-variable @key{RET} fill-column @key{RET} 75 @key{RET}
@end example

@noindent
sets @code{fill-column} to 75.

@node Edit Options, Locals, Examining, Variables
@subsection Editing Variable Values

@table @kbd
@item M-x list-options
Display a buffer listing names, values and documentation of all options.
@item M-x edit-options
Change option values by editing a list of options.
@end table

@findex list-options
  @kbd{M-x list-options} displays a list of all Emacs option variables, in
an Emacs buffer named @samp{*List Options*}.  Each option is shown with its
documentation and its current value.  Here is what a portion of it might
look like:

@smallexample
;; exec-path:
	("." "/usr/local/bin" "/usr/ucb" "/bin" "/usr/bin" "/u2/emacs/etc")
*List of directories to search programs to run in subprocesses.
Each element is a string (directory name) or nil (try default directory).
;;
;; fill-column:
	75
*Column beyond which automatic line-wrapping should happen.
Separate value in each buffer.
;;
;; find-file-hook:
	nil
*If non-nil specifies a function to be called after a buffer
is found or reverted from a file.
The buffer's local variables (if any) will have been processed
before the function is called.
;;
@end smallexample

@findex edit-options
  @kbd{M-x edit-options} goes one step farther and selects the @samp{*List
Options*} buffer; this buffer uses the major mode Options mode, which
provides commands that allow you to point at an option and change its
value:

@table @kbd
@item s
Set the variable point is in or near to a new value read using the
minibuffer.
@item x
Toggle the variable point is in or near: if the value was @code{nil},
it becomes @code{t}; otherwise it becomes @code{nil}.
@item 1
Set the variable point is in or near to @code{t}.
@item 0
Set the variable point is in or near to @code{nil}.
@item n
@itemx p
Move to the next or previous variable.
@end table

@node Locals, File Variables, Edit Options, Variables
@subsection Local Variables

@table @kbd
@item M-x make-local-variable
Make a variable have a local value in the current buffer.
@item M-x kill-local-variable
Make a variable use its global value in the current buffer.
@end table

@cindex local variables
  Any variable can be made @dfn{local} to a specific Emacs buffer.  This
means that its value in that buffer is independent of its value in other
buffers.  A few variables are always local in every buffer.  Every other
Emacs variable has a @dfn{global} value which is in effect in all buffers
that have not made the variable local.

  Major modes always make the variables they set local to the buffer.
This is why changing major modes in one buffer has no effect on other
buffers.

@findex make-local-variable
  @kbd{M-x make-local-variable} reads the name of a variable and makes it
local to the current buffer.  Further changes in this buffer will not
affect others, and further changes in the global value will not affect this
buffer.

@findex kill-local-variable
  @kbd{M-x kill-local-variable} reads the name of a variable and makes it
cease to be local to the current buffer.  The global value of the variable
henceforth is in effect in this buffer.  Setting the major mode kills all
the local variables of the buffer, except for those variables that are
always local to every buffer.

@node File Variables,, Locals, Variables
@subsection Local Variables in Files
@cindex local variables in files

  A file can contain a @dfn{local variables list}, which specifies the
values to use for certain Emacs variables when that file is edited.
Visiting the file checks for a local variables list and makes each variable
in the list local to the buffer in which the file is visited, with the
value specified in the file.

  A local variables list goes near the end of the file, in the last page.
(It is often best to put it on a page by itself.)  The local variables list
starts with a line containing the string @samp{Local Variables:}, and ends
with a line containing the string @samp{End:}.  In between come the
variable names and values, one set per line, as @samp{@var{variable}:@:
@var{value}}.  The @var{value}s are not evaluated; they are used literally.

  The line which starts the local variables list does not have to say just
@samp{Local Variables:}.  If there is other text before @samp{Local
Variables:}, that text is called the @dfn{prefix}, and if there is other
text after, that is called the @dfn{suffix}.  If these are present, each
entry in the local variables list should have the prefix before it and the
suffix after it.  This includes the @samp{End:} line.  The prefix and
suffix are included to disguise the local variables list as a comment so
that the compiler or text formatter will not be perplexed by it.  If you do
not need to disguise the local variables list as a comment in this way, do
not bother with a prefix or a suffix.@refill

  Two ``variable'' names are special in a local variables list: a value for
the variable @code{mode} really sets the major mode, and a value for the
variable @code{eval} is simply evaluated as an expression and the value is
ignored.  These are not real variables; setting such variables in any other
context has no such effect.  If @code{mode} is used in a local variables
list, it should be the first entry in the list.

Here is an example of a local variables list:
@example
;;; Local Variables: ***
;;; mode:lisp ***
;;; comment-column:0 ***
;;; comment-start: ";;; "  ***
;;; comment-end:"***" ***
;;; End: ***
@end example

  Note that the prefix is @samp{;;; } and the suffix is @samp{ ***}.  Note also
that comments in the file begin with and end with the same strings.
Presumably the file contains code in a language which is like Lisp
(like it enough for Lisp mode to be useful) but in which comments start
and end in that way.  The prefix and suffix are used in the local
variables list to make the list appear as comments when the file is read
by the compiler or interpreter for that	language.

  The start of the local variables list must be no more than 3000
characters from the end of the file, and must be in the last page if the
file is divided into pages.  Otherwise, Emacs will not notice it is there.
The purpose of this is so that a stray @samp{Local Variables:}@: not in the
last page does not confuse Emacs, and so that visiting a long file that is
all one page and has no local variables list need not take the time to
search the whole file.

  You may be tempted to try to turn on Auto Fill mode with a local variable
list.  That is a mistake.  The choice of Auto Fill mode or not is a matter
of individual taste, not a matter of the contents of particular files.
If you want to use Auto Fill, set up major mode hooks with your @file{.emacs}
file to turn it on (when appropriate) for you alone (@pxref{Init File}).
Don't try to use a local variable list that would impose your taste on
everyone.

@node Keyboard Macros, Key Bindings, Variables, Customization
@section Keyboard Macros

@cindex keyboard macros
  A @dfn{keyboard macro} is a command defined by the user to abbreviate a
sequence of keys.  For example, if you discover that you are about to type
@kbd{C-n C-d} forty times, you can speed your work by defining a keyboard
macro to do @kbd{C-n C-d} and calling it with a repeat count of forty.

@c widecommands
@table @kbd
@item C-x (
Start defining a keyboard macro (@code{start-kbd-macro}).
@item C-x )
End the definition of a keyboard macro (@code{end-kbd-macro}).
@item C-x e
Execute the most recent keyboard macro (@code{call-last-kbd-macro}).
@item C-u C-x (
Re-execute last keyboard macro, then add more keys to its definition.
@item C-x q
When this point is reached during macro execution, ask for confirmation
(@code{kbd-macro-query}).
@item M-x name-last-kbd-macro
Give a command name (for the duration of the session) to the most
recently defined keyboard macro.
@item M-x write-kbd-macro
Store the definition of a keyboard macro into a file.
@item M-x append-kbd-macro
Append the definition of a keyboard macro to the end of a file.
@end table

  Keyboard macros differ from ordinary Emacs commands in that they are
written in the Emacs command language rather than in Lisp.  This makes it
easier for the novice to write them, and makes them more convenient as
temporary hacks.  However, the Emacs command language is not powerful
enough as a programming language to be useful for writing anything
intelligent or general.  For such things, Lisp must be used.

  You define a keyboard macro while executing the commands which are the
definition.  Put differently, as you are defining a keyboard macro, the
definition is being executed for the first time.  This way, you can see
what the effects of your commands are, so that you don't have to figure
them out in your head.  When you are finished, the keyboard macro is
defined and also has been, in effect, executed once.  You can then do the
whole thing over again by invoking the macro.

@subsection Basic Use

@kindex C-x (
@kindex C-x )
@kindex C-x e
@findex start-kbd-macro
@findex end-kbd-macro
@findex call-last-kbd-macro
  To start defining a keyboard macro, type the @kbd{C-x (} command
(@code{start-kbd-macro}).  From then on, your keys continue to be
executed, but also become part of the definition of the macro.  @samp{Def}
appears in the mode line to remind you of what is going on.  When you are
finished, the @kbd{C-x )} command (@code{end-kbd-macro}) terminates the
definition (without becoming part of it!).  For example

@example
C-x ( M-F foo C-x )
@end example

@noindent
defines a macro to move forward a word and then insert @samp{foo}.

  The macro thus defined can be invoked again with the @kbd{C-x e} command
(@code{call-last-kbd-macro}), which may be given a repeat count as a
numeric argument to execute the macro many times.  @kbd{C-x )} can also be
given a repeat count as an argument, in which case it repeats the macro
that many times right after defining it, but defining the macro counts as
the first repetition (since it is executed as you define it).  So, giving
@kbd{C-x )} an argument of 4 executes the macro immediately 3 additional
times.  An argument of zero to @kbd{C-x e} or @kbd{C-x )} means repeat the
macro indefinitely (until it gets an error, or you type @kbd{C-g}).

  If you wish to repeat an operation at regularly spaced places in the
text, define a macro and include as part of the macro the commands to move
to the next place you want to use it.  For example, if you want to change
each line, you should position point at the start of a line, and define a
macro to change that line and leave point at the start of the next line.
Then repeating the macro will operate on successive lines.

  After you have terminated the definition of a keyboard macro, you can add
to the end of its definition by typing @kbd{C-u C-x (}.  This is equivalent
to plain @kbd{C-x (} followed by retyping the whole definition so far.  As
a consequence it re-executes the macro as previously defined.

@subsection Naming and Saving Keyboard Macros

@findex name-last-kbd-macro
  If you wish to save a keyboard macro for longer than until you define the
next one, you must give it a name or install it on a command sequence.  To
give the macro a name, use @kbd{M-x name-last-kbd-macro}.  This reads a
name as an argument using the minibuffer and defines that name to execute
the macro.  The macro name is a Lisp symbol, and defining it in this way
makes it a valid command name for calling with @kbd{M-x} or for binding a
key to with @code{global-set-key} (@pxref{Keymaps}).

@findex write-kbd-macro
@findex append-kbd-macro
  Once a macro has a command name, you can save its definition in a file.
Then it can be used in another editing session.

@example
M-x write-kbd-macro @key{RET} @var{macroname} @key{RET} @var{file} @key{RET}
@end example

@noindent
writes a Lisp expression for the definition of the keyboard macro named
@var{macroname} into the file @var{file}, replacing any previous contents.
The file can be loaded with @code{load} (@pxref{Lisp Libraries}).
The command @code{append-kbd-macro} is similar, but adds the definition
to the end of the file, in addition to the previous contents.  You might
want to add a macro in this way to your init file @file{~/.emacs}; then
it will automatically be defined when you run Emacs again.

@subsection Executing Macros with Variations

@kindex C-x q
@findex kbd-macro-query
  Using @kbd{C-x q} (@code{kbd-macro-query}), you can get an effect similar
to that of @code{query-replace}, where the macro asks you each time around
whether to make a change.  When you are defining the macro, type @kbd{C-x
q} at the point where you want the query to occur.  During macro
definition, the @kbd{C-x q} does nothing, but when the macro is invoked the
@kbd{C-x q} reads a character from the terminal to decide whether to
continue.

  The special answers are @key{SPC}, @key{DEL}, @kbd{C-d}, @kbd{C-l} and
@kbd{C-r}.  Any other character terminates execution of the keyboard macro
and is then read as a command.  @key{SPC} means to continue.  @key{DEL}
means to skip the remainder of this repetition of the macro, starting again
from the beginning in the next repetition.  @kbd{C-d} means to skip the
remainder of this repetition and cancel further repetition.  @kbd{C-l}
clears the screen and asks you again for a character to say what to do.
@kbd{C-r} enters a recursive editing level, in which you can perform
editing which is not part of the macro.  When you exit the recursive edit
using @kbd{C-M-c}, you are asked again how to continue with the keyboard
macro.  If you type a @key{SPC} at this time, the rest of the macro
definition is executed.  It is up to you to leave point and the text in a
state such that the rest of the macro will do what you want.@refill

  @kbd{C-u C-x q}, @kbd{C-x q} with a numeric argument, performs a different
function.  It enters a recursive edit reading input from the keyboard, both
when you type it during the definition of the macro, and when it is
executed from the macro.  During definition, the editing you do inside the
recursive edit does not become part of the macro.  During macro execution,
the recursive edit gives you a chance to do some particularized editing.
@xref{Recursive Edit}.

@node Key Bindings, Syntax, Keyboard Macros, Customization
@section Customizing Key Bindings

  This section deals with the @dfn{keymaps} which define the bindings
between keys and functions, and says how you can customize these bindings.
@cindex command
@cindex function
@cindex command name

  A command is a Lisp function whose definition provides for interactive
use.  Like every Lisp function, a command has a function name, a Lisp
symbol whose name usually consists of lower case letters and dashes.

  The bindings between characters and command functions are recorded in
data structures called @dfn{keymaps}.  Emacs has many of these.  One, the
@dfn{global} keymap, defines the meanings of the single keys that are
defined regardless of major mode.  Each major mode has another keymap, its
@dfn{local keymap}, which contains overriding definitions for the single
keys that are to be redefined in that mode.  Finally, each prefix key has a
keymap which defines the key sequences that start with that prefix.

@menu
* Keymaps::    Definition of the keymap data structure.
* Rebinding::  How to redefine one key's meaning conveniently.
* Disabling::  Disabling a command means confirmation is required
                before it can be executed.  This is done to protect
                beginners from surprises.
@end menu

@node Keymaps, Rebinding, Key Bindings, Key Bindings
@subsection Keymaps
@cindex keymap

@cindex global keymap
@vindex global-map
  The bindings between characters and command functions are recorded in
data structures called @dfn{keymaps}.  Emacs has many of these.  One, the
@dfn{global} keymap, defines the meanings of the single keys that are
defined regardless of major mode.  It is the value of the variable
@code{global-map}.

@cindex local keymap
@vindex c-mode-map
@vindex lisp-mode-map
  Each major mode has another keymap, its @dfn{local keymap}, which
contains overriding definitions for the single keys that are to be
redefined in that mode.  Each buffer records which local keymap is
installed for it at any time, and the current buffer's local keymap is the
only one that directly affects command execution.  The local keymaps for
Lisp mode, C mode, and many other major modes always exist even when not in
use.  They are the values of the variables @code{lisp-mode-map},
@code{c-mode-map}, and so on.  For major modes less often used, the local
keymap is sometimes constructed only when the mode is used for the first
time in a session.  This is to save space.

@vindex minibuffer-local-map
@vindex minibuffer-local-ns-map
@vindex minibuffer-local-completion-map
@vindex minibuffer-local-must-match-map
  There are local keymaps for the minibuffer too; they contain various
completion and exit commands.

@itemize @bullet
@item
@code{minibuffer-local-map} is used for ordinary input (no completion).
@item
@code{minibuffer-local-ns-map} is similar, except that @key{SPC} exits
just like @key{RET}.  This is used mainly for Mocklisp compatibility.
@item
@code{minibuffer-local-completion-map} is for permissive completion.
@item
@code{minibuffer-local-must-match-map} is for strict completion and
for cautious completion.
@end itemize

@vindex ctl-x-map
@vindex help-map
@vindex esc-map
  Finally, each prefix key has a keymap which defines the key sequences
that start with it.  For example, @code{ctl-x-map} is the keymap used for
characters following a @kbd{C-x}, and @code{help-map} is the keymap used
for characters following a @kbd{C-h}.  @code{esc-map} is the keymap used
for characters following @key{ESC}, and therefore for all Meta characters
(see below).  In fact, the definition of a prefix key is just the keymap to
use for looking up the following character.  Actually, the definition is
sometimes a Lisp symbol whose function definition is the following character
keymap.  The effect is the same, but it provides a command name for the
prefix key that can be used as a description of what the prefix key is for.
Thus, the binding of @kbd{C-x} is the symbol @code{Ctl-X-Prefix}, whose
function definition is the keymap for @kbd{C-x} commands, the value of
@code{ctl-x-map}.@refill

  Prefix key definitions of this sort can appear in either the global map
or a local map.  The definitions of @kbd{C-c}, @kbd{C-x}, @kbd{C-h} and @key{ESC}
as prefix keys appear in the global map, so these prefix keys are always
available.  Major modes can locally redefine a key as a prefix by putting
a prefix key definition for it in the local map.@refill

  A mode can also put a prefix definition of a global prefix character such
as @kbd{C-x} into its local map.  This is how major modes override the
definitions of certain keys that start with @kbd{C-x}.  This case is
special, because the local definition does not entirely replace the global
one.  When both the global and local definitions of a key are other
keymaps, the next character is looked up in both keymaps, with the local
definition overriding the global one as usual.  So, the character after the
@kbd{C-x} is looked up in both the major mode's own keymap for redefined
@kbd{C-x} commands and in @code{ctl-x-map}.  If the major mode's own keymap
for @kbd{C-x} commands contains @code{nil}, the definition from the global
keymap for @kbd{C-x} commands is used.@refill

@cindex sparse keymap
  A keymap is actually a Lisp object.  The simplest form of keymap is a
Lisp vector of length 128.  The binding for a character in such a keymap is
found by indexing into the vector with the character as an index.  A keymap
can also be a Lisp list whose car is the symbol @code{keymap} and whose
remaining elements are pairs of the form @code{(@var{char} . @var{binding})}.
Such lists are called @dfn{sparse keymaps} because they are used when most
of the characters' entries will be @code{nil}.  Sparse keymaps are used
mainly for prefix characters.

  Keymaps are only of length 128, so what about Meta characters, whose
codes are from 128 to 255?  A key that contains a Meta character actually
represents it as a sequence of two characters, the first of which is
@key{ESC}.  So the key @kbd{M-a} is really represented as @kbd{@key{ESC}
a}, and its binding is found at the slot for @samp{a} in
@code{esc-map}.@refill

@node Rebinding, Disabling, Keymaps, Key Bindings
@subsection Changing Key Bindings Interactively

  The way to redefine an Emacs key is to change its entry in a keymap.
You can change the global keymap, in which case the change is effective in
all major modes (except those that have their own overriding local
definitions for the same key).  Or you can change the current buffer's
local map, which affects all buffers using the same major mode.
@findex global-set-key
@findex local-set-key

@table @kbd
@item M-x global-set-key @key{RET} @var{key} @var{cmd} @key{RET}
Defines @var{key} globally to run @var{cmd}.
@item M-x local-set-key @key{RET} @var{key} @var{cmd} @key{RET}
Defines @var{key} locally (in the major mode now in effect) to run
@var{cmd}.
@end table

  For example,

@example
M-x global-set-key @key{RET} C-f next-line @key{RET}
@end example

@noindent
would redefine @kbd{C-f} to move down a line.  The fact that @var{cmd} is
read second makes it serve as a kind of confirmation for @var{key}.

  These functions offer no way to specify a particular prefix keymap as the
one to redefine in, but that is not necessary, as you can include prefixes
in @var{key}.  @var{key} is read by reading characters one by one until
they amount to a complete key (that is, not a prefix key).  Thus, if you
type @kbd{C-f} for @var{key}, that's the end; the minibuffer is entered
immediately to read @var{cmd}.  But if you type @kbd{C-x}, another
character is read; if that is @kbd{4}, another character is read, and so
on.  For example,@refill

@example
M-x global-set-key @key{RET} C-x 4 $ dictionary-other-window @key{RET}
@end example

@noindent
would redefine @kbd{C-x 4 $} to run the (fictitious) command
@code{dictionary-other-window}.

@findex define-key
  The most general way to modify a keymap is the function @code{define-key},
used in Lisp code (such in your @file{.emacs} file).  @code{define-key}
takes three arguments: the keymap, the key to modify in it, and the new
definition.  @xref{Init File}, for an example.

@node Disabling,, Rebinding, Key Bindings
@subsection Disabling Commands
@cindex disabled command

  Disabling a command marks the command as requiring confirmation before it
can be executed.  The purpose of disabling a command is to prevent
beginning users from executing it by accident and being confused.

  The direct mechanism for disabling a command is to have a non-@code{nil}
@code{disabled} property on the Lisp symbol for the command.  These
properties are normally set up by the user's @file{.emacs} file with
Lisp expressions such as

@example
(put 'delete-region 'disabled t)
@end example

@findex disable-command
@findex enable-command
  You can make a command disabled either by editing the @file{.emacs} file
directly or with the command @kbd{M-x disable-command}, which edits the
@file{.emacs} file for you.  @xref{Init File}.

  Attempting to invoke a disabled command interactively in Emacs causes the
display of a window containing the command's name, its documentation, and
some instructions on what to do immediately; then Emacs asks for input
saying whether to execute the command as requested, enable it and execute,
or cancel it.  If you decide to enable the command, you are asked whether to
do this permanently or just for the current session.  Enabling permanently
works by automatically editing your @file{.emacs} file.  You can use
@kbd{M-x enable-command} to enable a command permanently without
executing it.

  Whether a command is disabled is independent of what key is used to
invoke it; it also applies if the command is invoked using @kbd{M-x}.
Disabling a command has no effect on calling it as a function from Lisp
programs.

@node Syntax, Init File, Key Bindings, Customization
@section The Syntax Table
@cindex syntax table

  All the Emacs commands which parse words or balance parentheses are
controlled by the @dfn{syntax table}.  The syntax table says which
characters are opening delimiters, which are parts of words, which are
string quotes, and so on.  Actually, each major mode has its own syntax
table (though sometimes related major modes use the same one) which it
installs in each buffer that uses that major mode.  The syntax table
installed in the current buffer is the one that all commands use.  So we
will call it ``the syntax table''.  A syntax table is a Lisp object, a
vector of length 256 whose elements are numbers.

  The syntax table entry for a character holds six pieces of information:

@itemize @bullet
@item
The syntactic class of the character, represented as a small integer.
@item
The matching delimiter, for delimiter characters only.
The matching delimiter of @samp{(} is @samp{)}, and vice versa.
@item
A flag saying whether the character is the first character of a
two-character comment starting sequence.
@item
A flag saying whether the character is the second character of a
two-character comment starting sequence.
@item
A flag saying whether the character is the first character of a
two-character comment ending sequence.
@item
A flag saying whether the character is the second character of a
two-character comment ending sequence.
@end itemize

  The syntactic classes are stored internally as small integers, but are
usually described to or by the user with characters.  For example, @samp{(}
is used to specify the syntactic class of opening delimiters.  Here is a
table of syntactic classes, with the characters that specify them.

@table @samp
@item @w{ }
The class of whitespace characters.
@item w
The class of word-constituent characters.
@item _
The class of characters that are part of symbol names but not words.
This class is represented by @samp{_} because the character @samp{_}
has this class in both C and Lisp.
@item .
The class of punctuation characters that do not fit into any other
special class.
@item (
The class of opening delimiters. 
@item )
The class of closing delimiters. 
@item '
The class of expression-adhering characters.  These characters are
part of a symbol if found within or adjacent to one, and are part
of a following expression if immediately preceding one, but are like
whitespace if surrounded by whitespace.
@item "
The class of string-quote characters.  They match each other in pairs,
and the characters within the pair all lose their syntactic
significance except for the @samp{\} and @samp{/} classes of escape
characters, which can be used to include a string-quote inside the
string.
@item $
The class of self-matching delimiters.  This is intended for @TeX{}'s
@samp{$}, which is used both to enter and leave math mode.  Thus,
a pair of matching @samp{$} characters surround each piece of math mode
@TeX{} input.  A pair of adjacent @samp{$} characters act like a single
one for purposes of matching

@item /
The class of escape characters that always just deny the following
character its special syntactic significance.  The character after one
of these escapes is always treated as alphabetic.
@item \
The class of C-style escape characters.  In practice, these are
treated just like @samp{/}-class characters, because the extra
possibilities for C escapes (such as being followed by digits) have no
effect on where the containing expression ends.
@item <
The class of comment-starting characters.  Only single-character
comment starters (such as @samp{;} in Lisp mode) are represented this
way.
@item >
The class of comment-ending characters.  Newline has this syntax in
Lisp mode.
@end table

@vindex parse-sexp-ignore-comments
  The characters flagged as part of two-character comment delimiters can
have other syntactic functions most of the time.  For example, @samp{/} and
@samp{*} in C code, when found separately, have nothing to do with
comments.  The comment-delimiter significance overrides when the pair of
characters occur together in the proper order.  Only the list and sexp
commands use the syntax table to find comments; the commands specifically
for comments have other variables that tell them where to find comments.
And the list and sexp commands notice comments only if
@code{parse-sexp-ignore-comments} is non-@code{nil}.  This variable is set
to @code{nil} in modes where comment-terminator sequences are liable to
appear where there is no comment; for example, in Lisp mode where the
comment terminator is a newline but not every newline ends a comment.

@findex modify-syntax-entry
  @kbd{M-x modify-syntax-entry} is the command to change a character's
syntax.  It can be used interactively, and is also the means used by major
modes to initialize their own syntax tables.  Its first argument is the
character to change.  The second argument is a string that specifies the
new syntax.  When called from Lisp code, there is a third, optional
argument, which specifies the syntax table in which to make the change.  If
not supplied, or if this command is called interactively, the third
argument defaults to the current buffer's syntax table.

@enumerate
@item
The first character in the string specifies the syntactic class.  It
is one of the characters in the previous table.

@item
The second character is the matching delimiter.  For a character that
is not an opening or closing delimiter, this should be a space, or may
be omitted if no following characters are needed.

@item
The remaining characters are flags.  The flag characters allowed are

@table @samp
@item 1
Flag this character as the first of a two-character comment starting sequence.
@item 2
Flag this character as the second of a two-character comment starting sequence.
@item 3
Flag this character as the first of a two-character comment ending sequence.
@item 4
Flag this character as the second of a two-character comment ending sequence.
@end table
@end enumerate

@kindex C-h s
@findex describe-syntax
  A description of the contents of the current syntax table can be
displayed with @kbd{C-h s} (@code{describe-syntax}).  The description of
each character includes both the string you would have to give to
@code{modify-syntax-entry} to set up that character's current syntax, and
some English to explain that string if necessary.

@node Init File,, Syntax, Customization
@section The Init File, .emacs
@cindex init file

  When Emacs is started, it normally loads the file @file{.emacs} in your
home directory.  This file, if it exists, should contain Lisp code.
Here we describe how to do certain common things in the @file{.emacs} file.

  The @file{.emacs} file contains one or more Lisp function call
expressions.  Each of these consists of a function name followed by
arguments, all surrounded by parentheses.  For example, @code{(setq
default-fill-column 60)} represents a call to the function @code{setq}
which is used to set the variable @code{default-fill-column}
(@pxref{Filling}) to 60.

  The second argument to @code{setq} is an expression for the new value of
the variable.  This can be a constant, a variable, or a function call
expression.  In @file{.emacs}, constants are used most of the time.  They can be:

@table @asis
@item Numbers:
Numbers are written in decimal, with an optional initial minus sign.
@item Strings:
Lisp string syntax is the same as C string syntax with a few extra
features.  First, newlines and any other characters may be present
literally in strings.  Second, @samp{\e} may be used to stand for the
character @key{ESC}.  Third, @samp{\C-} can be used as a prefix for a
control character, as in @samp{\C-s} for ASCII Control-S, and
@samp{\M-} can be used as a prefix for a meta character, as in
@samp{\M-a} for Meta-A or @samp{\M-\C-a} for Control-Meta-A.@refill
@item Characters:
Lisp character constant syntax consists of a @samp{?} followed by
either a character or an escape sequence starting with @samp{\}.
Examples: @code{?x}, @code{?\n}, @code{?\"}, @code{?\)}.  Note that
strings and characters are not interchangeable in Lisp; some contexts
require one and some contexts require the other.
@item True:
@code{t} stands for `true'.
@item False:
@code{nil} stands for `false'.
@item Other Lisp objects:
Write a single-quote (') followed by the Lisp object you want.
@end table

  Here are some examples of doing certain commonly desired things with
Lisp expressions:

@itemize @bullet
@item
Make searches case sensitive:

@example
(setq default-case-fold-search nil)
@end example

Here we have a variable whose value is normally @code{t} for `true'
and the alternative is @code{nil} for `false'.

@item
Make Text mode the default mode for new buffers:

@example
(setq default-major-mode 'text-mode)
@end example

Note that @code{text-mode} is used because it is the command for entering
the mode we want.  A single-quote is written before it to make a symbol
constant; otherwise, @code{text-mode} would be treated as a variable name.

@item
Turn on Auto Fill mode automatically in Text mode and related modes:

@example
(setq text-mode-hook
  '(lambda () (auto-fill-mode 1)))
@end example

Here we have a variable whose value should be a Lisp function.  The
function we supply is a list starting with @code{lambda}, and a single
quote is written in front of it to make it (for the purpose of this
@code{setq}) a list constant rather than an expression.  Lisp functions
are not explained here, but for mode hooks it is enough to know that
@code{(auto-fill-mode 1)} is an expression that will be executed when
Text mode is entered, and you could replace it with any other expression
that you like, or with several expressions in a row.

@example
(setq text-mode-hook 'turn-on-auto-fill)
@end example

This is another way to accomplish the same result.
@code{turn-on-auto-fill} is a symbol whose function definition is
@code{(lambda () (auto-fill-mode 1))}.

@item
Load the compiled Lisp file @file{foo.elc}.

@example
(load "foo")
@end example

@item
Rebind the key @kbd{C-x l} to run the function @code{make-symbolic-link}.

@example
(global-set-key "\C-xl" 'make-symbolic-link)
@end example

Note once again the single-quote used to refer to the symbol
@code{make-symbolic-link} instead of its value as a variable.

@item
Do the same thing for C mode only.

@example
(define-key c-mode-map "\C-xl" 'make-symbolic-link)
@end example

@item
Make @kbd{C-x p} undefined.

@example
(global-unset-key "\C-xp")
@end example

One reason to undefine a key is so that you can make it a prefix.
Simply defining @kbd{C-x p @var{anything}} would make @kbd{C-x p}
a prefix, provided it is not otherwise defined.

@item
Make @samp{$} have the syntax of punctuation in Text mode.
Note the use of a character constant for @samp{$}.

@example
(modify-syntax-entry ?\$ "." text-mode-syntax-table)
@end example

@item
Enable the use of the command @code{eval-expression} without confirmation.

@example
(put 'eval-expression 'disabled nil)
@end example
@end itemize

@iftex
@chapter Correcting Mistakes (Yours or Emacs's)

  If you type an Emacs command you did not intend, the results are often
mysterious.  This chapter tells what you can do to cancel your mistake or
recover from a mysterious situation.  Emacs bugs and system crashes are
also considered.
@end iftex

@node Quitting, Lossage, Customization, Top
@section Quitting and Aborting
@cindex quitting

@table @kbd
@item C-g
Quit.  Cancel running or partially typed command.
@item C-]
Abort recursive editing level and cancel the command which invoked it
(@code{abort-recursive-edit}).
@item M-x top-level
Abort all recursive editing levels that are currently executing.
@item C-x u
Cancel an already-executed command, usually (@code{undo}).
@end table

  There are two ways of cancelling commands which are not finished
executing: @dfn{quitting} with @kbd{C-g}, and @dfn{aborting} with @kbd{C-]}
or @kbd{M-x top-level}.  Quitting is cancelling a partially typed command
or one which is already running.  Aborting is getting out of a recursive
editing level and cancelling the command that invoked the recursive edit.

@cindex quitting
@cindex C-g
  Quitting with @kbd{C-g} is used for getting rid of a partially typed
command, or a numeric argument that you don't want.  It also stops a
running command in the middle in a relatively safe way, so you can use it
if you accidentally give a command which takes a long time.  In particular,
it is safe to quit out of killing; either your text will @var{all} still be
there, or it will @var{all} be in the kill ring (or maybe both).  Quitting
an incremental search does special things documented under searching; in
general, it may take two successive @kbd{C-g} characters to get out of a
search.  @kbd{C-g} works by setting the variable @code{quit-flag} to
@code{t} the instant @kbd{C-g} is typed; Emacs Lisp checks this variable
frequently and quits if it is non-@code{nil}.  @kbd{C-g} is only actually
executed as a command if it is typed while Emacs is waiting for input.

  If you quit twice in a row before the first @kbd{C-g} is recognized, you
activate the ``emergency escape'' feature and return to the shell.
@xref{Emergency Escape}.

@cindex recursive editing level
@cindex aborting
@findex abort-recursive-edit
@kindex C-]
  Aborting with @kbd{C-]} (@code{abort-recursive-edit}) is used to get out
of a recursive editing level and cancel the command which invoked it.
Quitting with @kbd{C-g} does not do this, and could not do this, because it
is used to cancel a partially typed command @i{within} the recursive
editing level.  Both operations are useful.  For example, if you are in the
Emacs debugger (@pxref{Lisp Debug}) and have typed @kbd{C-u 8} to enter a
numeric argument, you can cancel that argument with @kbd{C-g} and remain in
the debugger.

@findex top-level
  The command @kbd{M-x top-level} is equivalent to ``enough'' @kbd{C-]}
commands to get you out of all the levels of subsystems and recursive edits
that you are in.  @kbd{C-]} gets you out one level at a time, but @kbd{M-x
top-level} goes out all levels at once.  Both @kbd{C-]} and @kbd{M-x
top-level} are like all other commands, and unlike @kbd{C-g}, in that they
are effective only when Emacs is ready for a command.  @kbd{C-]} is an
ordinary key and has its meaning only because of its binding in the keymap.

  @kbd{C-x u} (@code{undo}) is not strictly speaking a way of cancelling
a command, but you can think of it as cancelling a command already finished
executing.  @xref{Undo}.

@node Lossage, Bugs, Quitting, Top
@section Dealing with Emacs Trouble

  This section describes various conditions in which Emacs fails to work,
and how to recognize them and correct them.

@menu
* Stuck Recursive::    `[...]' in mode line around the parentheses
* Screen Garbled::     Garbage on the screen
* Text Garbled::       Garbage in the text
* Unasked-for Search:: Spontaneous entry to incremental search
* Emergency Escape::   Emergency escape---
                        What to do if Emacs stops responding
* Total Frustration::  When you are at your wits' end.
@end menu

@node Stuck Recursive, Screen Garbled, Lossage, Lossage
@subsection Recursive Editing Levels

  Recursive editing levels are important and useful features of Emacs, but
they can seem like malfunctions to the user who does not understand them.

  If the mode line has square brackets @samp{[@dots{}]} around the parentheses
that contain the names of the major and minor modes, you have entered a
recursive editing level.  If you did not do this on purpose, or if you
don't understand what that means, you should just get out of the recursive
editing level.  To do so, type @kbd{M-x top-level}.  This is called getting
back to top level.  @xref{Recursive Edit}.

@node Screen Garbled, Text Garbled, Stuck Recursive, Lossage
@subsection Garbage on the Screen

  If the data on the screen looks wrong, the first thing to do is see
whether the text is really wrong.  Type @kbd{C-l}, to redisplay the entire
screen.  If it appears correct after this, the problem was entirely in the
previous screen update.

  Display updating problems often result from an incorrect termcap entry
for the terminal you are using.  The file @file{etc/TERMS} gives the fixes
for known problems of this sort.  @file{INSTALL} contains general advice
for these problems in one of its sections.  Very likely there is simply
insufficient padding for certain display operations.  To investigate the
possibility that you have this sort of problem, try Emacs on another
terminal made by a different manufacturer.  If problems happen frequently
on one kind of terminal but not another kind, it is likely to be a bad
termcap entry, though it could also be due to a bug in Emacs that appears
for terminals that have or that lack specific features.

@node Text Garbled, Unasked-for Search, Screen Garbled, Lossage
@subsection Garbage in the Text

  If @kbd{C-l} shows that the text is wrong, try undoing the changes to it
using @kbd{C-x u} until it gets back to a state you consider correct.  Also
try @kbd{C-h l} to find out what command you typed to produce the observed
results.

  If a large portion of text appears to be missing at the beginning or
end of the buffer, check for the word @samp{Narrow} in the mode line.
If it appears, the text is still present, but marked off-limits.
To make it visible again, type @kbd{C-x w}.  @xref{Narrowing}.

@node Unasked-for Search, Emergency Escape, Text Garbled, Lossage
@subsection Spontaneous Entry to Incremental Search

  If Emacs spontaneously displays @samp{I-search:} at the bottom of the
screen, it means that the terminal is sending @kbd{C-s} and @kbd{C-q}
according to the badly designed xon/xoff ``flow control'' protocol.  You
should try to prevent this by putting the terminal in a mode where it will
not use flow control or giving it enough padding that it will never send a
@kbd{C-s}.  If that cannot be done, you must tell Emacs to expect flow
control to be used, until you can get a properly designed terminal.

  Information on how to do these things can be found in the file
@file{INSTALL} in the Emacs distribution.

@node Emergency Escape, Total Frustration, Unasked-for Search, Lossage
@subsection Emergency Escape

  Because at times there have been bugs causing Emacs to loop without
checking @code{quit-flag}, a special feature causes Emacs to be suspended
immediately if you type a second @kbd{C-g} while the flag is already set.
So you can always get out of GNU Emacs.  Normally Emacs recognizes and
clears @code{quit-flag} (and quits!)  quickly enough to prevent this from
happening.

  When you resume Emacs after a suspension caused by multiple @kbd{C-g}, it
asks two questions before going back to what it had been doing:

@example
Checkpoint?
Abort (and dump core)?
@end example

@noindent
Answer each one with @kbd{y} or @kbd{n} followed by @key{RET}.

  Saying @kbd{y} to @samp{Checkpoint?} causes immediate auto-saving of all
modified buffers in which auto-saving is enabled.

  Saying @kbd{y} to @samp{Abort (and dump core)?} causes an illegal instruction to be
executed, dumping core.  This is to enable a wizard to figure out why Emacs
was failing to quit in the first place.  Execution does not continue
after a core dump.  If you answer @kbd{n}, execution does continue.  With
luck, GNU Emacs will ultimately check @code{quit-flag} and quit normally.
If not, and you type another @kbd{C-g}, it is suspended again.

  If Emacs is not really hung, just slow, you may invoke the double
@kbd{C-g} feature without really meaning to.  Then just resume and answer
@kbd{n} to both questions, and you will arrive at your former state.
Presumably the quit you requested will happen soon.

@node Total Frustration,, Emergency Escape, Lossage
@subsection Help for Total Frustration
@cindex Eliza
@cindex doctor

  If using Emacs (or something else) becomes terribly frustrating and none
of the techniques described above solve the problem, Emacs can still help
you.

  First, if the Emacs you are using is not responding to commands, type
@kbd{C-g C-g} to get out of it and then start a new one.

@findex doctor
  Second, type @kbd{M-x doctor @key{RET}}.

  The doctor will make you feel better.  Each time you say something to
the doctor, you must end it by typing @key{RET} @key{RET}.  This lets the
doctor know you are finished.

@node Bugs, Manifesto, Lossage, Top
@section Reporting Bugs

@cindex bugs
  Sometimes you will encounter a bug in Emacs.  Although we cannot promise
we can or will fix the bug, and we might not even agree that it is a bug,
we want to hear about bugs you encounter in case we do want to fix them.

  To make it possible for us to fix a bug, you must report it.  In order
to do so effectively, you must know when and how to do it.

@subsection When Is There a Bug

  If Emacs executes an illegal instruction, or dies with an operating
system error message that indicates a problem in the program (as opposed to
something like ``disk full''), then it is certainly a bug.

  If Emacs updates the display in a way that does not correspond to what is
in the buffer, then it is certainly a bug.  If a command seems to do the
wrong thing but the problem corrects itself if you type @kbd{C-l}, it is a
case of incorrect display updating.

  Taking forever to complete a command can be a bug, but you must make
certain that it was really Emacs's fault.  Some commands simply take a long
time.  Type @kbd{C-g} and then @kbd{C-h l} to see whether the input Emacs
received was what you intended to type; if the input was such that you
@var{know} it should have been processed quickly, report a bug.  If you
don't know whether the command should take a long time, find out by looking
in the manual or by asking for assistance.

  If a command you are familiar with causes an Emacs error message in a
case where its usual definition ought to be reasonable, it is probably a
bug.

  If a command does the wrong thing, that is a bug.  But be sure you know
for certain what it ought to have done.  If you aren't familiar with the
command, or don't know for certain how the command is supposed to work,
then it might actually be working right.  Rather than jumping to
conclusions, show the problem to someone who knows for certain.

  Finally, a command's intended definition may not be best for editing
with.  This is a very important sort of problem, but it is also a matter of
judgment.  Also, it is easy to come to such a conclusion out of ignorance
of some of the existing features.  It is probably best not to complain
about such a problem until you have checked the documentation in the usual
ways, feel confident that you understand it, and know for certain that what
you want is not available.  If you are not sure what the command is
supposed to do after a careful reading of the manual, check the index and
glossary for any terms that may be unclear.  If you still do not
understand, this indicates a bug in the manual.  The manual's job is to
make everything clear.  It is just as important to report documentation
bugs as program bugs.

  If the on-line documentation string of a function or variable disagrees
with the manual, one of them must be wrong, so report the bug.

@subsection How to Report a Bug

@findex emacs-version
  When you decide that there is a bug, it is important to report it and to
report it in a way which is useful.  What is most useful is an exact
description of what commands you type, starting with the shell command to
run Emacs, until the problem happens.  Always include the version number
of Emacs that you are using; type @kbd{M-x emacs-version} to print this.

  The most important principle in reporting a bug is to report @var{facts},
not hypotheses or categorizations.  It is always easier to report the facts,
but people seem to prefer to strain to posit explanations and report
them instead.  If the explanations are based on guesses about how Emacs is
implemented, they will be useless; we will have to try to figure out what
the facts must have been to lead to such speculations.  Sometimes this is
impossible.  But in any case, it is unnecessary work for us.

  For example, suppose that you type @kbd{C-x C-f /glorp/baz.ugh
@key{RET}}, visiting a file which (you know) happens to be rather large,
and Emacs prints out @samp{I feel pretty today}.  The best way to report
the bug is with a sentence like the preceding one, because it gives all the
facts and nothing but the facts.

  Do not assume that the problem is due to the size of the file and say,
``When I visit a large file, Emacs prints out @samp{I feel pretty today}.''
This is what we mean by ``guessing explanations''.  The problem is just as
likely to be due to the fact that there is a @code{z} in the file name.  If
this is so, then when we got your report, we would try out the problem with
some ``large file'', probably with no @code{z} in its name, and not find
anything wrong.  There is no way in the world that we could guess that we
should try visiting a file with a @code{z} in its name.

  Alternatively, the problem might be due to the fact that the file starts
with exactly 25 spaces.  For this reason, you should make sure that you
inform us of the exact contents of any file that is needed to reproduce the
bug.  What if the problem only occurs when you have typed the @kbd{C-x C-a}
command previously?  This is why we ask you to give the exact sequence of
characters you typed since starting to use Emacs.

  You should not even say ``visit a file'' instead of @kbd{C-x C-f} unless
you @i{know} that it makes no difference which visiting command is used.
Similarly, rather than saying ``if I have three characters on the line,''
say ``after I type @kbd{@key{RET} A B C @key{RET} C-p},'' if that is
the way you entered the text.@refill

  If you are not in Fundamental mode when the problem occurs, you should
say what mode you are in.

  If the manifestation of the bug is an Emacs error message, it is
important to report not just the text of the error message but a backtrace
showing how the Lisp program in Emacs arrived at the error.  To make the
backtrace, you must execute the Lisp expression @code{(setq debug-on-error@ t)}
before the error happens (that is to say, you must execute that expression
and then make the bug happen).  This causes the Lisp debugger to run
(@pxref{Lisp Debug}).  The debugger's backtrace can be copied as text into
the bug report.  This use of the debugger is possible only if you know how
to make the bug happen again.  Do note the error message the first time the
bug happens, so if you can't make it happen again, you can report at least
that.

  Check whether any programs you have loaded into the Lisp world, including
your @file{.emacs} file, set any variables that may affect the functioning
of Emacs.  Also, see whether the problem happens in a freshly started Emacs
without loading your @file{.emacs} file (start Emacs with the @code{-q} switch
to prevent loading the init file.)  If the problem does @var{not} occur
then, it is essential that we know the contents of any programs that you
must load into the Lisp world in order to cause the problem to occur.

  If the problem does depend on an init file or other Lisp programs that
are not part of the standard Emacs system, then you should make sure it is
not a bug in those programs by complaining to their maintainers, first.
After they verify that they are using Emacs in a way that is supposed to
work, they should report the bug.

  If you can tell us a way to cause the problem without visiting any files,
please do so.  This makes it much easier to debug.  If you do need files,
make sure you arrange for us to see their exact contents.  For example, it
can often matter whether there are spaces at the ends of lines, or a
newline after the last line in the buffer (nothing ought to care whether
the last line is terminated, but tell that to the bugs).

@findex open-dribble-file
@cindex dribble file
  The easy way to record the input to Emacs precisely is to to write a
dribble file; execute the Lisp expression

@example
(open-dribble-file "~/dribble")
@end example

@noindent
using @kbd{Meta-@key{ESC}} or from the @samp{*scratch*} buffer just after starting
Emacs.  From then on, all Emacs input will be written in the specified
dribble file until the Emacs process is killed.

@findex open-termscript
@cindex termcript file
  For possible display bugs, it is important to report the terminal type
(the value of environment variable @code{TERM}), the termcap entry for the
terminal (since @file{/etc/termcap} is not identical on all machines), and
the output that Emacs actually sent to the terminal.  The way to collect
this output is to execute the Lisp expression

@example
(open-termscript "~/termscript")
@end example

@noindent
using @kbd{Meta-@key{ESC}} or from the @samp{*scratch*} buffer just
after starting Emacs.  From then on, all output from Emacs to the terminal
will be written in the specified termscript file as well, until the Emacs
process is killed.  If the problem happens when Emacs starts up, put this
expression into your @file{~/.emacs} file so that the termscript file will
be open when Emacs displays the screen for the first time.  Be warned:
it is often difficult, and sometimes impossible, to fix a terminal-dependent
bug without access to a terminal of the type that stimulates the bug.@refill

  The address for reporting bugs is

@format
GNU Emacs Bugs
545 Tech Sq, rm 703
Cambridge, MA 02139
@end format

@noindent
or, on Usenet, mail to @samp{mit-eddie!bug-gnu-emacs}.

  Once again, we do not promise to fix the bug; but if the bug is serious,
or ugly, or easy to fix, chances are we will want to.

@node Manifesto,, Bugs, Top
@unnumbered The GNU Manifesto

@unnumberedsec What's GNU?  Gnu's Not Unix!

GNU, which stands for Gnu's Not Unix, is the name for the complete
Unix-compatible software system which I am writing so that I can give it
away free to everyone who can use it.  Several other volunteers are helping
me.  Contributions of time, money, programs and equipment are greatly
needed.

So far we have a portable C and Pascal compiler which compiles for Vax and
68000 (though needing much rewriting), an Emacs-like text editor with Lisp
for writing editor commands, a yacc-compatible parser generator, a linker,
and around 35 utilities.  A shell (command interpreter) is nearly
completed.  When the kernel and a debugger are written, it will be possible
to distribute a GNU system suitable for program development.  After this we
will add a text formatter, an Empire game, a spreadsheet, and hundreds of
other things, plus on-line documentation.  We hope to supply, eventually,
everything useful that normally comes with a Unix system, and more.

GNU will be able to run Unix programs, but will not be identical to Unix.
We will make all improvements that are convenient, based on our experience
with other operating systems.  In particular, we plan to have longer
filenames, file version numbers, a crashproof file system, filename
completion perhaps, terminal-independent display support, and eventually a
Lisp-based window system through which several Lisp programs and ordinary
Unix programs can share a screen.  Both C and Lisp will be available as
system programming languages.  We will try to support UUCP, MIT Chaosnet,
and Internet protocols for communication.

GNU is aimed initially at machines in the 68000/16000 class, with virtual
memory, because they are the easiest machines to make it run on.  The extra
effort to make it run on smaller machines will be left to someone who wants
to use it on them.

To avoid horrible confusion, please pronounce the `G' in the word `GNU'
when it is the name of this project.

@unnumberedsec Why I Must Write GNU

I consider that the golden rule requires that if I like a program I must
share it with other people who like it.  Software sellers want to divide
the users and conquer them, making each user agree not to share with
others.  I refuse to break solidarity with other users in this way.  I
cannot in good conscience sign a nondisclosure agreement or a software
license agreement.  For years I worked within the Artificial Intelligence
Lab to resist such tendencies and other inhospitalities, but eventually
they had gone too far: I could not remain in an institution where such
things are done for me against my will.

So that I can continue to use computers without dishonor, I have decided to
put together a sufficient body of free software so that I will be able to
get along without any software that is not free.  I have resigned from the
AI lab to deny MIT any legal excuse to prevent me from giving GNU away.

@unnumberedsec Why GNU Will Be Compatible with Unix

Unix is not my ideal system, but it is not too bad.  The essential features
of Unix seem to be good ones, and I think I can fill in what Unix lacks
without spoiling them.  And a system compatible with Unix would be
convenient for many other people to adopt.

@unnumberedsec How GNU Will Be Available

GNU is not in the public domain.  Everyone will be permitted to modify and
redistribute GNU, but no distributor will be allowed to restrict its
further redistribution.  That is to say, proprietary modifications will not
be allowed.  I want to make sure that all versions of GNU remain free.

@unnumberedsec Why Many Other Programmers Want to Help

I have found many other programmers who are excited about GNU and want to
help.

Many programmers are unhappy about the commercialization of system
software.  It may enable them to make more money, but it requires them to
feel in conflict with other programmers in general rather than feel as
comrades.  The fundamental act of friendship among programmers is the
sharing of programs; marketing arrangements now typically used essentially
forbid programmers to treat others as friends.  The purchaser of software
must choose between friendship and obeying the law.  Naturally, many decide
that friendship is more important.  But those who believe in law often do
not feel at ease with either choice.  They become cynical and think that
programming is just a way of making money.

By working on and using GNU rather than proprietary programs, we can be
hospitable to everyone and obey the law.  In addition, GNU serves as an
example to inspire and a banner to rally others to join us in sharing.
This can give us a feeling of harmony which is impossible if we use
software that is not free.  For about half the programmers I talk to, this
is an important happiness that money cannot replace.

@unnumberedsec How You Can Contribute

I am asking computer manufacturers for donations of machines and money.
I'm asking individuals for donations of programs and work.

One consequence you can expect if you donate machines is that GNU will run
on them at an early date.  The machines should be complete, ready to use
systems, approved for use in a residential area, and not in need of
sophisticated cooling or power.

I have found very many programmers eager to contribute part-time work for
GNU.  For most projects, such part-time distributed work would be very hard
to coordinate; the independently-written parts would not work together.
But for the particular task of replacing Unix, this problem is absent.  A
complete Unix system contains hundreds of utility programs, each of which
is documented separately.  Most interface specifications are fixed by Unix
compatibility.  If each contributor can write a compatible replacement for
a single Unix utility, and make it work properly in place of the original
on a Unix system, then these utilities will work right when put together.
Even allowing for Murphy to create a few unexpected problems, assembling
these components will be a feasible task.  (The kernel will require closer
communication and will be worked on by a small, tight group.)

If I get donations of money, I may be able to hire a few people full or
part time.  The salary won't be high by programmers' standards, but I'm
looking for people for whom building community spirit is as important as
making money.  I view this as a way of enabling dedicated people to devote
their full energies to working on GNU by sparing them the need to make a
living in another way.

@unnumberedsec Why All Computer Users Will Benefit

Once GNU is written, everyone will be able to obtain good system software
free, just like air.

This means much more than just saving everyone the price of a Unix license.
It means that much wasteful duplication of system programming effort will
be avoided.  This effort can go instead into advancing the state of the
art.

Complete system sources will be available to everyone.  As a result, a user
who needs changes in the system will always be free to make them himself,
or hire any available programmer or company to make them for him.  Users
will no longer be at the mercy of one programmer or company which owns the
sources and is in sole position to make changes.

Schools will be able to provide a much more educational environment by
encouraging all students to study and improve the system code.  Harvard's
computer lab used to have the policy that no program could be installed on
the system if its sources were not on public display, and upheld it by
actually refusing to install certain programs.  I was very much inspired by
this.

Finally, the overhead of considering who owns the system software and what
one is or is not entitled to do with it will be lifted.

Arrangements to make people pay for using a program, including licensing of
copies, always incur a tremendous cost to society through the cumbersome
mechanisms necessary to figure out how much (that is, which programs) a
person must pay for.  And only a police state can force everyone to obey
them.  Consider a space station where air must be manufactured at great
cost: charging each breather per liter of air may be fair, but wearing the
metered gas mask all day and all night is intolerable even if everyone can
afford to pay the air bill.  And the TV cameras everywhere to see if you
ever take the mask off are outrageous.  It's better to support the air
plant with a head tax and chuck the masks.

Copying all or parts of a program is as natural to a programmer as
breathing, and as productive.  It ought to be as free.

@unnumberedsec Some Easily Rebutted Objections to GNU's Goals

@quotation
``Nobody will use it if it is free, because that means they can't rely
on any support.''

``You have to charge for the program to pay for providing the
support.''
@end quotation

If people would rather pay for GNU plus service than get GNU free without
service, a company to provide just service to people who have obtained GNU
free ought to be profitable.

We must distinguish between support in the form of real programming work
and mere handholding.  The former is something one cannot rely on from a
software vendor.  If your problem is not shared by enough people, the
vendor will tell you to get lost.

If your business needs to be able to rely on support, the only way is to
have all the necessary sources and tools.  Then you can hire any available
person to fix your problem; you are not at the mercy of any individual.
With Unix, the price of sources puts this out of consideration for most
businesses.  With GNU this will be easy.  It is still possible for there to
be no available competent person, but this problem cannot be blamed on
distibution arrangements.  GNU does not eliminate all the world's problems,
only some of them.

Meanwhile, the users who know nothing about computers need handholding:
doing things for them which they could easily do themselves but don't know
how.

Such services could be provided by companies that sell just hand-holding
and repair service.  If it is true that users would rather spend money and
get a product with service, they will also be willing to buy the service
having got the product free.  The service companies will compete in quality
and price; users will not be tied to any particular one.  Meanwhile, those
of us who don't need the service should be able to use the program without
paying for the service.

@quotation
``You cannot reach many people without advertising,
and you must charge for the program to support that.''

``It's no use advertising a program people can get free.''
@end quotation

There are various forms of free or very cheap publicity that can be used to
inform numbers of computer users about something like GNU.  But it may be
true that one can reach more microcomputer users with advertising.  If this
is really so, a business which advertises the service of copying and
mailing GNU for a fee ought to be successful enough to pay for its
advertising and more.  This way, only the users who benefit from the
advertising pay for it.

On the other hand, if many people get GNU from their friends, and such
companies don't succeed, this will show that advertising was not really
necessary to spread GNU.  Why is it that free market advocates don't want
to let the free market decide this?

@quotation
``My company needs a proprietary operating system
to get a competitive edge.''
@end quotation

GNU will remove operating system software from the realm of competition.
You will not be able to get an edge in this area, but neither will your
competitors be able to get an edge over you.  You and they will compete in
other areas, while benefitting mutually in this one.  If your business is
selling an operating system, you will not like GNU, but that's tough on
you.  If your business is something else, GNU can save you from being
pushed into the expensive business of selling operating systems.

I would like to see GNU development supported by gifts from many
manufacturers and users, reducing the cost to each.

@quotation
``Don't programmers deserve a reward for their creativity?''
@end quotation

If anything deserves a reward, it is social contribution.  Creativity can
be a social contribution, but only in so far as society is free to use the
results.  If programmers deserve to be rewarded for creating innovative
programs, by the same token they deserve to be punished if they restrict
the use of these programs.

@quotation
``Shouldn't a programmer be able to ask for a reward for his creativity?''
@end quotation

There is nothing wrong with wanting pay for work, or seeking to maximize
one's income, as long as one does not use means that are destructive.  But
the means customary in the field of software today are based on
destruction.

Extracting money from users of a program by restricting their use of it is
destructive because the restrictions reduce the amount and the ways that
the program can be used.  This reduces the amount of wealth that humanity
derives from the program.  When there is a deliberate choice to restrict,
the harmful consequences are deliberate destruction.

The reason a good citizen does not use such destructive means to become
wealthier is that, if everyone did so, we would all become poorer from the
mutual destructiveness.  This is Kantian ethics; or, the Golden Rule.
Since I do not like the consequences that result if everyone hoards
information, I am required to consider it wrong for one to do so.
Specifically, the desire to be rewarded for one's creativity does not
justify depriving the world in general of all or part of that creativity.

@quotation
``Won't programmers starve?''
@end quotation

I could answer that nobody is forced to be a programmer.  Most of us cannot
manage to get any money for standing on the street and making faces.  But
we are not, as a result, condemned to spend our lives standing on the
street making faces, and starving.  We do something else.

But that is the wrong answer because it accepts the questioner's implicit
assumption: that without ownership of software, programmers cannot possibly
be paid a cent.  Supposedly it is all or nothing.

The real reason programmers will not starve is that it will still be
possible for them to get paid for programming; just not paid as much as
now.

Restricting copying is not the only basis for business in software.  It is
the most common basis because it brings in the most money.  If it were
prohibited, or rejected by the customer, software business would move to
other bases of organization which are now used less often.  There are
always numerous ways to organize any kind of business.

Probably programming will not be as lucrative on the new basis as it is
now.  But that is not an argument against the change.  It is not considered
an injustice that sales clerks make the salaries that they now do.  If
programmers made the same, that would not be an injustice either.  (In
practice they would still make considerably more than that.)

@quotation
``Don't people have a right to control how their creativity is used?''
@end quotation

``Control over the use of one's ideas'' really constitutes control over
other people's lives; and it is usually used to make their lives more
difficult.

People who have studied the issue of intellectual property rights carefully
(such as lawyers) say that there is no intrinsic right to intellectual
property.  The kinds of supposed intellectual property rights that the
government recognizes were created by specific acts of legislation for
specific purposes.

For example, the patent system was established to encourage inventors to
disclose the details of their inventions.  Its purpose was to help society
rather than to help inventors.  At the time, the life span of 17 years for
a patent was short compared with the rate of advance of the state of the
art.  Since patents are an issue only among manufacturers, for whom the
cost and effort of a license agreement are small compared with setting up
production, the patents often do not do much harm.  They do not obstruct
most individuals who use patented products.

The idea of copyright did not exist in ancient times, when authors
frequently copied other authors at length in works of non-fiction.  This
practice was useful, and is the only way many authors' works have survived
even in part.  The copyright system was created expressly for the purpose
of encouraging authorship.  In the domain for which it was
invented---books, which could be copied economically only on a printing
press---it did little harm, and did not obstruct most of the individuals
who read the books.

All intellectual property rights are just licenses granted by society
because it was thought, rightly or wrongly, that society as a whole would
benefit by granting them.  But in any particular situation, we have to ask:
are we really better off granting such license?  What kind of act are we
licensing a person to do?

The case of programs today is very different from that of books a hundred
years ago.  The fact that the easiest way to copy a program is from one
neighbor to another, the fact that a program has both source code and
object code which are distinct, and the fact that a program is used rather
than read and enjoyed, combine to create a situation in which a person who
enforces a copyright is harming society as a whole both materially and
spiritually; in which a person should not do so regardless of whether the
law enables him to.

@quotation
``Competition makes things get done better.''
@end quotation

The paradigm of competition is a race: by rewarding the winner, we
encourage everyone to run faster.  When capitalism really works this way,
it does a good job; but its defenders are wrong in assuming it always works
this way.  If the runners forget why the reward is offered and become
intent on winning, no matter how, they may find other strategies---such as,
attacking other runners.  If the runners get into a fist fight, they will
all finish late.

Proprietary and secret software is the moral equivalent of runners in a
fist fight.  Sad to say, the only referee we've got does not seem to
object to fights; he just regulates them (``For every ten yards you run,
you can fire one shot'').  He really ought to break them up, and penalize
runners for even trying to fight.

@quotation
``Won't everyone stop programming without a monetary incentive?''
@end quotation

Actually, many people will program with absolutely no monetary incentive.
Programming has an irresistible fascination for some people, usually the
people who are best at it.  There is no shortage of professional musicians
who keep at it even though they have no hope of making a living that way.

But really this question, though commonly asked, is not appropriate to the
situation.  Pay for programmers will not disappear, only become less.  So
the right question is, will anyone program with a reduced monetary
incentive?  My experience shows that they will.

For more than ten years, many of the world's best programmers worked at the
Artificial Intelligence Lab for far less money than they could have had
anywhere else.  They got many kinds of non-monetary rewards: fame and
appreciation, for example.  And creativity is also fun, a reward in itself.

Then most of them left when offered a chance to do the same interesting
work for a lot of money.

What the facts show is that people will program for reasons other than
riches; but if given a chance to make a lot of money as well, they will
come to expect and demand it.  Low-paying organizations do poorly in
competition with high-paying ones, but they do not have to do badly if the
high-paying ones are banned.

@quotation
``We need the programmers desperately.  If they demand that we
stop helping our neighbors, we have to obey.''
@end quotation

You're never so desperate that you have to obey this sort of demand.
Remember: millions for defense, but not a cent for tribute!

@quotation
``Programmers need to make a living somehow.''
@end quotation

In the short run, this is true.  However, there are plenty of ways that
programmers could make a living without selling the right to use a program.
This way is customary now because it brings programmers and businessmen the
most money, not because it is the only way to make a living.  It is easy to
find other ways if you want to find them.  Here are a number of examples.

A manufacturer introducing a new computer will pay for the porting of
operating systems onto the new hardware.

The sale of teaching, hand-holding and maintenance services could also
employ programmers.

People with new ideas could distribute programs as freeware, asking for
donations from satisfied users, or selling hand-holding services.  I have
met people who are already working this way successfully.

Users with related needs can form users' groups, and pay dues.  A group
would contract with programming companies to write programs that the
group's members would like to use.

All sorts of development can be funded with a Software Tax:

@quotation
Suppose everyone who buys a computer has to pay x percent of
the price as a software tax.  The government gives this to
an agency like the NSF to spend on software development.

But if the computer buyer makes a donation to software development
himself, he can take a credit against the tax.  He can donate to
the project of his own choosing---often, chosen because he hopes to
use the results when it is done.  He can take a credit for any amount
of donation up to the total tax he had to pay.

The total tax rate could be decided by a vote of the payers of
the tax, weighted according to the amount they will be taxed on.

The consequences:

@itemize @bullet
@item
The computer-using community supports software development.
@item
This community decides what level of support is needed.
@item
Users who care which projects their share is spent on
can choose this for themselves.
@end itemize
@end quotation

In the long run, making programs free is a step toward the post-scarcity
world, where nobody will have to work very hard just to make a living.
People will be free to devote themselves to activities that are fun, such
as programming, after spending the necessary ten hours a week on required
tasks such as legislation, family counseling, robot repair and asteroid
prospecting.  There will be no need to be able to make a living from
programming.

We have already greatly reduced the amount of work that the whole society
must do for its actual productivity, but only a little of this has
translated itself into leisure for workers because much nonproductive
activity is required to accompany productive activity.  The main causes of
this are bureaucracy and isometric struggles against competition.  Free
software will greatly reduce these drains in the area of software
production.  We must do this, in order for technical gains in productivity
to translate into less work for us.

@node Glossary, Key Index, Intro, Top
@unnumbered Glossary

@table @asis
@item Abbrev
An abbrev is a text string which expands into a different text string
when present in the buffer.  For example, you might define a short
word as an abbrev for a long phrase that you want to insert
frequently.  @xref{Abbrevs}.

@item Aborting
Aborting means getting out of a recursive edit (q.v.@:).  The
commands @kbd{C-]} and @kbd{M-x top-level} are used for this.
@xref{Quitting}.

@item Auto Fill mode
Auto Fill mode is a minor mode in which text that you insert is
automatically broken into lines of fixed width.  @xref{Filling}.

@item Balance Parentheses
Emacs can balance parentheses manually or automatically.  Manual
balancing is done by the commands to move over balanced expressions
(@pxref{Lists}).  Automatic balancing is done by blinking the
parenthesis that matches one just inserted (@pxref{Matching,,Matching
Parens}).

@item Bind
To bind a key is to change its binding (q.v.@:).  @xref{Rebinding}.

@item Binding
A key gets its meaning in Emacs by having a binding which is a
command (q.v.@:), a Lisp function that is run when the key is typed.
@xref{Commands,Binding}.  Customization often involves rebinding a
character to a different command function.  The bindings of all keys
are recorded in the keymaps (q.v.@:).  @xref{Keymaps}.

@item Blank Lines
Blank lines are lines that contain only whitespace.  Emacs has several
commands for operating on the blank lines in the buffer.

@item Buffer
The buffer is the basic editing unit; one buffer corresponds to one
piece of text being edited.  You can have several buffers, but at any
time you are editing only one, the `selected' buffer, though several
can be visible when you are using multiple windows.  @xref{Buffers}.

@item Buffer Selection History
Emacs keeps a buffer selection history which records how recently each
Emacs buffer has been selected.  This is used for choosing a buffer to
select.  @xref{Buffers}.

@item C-
@samp{C} in the name of a character is an abbreviation for Control.
@xref{Characters,C-}.

@item C-M-
@samp{C-M-} in the name of a character is an abbreviation for
Control-Meta.  @xref{Characters,C-M-}.

@item Case Conversion
Case conversion means changing text from upper case to lower case or
vice versa.  @xref{Case}, for the commands for case conversion.

@item Characters
Characters form the contents of an Emacs buffer; also, Emacs commands
are invoked by keys (q.v.@:), which are sequences of one or more
characters.  @xref{Characters}.

@item Command
A command is a Lisp function specially defined to be able to serve as
a key binding in Emacs.  When you type a key (q.v.@:), its binding
(q.v.@:) is looked up in the relevant keymaps (q.v.@:) to find the
command to run.  @xref{Commands}.

@item Command Name
A command name is the name of a Lisp symbol which is a command
(@pxref{Commands}).  You can invoke any command by its name using
@kbd{M-x} (@pxref{M-x}).

@item Comments
A comment is text in a program which is intended only for humans
reading the program, and is marked specially so that it will be
ignored when the program is loaded or compiled.  Emacs offers special
commands for creating, aligning and killing comments.
@xref{Comments}.

@item Compilation
Compilation is the process of creating an executable program from
source code.  Emacs has commands for compiling files of Emacs Lisp
code (@pxref{Lisp Libraries}) and programs in C and other languages
(@pxref{Compilation}).

@item Completion
Completion is what Emacs does when it automatically fills out an
abbreviation for a name into the entire name.  Completion is done for
minibuffer (q.v.@:) arguments, when the set of possible valid inputs
is known; for example, on command names, buffer names, and
file names.  Completion occurs when @key{TAB}, @key{SPC} or @key{RET}
is typed.  @xref{Completion}.@refill

@item Continuation Line
When a line of text is longer than the width of the screen, it
takes up more than one screen line when displayed.  We say that the
text line is continued, and all screen lines used for it after the
first are called continuation lines.  @xref{Basic,Continuation,Basic
Editing}.

@item Control-Character
ASCII characters with octal codes 0 through 040, and also code 0177,
do not have graphic images assigned to them.  These are the control
characters.  Any control character can be typed by holding down the
@key{CTRL} key and typing some other character; some have special keys
on the keyboard.  @key{RET}, @key{TAB}, @key{ESC}, @key{LFD} and
@key{DEL} are all control characters.  @xref{Characters}.@refill

@item Current Buffer
The current buffer in Emacs is the Emacs buffer on which most editing
commands operate.  You can select any Emacs buffer as the current one.
@xref{Buffers}.

@item Current Line
The line point is on (@pxref{Point}).

@item Current Paragraph
The paragraph that point is in.  If point is between paragraphs, the
current paragraph is the one that follows point.  @xref{Paragraphs}.

@item Current Defun
The defun (q.v.@:) that point is in.  If point is between defuns, the
current defun is the one that follows point.  @xref{Defuns}.

@item Cursor
The cursor is the rectangle on the screen which indicates the position
called point (q.v.@:) at which insertion and deletion takes place.
Often people speak of `the cursor' when, strictly speaking, they mean
`point'.  @xref{Basic,Cursor,Basic Editing}.

@item Customization
Customization is making minor changes in the way Emacs works.  It is
often done by setting variables (@pxref{Variables}) or by rebinding
keys (@pxref{Keymaps}).

@item Default Argument
The default for an argument is the value that will be assumed if you
do not specify one.  When the minibuffer is used to read an argument,
the default argument is used if you just type @key{RET}.
@xref{Minibuffer}.

@item Default Directory
When you specify a file name that does not start with @samp{/} or @samp{~},
it is interpreted relative to the current buffer's default directory.
@xref{Minibuffer File,Default Directory}.

@item Defun
A defun is a list at the top level of parenthesis or bracket structure
in a program.  It is so named because most such lists in Lisp programs
are calls to the Lisp function @code{defun}.  @xref{Defuns}.

@item @key{DEL}
@key{DEL} is a character that runs the command to delete one character of
text.  @xref{Basic,DEL,Basic Editing}.

@item Deletion
Deletion means erasing text without saving it.  Emacs deletes text
only when it is expected not to be worth saving (all whitespace, or
only one character).  The alternative is killing (q.v.@:).
@xref{Killing,Deletion}.

@item Deletion of Files
Deleting a file means erasing it from the file system.  @xref{Misc
File Ops}.

@item Deletion of Messages
Deleting a message means flagging it to be eliminated from your mail
file.  This can be undone by undeletion until the mail file is expunged.
@xref{Rmail Deletion}.

@item Deletion of Windows
Deleting a window means eliminating it from the screen.  Other windows
expand to use up the space.  The deleted window can never come back,
but no actual text is thereby lost.  @xref{Windows}.

@item Directory
Files in the Unix file system are grouped into file directories.
@xref{ListDir,,Directories}.

@item Dired
Dired is the Emacs facility that displays the contents of a file
directory and allows you to ``edit the directory'', performing
operations on the files in the directory.  @xref{Dired}.

@item Disabled Command
A disabled command is one that you may not run without special
confirmation.  The usual reason for disabling a command is that it is
confusing for beginning users.  @xref{Disabling}.

@item Dribble File
A file into which Emacs writes all the characters that the user types
on the keyboard.  Dribble files are used to make a record for
debugging Emacs bugs.  Emacs does not make a dribble file unless you
tell it to.  @xref{Bugs}.

@item Echo Area
The echo area is the bottom line of the screen, used for echoing the
arguments to commands, for asking questions, and printing brief
messages (including error messages).  @xref{Echo Area}.

@item Echoing
Echoing is acknowledging the receipt of commands by displaying them
(in the echo area).  Emacs never echoes single-character keys; longer
keys echo only if you pause while typing them.

@item Error Messages
Error messages are single lines of output printed by Emacs when the
user asks for something impossible to do (such as, killing text
forward when point is at the end of the buffer).  They appear in the
echo area, accompanied by a beep.

@item @key{ESC}
@key{ESC} is a character, used to end incremental searches and as a
prefix for typing Meta characters on keyboards lacking a @key{META}
key.

@item Fill Prefix
The fill prefix is a string that should be expected at the beginning
of each line when filling is done.  It is not regarded as part of the
text to be filled.  @xref{Filling}.

@item Filling
Filling text means moving text from line to line so that all the lines
are approximately the same length.  @xref{Filling}.

@item Global
Global means `independent of the current environment; in effect
throughout Emacs'.  It is the opposite of local (q.v.@:).  Particular
examples of the use of `global' appear below.

@item Global Abbrev
A global definition of an abbrev (q.v.@:) is effective in all major
modes that do not have local (q.v.@:) definitions for the same abbrev.
@xref{Abbrevs}.

@item Global Keymap
The global keymap (q.v.@:) contains key bindings that are in effect
except when overridden by local key bindings in a major mode's local
keymap (q.v.@:).  @xref{Keymaps}.

@item Global Substitution
Global substitution means replacing each occurrence of one string by
another string through a large amount of text.  @xref{Replace}.

@item Global Variable
The global value of a variable (q.v.@:) takes effect in all buffers
that do not have their own local (q.v.@:) values for the variable.
@xref{Variables}.

@item Graphic Character
Graphic characters are those assigned pictorial images rather than
just names.  All the non-Meta (q.v.@:) characters except for the
Control (q.v.@:) characters are graphic characters.  These include
letters, digits, punctuation, and spaces; they do not include
@key{RET} or @key{ESC}.  In Emacs, typing a graphic character inserts
that character (in ordinary editing modes).  @xref{Basic,,Basic Editing}.

@item Grinding
Grinding means adjusting the indentation in a program to fit the
nesting structure.  @xref{Indentation,Grinding}.

@item Hardcopy
Hardcopy means printed output.  Emacs has commands for making printed
listings of text in Emacs buffers.  @xref{Hardcopy}.

@item @key{HELP}
You can type @key{HELP} at any time to ask what options you have, or
to ask what any command does.  @key{HELP} is really @kbd{Control-h}.
@xref{Help}.

@item Indentation
Indentation means blank space at the beginning of a line.  Most
programming languages have conventions for using indentation to
illuminate the structure of the program, and Emacs has special
features to help you set up the correct indentation.
@xref{Indentation}.

@item Insertion
Insertion means copying text into the buffer, either from the keyboard
or from some other place in Emacs.

@item Justification
Justification means adding extra spaces to lines of text to make them
come exactly to a specified width.  @xref{Filling,Justification}.

@item Keyboard Macros
Keyboard macros are a way of defining new Emacs commands from
sequences of existing ones, with no need to write a Lisp program.
@xref{Keyboard Macros}.

@item Key
A key is a character or sequence of characters which, when typed by
the user, fully specifies one action to be performed by Emacs.  For
example, @kbd{X} and @kbd{Control-f} and @kbd{Control-x m} are keys.
Keys derive their meanings from being bound (q.v.@:) to commands
(q.v.@:).  @xref{Keys}.

@item Keymap
The keymap is the data structure that records the bindings (q.v.@:) of
keys to the commands that they run.  For example, the keymap binds the
character @kbd{C-n} to the command function @code{next-line}.
@xref{Keymaps}.

@item Kill Ring
The kill ring is where all text you have killed recently is saved.
You can reinsert any of the killed text still in the ring; this is
called yanking (q.v.@:).  @xref{Yanking}.

@item Killing
Killing means erasing text and saving it on the kill ring so it can be
yanked (q.v.@:) later.  Most Emacs commands to erase text do killing,
as opposed to deletion (q.v.@:).  @xref{Killing}.

@item Killing Jobs
Killing a job (such as, an invocation of Emacs) means making it cease
to exist.  Any data within it, if not saved in a file, is lost.
@xref{Exiting}.

@item List
A list is, approximately, a text string beginning with an open
parenthesis and ending with the matching close parenthesis.  In C mode
and other non-Lisp mode groupings surrounded by other kinds of matched
delimiters appropriate to the language, such as braces, are also
considered lists.  Emacs has special commands for many operations on
lists.  @xref{Lists}.

@item Local
Local means `in effect only in a particular context'; the relevant
kind of context is a particular function execution, a particular
buffer, or a particular major mode.  It is the opposite of `global'
(q.v.@:).  Specific uses of `local' in Emacs terminology appear below.

@item Local Abbrev
A local abbrev definition is effective only if a particular major mode
is selected.  In that major mode, it overrides any global definition
for the same abbrev.  @xref{Abbrevs}.

@item Local Keymap
A local keymap is used in a particular major mode; the key bindings
(q.v.@:) in the current local keymap override global bindings of the
same keys.  @xref{Keymaps}.

@item Local Variable
A local value of a variable (q.v.@:) applies to only one buffer.
@xref{Locals}.

@item M-
@kbd{M-} in the name of a character is an abbreviation for @key{META},
one of the modifier keys that can accompany any character.
@xref{Characters}.

@item M-C-
@samp{M-C-} in the name of a character is an abbreviation for
Control-Meta; it means the same thing as @samp{C-M-}.
@xref{Characters,C-M-}.

@item M-x
@kbd{M-x} is the key which is used to call an Emacs command by name.
This is how commands that are not bound to keys are called.
@xref{M-x}.

@item Mail
Mail means messages sent from one user to another through the computer
system, to be read at the recipient's convenience.  Emacs has commands for
composing and sending mail, and for reading and editing the mail you have
received.  @xref{Sending Mail}.

@item Major Mode
The major modes are a mutually exclusive set of options each of which
configures Emacs for editing a certain sort of text.  Ideally, each
programming language has its own major mode.  @xref{Major Modes}.

@item Mark
The mark points to a position in the text.  It specifies one end of
the region (q.v.@:), point being the other end.  Many commands operate
on all the text from point to the mark.  @xref{Mark}.

@item Mark Ring
The mark ring is used to hold several recent previous locations of the
mark, just in case you want to move back to them.  @xref{Mark Ring}.

@item Message
See `mail'.

@item Meta
Meta is the name of a modifier bit which a command character may have.
It is present in a character if the character is typed with the
@key{META} key held down.  Such characters are given names that start
with @kbd{Meta-}.  For example, @kbd{Meta-<} is typed by holding down
@key{META} and typing @kbd{<} (which itself is done, on most terminals,
by holding down @key{SHIFT} and typing @kbd{,}).  @xref{Characters,Meta}.

@item Meta Character
A Meta character is one whose character code includes the Meta bit.

@item Minibuffer
The minibuffer is the window that appears when necessary inside the
echo area (q.v.@:), used for reading arguments to commands.
@xref{Minibuffer}.

@item Minor Mode
A minor mode is an optional feature of Emacs which can be switched on
or off independently of all other features.  Each minor mode has a
command to turn it on or off.  @xref{Minor Modes}.

@item Mode Line
The mode line is the line at the bottom of each text window (q.v.@:),
which gives status information on the buffer displayed in that window.
@xref{Mode Line}.

@item Modified Buffer
A buffer (q.v.@:) is modified if its text has been changed since the
last time the buffer was saved (or since when it was created, if it
has never been saved).  @xref{Saving}.

@item Moving Text
Moving text means erasing it from one place and inserting it in
another.  This is done by killing (q.v.@:) and then yanking (q.v.@:).
@xref{Killing}.

@item Named Mark
A named mark is a register (q.v.@:) in its role of recording a
location in text so that you can move point to that location.
@xref{Registers}.

@item Narrowing
Narrowing means creating a restriction (q.v.@:) that limits editing in
the current buffer to only a part of the text in the buffer.  Text
outside that part is inaccessible to the user until the boundaries are
widened again, but it is still there, and saving the file saves it
all.  @xref{Narrowing}.

@item Newline
@key{LFD} characters in the buffer terminate lines of text and are
called newlines.  @xref{Characters,Newline}.

@item Numeric Argument
A numeric argument is a number, specified before a command, to change
the effect of the command.  Often the numeric argument serves as a
repeat count.  @xref{Arguments}.

@item Option
An option is a variable (q.v.@:) that exists so that you can customize
Emacs by giving it a new value.  @xref{Variables}.

@item Overwrite Mode
Overwrite mode is a minor mode.  When it is enabled, ordinary text
characters replace the existing text after point rather than pushing
it to the right.  @xref{Minor Modes}.

@item Page
A page is a unit of text, delimited by formfeed characters (ASCII
Control-L, code 014) coming at the beginning of a line.  Some Emacs
commands are provided for moving over and operating on pages.
@xref{Pages}.

@item Paragraphs
Paragraphs are the medium-size unit of English text.  There are
special Emacs commands for moving over and operating on paragraphs.
@xref{Paragraphs}.

@item Parsing
We say that Emacs parses words or expressions in the text being
edited.  Really, all it knows how to do is find the other end of a
word or expression.  @xref{Syntax}.

@item Point
Point is the place in the buffer at which insertion and deletion
occur.  Point is considered to be between two characters, not at one
character.  The terminal's cursor (q.v.@:) indicates the location of
point.  @xref{Basic,Point}.

@item Prefix Key
A prefix key is a key (q.v.@:) whose sole function is to introduce a
set of multi-character keys.  @kbd{Control-x} is an example of prefix
key; thus, any two-character sequence starting with @kbd{C-x} is also
a legitimate key.  @xref{Keys}.

@item Prompt
A prompt is text printed to ask the user for input.  Printing a prompt
is called prompting.  Emacs prompts always appear in the echo area
(q.v.@:).  One kind of prompting happens when the minibuffer is used
to read an argument (@pxref{Minibuffer}); the echoing which happens
when you pause in the middle of typing a multicharacter key is also a
kind of prompting (@pxref{Echo Area}).

@item Quitting
Quitting means cancelling a partially typed command or a running
command, using @kbd{C-g}.  @xref{Quitting}.

@item Quoting
Quoting means depriving a character of its usual special significance.
In Emacs this is usually done with @kbd{Control-q}.  What constitutes special
significance depends on the context and on convention.  For example,
an ``ordinary'' character as an Emacs command inserts itself; so in
this context, a special character is any character that does not
normally insert itself (such as @key{DEL}, for example), and quoting
it makes it insert itself as if it were not special.  Not all contexts
allow quoting.  @xref{Basic,Quoting,Basic Editing}.

@item Read-only Buffer
A read-only buffer is one whose text you are not allowed to change.
Normally Emacs makes buffers read-only when they contain text which
has a special significance to Emacs; for example, Dired buffers.
Visiting a file that is write protected also makes a read-only buffer.
@xref{Buffers}.

@item Recursive Editing Level
A recursive editing level is a state in which part of the execution of
a command involves asking the user to edit some text.  This text may
or may not be the same as the text to which the command was applied.
The mode line indicates recursive editing levels with square brackets
(@samp{[} and @samp{]}).  @xref{Recursive Edit}.

@item Redisplay
Redisplay is the process of correcting the image on the screen to
correspond to changes that have been made in the text being edited.
@xref{Screen,Redisplay}.

@item Region
The region is the text between point (q.v.@:) and the mark (q.v.@:).
Many commands operate on the text of the region.  @xref{Mark,Region}.

@item Registers
Registers are named slots in which text or buffer positions or
rectangles can be saved for later use.  @xref{Registers}.

@item Replacement
See `global substitution'.

@item Restriction
A buffer's restriction is the amount of text, at the beginning or the
end of the buffer, that is temporarily invisible and inaccessible.
Giving a buffer a nonzero amount of restriction is called narrowing
(q.v.).  @xref{Narrowing}.

@item @key{RET}
@key{RET} is a character than in Emacs runs the command to insert a
newline into the text.  It is also used to terminate most arguments
read in the minibuffer (q.v.@:).  @xref{Characters,Return}.

@item Saving
Saving a buffer means copying its text into the file that was visited
(q.v.@:) in that buffer.  This is the way text in files actually gets
changed by your Emacs editing.  @xref{Saving}.

@item Scrolling
Scrolling means shifting the text in the Emacs window so as to see a
different part of the buffer.  @xref{Display,Scrolling}.

@item Searching
Searching means moving point to the next occurrence of a specified
string.  @xref{Search}.

@item Selecting
Selecting a buffer means making it the current (q.v.@:) buffer.
@xref{Buffers,Selecting}.

@item Self-documentation
Self-documentation is the feature of Emacs which can tell you what any
command does, or give you a list of all commands related to a topic
you specify.  You ask for self-documentation with the @key{HELP}
character.  @xref{Help}.

@item Sentences
Emacs has commands for moving by or killing by sentences.
@xref{Sentences}.

@item Sexp
A sexp (short for `s-expression') is the basic syntactic unit of Lisp
in its textual form: either a list, or Lisp atom.  Many Emacs commands
operate on sexps.  The term `sexp' is generalized to languages other
than Lisp, to mean a syntactically recognizable expression.
@xref{Lists,Sexps}.

@item Simultaneous Editing
Simultaneous editing means two users modifying the same file at once.
Simultaneous editing if not detected can cause one user to lose his
work.  Emacs detects all cases of simultaneous editing and warns the
user to investigate them.  @xref{Interlocking,,Simultaneous Editing}.

@item String
A string is a kind of Lisp data object which contains a sequence of
characters.  Many Emacs variables are intended to have strings as
values.  The Lisp syntax for a string consists of the characters in
the string with a @samp{"} before and another @samp{"} after.  A
@samp{"} that is part of the string must be written as @samp{\"} and a
@samp{\} that is part of the string must be written as @samp{\\}.  All
other characters, including newline, can be included just by writing
them inside the string; however, escape sequences as in C, such as
@samp{\n} for newline or @samp{\241} using an octal character code,
are allowed as well.

@item String Substitution
See `global substitution'.

@item Syntax Table
The syntax table tells Emacs which characters are part of a word,
which characters balance each other like parentheses, etc.
@xref{Syntax}.

@item Tag Table
A tag table is a file that serves as an index to the function
definitions in one or more other files.  @xref{Tags}.

@item Termscript File
A termscript file contains a record of all characters sent by Emacs to
the terminal.  It is used for tracking down bugs in Emacs redisplay.
Emacs does not make a termscript file unless you tell it to.
@xref{Bugs}.

@item Text
Two meanings (@pxref{Text}):

@itemize @bullet
@item
Data consisting of a sequence of characters.  The contents of an
Emacs buffer are always text in this sense.
@item
Data consisting of written human language, as opposed to programs,
or following the stylistic conventions of human language.
@end itemize

@item Top Level
Top level is the normal state of Emacs, in which you are editing the
text of the file you have visited.  You are at top level whenever you
are not in a recursive editing level (q.v.@:) or the minibuffer
(q.v.@:), and not in the middle of a command.  You can get back to top
level by aborting (q.v.@:) and quitting (q.v.@:).  @xref{Quitting}.

@item Transposition
Transposing two units of text means putting each one into the place
formerly occupied by the other.  There are Emacs commands to transpose
two adjacent characters, words, sexps (q.v.@:) or lines
(@pxref{Transpose}).

@item Truncation
Truncating text lines in the display means leaving out any text on a
line that does not fit within the right margin of the window
displaying it.  See also `continuation line'.
@xref{Basic,Truncation,Basic Editing}.

@item Undoing
Undoing means making your previous editing go in reverse, bringing
back the text that existed earlier in the editing session.
@xref{Undo}.

@item Variable
A variable is an object in Lisp that can store an arbitrary value.
Emacs uses some variables for internal purposes, and has others (known
as `options' (q.v.@:)) just so that you can set their values to
control the behavior of Emacs.  The variables used in Emacs that you
are likely to be interested in are listed in the Variables Index in
this manual.  @xref{Variables}, for information on variables.

@item Visiting
Visiting a file means loading its contents into a buffer (q.v.@:)
where they can be edited.  @xref{Visiting}.

@item Whitespace
Whitespace is any run of consecutive formatting characters (space,
tab, newline, and backspace).

@item Widening
Widening is removing any restriction (q.v.@:) on the current buffer;
it is the opposite of narrowing (q.v.@:).  @xref{Narrowing}.

@item Window
Emacs divides the screen into one or more windows, each of which can
display the contents of one buffer (q.v.@:) at any time.
@xref{Screen}, for basic information on how Emacs uses the screen.
@xref{Windows}, for commands to control the use of windows.

@item Word Abbrev
Synonymous with `abbrev'.

@item Word Search
Word search is searching for a sequence of words, considering the
punctuation between them as insignificant.  @xref{Word Search}.

@item Yanking
Yanking means reinserting text previously killed.  It can be used to
undo a mistaken kill, or for copying or moving text.  @xref{Yanking}.
@end table

@node Key Index, Command Index, Glossary, Top
@unnumbered Key (Character) Index
@printindex ky

@node Command Index, Variable Index, Key Index, Top
@unnumbered Command and Function Index
@printindex fn

@node Variable Index, Concept Index, Command Index, Top
@unnumbered Variable Index
@printindex vr

@node Concept Index, Screen, Variable Index, Top
@unnumbered Concept Index
@printindex cp

@summarycontents
@contents
@bye
