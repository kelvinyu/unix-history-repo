% run this through LaTeX

\input lcustom
\draftfalse
\input version

\documentstyle[12pt,DScustom,sfwmac]{article}
\setcounter{page}{0}
\pagestyle{empty}

\begin{document}

\title{The Rand MH Message Handling System:\\
	Myths about MH}
\author{Marshall T.~Rose\\
	Northrop Research and Technology Center\\
	One~Research Park\\
	Palos Verdes Peninsula, CA  90274}
\date{\ifdraft \versiondate/\\ Version \versiontag/\else \today\fi}
\maketitle
\footnotetext[0]{\hskip -\parindent
This document (version \versiontag/)
was \LaTeX set \today\ with \fmtname\ v\fmtversion.}%

\begin{abstract}
\noindent Mail handlers, like text editors, text formatters,
programming languages, and computer-communication network technologies,
have become the basis of religious wars.
Although the UCI version of the Rand Message Handling System is well beloved
(and deservedly so) by its disciples,
there remain those who spread non-truths about \MH/.
This document seeks to set the record straight.

Of course,
the gentle reader should understand that the author uses this forum primarily
as a non-violent release of hostility.
Note however,
that this document is not a case of ``me against them''.
Rather it is a case of ``us against them''.
There are a lot of \MH/ supporters,
though
I don't pretend that this paper represents anyone's views other than my own.
Furthermore,
I only write like this when I'm upset.
As such,
this paper shouldn't be taken {\em too\/} seriously.
\end{abstract}

\bop\pagestyle{plain}\pagenumbering{arabic}

\section*	{The Plain Facts}
I really hate writing, especially documentation.
I don't mind having written so much, but I don't like writing.

To put the reader in the right mood,
let us recall the words of Lord John Whorfin:

\begin{verse}
Sealed with a curse,\\
\qquad as sharp as a knife;\\
doomed is your soul,\\
\qquad and damned is your life.
\end{verse}

This is my standard response to people who want more documentation on \MH/.
The \MH/ documentation set is way too large as it is.

Now with that out of the way,
onto the plain facts.

\begin{enumerate}
\item	Speed versus Performance\hbreak
For some reason,
people think that \MH/ is {\em slow}.
This is not true.

It is true that \MH/ is {\em slower\/} than some monolithic user agents.
The reason for this is that since each \MH/ command is a \unix/ program,
there is a larger initial cost for running each \MH/ program.
Of course,
once an \MH/ program has loaded its state information,
it can execute quite quickly.
In particular,
a lot of time has gone into tuning \MH/ towards fast execution.
(Everyone owes a hearty thanks to Van Jacobson who did most of this work.)

It is important to distinguish between how fast a system runs,
what work that system does,
and the ratio between the two.
Although \MH/ may not be fast,
it does perform well.

\item	Hardware/Software Dependencies\hbreak
For some reason,
people think that \MH/ is riddled with \vax/, \bsd/~\unix/, and/or \SendMail/
dependencies.
Nothing could be further from the truth.

\MH/ does run on all \bsd/ releases of \unix/ since 4.1\bsd/.
It also runs on V7~\unix/ and various \xenix/ variants of \unix/.
Recently, the \MH/ distribution has taken to supporting the AT\&T variant of
\unix/, System~5.
\MH/ does not have support for System~3.
If someone would like to port \MH/ to System~3,
please let me know.

\MH/ does run on Digital Equipment Corporation's \vax/-family of computers
(providing that host is running a \unix/ mentioned above).
It also runs on a large number of other hosts,
such as SUNs, Integrated Solutions, Pyramid-90x's, Gould FireBreather's, 
ALTOS's, 3B2's, \pdp/-11's, and so on.

\MH/ does run with \SendMail/ as its message transport agent.
It also runs with \MMDFI/ and \MMDFII/,
and provides it's own stand-alone delivery system with \UUCP/ support
(e.g., \pgm{rmail\/}).
Furthermore,
if your host can make an SMTP connection to another host,
you don't even need to use any of these programs.
You can simply instruct \MH/ to open an SMTP connection,
using a flexible search-list of service hosts (and network),
to any service host accepting mail.
The bottom line is that \MH/ can really run with any 822--based message
transport agent.

\item	The BBoards channel and distribution lists\hbreak
For some reason,
people think that the UCI BBoards facility is good only for local receipt of
BBoards.
It is true that the UCI BBoards facility provides excellent support for local
BBoard subscribers.
However,
the UCI BBoards facility also supports continued distribution.

The BBoards channel does both local delivery and remote distribution for lists.
For local delivery,
it delivers messages into maildrops in a spool area.
In addition,
it allows the \MH/ user to shorten the typein of distribution list addresses.
For remote distribution,
it does the usual list exploding and error trapping.
\end{enumerate}

\section*	{The End}
That's it for now.

\bibliography{myths}

\showsummary

\end{document}
