% Reference Card for GNU Emacs version 18 on Unix systems
%**start of header
\newcount\columnsperpage

% This file can be printed with 1, 2, or 3 columns per page (see below).
% Specify how many you want here.  Nothing else needs to be changed.

\columnsperpage=1

% Copyright (c) 1987 Free Software Foundation, Inc.

% This file is part of GNU Emacs.

% GNU Emacs is free software; you can redistribute it and/or modify
% it under the terms of the GNU General Public License as published by
% the Free Software Foundation; either version 1, or (at your option)
% any later version.

% GNU Emacs is distributed in the hope that it will be useful,
% but WITHOUT ANY WARRANTY; without even the implied warranty of
% MERCHANTABILITY or FITNESS FOR A PARTICULAR PURPOSE.  See the
% GNU General Public License for more details.

% You should have received a copy of the GNU General Public License
% along with GNU Emacs; see the file COPYING.  If not, write to
% the Free Software Foundation, 675 Mass Ave, Cambridge, MA 02139, USA.

% This file is intended to be processed by plain TeX (TeX82).
%
% The final reference card has six columns, three on each side.
% This file can be used to produce it in any of three ways:
% 1 column per page
%    produces six separate pages, each of which needs to be reduced to 80%.
%    This gives the best resolution.
% 2 columns per page
%    produces three already-reduced pages.
%    You will still need to cut and paste.
% 3 columns per page
%    produces two pages which must be printed sideways to make a
%    ready-to-use 8.5 x 11 inch reference card.
%    For this you need a dvi device driver that can print sideways.
% Which mode to use is controlled by setting \columnsperpage above.
%
% Author:
%  Stephen Gildea
%  UUCP: mit-erl!gildea
%  Internet: gildea@erl.mit.edu
%
% Thanks to Paul Rubin, Bob Chassell, Len Tower, and Richard Mlynarik
% for their many good ideas.

% If there were room, it would be nice to see sections on
% Abbrevs, Rectangles, and Dired.

\def\versionnumber{1.9}
\def\year{1987}
\def\version{March \year\ v\versionnumber}

\def\shortcopyrightnotice{\vskip 1ex plus 2 fill
  \centerline{\small \copyright\ \year\ Free Software Foundation, Inc.
  Permissions on back.  v\versionnumber}}

\def\copyrightnotice{
\vskip 1ex plus 2 fill\begingroup\small
\centerline{Copyright \copyright\ \year\ Free Software Foundation, Inc.}
\centerline{designed by Stephen Gildea, \version}
\centerline{for GNU Emacs version 18 on Unix systems}

Permission is granted to make and distribute copies of
this card provided the copyright notice and this permission notice
are preserved on all copies.

For copies of the GNU Emacs manual, write to the Free Software
Foundation, Inc., 675 Massachusetts Ave, Cambridge MA 02139.

\endgroup}

% make \bye not \outer so that the \def\bye in the \else clause below
% can be scanned without complaint.
\def\bye{\par\vfill\supereject\end}

\newdimen\intercolumnskip
\newbox\columna
\newbox\columnb

\def\ncolumns{\the\columnsperpage}

\message{[\ncolumns\space 
  column\if 1\ncolumns\else s\fi\space per page]}

\def\scaledmag#1{ scaled \magstep #1}

% This multi-way format was designed by Stephen Gildea
% October 1986.
\if 1\ncolumns
  \hsize 4in
  \vsize 10in
  \voffset -.7in
  \font\titlefont=\fontname\tenbf \scaledmag3
  \font\headingfont=\fontname\tenbf \scaledmag2
  \font\smallfont=\fontname\sevenrm
  \font\smallsy=\fontname\sevensy

  \footline{\hss\folio}
  \def\makefootline{\baselineskip10pt\hsize6.5in\line{\the\footline}}
\else
  \hsize 3.2in
  \vsize 7.95in
  \hoffset -.75in
  \voffset -.745in
  \font\titlefont=cmbx10 \scaledmag2
  \font\headingfont=cmbx10 \scaledmag1
  \font\smallfont=cmr6
  \font\smallsy=cmsy6
  \font\eightrm=cmr8
  \font\eightbf=cmbx8
  \font\eightit=cmti8
  \font\eighttt=cmtt8
  \font\eightsy=cmsy8
  \textfont0=\eightrm
  \textfont2=\eightsy
  \def\rm{\eightrm}
  \def\bf{\eightbf}
  \def\it{\eightit}
  \def\tt{\eighttt}
  \normalbaselineskip=.8\normalbaselineskip
  \normallineskip=.8\normallineskip
  \normallineskiplimit=.8\normallineskiplimit
  \normalbaselines\rm		%make definitions take effect

  \if 2\ncolumns
    \let\maxcolumn=b
    \footline{\hss\rm\folio\hss}
    \def\makefootline{\vskip 2in \hsize=6.86in\line{\the\footline}}
  \else \if 3\ncolumns
    \let\maxcolumn=c
    \nopagenumbers
  \else
    \errhelp{You must set \columnsperpage equal to 1, 2, or 3.}
    \errmessage{Illegal number of columns per page}
  \fi\fi

  \intercolumnskip=.46in
  \def\abc{a}
  \output={%
      % This next line is useful when designing the layout.
      %\immediate\write16{Column \folio\abc\space starts with \firstmark}
      \if \maxcolumn\abc \multicolumnformat \global\def\abc{a}
      \else\if a\abc
	\global\setbox\columna\columnbox \global\def\abc{b}
        %% in case we never use \columnb (two-column mode)
        \global\setbox\columnb\hbox to -\intercolumnskip{}
      \else
	\global\setbox\columnb\columnbox \global\def\abc{c}\fi\fi}
  \def\multicolumnformat{\shipout\vbox{\makeheadline
      \hbox{\box\columna\hskip\intercolumnskip
        \box\columnb\hskip\intercolumnskip\columnbox}
      \makefootline}\advancepageno}
  \def\columnbox{\leftline{\pagebody}}

  \def\bye{\par\vfill\supereject
    \if a\abc \else\null\vfill\eject\fi
    \if a\abc \else\null\vfill\eject\fi
    \end}  
\fi

% we won't be using math mode much, so redefine some of the characters
% we might want to talk about
\catcode`\^=12
\catcode`\_=12

\chardef\\=`\\
\chardef\{=`\{
\chardef\}=`\}

\hyphenation{mini-buf-fer}

\parindent 0pt
\parskip 1ex plus .5ex minus .5ex

\def\small{\smallfont\textfont2=\smallsy\baselineskip=.8\baselineskip}

\outer\def\newcolumn{\vfill\eject}

\outer\def\title#1{{\titlefont\centerline{#1}}\vskip 1ex plus .5ex}

\outer\def\section#1{\par\filbreak
  \vskip 3ex plus 2ex minus 2ex {\headingfont #1}\mark{#1}%
  \vskip 2ex plus 1ex minus 1.5ex}

\newdimen\keyindent

\def\beginindentedkeys{\keyindent=1em}
\def\endindentedkeys{\keyindent=0em}
\endindentedkeys

\def\paralign{\vskip\parskip\halign}

\def\<#1>{$\langle${\rm #1}$\rangle$}

\def\kbd#1{{\tt#1}\null}	%\null so not an abbrev even if period follows

\def\beginexample{\par\leavevmode\begingroup
  \obeylines\obeyspaces\parskip0pt\tt}
{\obeyspaces\global\let =\ }
\def\endexample{\endgroup}

\def\key#1#2{\leavevmode\hbox to \hsize{\vtop
  {\hsize=.75\hsize\rightskip=1em
  \hskip\keyindent\relax#1}\kbd{#2}\hfil}}

\newbox\metaxbox
\setbox\metaxbox\hbox{\kbd{M-x }}
\newdimen\metaxwidth
\metaxwidth=\wd\metaxbox

\def\metax#1#2{\leavevmode\hbox to \hsize{\hbox to .75\hsize
  {\hskip\keyindent\relax#1\hfil}%
  \hskip -\metaxwidth minus 1fil
  \kbd{#2}\hfil}}

\def\threecol#1#2#3{\hskip\keyindent\relax#1\hfil&\kbd{#2}\quad
  &\kbd{#3}\quad\cr}

%**end of header


\title{GNU Emacs Reference Card}

\centerline{(for version 18)}

\section{Starting Emacs}

To enter Emacs, just type its name: \kbd{emacs}

To read in a file to edit, see Files, below.

\section{Leaving Emacs}

\key{suspend Emacs (the usual way of leaving it)}{C-z}
\key{exit Emacs permanently}{C-x C-c}

\section{Files}

\key{{\bf read} a file into Emacs}{C-x C-f}
\key{{\bf save} a file back to disk}{C-x C-s}
\key{{\bf insert} contents of another file into this buffer}{C-x i}
\key{replace this file with the file you really want}{C-x C-v}
\key{write buffer to a specified file}{C-x C-w}
\key{run Dired, the directory editor}{C-x d}

\section{Getting Help}

The Help system is simple.  Type \kbd{C-h} and follow the directions.
If you are a first-time user, type \kbd{C-h t} for a {\bf tutorial}.
(This card assumes you know the tutorial.)

\key{get rid of Help window}{C-x 1}
\key{scroll Help window}{ESC C-v}

\key{apropos: show commands matching a string}{C-h a}
\key{show the function a key runs}{C-h c}
\key{describe a function}{C-h f}
\key{get mode-specific information}{C-h m}

\section{Error Recovery}

\key{{\bf abort} partially typed or executing command}{C-g}
\metax{{\bf recover} a file lost by a system crash}{M-x recover-file}
\key{{\bf undo} an unwanted change}{C-x u {\rm or} C-_}
\metax{restore a buffer to its original contents}{M-x revert-buffer}
\key{redraw garbaged screen}{C-l}

\section{Incremental Search}

\key{search forward}{C-s}
\key{search backward}{C-r}
\key{regular expression search}{C-M-s}

Use \kbd{C-s} or \kbd{C-r} again to repeat the search in either direction.

\key{exit incremental search}{ESC}
\key{undo effect of last character}{DEL}
\key{abort current search}{C-g}

If Emacs is still searching, \kbd{C-g} will cancel the
part of the search not done, otherwise it aborts the entire search.

\shortcopyrightnotice

\section{Motion}

Cursor motion:

\beginindentedkeys

\paralign to \hsize{#\tabskip=10pt plus 1 fil&#\tabskip=0pt&#\cr
\threecol{{\bf entity to move over}}{{\bf backward}}{{\bf forward}}
\threecol{character}{C-b}{C-f}
\threecol{word}{M-b}{M-f}
\threecol{line}{C-p}{C-n}
\threecol{go to line beginning (or end)}{C-a}{C-e}
\threecol{sentence}{M-a}{M-e}
\threecol{paragraph}{M-[}{M-]}
\threecol{page}{C-x [}{C-x ]}
\threecol{sexp}{C-M-b}{C-M-f}
\threecol{function}{C-M-a}{C-M-e}
\threecol{go to buffer beginning (or end)}{M-<}{M->}
}
Screen motion:

\key{scroll to next screen}{C-v}
\key{scroll to previous screen}{M-v}
\key{scroll left}{C-x <}
\key{scroll right}{C-x >}

\endindentedkeys

\section{Killing and Deleting}

\paralign to \hsize{#\tabskip=10pt plus 1 fil&#\tabskip=0pt&#\cr
\threecol{{\bf entity to kill}}{{\bf backward}}{{\bf forward}}
\threecol{character (delete, not kill)}{DEL}{C-d}
\threecol{word}{M-DEL}{M-d}
\threecol{line (to end of)}{M-0 C-k}{C-k}
\threecol{sentence}{C-x DEL}{M-k}
\threecol{sexp}{M-- C-M-k}{C-M-k}
}

\key{kill {\bf region}}{C-w}
\key{kill to next occurrence of {\it char}}{M-z {\it char}}

\key{yank back last thing killed}{C-y}
\key{replace last yank with previous kill}{M-y}

\section{Marking}

\key{set mark here}{C-@ {\rm or} C-SPC}
\key{exchange point and mark}{C-x C-x}

\key{set mark {\it arg\/} {\bf words} away}{M-@}
\key{mark {\bf paragraph}}{M-h}
\key{mark {\bf page}}{C-x C-p}
\key{mark {\bf sexp}}{C-M-@}
\key{mark {\bf function}}{C-M-h}
\key{mark entire {\bf buffer}}{C-x h}

\section{Query Replace}

\key{interactively replace a text string}{M-\%}
\metax{using regular expressions}{M-x query-replace-regexp}

Valid responses in query-replace mode are

\key{{\bf replace} this one, go on to next}{SPC}
\key{replace this one, don't move}{,}
\key{{\bf skip} to next without replacing}{DEL}
\key{replace all remaining matches}{!}
\key{{\bf back up} to the previous match}{^}
\key{{\bf exit} query-replace}{ESC}
\key{enter recursive edit (\kbd{C-M-c} to exit)}{C-r}

\section{Multiple Windows}

\key{delete all other windows}{C-x 1}
\key{delete this window}{C-x 0}
\key{split window in 2 vertically}{C-x 2}
\key{split window in 2 horizontally}{C-x 5}

\key{scroll other window}{C-M-v}
\key{switch cursor to another window}{C-x o}

\metax{shrink window shorter}{M-x shrink-window}
\key{grow window taller}{C-x ^}
\key{shrink window narrower}{C-x \{}
\key{grow window wider}{C-x \}}

\key{select a buffer in other window}{C-x 4 b}
\key{find file in other window}{C-x 4 f}
\key{compose mail in other window}{C-x 4 m}
\key{run Dired in other window}{C-x 4 d}
\key{find tag in other window}{C-x 4 .}

\section{Formatting}

\key{indent current {\bf line} (mode-dependent)}{TAB}
\key{indent {\bf region} (mode-dependent)}{C-M-\\}
\key{indent {\bf sexp} (mode-dependent)}{C-M-q}
\key{indent region rigidly {\it arg\/} columns}{C-x TAB}

\key{insert newline after point}{C-o}
\key{move rest of line vertically down}{C-M-o}
\key{delete blank lines around point}{C-x C-o}
\key{delete all whitespace around point}{M-\\}
\key{put exactly one space at point}{M-SPC}

\key{fill {\bf paragraph}}{M-q}
\key{fill {\bf region}}{M-g}
\key{set fill column}{C-x f}
\key{set prefix each line starts with}{C-x .}

\section{Case Change}

\key{uppercase word}{M-u}
\key{lowercase word}{M-l}
\key{capitalize word}{M-c}

\key{uppercase region}{C-x C-u}
\key{lowercase region}{C-x C-l}
\metax{capitalize region}{M-x capitalize-region}

\section{The Minibuffer}

The following keys are defined in the minibuffer.

\key{complete as much as possible}{TAB}
\key{complete up to one word}{SPC}
\key{complete and execute}{RET}
\key{show possible completions}{?}
\key{abort command}{C-g}

Type \kbd{C-x ESC} to edit and repeat the last command that used the
minibuffer.  The following keys are then defined.

\key{previous minibuffer command}{M-p}
\key{next minibuffer command}{M-n}

\newcolumn
\title{GNU Emacs Reference Card}

\section{Buffers}

\key{select another buffer}{C-x b}
\key{list all buffers}{C-x C-b}
\key{kill a buffer}{C-x k}

\section{Transposing}

\key{transpose {\bf characters}}{C-t}
\key{transpose {\bf words}}{M-t}
\key{transpose {\bf lines}}{C-x C-t}
\key{transpose {\bf sexps}}{C-M-t}

\section{Spelling Check}

\key{check spelling of current word}{M-\$}
\metax{check spelling of all words in region}{M-x spell-region}
\metax{check spelling of entire buffer}{M-x spell-buffer}

\section{Tags}

\key{find tag}{M-.}
\key{find next occurrence of tag}{C-u M-.}
\metax{specify a new tags file}{M-x visit-tags-table}

\metax{regexp search on all files in tags table}{M-x tags-search}
\metax{query replace on all the files}{M-x tags-query-replace}
\key{continue last tags search or query-replace}{M-,}

\section{Shells}

\key{execute a shell command}{M-!}
\key{run a shell command on the region}{M-|}
\key{filter region through a shell command}{C-u M-|}
\metax{start a shell in window \kbd{*shell*}}{M-x shell}

\section{Rmail}

\key{scroll forward}{SPC}
\key{scroll reverse}{DEL}
\key{beginning of message}{. {\rm (dot)}}
\key{{\bf next} non-deleted message}{n}
\key{{\bf previous} non-deleted message}{p}
\key{next message}{M-n}
\key{previous message}{M-p}
\key{{\bf delete} message}{d}
\key{delete message and back up}{C-d}
\key{undelete message}{u}
\key{{\bf reply} to message}{r}
\key{forward message to someone}{f}
\key{send mail}{m}
\key{{\bf get} newly arrived mail}{g}
\key{{\bf quit} Rmail}{q}
\key{output message to another Rmail file}{o}
\key{output message in Unix-mail style}{C-o}
\key{show summary of headers}{h}

\section{Regular Expressions}

The following have special meaning inside a regular expression.

\key{any single character}{. {\rm(dot)}}
\key{zero or more repeats}{*}
\key{one or more repeats}{+}
\key{zero or one repeat}{?}
\key{any character in set}{[ {\rm$\ldots$} ]}
\key{any character not in set}{[^ {\rm$\ldots$} ]}
\key{beginning of line}{^}
\key{end of line}{\$}
\key{quote a special character {\it c\/}}{\\{\it c}}
\key{alternative (``or'')}{\\|}
\key{grouping}{\\( {\rm$\ldots$} \\)}
\key{{\it n\/}th group}{\\{\it n}}
\key{beginning of buffer}{\\`}
\key{end of buffer}{\\'}
\key{word break}{\\b}
\key{not beginning or end of word}{\\B}
\key{beginning of word}{\\<}
\key{end of word}{\\>}
\key{any word-syntax character}{\\w}
\key{any non-word-syntax character}{\\W}
\key{character with syntax {\it c}}{\\s{\it c}}
\key{character with syntax not {\it c}}{\\S{\it c}}

\section{Registers}

\key{copy region to register}{C-x x}
\key{insert register contents}{C-x g}

\key{save point in register}{C-x /}
\key{move point to saved location}{C-x j}

\section{Info}

\key{enter the Info documentation reader}{C-h i}
\beginindentedkeys

Moving within a node:

\key{scroll forward}{SPC}
\key{scroll reverse}{DEL}
\key{beginning of node}{. {\rm (dot)}}

Moving between nodes:

\key{{\bf next} node}{n}
\key{{\bf previous} node}{p}
\key{move {\bf up}}{u}
\key{select menu item by name}{m}
\key{select {\it n\/}th menu item by number (1--5)}{{\it n}}
\key{follow cross reference  (return with \kbd{l})}{f}
\key{return to last node you saw}{l}
\key{return to directory node}{d}
\key{go to any node by name}{g}

Other:

\key{run Info {\bf tutorial}}{h}
\key{list Info commands}{?}
\key{{\bf quit} Info}{q}
\key{search nodes for regexp}{s}

\endindentedkeys

\section{Keyboard Macros}

\key{{\bf start} defining a keyboard macro}{C-x (}
\key{{\bf end} keyboard macro definition}{C-x )}
\key{{\bf execute} last-defined keyboard macro}{C-x e}
\key{append to last keyboard macro}{C-u C-x (}
\metax{name last keyboard macro}{M-x name-last-kbd-macro}
\metax{insert lisp definition in buffer}{M-x insert-kbd-macro}

\section{Commands Dealing with Emacs Lisp}

\key{eval {\bf sexp} before point}{C-x C-e}
\key{eval current {\bf defun}}{C-M-x}
\metax{eval {\bf region}}{M-x eval-region}
\metax{eval entire {\bf buffer}}{M-x eval-current-buffer}
\key{read and eval minibuffer}{M-ESC}
\key{re-execute last minibuffer command}{C-x ESC}
\metax{read and eval Emacs Lisp file}{M-x load-file}
\metax{load from standard system directory}{M-x load-library}

\section{Simple Customization}

% The intended audience here is the person who wants to make simple
% customizations and knows Lisp syntax.

Here are some examples of binding global keys in Emacs Lisp.  Note
that you cannot say \kbd{"\\M-\#"}; you must say \kbd{"\\e\#"}.

\beginexample%
(global-set-key "\\C-cg" 'goto-line)
(global-set-key "\\e\\C-r" 'isearch-backward-regexp)
(global-set-key "\\e\#" 'query-replace-regexp)
\endexample

An example of setting a variable in Emacs Lisp:

\beginexample%
(setq backup-by-copying-when-linked t)
\endexample

\section{Writing Commands}

\beginexample%
(defun \<command-name> (\<args>)
  "\<documentation>"
  (interactive "\<template>")
  \<body>)
\endexample

An example:

\beginexample%
(defun this-line-to-top-of-screen (line)
  "Reposition line point is on to the top of
the screen.  With ARG, put point on line ARG.
Negative counts from bottom."
  (interactive "P")
  (recenter (if (null line)
                0
              (prefix-numeric-value line))))
\endexample

The argument to \kbd{interactive} is a string specifying how to get
the arguments when the function is called interactively.
Type \kbd{C-h f interactive} for more information.

\copyrightnotice

\bye

% Local variables:
% compile-command: "tex refcard"
% End:
