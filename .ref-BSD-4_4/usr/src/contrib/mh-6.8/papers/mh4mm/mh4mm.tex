\documentstyle [12pt]{article}
\parindent=0pt
\parskip=7pt plus 2pt

\def\oneline#1{\par\bigskip\leftline{\tt\hskip.75in#1}}
\def\command#1{\par\bigskip\leftline{\tt\hskip.75in\% #1}}
\def\comarg#1#2{\par\bigskip\leftline{\tt\hskip.75in\% #1 {\it#2}}}
\def\tcom#1{\par\bigskip\leftline{\tt\hskip.75in@#1}}
\def\MH/{{\sf MH}}
\def\tops20/{{\sc Tops-20}}
\def\unix/{{\sc Unix}}
\def\MM/{{\sc mm}}
\newfont{\itt}{cmti10}
\sloppy
\begin{document}

\title{MH for MM Users}
\author{Mary Hegardt \and Tim Morgan}
\maketitle

\section{Introduction}

This document is another in a continuing series on use of the \MH/ mail
system at UCI.  It is intended for those users accustomed to the \MM/ ``user
agent'' (mail program) under \tops20/ and who are already familiar with
network mail, but who may not be experienced \unix/ users.  For an
introduction to \MH/, see ``MH For Beginners'' by Mary
Hegardt and Tim Morgan.  For complete, detailed information on the \MH/
system, see {\sl The Rand MH Message Handling System: User's Manual\/} by
Marshall~T. Rose and John~L. Romine. Both documents are available for
Xeroxing in suite CS408.

\subsection{UNIX Versus Tops-20}

The \unix/\footnote{\unix/ is a trademark of AT\&T Bell Laboratories}
paradigm is that each command, or program, should perform only one
function.  An extension of this idea is that the operating system
implements only those functions which are necessary, but it does so in
a very general way so that programs may still accomplish their functions.
This philosophy probably evolved because the original versions
of \unix/ ran on PDP-11 minicomputers which had only a small memory
space for each process.

Consequently, all commands in \unix/, with a very few exceptions, are
in actuality programs.  On \tops20/, in contrast, many of the most frequently
used commands are built into the user's shell, or {\em exec.}  Both the
Exec and {\it csh,} which is typically the user's command interface on
\unix/, will accept and parse command lines, attempting to invoke a command
as a program if it is not one of the built-in commands.  \unix/ and \tops20/
are surprisingly similar internally: the use of the shell, separate
processes for each command or program to execute, standard input and output
for each program, and many other ideas are common to both operating
systems.  Users should be familiar with the capabilities of the shell,
which is described in the document ``Introduction to the Csh.''

\subsection{The MH User Interface}

The \MH/ mail ``user agent'' is different from most other mail systems.
\MM/, for example, is a {\em monolithic\/} system because one program
implements all the mail-related functions.  The disadvantages of monolithic
systems are that (a)~they are large, so they tend to put more burden on the
computer system, and (b)~they allow for much less flexibility.  In contrast,
\MH/ implements each mail command as a separate program: there is no single
program called {\it mh.}  This approach facilitates interspersing mail
commands with other, perhaps unrelated, commands.

Another unique feature of \MH/ is that it takes advantage of the facilities
provided by the operating system.  Most mail agents, such as \MM/, maintain
a file containing the user's mail in a special, usually undocumented, format.
When a message is deleted, \MM/ must take care of compacting the mail file.
It must be able to distinguish the separate messages contained in the file.
\MM/ must also implement a simple text editor to allow the user to enter
and modify a message while it is being composed.
These functions are essentially those provided by the operating system when
separate files are stored within a directory.  Therefore, the approach
taken by \MH/ is that
each message is kept in a separate file.  This file simply contains the
message, with no other special formatting characters or requirements.
All the messages are stored within a normal \unix/ directory.  This approach
makes it easy to add new \MH/ commands, to edit messages using standard
text editors, etc.

All your \MH/ related files are stored in a directory within your home
directory.  Usually this directory is called {\tt Mail} or {\tt mhbox},
although you are free to name it as you choose.  Another file in your home
directory called \verb|.mh_profile| is equivalent to the {\tt MM.INIT} file
under \MM/.  It contains all the options which you prefer for the various
\MH/ commands.  It also contains the name of the \MH/ directory and the
name that you want on your outgoing mail in the {\tt From:} field (your
``{\em signature\/}'').

\section{Getting Started}

\subsection{Incorporating Mail}

Another important difference between \MM/ and \MH/ is the concept of the
{\em maildrop\/} file.  Under \tops20/, the mail transport system delivers
new messages directly into the recipient's {\tt MAIL.TXT} file, where they may
then be processed with \MM/.  In contrast, the \unix/ mail transport system,
called {\it MMDF-II,} makes no assumptions about the user agent used by the
recipient.  Instead, it puts all new mail into a special file called the
{\em maildrop.}  This file is in the \verb|/usr/spool/mail| directory.
When you log in, if there is new mail for you in your
maildrop, you will be so notified by the message

	\oneline{You have new ZOTnet mail -- type inc (or mail)}

When you are ready to process this new mail, you may type the command

	\command{inc}

(``incorporate'') which will copy the new mail into separate files, one per
message, stored in your ``inbox'' folder.  A folder is a subdirectory
beneath your \MH/ directory which is used to store related messages.
Additional information on folders is given in Section~\ref{Folders},
page~\pageref{Folders}.  The
``inbox'' is a distinguished folder because by default {\it inc\/} will
always copy new mail into that folder, removing it from the maildrop.

If this is the first time you have used {\it inc\/} or any other \MH/
command, the {\it mh-install\/} program  will inform you that it is
creating your {\tt Mail} directory. It will also create
the ``inbox'' folder directory, and \verb|.mh_profile|.  

\subsection{Message Numbers}\label{Scan}

As {\it inc\/} processes each message, it prints a ``scan'' listing showing
the message number, the date the message was sent,
the name of the sender, the subject, and sometimes the initial part of the
text of the message.  A ``scan'' listing is thus similar to the output of
the {\tt HEADERS} command in \MM/.  Each message within a folder is given a
number, starting with 1, by which it can be referenced.  {\it
Inc\/} will display the numbers assigned to each new message in its ``scan''
listing.

As in \MM/, there is a ``current message'' number which usually identifies the
message most recently manipulated by the user.  With most \MH/ commands,
this will be the default message if no messages are explicitly specified in
a command.  {\it Inc\/} makes the first new message the current message,
which is indicated by a ``{\tt+}'' character in the scan listing, just after
the message number.

Many \MH/ commands take a list of messages to process. A message designation
is either a single message number, two message numbers separated by a dash.
The dash format indicates a range of messages including the endpoints. A
message list consists of one or more message designations separated by
spaces. For example, messages 11 through 15 and message 17 may be indicated
by typing

	\oneline{11-15 17}

as the argument to some command.  There are also several predefined names
for messages or lists of messages which may be used in place of message
numbers:

\bigskip
\begingroup\parskip=0pt
\def\titem[#1]{\par\hangafter=1\hangindent=.7in\noindent
	\hbox to\hangindent{\it\hfil#1\qquad}\ignorespaces}

\titem[cur] The current message (the last one that was handled).  Equivalent
  to ``.'' or ``{\tt CURRENT}'' in \MM/.
\titem[next] The next message 
\titem[prev] The previous message
\titem[first] The first message in the current folder.
\titem[last] The last message in the folder.  Equivalent to {\tt\%} or
  {\tt*} in \MM/.
\titem[all] All messages ($\it first-last$).  Same as in \MM/.
\endgroup

It is also possible for you to define your own named {\em sequences\/} of
messages.  See the {\it pick\/} command description for more details.

\section{Processing Messages}

This section contains a list of the common \MM/ commands and their
equivalents in the in \MH/ mail system.  A short textual note describes how
the \MH/ commands differ from their \MM/ counterparts.

\subsection{Listing Messages}

As mentioned in Section~\ref{Scan},
the {\it scan\/} command may be used to summarize the
messages in a folder, similar to the {\tt HEADERS} command in \MM/.  Unlike
most \MH/ commands, however, {\it scan\/} defaults to {\em all\/} the
messages in the current folder unless you specify one or messages on the
command line to be scanned.  So simply typing

	\command{scan}

is equivalent to typing {\tt HEADERS ALL} (or {\tt H A}) in \MM/.

\subsection{Reading Mail}

Unlike the {\tt READ} command in \MM/, in \MH/ there is no special
mail-reading mode (indicated in \MM/ by the {\tt R>} prompt).
The command to read messages in \MH/ is {\it show.} If no message list is
specified, then the current message is displayed.  The message is displayed
by your ``showproc'', as specified in the \verb|.mh_profile|, described
in Section~\ref{Tailoring}.
Normally, your ``showproc'' will be {\it more\/} or {\it mhless.}
Both of these programs will display your messages one screenful at a time.
You press the \fbox{space bar} on your terminal to see the next screenful,
or the \fbox{return} key to see the next line.

The command

	\command{show next}

(which will display the first message following the current message in the
current folder) can be abbreviated as simply

	\command{next}

Similarly, the command ``{\tt show prev}'' can be abbreviated as simply ``{\tt
prev}''.

To get a paper copy of a mail message, take the output from the {\it show\/}
command and pipe it into the {\it imprint\/} command.

	\command{show 5 | imprint}

See the manual page for {\it imprint\/} for more information.


\subsection{Deleting Messages}

The equivalent of the {\tt DELETE} command in \MM/ is {\it rmm\/} in \MH/
(remove messages).  It acts on the current message unless messages are
specified on the command line. Unlike \MM/, the deleted messages will
no longer show up in a ``scan'' listing. But the messages are not 
completely removed; they are renamed to have a comma
prepended to the name of the file containing the message within its folder
directory.  Therefore, if you need to recover a message, it is possible to
go into the directory and rename the message back.  Be careful in doing this
not to overwrite a new message with the same message number!  It is a \unix/
convention that files whose names begin with a comma will be removed from
disk ({\em expunged\/}) early each morning. Therefore, your deleted
messages will be available for the rest of the day, unless you remove another
message subsequently which has the same message.  Then the previously deleted
message is gone.

\subsection{Replying to Mail}

The equivalent of the {\tt REPLY} command in \MM/ is {\it repl\/} in \MH/.
{\it Repl\/} may be given the number of the message to which you wish to
reply, or it will default to the current message.  When replying in \MM/,
you are prompted asking if you wish to reply to all the recipients of the
message to which you are replying, or only to its sender.  In \MH/, normally
the reply will be constructed to be sent to all the recipients.  You may
select which recipients receive copies of your reply by using the {\tt
-query} option on {\it repl,} or by putting this option in your
\verb|.mh_profile|, as described in Section~\ref{Tailoring}. If you
wish a reply to go to everyone but yourself, you can 
use \verb|repl -nocc me|.

\subsection{Sending Mail}

The equivalent of the {\tt SEND} \MM/ command is {\it comp\/} (``compose'')
in \MH/.  These two commands are fairly similar, except that the recipient of
the message cannot be specified on the {\it comp\/} command line.
The {\it comp\/} program invokes a simple editor called {\it prompter\/}
which will prompt you for the {\tt To:}, {\tt Cc:},
and {\tt Subject:} fields of the message.  Then a line of dashes is typed,
and you may enter the body of your message (its {\em text,} in \MM/ terms).
When you are finished, type \fbox{ctrl}--D (equivalent to typing \fbox{ESC} or
control-Z in \MM/).  Then you'll receive the prompt

	\oneline{What now?}

which is similar to \MM/'s {\tt S>} prompt.  You may receive a list of the
options that you have at this point by typing ``{\tt?}'' followed by
\fbox{return}.  Here is a short list of the options and their meanings.
Notice that, unlike \MM/, there are very few commands to modify the message
(such as the {\tt TEXT}, {\tt TO}, {\tt CC}, etc., commands which may be
typed at the {\tt S>} prompt in \MM/).  In place of these commands, you use
the {\tt edit} command to invoke your favorite text editor on the message,
and you use it to make the equivalent changes.  You also use your editor
to include other files into the body of the message, rather than using
control-B, as in \MM/.  One additional use of the {\tt edit} command is
for spelling checking.  In \MM/, you may use the command {\tt SPELL} 
for this purpose.  
In \MH/, you type ``{\tt edit spell}''\footnote{ Actually, any
program named after the ``{\tt edit}'' command will be invoked with
whatever arguments you have given and a path to the file containing
the message you are editing.} instead.  This will
cause the spelling checker to be run, giving you a list of the possibly
misspelled words in your message.

\bigskip

\begingroup
\def\titem[#1]{\par\hangafter=1\hangindent=1.4in\noindent
	\hbox to\hangindent{\hfil#1\qquad}\ignorespaces}

\titem[\tt edit \it editor\/] Edit the message using the specified
				editor.  When you exit, you will be
				back at \verb|What now?|.

\titem[\tt list] Shows the message you just typed

\titem[\tt whom -check] Verifies that the addresses you have
				used are valid as far as our system
				can tell

\titem[\tt send] Sends the message to the recipients

\titem[\tt push] Sends the message in the background

\titem[\tt quit] Quits without sending the message.
				Saves the text of the message as
				a ``draft''. Type \verb|comp -use| to
				get back to that draft later.

\titem[\tt quit -delete] Quit, throwing away the draft
\endgroup

\bigskip

\subsection{Forwarding Mail}

The {\it forw\/} command is used in \MH/ to forward messages.  It will take
a list of messages on the command line to be forwarded, or it will default
to the current message if none are specified.  It will prompt you like {\it
comp\/} does for the {\tt To:}, {\tt Cc:}, and {\tt Subject:}\ fields.  Note
that, unlike \MM/'s {\tt FORWARD} command, {\it forw\/} will not construct a
subject line automatically.  Also as with {\it comp,} you will have the
opportunity to add additional text to the message(s) which you are
forwarding, ended with a control-D.  

\subsection{Resending Mail}

The equivalent of the {\tt RESEND} command in \MM/ is the {\it dist\/}
(``distribute'') command in \MH/.  {\it Dist\/} works very much like the
{\it forw\/} command, except that the prompts will be {\tt Resend-To:},
{\tt Resend-Cc:}, etc. After filling in the headers, a line of dashes
is typed giving the impression that additional text can be entered.
Nothing could be further from the truth; if you add any text at this
point the {\it dist\/} will fail.  Your only opportunity to add text
is in the {\tt Resend-Note:} field.

\section{Advanced Topics}

\subsection{Selecting Messages}

In \MM/, you may use several reserved command words to select messages
in place of an explicit list of message numbers.  For example, you can
type ``{\tt DELETE FROM SMITH}'' to remove all the messages from a user named
``Smith''.  Rather than building such a capability into each \MH/ program
which can process message lists, a special program called {\it pick\/} is
used instead.  Just as there are predefined sequences of messages, such as
``{\tt all}'', ``{\tt cur}'', etc., you may use {\it pick\/} to define your
own sequences.  {\it Pick\/} is capable of selecting messages from a folder
based on the {\tt To:}, {\tt From:}, {\tt Subject:}, {\tt Cc:}, or {\tt
Date:} fields, or by searching the body of the message.  The patterns to be
searched for may include full regular expressions (see the ``man'' page for
{\it ed(1)\/} for more information) or simple strings.

{\it Pick\/} may be used in one of two ways.  First, it may output the
sequence of message numbers which match the search parameters.  Using the
backquoting mechanism of the shell, these message numbers may then become
the arguments to other \MH/ programs.  The second way to use {\it pick\/} is
to have it define a new sequence name which will be the messages which were
selected.  Only this second method of using {\it pick\/} will be described
here; see {\it pick(l)\/} if you wish to use the first method.

In your \verb|.mh_profile|, add the line

	\oneline{pick: -seq sel}

Then each time you use the {\it pick\/} command, it will define the
resulting sequence of messages to be called ``sel''.  Then to ``pick'' all
the messages in the current folder which are from ``Smith'', just type

	\command{pick -from smith}

To see a summary of those messages, type

	\command{scan sel}

Then to the remove the messages, type the command

	\command{rmm sel}

You can {\it pick\/} messages according to any of the headers ({\tt -to
-from -subj -cc {\rm or} -date}) or just search all the messages for a given
word ({\tt -search}).

\subsection{Customizing Your Mail Environment}\label{Tailoring}

In \MM/, you use the {\tt PROFILE} command to tailor your mail environment.
This command writes a file called {\tt MM.INIT} in your home directory which
is then read by subsequent executions of \MM/.  In the \MH/ system, the file
\verb|.mh_profile| serves the same purpose.  It is edited with any normal
text editor, rather than using a special-purpose command to modify it.  The
format of the file is line oriented, one line per \MH/ program or \MH/
option to be set.  The only required line in the profile is the name of the
primary \MH/ mail directory, which is by default {\tt Mail}. This
information is specified by the line

	\oneline{Path: Mail}

The textual name you would like to have on your outgoing mail is specified
by the {\tt Signature:} line.  For example,

	\oneline{Signature: Mary Hegardt}

The BBoards which you like to read should also be listed in the
\verb|.mh_profile| (see Section~\ref{BBoards}, page~\pageref{BBoards}, for
additional information).  For example, if you read the ``system'' BBoard
(where all important announcements are posted), as well as ``whimsey''
and ``imagen-users'' BBoards, your \verb|.mh_profile| should contain the line

	\oneline{bboards: system whimsey imagen-users}

Other options may be specified on a per-program basis.  The format for these
lines is the same.  First, the program name is given followed by a colon.
Then any flags which are to be the default options for that program are
given.  Here is a short list of the most common options which you may want
to set in your \verb|.mh_profile|:

	\oneline{showproc: mhless}

The {\it showproc\/} is the program used to show messages to you.  By
default, it is the {\it more\/} command.  {\it Mhless\/} is the same as
{\it more\/} except that it omits the headers of the messages which you
indicate that you wish not to see.  Type

	\command{man mhless}

for more information about this program.

	\oneline{msh: -scan}

Selecting this option causes an automatic scan of new messages on BBoards to be
made when reading BBoards with {\it bbc,} similar to the scan listing
produced by {\it inc.}

	\oneline{repl: -query}

causes {\it repl\/} to ask for each address in the message being replied to
if it should be included in the {\tt To:} or {\tt Cc:} fields of the reply
being composed.

	\oneline{pick: -seq sel}

This line will cause messages ``picked'' by the {\it pick\/} command to be
put into a sequence named ``sel''.  This sequence name may then be used
just as the built-in sequences (``last'', ``first'', etc.).

\subsection{Aliases}

Using \MH/, you may specify your own private mail aliases.  This feature
allows you to store lists of addresses or long internet addresses of people
with whom you frequently correspond in one file, and then to address them
using short mnemonic names.  Typically, you will call your alias file ``{\tt
aliases}''; it must be stored in your \MH/ directory.  The format of this
file is simple. The alias is given, followed by a colon, followed by one or
more legal mail addresses separated by commas.  For example, you might for
some reason have an alias for all the users named ``Rose'' in the ICS
department:

	\oneline{roses: prose, srose, mrose, drose}

In addition to your ``{\tt aliases}'' file, you will need to modify your
\verb|.mh_profile| in order to use aliases.  You should add the flag
``{\tt -alias aliases}'' to the entries for the commands {\it ali, whom,
send,} and {\it push,} creating entries for these programs if they aren't
already in your \verb|.mh_profile|.
Now, messages addressed to ``{\tt roses}'' will be distributed to all
the people listed in the alias.

The {\it ali\/} command is used to show you what an alias expands to.
You just type

	\comarg{ali}{alias}

and {\it ali\/} will respond with the expansion of the {\it alias.}  {\it
Ali\/} searches the system aliases file in addition to your private ones.

\subsection{Blind Lists}

There are two different types of so-called ``blind addressing'' of messages.
Users of \MM/ may already be familiar with the ``Blind Carbon Copy'', or
{\tt BCC:} field.  It allows you to add recipients to your message just
like those who are CC'd, but the normal recipients will not see that the
BCC recipients were copied on the message, their replies will not go to the
blind recipients, and the blind recipients cannot (easily) reply to the
message.

The second type of blind mailing is actually called a ``group address list'',
although it is commonly referred to as a ``blind list''.
The format of this type of address is

	\oneline{{\it phrase\/}: {\it address\_list\/};}

where the ``{\it phrase\/}'' is any English phrase of one or more words,
and the {\it address\_list\/} consists of one or more addresses separated by
commas.  The recipients of a message addressed in this fashion will
see simply

	\oneline{{\it phrase\/}: ;}

so when they reply to the message, their reply will come only to the sender
(or the {\tt Reply-To:} field, if one was specified), rather than going to
all the recipients of the original list. For example, to use a group
address list for the ``{\tt roses}'' alias you would type:

	\oneline{To: People Named Rose: roses;}

This type of group address is very
useful for making up lists of related people, such as all the people working
on a particular research project.

\subsection{Folders}\label{Folders}

As mentioned previously, folders are directories within your \MH/ directory
used to store related messages.  There is no equivalent of the folder
concept in the \MM/ system. Usually, you will use only the folder ``inbox'',
so you won't have to worry about folders.  However, if you process a large
volume of mail, then folders become invaluable in managing the messages
which you wish to keep for future reference.

Just as there is a ``current message,'' \MH/ maintains a ``current folder,''
which will normally be ``inbox''.  You can change folders either by
specifying the folder on the command line of \MH/ programs which take a
list of messages as an argument, or by using the {\tt folder} command:

	\command{folder +{\it folder\_name}}

In general, the folder name is indicated by a ``{\tt+}'' sign followed
immediately by the folder name, all preceeding any list of messages. For
example, you may read the most recent message in a folder called
``job\_offers'' using the command

	\command{show +job\_offers last}

This command will have the side-effect of setting the current folder to be
``job\_offers''. You may move messages from the current folder into the
``job\_offers'' folder using the command

	\comarg{refile +job\_offers}{messages}

where, as usual, the {\it messages\/} list will default to the current
message in the current folder if none are specified.  Note that, in contrast
with the {\it show\/} command and most other \MH/ commands, the {\it
messages\/} are {\em not\/} considered to be in the folder ``job\_offers''.
You may obtain a summary of all your folders by typing the command

	\command{folders}

When you remove messages from a folder using the {\it rmm\/} command,
the deleted messages will show up as a ``hole'' in the message numbering.
The command

	\command{folder -pack}

will cause all the messages within one folder to be renumbered starting
with~1. Similarly, the command

	\command{folders -pack}

will do the same thing for all your folders.

To remove an empty folder, use the command

	\command{rmf +{\it folder}}

\subsection{Reading BBoards}\label{BBoards}

Two special-purpose programs are utilized in reading BBoards.  The first is
{\it bbc,} which is used to check a list of BBoards for new messages.
The list of messages may be given on the command line, or if not, it will be
taken from the {\tt BBoards:} list in your \verb|.mh_profile|.  You may
obtain a list of all the available BBoards by typing the command

	\command{bbc -topics}

For each BBoard with unseen messages, {\it bbc\/} will invoke the \MH/
shell, {\it msh,} whose prompt is

	\oneline{(msh)}

The {\it msh\/} program allows you to read BBoard mail as if it were normal
messages in one of your folders.  Almost all the \MH/ commands will work
just as the normally do.  Typing the command ``{\tt quit}'' to {\it msh\/}
causes it to stop reading the current BBoard and go on to the next one
containing unseen messages, or to exit if there are no more such BBoards.
Typing control-D causes {\it msh\/} to exit unconditionally.  The command
{\tt mark} followed by a message number causes {\it msh\/} to act as if you
have seen that message and all previous ones.

\subsection{Checking for Mail}

Under \unix/, there are about a dozen different ways to check for new mail.
The easiest way to do it is to set the {\it csh\/} variable named ``mail''
to tell {\it csh\/} to check for new mail for you periodically.  To do this,
add the line

	\oneline{set mail=(60 /usr/spool/mail/\$USER)}

to your {\tt .login} file in your home directory.  This command says to
check for mail if {\it csh\/} is about to prompt you with a {\tt\%} sign,
and if it has been at least 60 seconds since it last checked for mail.
The advantage of this method of mail notification, besides simplicity, is
that you will never be interrupted by a mail notification.  You will only be
notified of new mail when you are between commands, when the shell is about to
prompt you.

If you desire asynchronous mail notification, which will print to your
terminal regardless of what you are currently doing, you may make use of a
``Receive Mail Hook'' called ``rcvtty''.  To do this, create a file in your
home directory called ``{\tt .maildelivery}''.  In this file, put the line

	\oneline{* - pipe R /usr/uci/lib/mh/rcvtty}

Then each time new mail arrives, you will receive a one-line ``scan''
listing of the mail if your terminal is world-writable. For more information
on ``maildelivery'' files, type:

	\command{man 5 maildelivery}


\subsection{Saving Drafts}

Normally when you use {\it comp,} it creates the message being composed in a
file called ``{\tt draft}'' in your \MH/ directory. If you use the ``{\tt
quit}'' option at the ``{\tt What now?}'' prompt, this file will remain
there.  You may later use the command

	\command{comp -use}

to resume composing the message.

If you begin composing a new message and there is already a ``draft'' file,
you will be asked for the disposition of this file.  Typing {\tt?}
\fbox{return} will give you a list of the options at this point.  Normally
you will either {\tt replace} (delete) the old draft and begin
a new one or {\tt use} the old one.

The {\tt -file} switch to {\it comp\/} may be used to specify the name of a
draft other than ``{\tt draft}''.  For example, one might type

	\command{comp -file mary}

to begin composing a message maintained in the draft file ``{\tt mary}''.
Typing

	\command{comp -file mary -use}

would cause {\it comp\/} to resume composing this same draft after a
``{\tt quit}'' command to the ``{\tt What now?}'' prompt.

Very advanced users of \MH/ maintain multiple draft files in a {\it draft
folder.}  This is a normal folder which holds all your drafts, rather than
having just one draft in your \MH/ directory named ``{\tt draft}''.  If you
feel that you need to use draft folders, you should consult the {\sl MH
User's Manual\/} for additional information.

\end{document}
